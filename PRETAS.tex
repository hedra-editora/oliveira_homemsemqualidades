
\textbf{Robson José Feitosa de Oliveira} nasceu numa família camponesa numa pequena cidade serrana do Ceará, de onde saiu aos 16 anos para estudar em Fortaleza. É tradutor e professor da Casa de Cultura Francesa da Universidade Federal do Ceará. Tendo estudado Letras, mas também Psicologia Social, seus estudos buscam aprofundar uma crítica categorial da forma"-social mercantil numa relação dialética com o processo de subjetivação na modernidade.


\textbf{\emph{O homem sem qualidades} à espera de Godot} encerra em seu título uma tese em ciências humanas, fruto do caráter irrequieto de um autor que não vislumbra na marcha incessante da Modernidade o movimento rumo a uma pretensa perfectibilidade humana, mas muito mais um movimento secular de subsunção da vida social e subjetiva concretas respectivamente pela forma"-mercadoria e a forma"-sujeito burguesa: o tornar"-se mercadoria do mundo representa o tornar"-se vazio da própria subjetividade plasmada à segunda natureza, onde reina o \emph{a priori} tácito das leis mercantis que dominam nossos atos mais banais.

