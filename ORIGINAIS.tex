%\emph{RESUMO}

%OLIVEIRA, R.J.F. \emph{O} homem sem qualidades \emph{à espera de Godot.
%Molière, Musil, Beckett,} Macunaíma \emph{e o devir vazio tautológico na
%modernidade} 2016. 323 f. Tese (Doutorado em Psicologia Social) --
%Instituto de Psicologia, Universidade do Estado do Rio de Janeiro, Rio
%de Janeiro, 2016.

%O nosso objetivo nessa tese é discutir o quanto o processo de
%entronização da mercadoria e sua lógica na vida social podem significar
%uma tendência destrutiva para a vida social concreta e para a própria
%subjetividade. Nossa tese é de que a forma-social mercantil é uma forma
%de sociedade \emph{ideal} que procura se realizar na história concreta
%avançando sobre as formas de vida não fundadas na mercadoria. Do mesmo
%modo que a forma-sujeito burguesa é uma forma de subjetividade
%\emph{ideal} que procura se realizar nas subjetividades concretas,
%tentando colonizar de forma cada vez maior os recantos conscientes e
%inconscientes e criar uma subjetividade cada vez mais em sintonia com a
%dinâmica mercantil. Dito de outra forma, trataremos de refletir em que
%medida o tornar-se mercadoria do mundo é também o tornar-se vazio da
%subjetividade. Para nosso intento, fazemos um diálogo com algumas obras
%literárias para refletir o quanto esse processe chamado de modernização
%objetiva e subjetiva se expressou indireta e inconscientemente nessas
%obras. Partimos de uma análise do paulatino aumento das trocas mercantis
%no século XI, quando uma forma subjetividade fundada no cálculo vai
%começar a vagarosamente tensionar com a subjetividade calcada no
%universo simbólico-religioso. A análise desses primeiros tensionamentos
%que demorarão séculos para ganhar forma -- pelo fato de a sociedade ser
%fundada ainda num universo simbólico não-mercantil -- vão nos levar ao
%teatro de Molière, especificamente a três peças em que o tensionamento
%entre o ideal de subjetividade nobre e a nova forma de subjetividade
%burguesa estão em tensão. Depois, fazemos uma reflexão sobre a
%racionalidade e a forma-sujeito modernas para desembocar numa análise de
%alguns elementos na obra de Musil, \emph{O homem sem qualidades.} Nela,
%refletimos sobre um novo entrecruzamento no alvorecer do século XX entre
%uma subjetividade ancorada no passado e outra que pretende avançar para
%o futuro. Em seguida, tentamos nos perguntar sobre o caminho percorrido
%pela forma-social e a forma-sujeito burguesas no Brasil, um caminho
%sempre difícil pelas vicissitudes que a vida social concreta impunha.
%Discutiremos criticamente o movimento chamado de Modernismo paulista e o
%quanto algumas ideias desse movimento desembocaram numa espécie de
%síntese mal-acabada que chamamos de \emph{indigeno-burguesa.} Na última
%parte, fazemos uma reflexão sobre o mundo contemporâneo em que a
%mercadoria passou a figurar não apenas como fundamento da vida social
%enquanto mediador inexorável da sobrevivência, mas também enquanto
%pretensa portadora de atributos subjetivos passíveis de serem comprados
%pelos sujeitos que, num processo de mímese, se veriam embebidos do poder
%desse objeto fetiche. Refletimos sobre uma sociedade em que a mercadoria
%deixa um rastro de ruína não só objetivamente, na natureza e nas
%relações sociais, mas também subjetivamente. É nesse sentido que
%refletimos sobre três peças de Samuel Beckett como um teatro que põe em
%cena uma vida que gira em falso, cujo sentido pena para se sustentar.

%Palavras-chave: Fetichismo da mercadoria, forma-sujeito burguesa,
%subjetividade, literatura.

\chapter*{}

\vspace*{\fill}

\thispagestyle{empty}
\begin{flushright}
\emph{A Célia Zanetti}
\end{flushright}

\chapter{Prefácio}


\begin{flushright}
\emph{Anselm Jappe}
\end{flushright}

Não há novidade nenhuma em se afirmar que vivemos hoje uma situação de
crise permanente e que a crise ecológica e a crise econômica -- a
devastação das bases naturais da vida e a pobreza crescente em sociedade
-- convivem numa atmosfera de catástrofe cada vez mais próxima. Enquanto
as ameaças parecem se renovar de modo incessante e nós nos vemos diante
de perigos de cuja existência nem tínhamos conhecimento até uma data
recente -- como as mudanças climáticas -- e que imaginávamos
completamente superados -- como os movimentos políticos de tonalidade
fascista -- o pensamento que deveria fazer frente a tais ameaças não se
renovou com a mesma velocidade e o mesmo vigor durante as últimas
décadas (é mínimo que se pode dizer). No mais das vezes, intentou"-se
compreender situações historicamente inéditas com as categorias herdadas
do marxismo tradicional e do liberalismo, da teoria do desenvolvimento
ou do subdesenvolvimento, da justiça social redistributiva e da
democracia representativa.

Entre as raras tentativas de repensar globalmente o que está nos
acontecendo, encontramos a ``crítica do valor''. Ela consiste numa
crítica radical do valor mercantil e do dinheiro, do trabalho e da
mercadoria, do Estado e do fetichismo da mercadoria que constituem as
categorias centrais do capitalismo desde seus primórdios. A crítica do
valor também analisa a crise irreversível na qual essas categorias se
encontram hoje. Trata"-se de uma abordagem inspirada em Marx, mas de modo
algum ``ortodoxa''. Nascida na Alemanha nos anos 1980 (e de maneira
similar, mas independente, nos Estados Unidos com a obra de Moishe
Postone) em torno da revista \emph{Krisis}, a crítica do valor teve uma
repercussão particularmente importante no Brasil. Seu principal autor,
Robert Kurz, sempre esteve presente na imprensa escrita e nas universidades
brasileiras nos anos 1990 -- antes que o \emph{boom} dos anos Lula
criasse a sensação de que estavam equivocados aqueles que falavam de uma
crise fundamental e definitiva do capitalismo. Em seguida, a crise
voltou, no Brasil mais forte ainda que em outros cantos -- e agora resta
saber se o pensamento crítico também vai despertar.

A crítica do valor, que é uma abordagem, um método, um paradigma, e não
uma ``escola de pensamento'', já deu inúmeros frutos. Por um lado,
dezenas de livros e centenas de artigos escritos por Robert Kurz
(falecido em 2012) e outros autores. A maioria foi redigido em alemão,
às vezes em francês e inglês; mas o português é -- de longe -- a língua
na qual encontramos mais traduções. Por outro, a parte lusófona do mundo
é também a região onde se pode encontrar o maior número de elaborações
e continuações originais dessa abordagem. As revistas, as publicações,
os blogs, os grupos de discussão e os cursos universitários são muitos.
Não se trata somente de traduções e de divulgação, mas também de novos
estudos, tanto sobre temas relacionados especificamente ao Brasil como
também sobre outros temas. O fato de podermos notar amiúde uma tendência
a combinar a crítica do valor com outras abordagens, com frequência
oriundas do marxismo tradicional, deve ser considerado como o destino
comum de teorias muito divulgadas.

Não se trata apenas de um aumento quantitativo das pesquisas baseadas na
crítica de categorias aparentemente tão ``naturais'' como são a
mercadoria e o valor, o dinheiro e principalmente o trabalho em sua
dupla natureza (abstrato e concreto). O aumento é também qualitativo:
novas esferas da vida, do que se costumou chamar por convenção de
``ciência humanas'', são submetidas a análises do tipo
\emph{wertkritisch.}

A crítica do valor nasceu historicamente como retomada da ``crítica da
economia política'' de Marx: não se trata de teoria econômica, mas de
uma crítica do totalitarismo econômico, da dominação total da economia
mercantil sobre a vida, algo que caracteriza intrinsecamente o
capitalismo. Não se tratava em absoluto de se limitar a ``análises
econômicas''. Mas de todo modo a crítica do valor, e as contribuições de
Robert Kurz em especial, privilegiaram durante um longo período a
análise da vertente econômica da crise do capitalismo e suas
consequências políticas, assim como um confronto com a obra de Marx e
seus intérpretes. A introdução dos conceitos de ``forma"-sujeito'' e
``valor"-dissociação'' no arsenal da crítica do valor ampliaram em
seguida os horizontes. Mas ainda faltava quase que completamente um
aspecto essencial para toda e qualquer teoria com pretensões globais:
pesquisas sobre a literatura, as artes, a música -- em suma, sobre a
chamada ``cultura'' em sentido estrito.

Tal tarefa era tão necessária quanto difícil. Difícil porque é preciso
medir forças com o peso da tradição marxista nesse terreno. Marx e
Engels voltaram"-se muito pouco para as questões culturais, limitando"-se
muito mais a algumas amostras. Mas o caminho havia sido aberto: era o
materialismo histórico. As criações culturais, principalmente aquelas
que caracterizam realmente uma época, seriam o ``reflexo'' dos conflitos
de classe dessa dada época. É o esquema bem conhecido de ``base'' e
``superestrutura'', do ``ser'' \emph{(Sein} em alemão) que determina a
consciência (Bewußtsein). Entretanto, os próprios fundadores dessa
abordagem tinham também indicado a possibilidade de uma relação não
mecânica entre esses fatores e de uma ``autonomia relativa'' das
superestruturas. A partir dos anos 1920, quando as ideias de Marx foram
adotadas fora do movimento operário e começaram a encontrar as outras
ciências humanas, inúmeros autores utilizaram, de maneira mais ou menos
``ortodoxa'', os métodos de Marx e de Engels, e às vezes também seus
comentários no mínimo sucintos sobre algumas obras culturais, para
elaborar uma teoria marxista da literatura. G. Lukács e J.-P. Sartre, H.
Lefebvre e Lucien Goldmann, Theodor Adorno e Walter Benjamin, E. Bloch e
H. Marcuse, L. Althusser e F. Jameson estão entre os representantes mais
conhecidos desse debate (sem contar os estudos realizados nos países do
Leste, que são muitas vezes as coisas menos desinteressantes que ali
foram feitas, sem falar de M. Bakhtin e sua escola). Frente a frente,
principalmente os defensores do ``realismo'', como Lukács, e os autores
mais abertos às experiências da literatura moderna e experimental, como
Adorno (que não se definia como ``marxista'' em sentido estrito). O
debate não dizia respeito aliás somente à interpretação a ser dada a
obras passadas e presentes, mas assumia também, principalmente entre os
``ortodoxos'', um forte valor normativo: tratava"-se de determinar em que
deve consistir uma literatura ``revolucionária'' ou ``socialista'',
chegando"-se ao ponto nos países ``comunistas'' de se proibirem obras
literárias por não serem suficientemente comunistas. A história desses
debates, que continuaram até os anos 1970 para se interromper
bruscamente em seguida, é bastante rico. No Brasil, as discussões de
teoria literária mais ou menos influenciadas por Marx jogaram um papel
especialmente grande. Porém, o aporte de Marx nesses debates -- quer
fosse preponderante, como em Lukács, ou somente um elemento entre
outros, como em Adorno -- consistia inevitavelmente em assumir que a
literatura traduz os conflitos sociais de uma época, as estratégias dos
atores sociais, os esforços de emancipação. Para tudo que não entrava
nesse esquema, os marxistas não"-ortodoxos pediam o socorro de outras
ciências, como a semiótica.

A crítica do valor revisitou de cabo a rabo a herança de Marx e
introduziu a distinção capital entre o ``Marx esotérico'' (teórico do
fetichismo da mercadoria, que se exprime principalmente no primeiro
capítulo do \emph{Capital} e cujas análises, que tocam no coração mesmo
da sociedade mercantil, são mais atuais do que nunca) e o ``Marx
exotérico'', que colocou no centro de suas análises a luta de classes
tais como existiam em seu tempo. Mas que olhar devemos lançar para a
esfera cultural? A questão durante muito tempo não mereceu muita atenção
por parte das publicações inspiradas pela crítica do valor. Teria sido
possível investigar as descrições literárias em relação à resistência
popular ao trabalho, tema evidentemente negligenciado pelo marxismo
tradicional que se fixa no papel positivo do trabalho. Na literatura
brasileira, aliás, encontramos bons exemplos, aos quais também faz
alusão o livro de Oliveira. Mas é fácil entender que uma tal perspectiva
só alcançaria um campo limitado.

É aqui que o conceito de ``forma"-sujeito'' mostra todo seu poder
heurístico. A partir do começo dos anos 1990, a crítica do valor
aprofundou progressivamente o fato que o ``sujeito'' não é um dado
supra"-histórico que se viu colonizado pelo capitalismo. O sujeito é ele
próprio uma forma fetichista: a forma"-sujeito constitui uma ``forma a
priori'' como são o valor, o trabalho abstrato, o dinheiro. Uma forma
historicamente determinada, mas inconsciente e que estrutura os
comportamentos e os pensamentos das pessoas -- dos ``sujeitos'' -- à sua
revelia. O sujeito não é o polo contrário da dominação capitalista, ele
nasceu e se desenvolveu juntamente com ela. A forma"-sujeito é
aparentemente abstrata e vazia de conteúdo, assim como o valor criado pelo
lado abstrato do trabalho. Ela considera o mundo somente pelo prisma da
quantificação e da abstratificação de todo e qualquer conteúdo. Ao mesmo
tempo, seu caráter abstrato esconde o fato de que a forma"-sujeito está
essencialmente ligada a uma figura histórica precisa: o homem branco,
macho e ocidental que conquistou o mundo e submeteu a natureza a partir
do século \versal{XV}. O sujeito não é idêntico à ``pessoa'' ou ao ``indivíduo''.
Ele é o indivíduo que vestiu a forma"-sujeito como se veste uma camisa de
força ou como se entra num leito de Procrusto. Mas esse sujeito, e sua
forma, são vieram ao mundo já com todas as suas formas delineadas. Eles
tiveram sua história, uma história que ainda continua.

É o grande mérito do livro de Robson de Oliveira, ter empreendido a
primeira análise de diferentes episódios da história da literatura
mundial relacionando"-os com as etapas do desenvolvimento da
forma"-sujeito. Ele fala de uma verdadeira ``dupla acumulação
primitiva'': objetiva -- a subordinação da vida social à lógica do
capital, de forma crescente e incessante durante a modernidade -- e
subjetiva: a importância crescente da abstratificação e da indiferença
(o autor retira de Georg Simmel o conceito de ``blasé'') nas estruturas
psíquicas dos portadores dessa modernidade. Isso diz respeito, deve"-se
sublinhar, a todas as classes sociais, embora nem sempre da mesma forma:
o ``sujeito burguês'' é uma categoria mais ampla do que apenas a classe
burguesa. De maneira bem convincente, Robson de Oliveira mostra que a forma"-sujeito,
que é uma pura abstração, se constitui pelo fim da Idade Média na
Europa, paralelamente à emergência do dinheiro, e em relação estreita
com uma nova maneira de conceber o tempo, em prelúdio à sua futura
abstratificação e aceleração. Mas esse sujeito abstrato permanece
durante séculos mesclado às formas concretas de socialização e não se
liberta senão gradualmente de sua imbricação e de seus compromissos com
as formas pré-burguesas e concretas (que aliás não são necessariamente
melhores, como Oliveira chama a atenção).

Ao falar do teatro de Molière, o livro de Oliveira evoca a oposição
entre a velha aristocracia e a nova burguesia, rica em dinheiro mas
pobre em cultura e \emph{savoir"-vivre.} Essa oposição, mil vezes
analisada em termos históricos e sociológicos, é apresentada por Robson
de Oliveira como duas etapas de um conflito que vai ser o esteio da
evolução do sujeito moderno: uma que remete a esse sujeito moderno em
vias de formação (a burguesia) e outra que remete para uma mentalidade
ainda estruturada pelas formas pré-modernas de socialização. Quase três
séculos depois, \emph{O homem sem qualidades} (1930-1942) de Robert
Musil representa para Robson de Oliveira um estágio histórico em que o
sujeito abstrato, o sujeito da mercadoria e do dinheiro, destinado à
pura quantificação, chega a uma vitória quase completa e não se defronta
senão com os últimos restos de uma mentalidade pré-capitalista. A
ausência do sentido de limites, o desaparecimento da dimensão simbólica
nos atos cotidianos e nas trocas, onde o dinheiro toma o lugar do dom (no
sentido de Marcel Mauss), e o desaparecimento da experiência
transmissível, que tanto preocupava Walter Benjamin, contam entre suas
características fundamentais. Evidentemente, tanto nesse como em outros
casos citados, não são as intenções explícitas dos autores que contam,
mas o que se pode haurir de suas obras enquanto testemunhos ou
sintomas.

O livro de Oliveira é também precioso por mostrar que o arcabouço dessa
análise pode igualmente dar resultados importantes no que diz respeito
ao Brasil e seus aspectos particulares, que não são de modo algum um
simples prolongamento de situações europeias. Ele constata ali, como
outros já o fizeram antes dele, a ausência de uma classe burguesa e de
sua subjetividade. Mas em vez de lamentar e desejar uma ``modernização''
das consciências como pressuposto necessário do ``progressos social'',
de Oliveira lembra -- citando também suas experiência pessoais no sertão
-- as devastações produzidas por essa modernização -- sem entretanto
idealizar, nem nesse caso nem em qualquer outro em seu livro, as formas
sociais pré-capitalistas. Nesse contexto, ele mostra também a
importância da cooptação de formas inicialmente de contestação para a
própria renovação do capitalismo. Criticando Antônio Cândido, ele aponta
as ambiguidades da ``superação'' da subjetividade burguesa: a
malandragem não aparece finalmente senão como uma via alternativa à
sociedade de concorrência. Essa via era menos ``eficiente'' na época do
``primeiro espírito do capitalismo'', weberiano, puritano e nórdico, mas
agora vive sua revanche com o ``terceiro espírito'' (Luc Boltansky),
pós"-moderno, narcísico e globalizado. Essa excursão nas particularidades
da forma"-sujeito brasileira em relação às formas europeias pode aliás
explicar -- será que é preciso lembrar? -- muitos elementos da história
recente do país. E será interessante -- embora não agradável -- ver qual
será a importância relativa do componente liberal"-pós"-moderno e do
componente autoritário e conservador nas novas formas de dominação
social que estão se configurando no Brasil e alhures.

É importante destacar que essas interpretações são muito originais; é
difícil encontrar algo comparável na literatura crítica. Além do mais, a
utilidade dessas pesquisas é dupla: o conceito de forma"-valor permite
jogar uma nova luz sobre a história da literatura e das formas de
consciência em geral. Ao mesmo tempo, os fenômenos culturais constituem
um ótimo prisma para melhor distinguir e compreender as etapas da
evolução da forma"-sujeito. É do mesmo modo digno de nota que tais
análises podem contribuir a ir além da oposição, tão velha quanto
inútil, entre ``materialismo'' e ``idealismo'', entre ``ser'' e
``representações''. O conceito de fetichismo se situa com efeito para
além dessa dicotomia do qual o marxismo tradicional fez um de seus
cavalos de batalha.

O último autor tratado neste livro é Samuel Beckett com \emph{Esperando
Godot} (1952), \emph{Fim de Partida} (1957) e \emph{Dias Felizes}
(1961). Embora \emph{Godot} esteja mais próximo, em anos, da obra de
Musil do que de nossa época, temos a impressão de que Musil descreve um
mundo findo, enquanto que o mundo de Beckett parece ser de uma
desconcertante atualidade. Aqui, o sujeito vazio e sem conteúdo triunfou
sobre tudo que pertence ao mundo concreto e reina soberano. Poderíamos
ver aí uma realização integral da forma, uma realização definitiva do
sujeito moderno: mas como muito bem mostra a obra de Beckett, o triunfo
do sujeito mercantil coincide necessariamente com seu desabamento. Os
homens mutilados de Beckett não constituem a negação do sujeito burguês,
mas sua realização. Exatamente como a vitória da forma"-valor na vida
econômica e social coincide com sua ruína. Com efeito, a abstração só
pode viver às expensas do concreto; se ela conseguir devorá-la
completamente, perde sua própria base. O sujeito realizado destrói a si
mesmo tanto quanto o capitalismo que aboliu quase todas as formas de
vida pré-capitalista provoca sua própria crise: o valor se mostra desvairado perante
todo e qualquer conteúdo e não pode senão devastar o mundo social e
natural. Beckett pinta um quadro sem piedade do \emph{waste land} do
capitalismo conseguindo coincidir consigo mesmo. Como Oliveira sublinha,
Beckett não evoca de modo algum uma situação existencial ``absurda''
atemporal que caracterizaria o ser humano enquanto tal, nem, ao
contrário, descreve apenas a situação do pós"-guerra na Europa, mas
anuncia precisamente o estágio final do sujeito burguês -- que se tornou
a forma"-sujeito de todos os membros das sociedades modernas, sem grandes
diferenças entre os grupos sociais.

Aqui, o círculo se fecha. A crítica do valor, depois das análises
literárias, retorna à análise do mundo contemporâneo, que era seu ponto
de partida, e que não consiste de forma alguma numa consideração
deslocada, mas num grito de alerta. A crise, o declínio e a
autodissolução do sujeito com toda segurança não fazem parte de um
processo pacífico que dará lugar automaticamente a formas melhores -- do
mesmo modo que o desabamento gradual do capitalismo não implica
necessariamente a passagem para uma sociedade emancipada. Ele apenas
abre a possibilidade. Compreender a evolução histórica da forma"-sujeito
e a catástrofe final à qual ela conduz não serve apenas para entender a
literatura, mas também os desafios que estão à nossa espera.

Quer se trate da incapacidade geral de reagir contra a catástrofe
climática -- incapacidade que assume hoje dimensões suicidas -- ou do
aumento de crimes absolutamente irracionais como os massacres nas
escolas ou outros lugares públicos (school shooting, amok), quer se
trate do aumento contínuo do narcisismo e outros distúrbios psíquicos ou
da propagação do ódio sob forma de racismo e feminicídio, de guerra
contra os pobres e de nostalgia da tortura e da pena de morte: trata"-se
de fenômenos cotidianos que não podem ser explicados apenas pelas razões
``materialistas'', evocando os ``interesses de classe'' ou as
estratégias dos dominantes. O capitalismo entrou há décadas numa fase de
autodestruição. Essa autodestruição sempre existiu em germe em sua
própria essência: a transformação, vazia de sentido, de trabalho em
dinheiro, sem qualquer relação com o conteúdo. Os sujeitos são em grande
medida (mas não completamente) afetados por essa lógica autodestrutiva.
As populações votam espontaneamente no tipo de opressor que, há algumas
décadas, só teriam como chegar ao poder pela força. A forma"-valor e a
forma"-sujeito, dois lados da mesma forma de base, remetem uma para a
outra, cada uma sendo tanto o pressuposto como a consequência da outra.

Todavia, nem tudo está perdido. Não estamos diante de uma condição
imutável do ser humano, mas de forma históricas. Assim como essas formas
vieram ao mundo, também podem desaparecer. Entendê-las é o primeiro
passo para se libertar delas um dia. Este livro constitui uma importante
contribuição a essa compreensão. A erudição de que dá mostras não
constitui um fim em si, mas está a serviço da compreensão da atualidade
mais palpitante -- sobretudo com relação às particularidades da
forma"-sujeito no Brasil.

Como diz o autor, ainda não somos homens completamente sem qualidades,
ainda não somos puro vazio -- e ainda é possível evitar que nos
tornemos.

\chapter*{Introdução}
\addcontentsline{toc}{chapter}{Introdução
\bigskip}
\hedramarkboth{Introdução}{}

\begin{flushright}
\scriptsize{Mas não é particularmente por instituições\\
políticas que se manifestará a ruína universal,\\
ou o progresso universal; pouco importa o\\
nome. Será pelo aviltamento dos corações.\\
\emph{Baudelaire}}\\
\end{flushright}

\begin{flushright}
\scriptsize{Vai pássaro encantado,\\
abre a porta\\
e voa à minha amada,\\
aninha"-te em seu peito\\
e conta a ela que sigo\\
vivo, mas putrefeito.\\
\versal{CLOV}-\emph{Beckett}}
\end{flushright}

O canto de Clov, no final da peça \emph{Fim de Partida} de Beckett,
apesar de aparecer como que deslocada do ambiente geral da
peça, de onde não se esperaria tal lírica, exprime o espírito que
sobressai também em \emph{Dias Felizes} e \emph{Esperando Godot}:
viver putrefeito. Um viver putrefeito que vai ao encontro das
palavras de Benjamin com relação a outro contexto por outras razões
também dramático: ``Ficamos pobres. Abandonamos uma depois da outra
todas as peças do patrimônio humano, tivemos que empenhá-las muitas
vezes a um centésimo do seu valor para recebermos em troca a moeda miúda
do `atual'" (\versal{BENJAMIN}, 2010, p. 119). Ao analisarmos a marcha
incessante em busca do ``desenvolvimento'', cada vez mais premente e
apagando os rastros do passado, quer dizer, ao analisarmos os poucos
séculos de desdobrar"-se da guerra concorrencial capitalista mediada pela
mão invisível do mercado, não podemos vislumbrar outra divisa. Mas com
algumas ressalvas: ficamos pobres em termos do que se tem considerado
minimamente humanidade, embora estejamos cada vez mais empanturrados de
mercadorias e de suas imagens idealizadas. Será que no avançar da vida
moderna, tida como o nível mais alto da sociabilidade humana, não
estariam escondidos os escombros deixados pelo desdobrar do movimento
incessante de multiplicação do dinheiro? Se a resposta for positiva, de
qualquer modo, os escombros contemporâneos e do porvir também se
anunciam como subjetivos, restos de pensamento reflexivo. Dito de outro
modo, será que se poderia dizer que a realização da promessa capitalista
de um contínuo progresso emaranhado no devir"-mercadoria do mundo está
ligada a um contínuo devir"-mercadoria do homem, forma
dessubstancializada? Ou ainda, será que a ``nova barbárie'', o começar
do zero a que se referia otimistamente Benjamin ao ver
certa positividade no apagamento da memória, poderia ser chamada de
\emph{positiva}? Ou a nova barbárie seria antes o estágio a que chega
uma sociedade cujo desejo insaciável é submeter a realidade concreta ao
movimento \emph{ideal} do acúmulo de riqueza abstrata?

Abordo neste livro a temática desse devir"-sujeito"-dessubstancializado
de fato como um devir, ou seja, nunca como um dado absoluto e acabado,
mas como um processo que vem se desdobrando na história da modernidade.
Esse desdobramento significa que o vazio já rondava desde o Renascimento
ao capitalismo do século \versal{XIX} o sujeito pensante, ele próprio
apressadamente chamado sujeito burguês. Veremos que o sujeito burguês
não é uma forma de subjetividade estática; que o chamado sujeito burguês
do século \versal{XIX} era em verdade um entrecruzamento de formas de
subjetividade mais modernas e formas mais fincadas na tradição anterior
à modernidade, e que, numa dialética de superações de formas de
subjetividade menos adaptadas ao cálculo, à frieza, ao formalismo do
pensamento, foi sendo superada por formas mais próximas desse caráter
que foi se desdobrando. Seremos levados talvez a nos perguntar por que
esse devir"-esvaziamento da subjetividade se dá como processo paulatino
na modernidade e não em outra época histórica. Nossa hipótese primeira é
que, para responder a isso, temos que pensar no papel histórico bastante
diverso que desempenhou nas várias formas sociais aquele que é o
equivalente geral e esvaziador geral das especificidades das coisas e
pessoas, aquele que é nosso velho conhecido, que parece ser tão trivial
à primeira vista, embora em verdade seja muito cheio de manhas
teológicas: o dinheiro, que corre nas veias da mercadoria. Do mesmo modo
é preciso pensar sobre a cimentação simbólica das sociedades anteriores,
uma cimentação que não permitia que o dinheiro tenha se tornado desde
sempre o que ele em verdade \emph{é}. Acho que não erraria em dizer que
o dinheiro, de formas distintas, compõe o fio que tece o caminho reflexivo
que perfazemos por obras teóricas e literárias.

Ora, seria possível perguntar: estará o sujeito moderno, autocentrado,
reflexivo, o sujeito que saiu das trevas e dos grilhões da religião e do
absolutismo para se entregar de braços abertos à democracia moderna,
sendo destruído e dando vez a um sujeito vazio, narcísico, de desejos
imperiosos? Ou será que, em vez de estar sendo destruído, ele está em
seu processo de desdobramento, de realização de seus desígnios como um
paroxismo da modernidade, ou do processo de modernização desencadeada
pela Razão (instrumental"-mercantil)?

O presente livro pretende constituir"-se como uma reflexão sobre quatro
tempos, quatro momentos distintos do desenvolvimento da forma"-sujeito
moderna coadunada com a forma"-social mercantil. Além das obras teóricas,
veremos como esse desenvolvimento da modernidade ecoa em obras de quatro
autores da literatura: Molière, Musil, Beckett e Mário de Andrade. Este
último apenas nos centraremos em seu \emph{Macunaíma.} Um dos conceitos
basilares neste estudo é o de \emph{tensão} em termos de psicologia
social, ou seja, em termos de tensão dialética entre subjetividade e
sociedade. Essa tensão se constitui como fundamento das relações
sociais, como elemento essencial da constituição subjetiva. Isso
significa que sociedade e individualidade se entrecruzam. No entanto,
essa relação é mais problemática especificamente na modernidade, pois
nela intervém uma forma de subjetividade abstrata que chamamos de
forma"-sujeito burguesa, uma \emph{fôrma} de subjetividade adaptada à
vida moderna.

Significa dizer que os quatro tempos da forma"-sujeito moderna sobre os
quais tentaremos refletir se ligam a uma tendência histórica ao
esvaziamento do mundo pelo fato de aquilo que paulatinamente foi
ocupando o nexo social ser também uma forma sem conteúdo, pura mônada de
trabalho abstrato, de valor mercantil, portanto de dinheiro: a
mercadoria. Tentaremos desenvolver a ideia de que a forma"-sujeito
burguesa, que se desenvolveu embrionariamente na prática com os
mercadores medievais, nunca pôde nem pode existir em estado puro, mas
sempre no entrecruzamento com uma subjetividade passada, presente e
futura.

Falar de tensão significa reexaminar a ideia do materialismo tradicional
de base marxista. De fato, para entender o ser humano de uma determinada
época, temos que apreender as relações sociais de dada época. Mas na
modernidade, há uma especificidade, pois a dinâmica por ela introduzida
não tem apenas um matiz de pura reprodução social como em outras
sociedades, mas está sempre apontando para o ``progresso'', com fases
que não se sucedem simplesmente, mas em que uma fase posterior demonstra
\emph{in destruens} o desdobrar"-se da fase anterior -- evidentemente com
os percalços que a vida concreta impõe -- sempre caminhando no mesmo
sentido. É por isso que autores como Robert Kurz, seguindo a linha da
Teoria Crítica, consideram a modernidade capitalista como uma metafísica
real. No sentido de ela ser uma espécie de conceito cuja tendência
histórica é se materializar na realidade concreta, que é encarada por
dito conceito como puro suporte para sua realização. Dito de outro modo,
a modernidade não pode ser entendida nem como simples relações
materiais, nem como simples ideias que ficam no campo ideal. Há na
modernidade -- pela sua característica dinâmica inédita na história --
uma constante tensão entre concreto e abstrato. Esse concreto é tanto a
vida material, quanto a própria subjetividade. No entanto, na
modernidade, tanto a vida material quanto a subjetividade estão
constantemente acossadas por novas formas, novos desdobramentos, novas
exigências. E essas exigências provêm do ideal abstrato de uma sociedade
que precisa sempre ``evoluir'' para ficar de pé. O ideal inconcretizável
-- uma vez que a forma"-social burguesa pura não é possível sem uma
aniquilação da realidade -- e invivível de sociedade mercantil seria um
crescimento econômico incessante. O ideal de sujeito burguês
inconcretizável -- uma vez que uma forma"-sujeito burguesa pura seria a
aniquilação da subjetividade em proveito da realização da segunda
natureza -- e insuportável seria o sujeito que se reduz simplesmente a
trabalhar e comprar incessantemente. Essa abstração tenta a todo custo
se impregnar da realidade, fazer da realidade o seu campo. É assim que a
subjetividade sofre a pressão para adaptar"-se a uma forma abstrata -- a
forma"-sujeito moderna --, uma verdadeira \emph{máscara de caráter}
(\versal{MARX}). Essa forma"-sujeito tenta se realizar nas subjetividades ou
individualidades concretas. É nesse sentido que se pode falar do
capitalismo, da vida social moderna, como uma metafísica real -- ou em
realização permanente -- tanto do ponto de vista objetivo quanto do
ponto de vista da subjetividade.

No entanto, essa abstração social e subjetiva não encontra sempre o
terreno aberto para simplesmente se realizar. O concreto -- em termos de
relações sociais materiais e da subjetividade -- impõe certos limites
aos caprichos da abstração que quer simplesmente se efetivar. É o que
chamamos aqui de tensões subjetivas, ou entrecruzamento subjetivo a cada
passo do desenvolvimento capitalista, é um entrecruzamento que sempre
guarda, além dos momentos capitalistas da vida, momentos não
necessariamente capitalistas, que podem ser outras formas de ver o
mundo, outros símbolos sociais. Historicamente o capitalismo se impôs
junto com essa tensão que só existe com a memória que se mantém das
várias experiências de vida ou daquelas transmitidas pelas gerações de
que a vida não se resume a vender, comprar, a tirar vantagem e realizar
imediatamente seus desejos mercantis. Ou seja, se as subjetividades
interiorizam, reproduzem e/ou transformam, como diz o materialismo, elas
também tensionam, ou se veem em meio a relações em que novas formas
subjetivas e objetivas tensionam com suas ``velhas'' concepções. Dito de
outro modo, é preciso nuançar a famosa afirmação de Marx de que ``não é
a consciência dos homens que determina o seu ser, mas, ao contrário, é o
seu ser social que determina a sua consciência'' (\versal{MARX}: 1982, p. 25). Em
abstrato essa afirmação não causa estranhamento, mostra"-se correta. Mas
quando analisamos a história concreta das relações sociais e das
subjetividades, vemos que tanto o ser social determina a consciência
quanto a consciência que se adquire determina esse próprio ser numa
trama dialética que, como dissemos, é mais complexa na modernidade,
quando um terceiro elemento entra nessa relação: um poder impessoal que
tece o fio que entrelaça o ser social e sua consciência.

Para esse intento de fazer uma espécie de viagem ao coração dessas
tensões subjetivas ocorridas no desdobrar"-se da modernidade até a
contemporaneidade, quando ela parece dar mostras de ter derrotado toda e
qualquer oposição que não seja o melhoramento da suas próprias
estruturas, teremos que refletir minimamente sobre a gestação das
tensões rumo à modernidade.

Começaremos nossa viagem no século \versal{XI}, com o impulso que tiveram os mercadores, sem
pretender que já se tratava de capitalismo, ou que o dinheiro tinha o
mesmo papel que terá no capitalismo, mas simplesmente porque o espírito
desses mercadores medievais apontava para questões que extrapolavam o
espírito medieval -- não esquecendo que eles eram uma minoria
desclassificada no começo, no seio de uma sociedade cheia de mediações
sociais e ainda regida pelo universo simbólico"-religioso. Alguém poderá
objetar: e por que não começar com os mercadores na antiguidade, que
também introduziram tensões na vida social, como em Atenas? Uma
justificativa é que aqueles mercadores viviam mergulhados, como toda a
sociedade, em tradições muito profundas para que pudessem tensionar como
o fizeram os mercadores medievais, que começaram a influenciar uma
reflexão sobre a liberdade de iniciativa e de ganho. Não havia uma
dinâmica social incessante para minar essas tradições que somente um
sistema produtor de mercadoria pode causar; tampouco houve algo tão
decisivo quanto o chamado \emph{big bang} da modernidade, que foi, nas
palavras de Kurz, a invenção das armas de fogo. Tanto que havia
meios políticos de se defender do dinheiro: ``Esparta proibia aos
particulares a posse do ouro; como meio de circulação os espartanos
usavam barras de ferro.'' (\versal{JAPPE}, 2006, p. 186). Era, portanto, difícil
acumular. Outra justificativa é que de fato não foi dali que surgiu a
modernidade. O fosso entre a Antiguidade e a Alta Idade Média já é
várias vezes duradouro do que o próprio capitalismo.

Mesmo com o grande desabrochar dos mercados de troca, houve um processo
muito lento de preparação à modernidade capitalista. Partindo desse
impulso medieval, chegaremos ao Renascimento, visto como passagem
fundamental, como preparação burguesa fundamental, embora o humanismo
renascentista seja um grande amálgama de passado medieval e futuro
moderno, não se podendo chamar o humanista simplesmente de sujeito
burguês. Essa preparação inclui o que Marx chamou de \emph{acumulação
primitiva do capital}, mas também uma espécie de
\emph{acumulação} \emph{primitiva} em termos subjetivos sempre contínua.

Minha pretensão é, portanto, refletir o quanto os aspectos modernos foram
sendo ``\emph{mijotés}'' na vida social e nas subjetividades e o quanto
o peso da tradição e de aspectos simbólicos são empecilhos a uma
sociedade que se pretende livre de qualquer amarra, inclusive da amarra
do outro.

Os sujeitos que veremos com Molière estão nesse
entrecruzamento entre pré-moderno e moderno, tendo ainda como ideal
subjetivo o nobre, mas um nobre que já tem a influência da moral
mercantil, que já sofre uma \emph{corrosão de caráter} -- e podemos
notar a lentidão das mudanças sociais, apesar de tudo. Ao mesmo tempo
que ama seu dinheiro, o mercador idealiza a subjetividade nobre, o
\emph{status} nobiliárquico, além dos privilégios. Ao mesmo tempo que
não tem honra, nem moral, nem ética no mercado, e que é capaz de
destruir os povos da colônia ou escravizar e vender os negros, lê os
tratados de comportamento do Renascimento, participa do processo
\emph{civilizador} (\versal{ELIAS}) e almeja ao estatuto de \emph{Courtois}. Ou
seja, no desenvolvimento da modernização, \emph{civilização} e
\emph{barbárie} aparecem como dois lados de uma moeda moderna: ``Nunca
houve um monumento da cultura que não fosse também um monumento de
barbárie.'' (\versal{BENJAMIN}, 2010, p. 225). Mas, na contemporaneidade, no
momento do capitalismo desenvolvido -- que coincide com sua crise lógica
-- os monumentos de cultura ou civilização são já monumentos de
barbárie. A barbárie não é mais, como antes, um efeito colateral, um
preço que se deveria pagar -- segundo a concepção dos defensores do
progresso. A barbárie é a forma de organização da sociedade moderna
desenvolvida (\versal{MENEGAT}, 2013, p. 154)\footnote{Muito interessante é a
  análise desse autor sobre alguns intérpretes do Brasil, que poderia
  ser estendida a outros da América Latina como o caso de Domingo
  Sarmiento ou até Euclides da Cunha: ``A posição crítica dos
  intérpretes ficou no plano da crítica da ideologia, em que a
  participação no fundo comum da civilização (cf. Mauss) ainda é
  reivindicada em oposição à barbárie. O processo de modernização e a
  sociedade burguesa são criticados na forma que assumem na periferia,
  mas não em seus fundamentos que, afora alguns transtornos aqui e ali,
  decorrentes de alguma possibilidade de originalidade nacional, eram
  aceitos. Em outros termos, os intérpretes formularam uma crítica às
  formas de distribuição de riqueza da civilização ocidental burguesa,
  mas não de seu modo de produção (incluindo suas forças produtivas).''
  (\versal{MENEGAT}, 2011, p. 36)}. A barbárie não pode mais ser colocada como
falta de civilização moderna, mas como excesso de civilização moderna.
Não apenas barbárie em termos objetivos, no sentido de uma profunda
disfunção na organização social que cria um quadro propício à violência
-- como o fato de não haver mais trabalho para todos -- mas sobretudo em
termos subjetivos, em termos de um vazio de pensamento comum a todas as
classes sociais, que não significa seres humanos vazios como uma tábula
rasa, mas o mover"-se na pura imanência de uma segunda natureza -- um
reino em que o pensamento não consegue transcender o automatismo de uma
vida que pretende imitar a vida natural, embora seja social. Eis a nova
barbárie, que não é portadora de nada de positivo, nem de qualquer
esperança. A esperança está do outro lado, só pode estar na recusa dessa
nova barbárie, uma recusa que, por sua vez, poderá advir do processo
árduo de aquisição da consciência necessária à transcendência a essa
segunda natureza.

Com Musil, veremos a forma"-sujeito moderna de forma mais delineada, um
sujeito já mais despojado do passado, mais apto à Razão
instrumental"-mercantil, sem ao mesmo tempo deixar de haver tensão com as
sobras de honra, as sobras de realeza, sobras do antigo regime. Sobras
de outra temporalidade. Veremos aqui a vitória da modernidade sobre as
outras formas de vida social, vitória essa que vai se consolidar de vez
no século \versal{XX}. Veremos aqui o quanto o indivíduo ou a subjetividade é
algo distinto do Indivíduo moderno ou da forma"-sujeito moderna.

Na terceira parte deste estudo, refletiremos sobre o desenvolvimento da
forma"-sujeito burguesa no Brasil como sendo bem distinto do processo
europeu. Veremos que, embora a colônia desde cedo tenha estabelecido
trocas mercantis com a Europa, isso não foi suficiente para o deslanchar
de um mínimo sistema mercantil internamente. Ou seja, a forma"-social e a
forma"-sujeito burguesas somente no passar do século \versal{XIX} para o \versal{XX}
puderam ter certo impulso interno. É nesse sentido que lemos o
Modernismo paulista, em especial \emph{Macunaíma} de Mário de Andrade,
como uma tentativa vanguardista de apreensão e impulso de uma
subjetividade \emph{sui} \emph{generis,} tipicamente brasileira, como
síntese inconfessa \emph{indigeno}-\emph{burguesa}, numa prefiguração,
essa sim vanguardista, do desdobramento do capitalismo como amálgama
entre cultura e natureza. Como nas outras obras, não se trata de buscar
interpretar o que o autor ou o movimento quis dizer naquele momento
histórico, mas o que na contemporaneidade se pode interpretar dessas
tentativas vanguardistas que nunca são somente estéticas.

Com o teatro de Beckett, analisaremos talvez o ponto culminante do
desenvolvimento moderno sem freios. A \emph{forma} do teatro de
Beckett, bem como seu
\emph{conteúdo} -- a falta de conteúdo -- podem dar uma medida do que
significa a vida social se fundamentar no movimento repetitivo de
valorização do dinheiro. Nesse momento da reflexão, destacaremos o
advento da sociedade de consumo, não como nova sociedade, mas como
desdobrar"-se necessário do movimento dinâmico da sociedade moderna. Em vez de
confundir o fetichismo da mercadoria de Marx com o fetichismo do
consumo, refletiremos o quanto são dois níveis de fetichismo distintos.

Do mesmo passo, de formas bem diversas, poderíamos afirmar que tanto
Molière -- embora seu teatro não pareça demonstrar de modo algum tensão
quanto a isso, limitando"-se a pintar o quadro social da época
ridicularizando os \emph{sem} \emph{qualidades}, às vezes,
devotando"-lhes benevolência, em suma, mostrando em certa medida os
valores que sua época cultivava, valores que já parecem paisagem
distante na contemporaneidade -- tanto Musil quanto Beckett tratam da
questão da substância ou da falta dela em relação a qual se constrói a
subjetividade, uma questão bastante cara à contemporaneidade vivida por
nós, uma questão que diz respeito diretamente aos processos de
subjetivação numa sociedade mercantil desenvolvida. Se os processos de
subjetivação mantêm estreita relação dialética com o quadro objetivo da
sociedade, significa que o avançar da sociedade poderá apontar novas
questões a esses processos.

Além disso, reafirmando o aspecto dialético da relação entre sociedade e
subjetividade, se a questão da falta de substância é colocada ao sujeito
é porque ela se coloca também na base na própria sociedade, embora
precise de um tempo de amadurecimento. E é interessante como na
literatura desses autores ecoam esses momentos. Mas de formas distintas,
por se tratar de momentos distintos de um mesmo processo e de autores de
características e experiências distintas. Como diz Todorov: ``Como a
filosofia e as ciências humanas, a literatura é pensamento e
conhecimento do mundo psíquico e social em que vivemos. A realidade que
a literatura aspira compreender é simplesmente, [\ldots{}] a experiência
humana''. (2009, p. 77).

Mesmo sem ser um documento psicológico, filosófico, teórico ou histórico
em si, a literatura desses autores não pode ser considerada mera ficção,
mera invenção sem relação com o momento histórico, social e subjetivo em que
foram escritas ou lidas e interpretadas. As formas estéticas também são mediadas
socialmente. É comum ver obras literárias, por mais absurdas que possam parecer,
transbordarem uma tensão com a realidade concreta, quer seja ela passada,
presente, ou até futura, como obras que só fazem sentido tempos depois,
ou que ganham mais força tempos depois, como o caso das peças
beckettianas escolhidas aqui. Ou seja, essa tentativa de reflexão tenta
contribuir também para ressaltar a literatura como fonte de pesquisa
qualitativa.

Mas é importante ficar claro que este estudo não é literário, mas
psicossocial. Dizer que não se trata de estudo literário significa que
as obras literárias não são o ponto de partida para uma análise
imanente, como se desenvolve cada vez mais contemporaneamente, mas o
ponto de aprofundamento. Significa dizer que não busco nas obras as
teses defendidas aqui -- o que seria com razão repreendido pelos
estetas. Tampouco usamos as obras como ilustração do conteúdo. O que
está em questão aqui é muito mais o que se poderia chamar de leitura da
obra literária como sintoma, ou leitura dos ecos da vida social que
repercutem na obra literária. Por isso, não deve se incomodar o leitor
ao ver que as páginas dedicadas à análise das obras estão às vezes em
menor número do que aquelas dedicadas à reflexão mais geral. É porque
nosso método pressupõe uma introdução histórico"-lógica antes de adentrar
pela obra literária em si. Não para que a obra seja ilustração do que
precedeu, mas simplesmente para deixar cristalino, já que a
interpretação é uma atividade pessoal, de onde o autor desse estudo joga
seu olhar para as obras literárias. Ou seja, ao deparar com a reflexão
histórico"-lógica, o leitor não deve ralhar e jogar às Gemônias o texto
por fazer mal uso da literatura, mas deve abrir"-se ao caminho que o
autor percorre para construir o universo a partir do qual ele lê as
obras literárias, sem pretender que seu desenvolvimento lógico"-histórico
esteja refletido nas obras literárias. A questão é mais complexa e
dialética. Até porque, como diz Blanchot: ``Não há arte literária que,
direta ou indiretamente, não queira afirmar ou provar uma verdade''
(1997, p. 187). Boa ou má. Conscientemente ou não. Uma verdade que não
existe necessariamente apenas no tempo histórico em que a obra foi
escrita, mas que pode ultrapassar o seu próprio tempo. E é perante essas
obras que ultrapassam seu próprio tempo que não podemos ficar em
silêncio, porque não são só obras interessantes, mas principalmente
acrescentam muito sobre sua época e nossa época (\versal{SARTRE}, 1999, p. 128),
embora não tenham necessariamente tido o intuito de fazê-lo.

E nossa época, como um processo de desembocar da modernidade ou da
modernização, Lipovetsky, por mais controversas que sejam suas análises,
mais apologéticas do que críticas, resumiu como a \emph{Era do
vazio}: ``A realização definitiva do indivíduo coincide com sua
\emph{dessubstancialização}, com a emergência de átomos flutuantes
esvaziados pela circulação de modelos e, por isso mesmo, continuamente
recicláveis'' (2008, p. 154). De certo modo, esse trecho resume nosso
desafio reflexivo.

Esse vazio do pensamento também Hannah Arendt problematiza ao denunciar
a crise da cultura, como algo típico das sociedades de massa -- embora
se queira fazer crer hoje na individualização pelo consumo de
mercadorias pretensamente individuais e pelo protagonismo proporcionado
pelas redes sociais. Se cada nova geração deve descobrir laboriosamente
a atividade de pensar e durante muito tempo, para desenvolver essa
atividade fundamental para a sociedade, recorreu"-se à tradição, para a
autora, esta mesma tradição sofre uma usura na modernidade e os
indivíduos precisam descobrir essa atividade por conta própria (\versal{ARENDT},
2006). Se como ela mesma afirma, o homem na modernidade deve descobrir
por conta própria a atividade do pensamento, o desenvolvimento contínuo
desse processo de pretensa ``subjetividade nova'', desse ``homem novo''
moderno despregado de uma retirada de sentido da tradição acaba pendendo
para uma falta de essência exatamente porque não há quase referência que
se constitua na construção de significados na relação social mesma a não
ser aquelas lançadas de forma heterônoma e acima da própria sociedade --
que as aceita. E a sociedade pretende -- quanto mais se desenvolve --
assegurar aos sujeitos a base sólida de não precisar se ``prender'' a
qualquer ``essência'' --- que seria encarada quase como uma prisão ---,
mas esse ``não prender"-se a nada de permanente'' é o mais movediço e
inseguro chão em que se vê o sujeito contemporâneo.

Tradição aqui precisa ser entendido em sentido mais amplo do que simples
perpetuação do poder, em si criticável, mas como acúmulo de experiência
construída socialmente, experiência que se constitui em referenciais de
reflexão. Mas como descobrir essa atividade numa sociedade do
\emph{prêt"-à-porter} e do \emph{prêt"-à-penser}, da novidade incessante,
onde a ideia de experiência e transmissão de saberes passam por
totalitários para sujeitos que se creem tudo saber desde a mais tenra
idade, onde os esforços de raciocínios são penosos diante do mundo de
pretensas facilidades, pretensas saciedades contínuas, um mundo sem
renúncia como pretende se nos apresentar a sociedade contemporânea? Esse
vazio de pensamento guarda relação com o conceito de memória de Benjamin
que se poderia resumir como a capacidade de \emph{retirar
sentido} do que se vive e construir experiências de significado para
nossa existência. Memória significaria, em suma, capacidade de manter
sentido de história, de manter o sentido de passado, presente e futuro,
capacidade de passar por um crivo crítico esse mesmo passado, mas também
o presente, para melhor delinear o futuro, em vez se simplesmente
aceitá-lo como simplesmente a obviedade do que vem depois.

Trata"-se, assim, de uma capacidade de diferenciação, capacidade de não
abstrair a especificidade de coisas, pessoas e formas de vida. E não
teria sido esse o processo paulatino da modernidade?

O empreendimento da reflexão sobre as tendências subjetivas na
contemporaneidade, nessa \emph{era do vazio}, como uma subjetividade
\emph{dessubstancializada}, constitui"-se em tarefa tanto difícil quanto
necessária à própria socialização. Isto porque, a sustentar"-se a
tese de um ponto culminante do \emph{vazio tautológico} na vida social
contemporânea e no processo de subjetivação na contemporaneidade é o próprio homem que
pode se tornar supérfluo (\versal{ARENDT}). Este estudo pode vir a ser um pequeno
contributo nesse sentido. O que proponho aqui ao leitor é decerto um texto teórico,
mas que pretende tensionar com o tempo presente, com a vida que levamos nesta
comunidade mercantil dentro da qual acordamos todos os dias.

\chapter*{Molière e \emph{o burguês fidalgo}: entrecruzamento entre fidalguia~e~burguesia}
\addcontentsline{toc}{chapter}{\large\versal{MOLIÈRE E \emph{O BURGUÊS FIDALGO}:\\ \small{ENTRECRUZAMENTO ENTRE FIDALGUIA E BURGUESIA}}}
\hedramarkboth{Molière e \emph{o burguês fidalgo}}{}

Numa sociedade marcada por uma dinâmica interna, a forma de
subjetividade nunca se dá em estado puro, mas em constante desdobrar"-se
na história. Hoje, podemos ter a medida de que o desenrolar da
modernidade trouxe consigo uma dialética constante entre formas de
reprodução material imersas e submetidas à tradição mítico"-religiosa no
sentido lato -- como a que dominou largamente na Idade Média e em grande
parte subsistiu na modernidade -- e formas baseadas no ganho financeiro
materializadas primeiramente no desenvolvimento das trocas mercantis, em
especial na Idade Média. Do mesmo modo, podemos ter a medida hoje de que
a modernidade se desenvolveu tendo em seu seio também uma dialética da
subjetividade. Significa dizer que a forma"-sujeito moderna, burguesa ou
mercantil, do Indivíduo abstrato, objetivante e objetivado, que anda no
mundo a fazer cálculos, não se implantou de
uma vez por todas nas individualidades concretas, mas num processo
contínuo de tensões e superações.

Nesse estudo, situaremos o surgimento da \emph{tensão} moderna no grande
desenvolvimento urbano e comercial que a Europa conheceu nos séculos \versal{XI},
\versal{XII} e \versal{XIII}, que fez ganhar importância a figura do mercador,
o ancestral do burguês moderno. Não porque o capitalismo fosse
inevitável a partir daquele desenvolvimento das trocas mercantis. Também
na Antiguidade havia trocas entre objetos. A diferença é que as trocas
na Baixa Idade Média, contrariamente a outros períodos
históricos, vieram acompanhadas de um movimento lento, mas contínuo, de
tensionamento com o universo simbólico que fundamentava a vida subjetiva
e as relações entre os homens. Para citar um exemplo, como afirma Crosby
(1999, p. 53): ``No fim do século \versal{XI} e durante o século \versal{XII}, os tratados
sobre cálculos elementares eram, em sua maioria, tratados sobre o uso da
tábua de calcular, e havia surgido um novo verbo, \emph{tabular}, que
significa fazer contas.''

Nesse sentido, não falamos de surgimento do capitalismo ou do sujeito
moderno por essa época, mas do aparecimento ainda tímido de uma forma de
subjetividade que desde seu aparecimento já causava estranhamento em
relação às formas que reinavam soberanas. Essa forma de subjetividade é
típica do mercador, aquele que tem como \emph{habitat} a cidade.
O que já é um estranhamento, visto que a vida se dava em sua grande
maioria nos espaços rurais.

O entrecruzamento subjetivo e objetivo a que me refiro se mostra, por
exemplo, no fato de que, apesar de ``as diferenças de estatuto, de
origens, de ordens e estados, enfim, de qualidades [introduzirem]
clivagens que se superpunham às da fortuna, de qualquer maneira, o
modelo social urbano é o burguês, e o critério de diferenciação
essencial é o dinheiro'' (\versal{ROUSIAUD}, 1989, p. 169). Ou
seja, as fraturas sociais introduzidas pela maior importância do papel
do mercador e daquilo com o que ele lida, o dinheiro, puderam conviver
com formas que condenavam essa conduta por terem base noutras relações
durante muito tempo, e era exatamente por essas relações estarem
fundamentadas numa trama diferente que criavam tensão com essa nova
figura social, em vez de simplesmente despertar nas pessoas o desejo de
construir também uma riqueza mercantil.

\begin{quote}
Assim, a cidade, por sua economia, seu ambiente, sua ética é, para a
massa, destruidora dos laços familiares; as epidemias atacam, os laços
de solidariedade se afrouxam, os perigos morais espreitam, a autoridade
do chefe de família é posta em perigo. O citadino, não raro, sem
ancestrais nem bens, não pode contar com ``seus amigos carnais''; essa
fragilidade estrutural não é causada só pela quantidade de homens ``sem
nome nem família'', mas pela natureza mesma da riqueza citadina. Essa
riqueza repousa no dinheiro. (\emph{Idem}, 1989, p. 168)
\end{quote}

O mercador que vive na cidade, assim como o burguês que lhe sucedeu, era
amplamente desprezado pela sociedade em geral, não porque não estivesse
passando a ocupar um papel importante, sobretudo a partir do século
\versal{XIII}, mas porque a sociedade era largamente alicerçada na tradição
simbólico"-religiosa, uma tradição que fez com que os teóricos da
sociedade feudal, segundo Le Goff (1999a, p. 79) e Gourevitcht (2009, p.
269), falassem apenas de três camadas principais na sociedade, com o
monarca à frente: os que rezam (o clérigo), os que combatem pela
comunidade (os cavaleiros) e os que cultivam a terra (os camponeses).
Notemos de passagem que não há trabalhadores. ``A desconfiança que
nutriam os camponeses em relação aos mercadores e o desdém que lhe
endereçavam os nobres'' (\emph{Idem}, p. 270) baseava"-se
na doutrina da igreja -- quem fornecia as normas sólidas que
fundamentavam o nexo social (\versal{SIMMEL}, 2009, p. 287).

Mas esse desprezo explícito, essa condenação explícita da atividade do
mercador vai começar a ganhar contornos mais ambíguos, e a relação da
sociedade com o mercador, principalmente após o século \versal{XIII}, torna"-se
contraditória: ela o menospreza, enquanto ele ganha importância social:
``O rico suscita inveja e hostilidade, sua responsabilidade e sua
honestidade são postas em dúvida. Em geral, o mercador permaneceu,
seguindo as palavras do historiador contemporâneo, um `paria' da
sociedade medieval na fase inicial de seu desenvolvimento'' (\versal{GOUREVITCH},
1989, p. 270).

Evidentemente, do século \versal{XIII} ao \versal{XVII}, quando Molière escreverá suas
peças para a monarquia e a corte, o processo de afirmação do mercador,
que vai amadurecer na figura mais delineada do burguês como tipo social,
só vai se desenvolver. O mercador itinerante vai perder espaço para o
mercador sedentário que cria uma rede mais complexa de associados e
empregados. Com a amplificação dos negócios, ele vai precisar procurar
capitais além de seus negócios. É
o caráter desmedido do dinheiro, da riqueza que ele representa e da
subjetividade que ele engendra, que fincava lentamente raízes.
Gourevitch também atesta que é só no século \versal{XIV} e \versal{XV} que o
mercador vai ter espaço na literatura não apenas como seres ridículos e
desprezíveis. No Decameron de Boccaccio, o mercador ``tem acesso ao rol
de herói no lugar daquele que anteriormente era a encarnação do
princípio heroico, o cavaleiro heroico'' (p. 306).

Nas peças de Molière que problematizaremos adiante, os mercadores vão se
tornar burgueses de fato, verdadeiros homens sem qualidades, mas com
dinheiro para comprar algumas. Ele passa a dirigir as cidades. No
entanto, a dinâmica social ainda não é comparável com o \emph{boom} que
ocorrerá após a Revolução Industrial e Francesa. A subjetividade ainda é
marcada largamente pelo entrecruzamento entre uma forma"-sujeito
burguesa, tendente a moderna, ainda embaçada, que vai se desenvolver e
se explicitar sempre mais, exigindo o apagamento dos rastros do passado,
uma abertura ao novo, rumo a uma liberação das amarras pessoais e da
tradição, e uma subjetividade medieval cujas bases sólidas
simbólico"-religiosas vão passo a passo sendo corroídas. Esse trecho de
Gourevicht é muito significativo da tensão a que me refiro:

\begin{quote}
Elemento notável, mas secundário, de uma sociedade fundamentalmente
agrária no início da Idade Média, o mercador torna"-se pouco a pouco uma
figura de primeiro plano, o iniciador de comportamentos novos que minam
os fundamentos tradicionais do feudalismo. [\ldots{}] A mentalidade
do mercador se distinguia radicalmente da dos cavaleiros, do clérigo ou
dos camponeses. A representação do mundo que toma forma pouco a pouco na
consciência dos mercadores, à medida que se desenvolvem, entra em
contradição com a representação do mundo que nutrem as outras camadas e
categorias da sociedade feudal. A profissão e o modo de vida dos homens
de negócios favoreceram a elaboração de novos princípios éticos e de um
modo de conduta diferente. (1989, p. 267).
\end{quote}

Aos poucos, vai ficando difícil simplesmente ignorá-los. A Igreja vai
precisar deslocar seu centro gravitacional para as cidades no século
\versal{XIII}, de onde ela precisava extirpar as heresias que eclodiam. A relação
da Igreja, portanto, com os mercadores é ambígua: ao mesmo tempo que o
condena, vai reconhecendo sua importância para a sociedade (\emph{Idem},
p. 272). Simmel, em seu livro \emph{Filosofia do Dinheiro},
defende a ideia de que, embora o dinheiro seja muito antigo -- século
\versal{VII} A. C. -- ele vai ganhar muita importância em sociedades em que a
religião ascende ao grau absoluto de objetivo final da existência (p.
281). Segundo ele, há uma relação forte entre o dinheiro e a ideia de
Deus: a essência profunda do pensamento divino, assim como a do
princípio do dinheiro, é unir nele todas as diversidades e as
contradições do mundo. Evidentemente, no caso do monoteísmo:

\begin{quote}
Enquanto o dinheiro se torna cada vez mais a expressão absolutamente
suficiente e equivalente de todos os valores, ele se eleva a uma altura
abstrata que se põe acima de toda a vasta multiplicidade de objetos, ele
se torna o centro onde as coisas mais opostas, as mais estranhas, as
mais afastadas encontram seu ponto comum e entram em contato. (p. 281)
\end{quote}

O dinheiro -- que é algo somente moderno, diferentemente da
moeda que é muito antiga -- só tem como adentrar pelo seio social, ou
seja, só tem como realizar seu desígnio, corroendo as bases simbólicas do
universo religioso como forma"-social total, que era o quadro social até
a Idade Média. Dito de outro modo, o dinheiro não aceita outro deus,
porque ele é que passa a ser o fundamento da forma"-social total nascente. O fato
notável é que as religiões monoteístas, pela experiência de uma
substância única como fundamento da vida, são o campo fértil para que
ele se imiscua sem ser notado como deus único, como o simbólico que
fundamenta as relações, enquanto desloca o deus anterior para a esfera
da fé privada.

Além disso, importa ressaltar aqui que o processo de
modernização é um processo de superação paulatina do universo
simbólico"-religioso pelo universo simbólico"-mercantil. São duas formas
de vida social distintas. As mediações necessárias para se aproximar de
deus criavam certas qualidades, até por serem mediações, uma conduta que
pretendia agradar a esse absoluto. Em suma, deus expressa um conteúdo
social e permanece apenas na cabeça dos crentes. Já o dinheiro é um deus
do vazio, um deus que não exige qualquer postura social para lhe agradar
a não ser servindo"-o no mercado como sujeito autônomo, frio,
concorrente, como produtor de uma riqueza que tem forma sempre idêntica
e que só quer aumentar -- embora à consciência imediata possa aparecer
como desenvolvimento. Além disso, a lógica do dinheiro se materializa no
social, como se o deus descesse à terra, mas vivesse escondido,
sussurrando desavenças nos ouvidos dos homens.

Esse idêntico, essa substância comum, que é mera forma sem conteúdo,
essa anulação dos caracteres específicos dos objetos, que se tornam
assim mercadorias, meros representantes de trabalho abstrato, foi o que
Marx denominou como valor\footnote{Esse conceito será problematizado na
  segunda parte desse estudo, quando analisaremos a Razão
  instrumental"-mercantil}, mera quantidade abstrata de tempo abstrato de
trabalho necessário para a produção das mercadorias.

\section*{O ``renascimento'' filosófico e~científico~intelectual~na idade~média:~tensionamentos}
\addcontentsline{toc}{section}{O ``renascimento'' filosófico e científico intelectual na idade média: tensionamentos}

Além do mercador, há outra figura que se destaca nas cidades medievais:
o intelectual. É no desenvolvimento do comércio, sobretudo do
artesanato, que ``ele aparece com um dos homens de ofício que se
instalam nas cidades onde se impõe a divisão do trabalho.'' (Le Goff,
1984, p. 11). Notemos com Le Goff que a divisão tradicional entre os que
rezam, os que guerreiam e os que laboram não significava uma verdadeira
especialização dos homens. O servo cultiva a terra, mas também é
artesão. O nobre, guerreiro, era proprietário, administrador, juiz. Os
clérigos podiam, muitas vezes, ser isso ao mesmo tempo, sendo a
atividade do espírito apenas uma das atividades. Poderíamos até dizer
o mesmo do mercador, cuja atividade essencial
é especificamente a do ganho, mas que tem outras acidentais. Mas um
homem cuja profissão seja escrever e ensinar, o sábio, o intelectual, só
aparece com as cidades. Não se fala aqui de cidade senão no sentido de
um verdadeiro renascimento urbano denso, o que não se dá antes do século
\versal{XII}, segundo Le Goff. Mas estará esse ``renascimento intelectual''
ligado apenas ao desenvolvimento das cidades, ou ambos os fatores estão
ligados também a fatores mais profundos?

Sabemos que o pensamento está sempre relacionado com a qualidade, a
distinção, a diferenciação, embora em seu núcleo possa estar a abstração
de determinações específicas para a consecução de regras gerais. De
qualquer maneira, o saber e o pensamento surgem da relação tensa no seio
de uma vida social e subjetiva. As sociedades historicamente sempre
pareceram estar de tal modo fincadas no concreto, num mundo simbólico
arraigado, que a abstração, apesar de sua força, não tomava conta da
organização social e ficava somente nas cavernas do pensamento.

No artigo \emph{Será que o dinheiro nos pensa}, quando analisa o
pensamento do alemão Alfred Sohn"-Rethel, companheiro de estrada de
Adorno e Benjamin, Anselm Jappe reflete sobre a relação estabelecida
pelo autor entre o surgimento do dinheiro no século \versal{VII} antes de Cristo
nas cidades jônicas e o surgimento da filosofia na Grécia. Ele explica,
com base em Rethel, que o comércio florescente nos \versal{III} e \versal{II} milênios
antes de nossa era deu nascimento à troca e, portanto, a uma primeira
abstração. Para se trocar, é preciso fazer abstração das qualidades
específicas que cada objeto possa ter e igualá-los a um terceiro
elemento. Mas essa abstração permaneceu durante muito tempo apenas
embrionária ou virtual, sem tendência para a totalidade social. Aos
poucos, foram sendo introduzidos objetos para desempenhar o papel do
terceiro elemento equivalente. Boi, sal, ferro, prata, ouro. Todos esses
elementos são reais e tem um uso concreto, logo, não são adequados à
abstração.

\begin{quote}
Tudo isso mudou no momento em que se começou a cunhar a moeda: um
equivalente sem utilidade material, que não servia senão para as trocas
e cujo valor não era determinado por suas qualidades físicas reais (peso
do metal), mas pela garantia dada com a cunhagem efetuada por uma
autoridade. Este acontecimento, que segundo Sohn"-Rethel perturbou a
história humana, tem uma origem bem precisa: nas cidades gregas de Jônia
onde apareceu a moeda, no século \versal{VII} antes de nossa era, em um período
de grandes transformações sociais (passagem da feudalidade fundiária à
burguesia comerciante). Sabe"-se que depois disso houve um grande impulso
comercial que abalou toda a história greco"-romana. (2012, p. 182-183)
\end{quote}

Como indica o autor, tratava"-se ali de moeda e, por mais que esse
aparecimento tenha causado transtornos sociais, a relação com ela estava
submetida a um universo simbólico que a tornava relativa e não com
tendência ao absoluto, como é o caso do dinheiro moderno. O autor
destaca que foi nessas mesmas cidades que surgiram os primeiros
filósofos uma ou duas gerações depois: Tales, Anaxímenes, Anaximandro.
Do mesmo modo, marca"-se o passo decisivo na passagem do pensamento
mítico ao pensamento racional. Jappe cita Parmênides como o primeiro a
elaborar ``a noção ontológica de uma substância unitária, perfeita,
inalterável, onipresente. Onde, pergunta Sohn"-Rethel, poderia ele ter
feito a experiência de uma tal substância, que não existe na natureza?''
(\emph{Idem}, 2012, p. 185). A resposta: somente tendo a experiência do
dinheiro.

\begin{quote}
Dessa forma, Sohn"-Rethel estabelece um liame entre aquilo que é
considerado pela tradição ocidental como o aspecto mais ``nobre'' do
homem, a saber, sua pura atividade de pensamento, a causas bastante
``baixas'': ao dinheiro, à mercadoria, a todos os atos cotidianos ---
tão banais --- de comprar e vender. O ``milagre grego'', que fez tanta
tinta ser gasta, é aqui relacionado com o aparecimento da moeda, um fato
considerado por muito tempo indigno da atenção dos filósofos. (2012, p.
184).
\end{quote}

Não é meu intento, como já preveni, entrar pelos caminhos da história
até a Antiguidade, já que, concretamente, aquelas sociedades não tinham
a mínima dinâmica interna que justifique o estabelecimento de um liame
com as sociedades modernas. Mas posso justificar ao leitor essa rápida
excursão por certa coincidência.. Os séculos \versal{XI},
\versal{XII} e \versal{XIII} foram considerados também
séculos de desenvolvimento não só comercial, mas também técnico e
intelectual. Haveria alguma coincidência? Seria apenas um quadro
material mais favorável ao pensar? Ao que parece, na Antiguidade, a
relação entre filosofia, pensamento abstrato e surgimento da moeda se dá
num plano de fato abstrato. Já no caso dos séculos a que nos referimos,
na Baixa Idade Média, essa experiência de pensamento relacionada com o
desenvolvimento das trocas mercantis já se relacionava com o
desenvolvimento paulatino de uma forma de pensar que tensionava mais
concretamente com o próprio fundamento simbólico"-religioso -- embora não
se chocasse com ele diretamente.

Jean Gimpel, em a \emph{Revolução industrial na Idade Média} (1975)
mostra o quanto esses três séculos foram de desenvolvimento racional,
embora o domínio da religião fosse enorme, a ponto de não passar pela
cabeça dos intelectuais que ``um dia os europeus viveriam sem Deus'' (p.
161). Para esse autor, homens como Giovanni di Dondi, relojoeiro e
astrônomo, Villard de Honnecourt, arquiteto"-engenheiro e Walter de
Henley, agrônomo, sem saber, contribuíram para forjar um mundo novo.
Esse impulso se deu entre os séculos \versal{XII} e \versal{XIII} quando houve um grande
esforço de casar religião e razão. ``Durante esses 150 anos
excepcionais, antes que a Igreja impusesse realmente seus dogmas, os
homens aprenderam a usar a razão e a disputá-la intelectualmente. Essa
liberdade adquirida de forma nova é o ponto de partida do espírito
científico moderno'' (\emph{Idem}, p. 161). Evidentemente, aqui, ao
falar da imposição dos dogmas, ele faz referência ao acontecimento de
1277, quando a Igreja execrou trabalhos de pesquisa em Paris, o que foi
um golpe para o desenvolvimento intelectual. Não demorou muito para
começar a caça às bruxas e aos hereges. Novamente, vemo"-nos em contato
com um poder concreto da tradição e da força -- não positivos nesse caso
-- que conseguiam dar um freio nesse impulso.

Quanto a Aberlardo, ele é visto como o primeiro intelectual europeu,
aquele que teria escrito o primeiro \emph{discurso do método}, com sua
dúvida sistemática. Não por acaso, estava entre os ``rebeldes''
goliardos, que veremos adiante. Corajosamente, foi ele quem citou 158
contradições encontradas nas Escrituras e nos padres da Igreja, todos
relacionados a pontos da doutrina, no seu \emph{Sic} \emph{et}
\emph{Non}. Incrivelmente, para a época, ele propõe uma ``crítica
racional dos textos'' (\versal{GIMPEL}, 1975, p. 162-163). Sua postura
intelectual é bastante avançada e por isso tensiona com o pensamento da
época. Como duvidar dos textos sagrados? Como era possível apontar
contradições neles? Essa nova forma de pensar de Abelardo deverá ser
abafada por ordem da Igreja: ``O Anti"-intelectualismo de são Bernardo,
para quem era escandaloso a aplicação do raciocínio lógico ao domínio
espiritual, explica"-se pelo desejo de proteger a fé e o misticismo
cristãos ameaçados por essas obras traduzidas do grego e do árabe
[\ldots{}]'' (\emph{Idem}, p. 165).

A atitude e a atividade de Abelardo ganham maior relevo quando lembramos
que a teologia é a ciência suprema na Idade Média, e a autoridade
suprema é a Escritura (\versal{LE} \versal{GOFF}, 2008, p. 299). E não se pode deixar de
notar que essa secularização da Escritura é um impulso racional moderno.

Pode"-se falar, então, de um ``Renascimento filosófico e científico'' no
século \versal{XII}, diferente daquele do século \versal{XV} que será de matiz mais
literário e artístico -- sem deixar de ser também uma continuação
daquele renascimento filosófico. Foi a época em que grandes textos
científicos gregos foram traduzidos do árabe. ``Equipes de eruditos
cristãos, judeus e árabes traduziam em latim textos gregos e árabes
acerca de medicina, astronomia, aritmética, álgebra e trigonometria.''
Para Gimpel, são essas traduções que vão preparar o salto da ciência
moderna.

É por representarem aspectos intelectuais que se opõem, que tensionam
com a forma social tradicional, que trazemos aqui esses aspectos. Por
exemplo: qual o tamanho do choque para a visão tradicional da afirmação
de Thierry de Chartres que tenta explicar racionalmente a criação, que
diz ser impossível compreender a Gênese sem a ajuda da matemática, onde
se encontraria a explicação racional do universo? Lembremos que essa é
uma concepção do século \versal{XII}. No mesmo sentido, Guillaume de Conches
sustenta a onipotência da razão. Para ele, se Deus ``criou a
natureza, ele respeita suas leis'' (\versal{GIMPEL}, 1975, p. 169). Como Deus
pode respeitar o que ele próprio criou? Isso já parece um corte do ponto
de vista da racionalidade de grande envergadura. Textos como estes
aceleram o processo de dessacralização da natureza que o próprio
cristianismo começara, uma dessacralização que ``é um dos fatores que
explica a invenção tecnológica na Idade Média'' (\emph{Idem}, p. 169).

Com toda evidência, a palavra tecnologia parece um tanto exagerada de
ser aplicada na Idade Média quando invenções importantes tiveram lugar,
mas sempre num quadro não totalizante na sociedade. Eram ainda técnicas
sem um todo racional próprio, ainda submetidas ao quadro
simbólico"-religioso. No entanto, há que se reconhecer a mudança
racional, intelectual que é necessária para se conceber e fazer uso
dessas técnicas. Além disso, quando a natureza não é mais sagrada, ela
não só pode ser explorada, como pode ser reinventada e sofrer
intervenção humana. Mas destaquemos ainda uma vez a lentidão dos avanços
científicos e da corrosão do universo simbólico tradicional religioso.

De todo modo, não há como a ciência avançar sem a dessacralização dos
objetos de estudo, algo que pressupõe uma ruptura com o mundo simbólico
anterior, portanto, uma dessacralização do que era sagrado no universo
de sentido anterior. É o caso, por exemplo, da experimentação da
dissecação levada a cabo por William Harvey, ``que encontrou uma
explicação fisiológica para os movimentos do coração e da circulação
sanguínea.'' (\versal{NEF}, 1954, p 69). E para isso foi preciso esperar o
período de 1570 e 1600, quase três séculos depois da primeira
dissecação, um novo período de desenvolvimento científico, apesar da
Contrarreforma.

Mesmo ainda estando num contexto completamente mergulhado na tradição
religiosa, é essa tradição religiosa que vai mudando paulatinamente para
adaptar"-se ao espírito do tempo. Prova disso é que Thierry de Chartres,
chanceler (reitor) da Escola de Chartres, mandou colocar, entre as
esculturas do portal real da Catedral, estátuas personificando as artes
liberais e seus atributos: gramática, dialética, retórica, aritmética,
astronomia, geometria e música. Pressupõe"-se que o ensino dessas
matérias não deixaria intacta a doutrina religiosa, da tradição, e os
questionamentos, é fato histórico, surgiram do seio da própria Igreja.

E esse processo de relacionamento entre razão e fé, que não tardará a
dar origem a tensionamentos, tanto subjetivos quanto também objetivos,
não se dá somente em Chartres, mas também em Paris e Oxford.

\begin{quote}
Paris é, na realidade e simbolicamente, para uns a cidade"-farol, a fonte
de todo gozo intelectual, e, para os outros, o antro do diabo onde se
misturam a perversidade dos espíritos conquistados pela depravação
filosófica e as dissipações de uma vida entregue ao jogo, ao vinho, às
mulheres. A grande cidade é o lugar da perdição, Paris é a Babilônia
moderna. (\versal{LE} \versal{GOFF}, 1984, p. 25)
\end{quote}

O fato de se introduzir nessa metrópole do pensamento medieval as obras
de Aristóteles e os comentários em árabe de Avicena e Averróis deixavam
a jovem universidade de Paris perante um sistema de pensamento
científico praticamente completo (\versal{GIMPEL}, 1975, p. 172). É assim que a
dialética de Aristóteles vem a ser considerada equivalente da ciência e
da razão. O confronto com esses filósofos vai obrigar a universidade a
um trabalho de síntese unificadora do aristotelismo e do cristianismo

\begin{quote}
Numa soma que englobaria a totalidade das verdades humanas e divinas, a
razão e a fé, a filosofia e a religião, a faculdade das artes e a
faculdade de teologia. Os pensadores medievais mais eminentes, Alexandre
de Hales, Albert Magno, Tomás de Aquino, Siger de Brabant tentarão essa
síntese com sucessos variados. Mas, temendo que essa síntese perturbe a
fé, o papa e os teólogos incentivam o bispo de Paris, Étienne Tempier, a
condenar em 7 de março de 1277 os 219 `erros execráveis que alguns
estudantes da faculdade de artes têm a temeridade de estudar e discutir
nas escolas'. Essa condenação vai frear o progresso das ciências e da
razão em Paris no século \versal{XIII}. (\emph{Idem}, 1975, p. 172)
\end{quote}

Apesar desse acontecimento não anódino -- que diz muito dos
tensionamentos criados pelo desenvolvimento de um pensamento que,
sorrateiramente, sem que se tivesse ainda a exata medida, não poderia
desembocar senão no questionamento do poder da Igreja, embora somente
por gerações muito futuras, que dirigiriam sua fé para o progresso -- em
Oxford, a ciência, especialmente experimental, vai progredir. Roger
Bacon e seu mestre, Grossetête, vão defender a ciência empírica para
provar os raciocínios.

Mesmo com a diminuição do ritmo intelectual reiniciado no século \versal{XI}, que
sofreu com a execração dos estudos dos estudantes de Paris em 1277, que
se fragilizou com a grande fome depois de 1315 e com a peste da metade
do século \versal{XIV}, as bases para o que se convencionou chamar de
Renascimento estavam lançadas. Será mero acaso que ele tenha nascido
numa região que contava com importantes cidades mercantis do mundo
medieval: Florença, Veneza, Gênova, Siena?

De todo modo, salta aos olhos o ritmo lento apesar de tudo do
desenvolvimento intelectual e técnico, o que só pode ser explicado pela
grande impregnação da tradição religiosa nas ciências e nas cabeças
comuns, pelo peso das relações sociais, e pelo pouco peso social dessas
invenções. Sem esquecer que o intelectual é antes de tudo um religioso,
imerso no universo simbólico religioso.

Além desses tensionamentos vindos do casamento não ameno entre razão e
fé, havia aqueles que faziam parte do que Le Goff (1984, p. 28-29)
descreve como \emph{vagabundagem intelectual,} um estranho grupo de
intelectuais para os quais Paris é o paraíso na terra, o bálsamo do
universo.

São os chamados Goliardos, clérigos ou errantes que são tratados de
vagabundos, charlatães e bobos, por vezes como jovens inconsequentes,
por vezes como ``perversores da ordem'', gente perigosa: ``Outros veem
neles uma espécie de \emph{intelligentsia urbana,} um meio
revolucionário, aberto a todas as formas de oposição declarada ao
feudalismo'' (\emph{Idem}, p. 28). São de origem urbana, camponesa e até
nobre, frutos do surto demográfico que faz das cidades também o ponto de
encontro de audaciosos, marginais e infelizes. Os Goliardos apareceram
no século \versal{XI}, são essencialmente ``estudantes pobres e clérigos que não
puderam encontrar uma função ou uma prebenda, embora a eles também se
juntem por vezes marginais.'' (\versal{WOLFF}, 1995, p. 19). Eles vagam de cidade
em cidade, seja para achar um generoso protetor ou até para seguir um
mestre de quem tenham gostado, ou até mesmo para escapar de proibições
ou excomunhões. São considerados gente perigosa pela Igreja, pois, sendo
``clérigos ou se passando por tais'', \emph{clerici vagantes}, dão"-lhe
uma visão negativa (p. 20). O estranhamento que causavam na
sociedade era porque viviam, perdoem"-me o exagero, uma espécie
de contracultura \emph{avant} \emph{la} \emph{lettre.} Se a Alta Idade
Média impunha a vinculação de cada um ``ao lugar que lhe pertencia, à
sua tarefa, à sua ordem, ao seu estado, os Goliardos são evadidos.
Evadidos, sem recursos, vão formar nas escolas urbanas esses bandos de
estudantes pobres que vivem de expedientes, se fazem criados dos
condiscípulos endinheirados, vivem da mendicidade [\ldots{}]''
(\emph{Ibidem}, p. 29). Esses estudantes pobres, chamados \emph{vagi
scolaris,} vão atacar severamente os representantes da ordem
estabelecida na época em seus poemas -- chamados \emph{Carmina}
\emph{Burana} --, como por exemplo em sua paródia chamada \emph{Início do
Evangelho segundo são Marco de prata}, em que expõem toda a corrupção da
Igreja, a venda de indulgências e a importância do dinheiro em relação à
fé verdadeira:

\begin{quote}
Naquele tempo, o papa disse aos romanos: ``Quando o Filho do homem vier
a nosso trono de glória, dizei"-lhe antes de tudo: `Amigo, qual o
objetivo de tua vinda?' Mas se ele persistir em bater à vossa porta sem
nada vos dar, jogai"-o às trevas do mundo exterior. [\ldots{}] Em
verdade, em verdade te digo [ao pobre]: Tu não entrarás na alegria
de teu senhor enquanto não deres teu último centavo. [\ldots{}] Se
deres marcos [dinheiro] até encher os cofres papais, serás absorvido
de todas as faltas [\ldots{}] A cúria romana não quer ovelha sem
lã.'' (\versal{CARMINA} \versal{BURANA}, 1995, p. 112-113).
\end{quote}

Do mesmo modo, vão louvar a volúpia carnal, o jogo e o gozo terreno.
Vejamos um trecho destacado por Le Goff (1984, p. 32):

\begin{quote}
\forceindent{}Mais ávido de volúpia que da salvação eterna

Com a alma morta, só a carne me preocupa

[\ldots{}]

Como é duro domar a natureza!

E, em face das belas, permanecer puro de espírito.

Os jovens não podem seguir uma tão dura lei

Nem esquecer o corpo disponível.
\end{quote}

Pôr a volúpia em lugar de maior destaque do que a salvação eterna não
pode senão chocar os espíritos da época. Mas apesar desse choque, desse
elogio do erotismo, dessa ode à transgressão, ao imoral, é possível dizer
dizer que eles representam uma revolução no pensamento ou mais uma
revolta, um tensionamento? Para Étienne Wolff, os Goliardos não constituem um
meio social de oprimidos. São homens de vários meios sociais unidos pela
idade e o estudo e que ainda não encontraram seu lugar na vida:
``Criticam a ordem social, mas sonham em obter as vantagens que lhes são
por enquanto recusadas ou que sua má conduta lhes fez perder: a proteção
de um mecenas generoso, uma gorda prebenda, uma vida doce e tranquila.''
(1995, p. 21). Talvez se possa dizer que suas ações e pensamentos mais
criem uma tensão com a tradição sem, no entanto, abalar a raiz da
doutrina tradicional, mesmo porque são uma minoria negligenciável no
quadro geral. Além disso, para Le Goff (1984, p. 32), falta a eles um
sentido de progresso e um sentido de história, típico de um pensamento
da revolução: ``[\ldots{}] a Roda da Fortuna que gira e preside a um
eterno retorno, ou o acaso cego que confunde os êxitos, não são, na sua
essência, temas revolucionários [\ldots{}]''. Era, nesse sentido, uma
busca de romper esse enquadramento e buscar os prazeres terrenos, o que
já tensionava com o quadro social existente. Mas, como lembra Étienne
Wolff, ``os goliardos atacam com virulência a Igreja, não a religião.''
(1995, p. 21). Tanto que os poemas de autores diversos não são apenas um
tensionamento com o quadro social existente, mas se revestem muitas
vezes de uma tonalidade moral em relação à Igreja por ela não cumprir
corretamente os preceitos do Evangelho. Se há poemas eróticos e de
elogio dos gozos terrenos, há também aqueles que destacam os sete
pecados capitais, bem como a superação do próprio prazer carnal.

De qualquer maneira, se por um lado a crítica dos Goliardos se volta
para os monges, vistos como sinônimos de luxúria, preguiça e gula, por
outro, volta"-se para alguns aspectos do cristianismo não de pouca
importância. Aqueles que dizem respeito à solidão, ao ascetismo, à
pobreza, à continência, à ignorância considerada como renúncia aos bens
espirituais. Uma crítica não menos importante e avançada para o tempo é
a que se dirige contra os nobres, como atestam os versos: ``Nobre é
aquele que a virtude enobreceu; Degenerado, aquele que nenhuma virtude
embelezou.'' (\versal{CARMINA} \versal{BURANA}, 1995, p. 56). Eis um tipo de pensamento
chocante para uma época em que o berço já designava todas as qualidades,
a nobreza. Colocar a nobreza ou a vileza como acessíveis a qualquer
mortal não deixa de ser um traço anunciador da liberdade moderna. Apesar
de ser aqui a qualidade o aspecto distintivo, o chamado
à transcendência àquele enquadramento, em proveito de uma relação ética
independentemente do berço. O que não tem nada a ver com estar livre
desse enquadramento religioso para buscar uma dignidade no mercado. Deve
estar claro que minha insistência nesses tensionamentos germinados no seio
social não implica que se trate de tensionamentos conscientes no sentido de romper com o
quadro simbólico reinante. Antes se trata de algo que vai amadurecendo
lentamente no subterrâneo social, sem que se trate ainda de movimentos
conscientes.

Esses tensionamentos, que aqui só brevemente esboçamos, podem ser
considerados como um impulso ao Renascimento que amadurecerá pouco
depois. Um Renascimento que, apesar de tudo, permanece sobre uma base
ainda religiosa, como tentativa de unir fé e razão.

As idas e vindas, o entrecuzamento subjetivo e objetivo na sociedade se
comprova também pelo misticismo e o início da caça às bruxas e aos
hereges que ganhou força após a execração de 1277. Esses entrecruzamentos
reforçam ainda mais a dificuldade de se concordar sem qualquer nuança com a
afirmação materialista da determinação da consciência pelo ser social,
uma vez que esse próprio ser social não se mostra necessariamente de
forma clara, mas sempre amalgamado, imerso em tensões que determinam a
consciência e o próprio ser social. Ou seja, depois desse \emph{élan}
intelectual, pôde"-se ter um intervalo de misticismo e diminuição da
razão em prol da justificação de uma forma social pela fé, cujos desvios
precisavam ser combatidos, extirpados, sob pena de a sociedade sucumbir
perante peste, fome, guerra e até impotência: ``Muitos homens que tinham
se tornado impotentes, sem razões aparentes, acusavam suas amantes de
tê-los enfeitiçado'' (\versal{GIMPEL}, 1975, 192). Astrólogos e bruxas estavam
entre os prediletos na perseguição. Não esqueçamos que povoa as
mentalidades e as sensibilidades um sentimento de insegurança (\versal{LE} \versal{GOFF},
2008, p. 298). Uma insegurança que está inclusive na base da coesão
social juntamente com a ideia de salvação. A ideia de sentir"-se seguro
jogará grande papel. E a Igreja trazia em grande parte essa segurança
subjetiva, mesmo a consequência sendo o que Kant chama de menoridade
intelectual, que é a consequência de todo fetichismo, inclusive o da
mercadoria. Mas como nosso intuito maior é ressaltar os aspectos que
tensionam com essa sociedade rumo à modernidade, passemos a um
acontecimento de grande importância para o avançar objetivo e subjetivo
rumo à modernidade: o transtorno do sentido de tempo.

\section{Mudança no sentido social e subjetivo do tempo: uma revolução rumo à modernidade}

O tempo sempre foi objeto de reflexões filosóficas. É bem conhecida a
diferença entre \emph{Chronos} e \emph{Kairos.} O que é comumente dito é
que \emph{chronos} é o tempo quantitativo, enquanto \emph{Kairos}, o
tempo qualitativo. Mas nem mesmo o \emph{chronos} tem relação com o
tempo moderno, não poderia ser visto como simplesmente quantitativo, uma
vez que a base de sua medida, embora sequencial e linear, é a
ciclicidade da natureza, que conhece distinções qualitativas. O tempo
\emph{chronos} não é um tempo coercitivo, que existe \emph{a priori}
como é a essência do tempo moderno. Ele mede a sequência de
acontecimentos, não são os acontecimentos que são medidos por ele. Certo
mesmo é que a ideia de considerar o tempo sob seu aspecto meramente
quantitativo somente nasceu na soleira da modernidade. Embora seja fato
histórico que os antigos teriam podido fazer sua clepsidra marcar
intervalos quantitativamente iguais, aquele contexto social, aquelas
subjetividades não viam qualquer sentido nessa medição que não se
coadunava com as mediações sociais reinantes.

Muitas foram as invenções técnicas que impulsionaram as relações
mercantis ainda na Idade Média, como o pisão de tecidos que dispensava
40 operários, a importância da energia hidráulica não só para o pisão
como para a moedura de cereais -- atividade da mais alta importância.
Tanto que houve proibições no século \versal{XIII} de moer grãos e bater tecidos
em casa, para que os camponeses fossem obrigados a pagar a renda dos
moinhos. Apesar disso, a maior parte da renda era paga em \emph{natura}
e só raramente em dinheiro (\versal{GIMPEL}, 1975, p. 20) e ainda estamos numa
sociedade largamente baseada numa reprodução social em que os
privilegiados abocanham a maior parte do produto social -- é por isso
que se pode falar de uma dominação pessoal, embora mediada por uma forma
de fetichismo.

Não se deve imaginar que se a Antiguidade não deu tanta importância a
mecanizações ou à utilização da energia hidráulica, preferindo o
trabalho escravo, foi devido simplesmente a uma desumanidade e que o uso
na Idade Média, que dispensava em alguns poucos casos a mão"-de"-obra, era
um exemplo maior de humanidade. Seria ingênuo. Se a Antiguidade mostrava
preocupação com a coesão social, alguns indivíduos na Idade Média, sob a
influência da visão que o comércio florescente trouxe à cena, tendiam a
pensar na ampliação dos ``lucros''. (\emph{Idem}, 1975, p. 14). Do mesmo
modo, seguindo Gimpel\footnote{É importante notar que esse autor não dá
  mostras de diferenciar o corte fundamental que significou a
  modernidade, apesar das preparações e antecedentes. Para ele, a
  modernidade é apenas a continuação do que vinha se dando na Idade
  Média. Nosso estudo, diferentemente, tentará mostrar que somente
  preparações intelectuais e técnicas não teriam como sozinhas
  desembocar na modernidade tal qual se configurou de modo mais
  cristalino com a Revolução Industrial e francesa.}, é um traço
marcante da técnica chinesa o fato de as grandes invenções, como a
pólvora de canhão, a imprensa, a bússola, não terem jogado um papel na
evolução do país. Um objeto tão indispensável hoje na indústria, o eixo
de comando de válvulas, já tinha sido idealizado pelos chineses antes de
Cristo (\emph{Ibidem,} p. 18). Todas essas invenções, no entanto,
jogaram um papel fundamental na germinação da modernidade ocidental. O
que prova mais uma vez que não é invenção técnica que determina por si
os rumos de uma sociedade, mais a confluência histórica de fatores
objetivos e subjetivos. E nessa confluência histórica há um aspecto
fundamental: a mudança ocorrida na forma de encarar o tempo, algo que surgiu
nas cidades de forte impulso comercial. Ora, o relógio tinha sido
construído por Su Song na China do século \versal{XI} sem que viesse ao
conhecimento do Ocidente. Ao que tudo indica, na China antiga e
medieval, o imperador tinha a prerrogativa de promulgar calendário, o
que equivalia a cunhar moedas no ocidente. Aceitar o calendário era
aceitar a autoridade. Todo aquele que se mostrava interessado nas
estrelas era objeto de desconfiança. (\emph{Ibidem}, 1975, p. 143).

No Ocidente, foi Giovanni di Dondi quem realizou o relógio astronômico,
ainda no século \versal{XIII}. Mas foi somente no século seguinte que começou a
se espalhar pelas fachadas de igrejas e torres municipais. Pádua, em
1344, onde está o pêndulo de Dondi, Gênova, em 1353, Florença, em 1355,
Bolonha, em 1356, Ferrara, em 1362, Paris, em 1370, soam horas iguais. O
rei Charles V obriga a autoridade eclesiástica a tocar o sino em
intervalos regulares de 60 minutos, o que era um ataque aos hábitos
religiosos com seus ritmos próprios. (\emph{Ibidem}, 1975, p. 159). E
essa nova temporalidade vai tensionar com o tempo predominante: o
canônico.

A Idade Média estava assistindo, portanto, ao advento do tempo abstrato,
do tempo que até os próprios mercadores tinham que aprender a considerar
como seu.

\subsection{Tempo abstrato e tempo concreto}

Se expus aqui questões relacionadas com o tempo, é porque a dinâmica
temporal abstrata é um dos aspectos fundamentais de uma tautologia, de
uma temporalidade pseudocíclica. A mudança de um tempo concreto para um
tempo abstrato vai dizer muito sobre a tautologia do vazio na
contemporaneidade. Esse tempo abstrato,
que faz abstração das qualidades concretas da vida social, vai, aos
poucos, corroendo as próprias qualidades sociais e dos indivíduos, que
vão cada vez mais raciocinar em termos da abstração que o próprio tempo
opera.

Moishe Postone, no Livro \emph{Tempo, trabalho e dominação social, uma
reinterpretação da teoria de Marx} faz uma incursão histórica
interessante para fundamentar sua distinção entre tempo concreto e tempo
abstrato e ao mesmo tempo localizar o aparecimento deste último como não
sendo algo casual ou técnico, mas devido a uma nova forma social que se
impunha no final da Idade Média: a forma social paulatinamente dominada
pela mercadoria.

Para ele, o tempo concreto significa, em verdade, diversos tipos de
tempo em função dos acontecimentos. É um tempo que se relaciona e deve
ser compreendido a partir dos ciclos naturais e os períodos da vida
humana. Assim, pode"-se falar de um tempo de cozinhar um arroz ou de
dizer um Pai"-nosso, ou o tempo de colher, de plantar, de coser uma
camisa. É um tempo que pode ser definido qualitativamente, posto que
dependente dos acontecimentos, como tempo bom, ruim, de paz, de guerra,
sagrado, laico, profano. (\emph{Idem}, p. 298)

Os modos de contagem ligados ao tempo concreto não dependem de uma
sucessão contínua de unidades temporais constantes, mas se fundam nos
acontecimentos principalmente naturais repetitivos, tais como dias,
ciclos lunares ou estações, ou seja, unidades temporais que variam. Para
Postone, que fundamenta seu estudo em vários outros autores, esse tipo
de tempo concreto predominou até o século \versal{XIV} na Europa e já dominara no
Egito Antigo, no Mundo Antigo, no Extremo Oriente e nos impérios
Islâmicos. Seria exagero dizer que esse tempo teria deixado de ser
predominante com o desenvolvimento do tempo abstrato no século \versal{XIV}. Até
porque, esse tempo abstrato foi uma experiência ainda bastante
localizada a áreas em que a manufatura de tecido tinha importância.
Portanto, essa experiência com o tempo abstrato, como os outros aspectos
que tensionam rumo à modernidade, começaram de modo restrito e foram
amadurecendo lentamente no seio social. A enorme maioria da população
continuava vivendo os ritmos ditados pelas horas canônicas, portanto,
pelo tempo concreto.

Essa forma de contagem do tempo concreto aparece ligada aos modos de
vida social estreitamente dominados pelos ritmos de vida ``naturais'',
agrários e pela atividade produtiva fundada no ciclo das estações, do
dia e da noite. (\emph{Ibidem}, p. 299). O fato de que a unidade de
tempo não seja constante, mas antes variável, indica que esta forma de
tempo é uma variável dependente, existindo em função dos acontecimentos
ou das ações.

O tempo abstrato é caracterizado, ao contrário, como um tempo vazio,
homogêneo, contínuo, uniforme, independente dos acontecimentos. Ele se
constitui num quadro independente em cujo seio o movimento, os
acontecimentos ou as ações acontecem. O tempo abstrato não depende dos
acontecimentos, são os acontecimentos que acontecem num tempo que já é
estabelecido de uma vez por todas e as ações são ``julgadas'' por esse
tempo que é divisível em unidades não qualitativas, constantes, iguais.

\begin{quote}
As horas não eram limitadas por nenhum evento natural, sendo antes
durações arbitrárias e passíveis de definição arbitrária. Os dias, em
contraste, tinham essas fronteiras na escuridão e na luz, e, além disso,
os calendários eram artefatos de milênios de civilização, repletos de
incrustações dos costumes e da religião. (\versal{CROSBY}, 1999, p. 82)
\end{quote}

Segundo Postone, o tempo abstrato foi muito bem definido, tempos depois,
no século \versal{XVIII} por Isaac Newton, que também era teólogo, em seus
\emph{Principia Mathematica:} ``o tempo absoluto, verdadeiro e
matemático que passa de forma igual, sem nenhuma relação com o que quer
que exista de exterior a ele'' (\emph{Ibidem}, p. 300). Embora não se
possa dizer que é esse tempo que começa a dominar, é o tempo que começar
a criar um tensionamento contínuo com a temporalidade dominante entre o
séculos \versal{XIV} e \versal{XVI} na Europa.

Este tempo abstrato, segundo Postone, que não existira até a Baixa Idade
Média, deve ser relacionado a uma forma de prática social estruturada
bem determinada que introduziu uma transformação da significação social
do tempo em certas esferas da sociedade europeia do século \versal{XIV} e que se
tornaria hegemônica no século \versal{XVII}: o desenvolvimento desse tempo está
ligado ao processo de generalização das relações mercantis. E por quê?

Postone explicita que no século \versal{XIV} o tempo concreto sofreu uma mudança
tão grande que a literatura do século \versal{XV} já tinha substituído as
``horas'' variáveis e as \emph{horae canonicae}, bem como o ano segundo
as estações e o zodíaco. (\emph{Ibidem}, p. 301). Como diz Crosby:
``Logo no início da Idade Média, havia apenas três dessas horas
[canônicas], que depois passaram a ser cinco e, finalmente, sete, e
que nunca se ancoraram solidamente no tempo dos relógios. (1999, p. 43).

Embora, para o autor, a passagem para o sistema de horas invariáveis,
intercambiáveis e mensuráveis tenha estreita relação com o
desenvolvimento do relógio mecânico no final do século \versal{XIII} e início do
século \versal{XIV}, não se pode explicar o surgimento do tempo abstrato pelo
surgimento de uma técnica, e já demos o exemplo da clepsidra. O
indicador desses marcadores temporais variava mais ou menos de acordo
com as estações. Pode"-se então reforçar a ideia de que o surgimento do
tempo abstrato é sócio"-cultural e não técnico. (\versal{POSTONE}, 2009, p. 303)

A prova de que o significado social do tempo é o determinante pode se
exemplificar no seguinte fato. Um missionário, um certo Matteo Ricci,
introduz o relógio mecânico na China no final século \versal{XVI}, sendo outras
cópias importadas, mas esses relógios não encontram uma verdadeira
utilidade e acabam sendo usados como brinquedos ou adornos. No Japão, as
antigas horas variáveis foram mantidas mesmo após a introdução do
relógio mecânico no século \versal{XVI}. Nessa época, os japoneses adaptaram os
ponteiros com números móveis. Assim, quando o Japão, no último terço do
século \versal{XIX}, introduziu a hora constante, não se pode encontrar a razão
no relógio, mas em um programa de ajustamento ao capitalismo.
(\emph{Ibidem}, p. 306)

O duplo problema do surgimento do tempo abstrato e do desenvolvimento do
relógio mecânico deve ser entendido segundo as circunstâncias nas quais
as horas constantes tornam"-se formas que de fato fazem sentido para a
organização social: ``[\ldots{}] as horas tinham um importância
central para os habitantes das cidades, já iniciados na onda da
quantificação pela compra e a venda.'' (\versal{CROSBY}, 1999, p. 82). Dois
lugares na Europa marcados pela vida coletiva, segundo Postone e também
Jappe (2006), interessaram"-se pelo tempo e sua medida: os mosteiros e os
centros urbanos.

Nos mosteiros, as rezas eram ordenadas temporalmente e ligadas às horas
variáveis segundo a regra Beneditina do século \versal{VI}. No entanto, a partir
dos séculos \versal{XI}, \versal{XII} e \versal{XIII}, o respeito à disciplina do tempo é
reforçado. Como explica Jappe, o trabalho na vida monástica era um dever
cristão a ser executado voluntariamente como expiação dos pecados e
mortificação da carne. Ou seja, não era mais, ``como na moral
pré-cristã, um mal necessário para alcançar uma finalidade, algo que se
delega noutros indivíduos, se possível. Pela primeira vez atribuía"-se ao
trabalho um significado moral.'' (2006, p. 189). Isto é particularmente
verdade na ordem Cisterciana fundada no século \versal{XII}. Essa ordem lançou
vários projetos agrícolas, manufatureiros, mineiros, numa escala
relativamente grande e insistia na organização temporal do trabalho, da
reza, da comida e do sono -- bem como no castigo aos desvios. Nos
mosteiros, a disciplina do tempo ganha grande importância juntamente com
o ``trabalho''. (\versal{POSTONE}, 2009, p. 307). Os monges criaram uma
organização econômica repousando numa excelente administração. ``Eles
faziam funcionar `as fábricas' mais modernas da Europa''. Além do mais,
jogaram grande papel no ``desenvolvimento, na Europa, da energia
hidráulica, da metalurgia nascente e do tratamento do mineral de ferro.
No domínio da agricultura, vemos os mosteiros ingleses criarem uma
economia dirigida para a exportação da lã.'' (\versal{GIMPEL}, 1975, p. 66). Dos
da França e da Alemanha, saía o vinho. Segundo Jappe, essa organização
também fazia parte desse fenômeno mais vasto do desenvolvimento do tempo
abstrato. O que poderia significar que o tempo abstrato não se opõe
tanto ao tempo canônico. Mas em verdade, como veremos no decorrer deste
estudo, não há como as horas abstratas se imporem sem deslocarem para um
plano secundário as horas canônicas. Por mais que a vida nos mosteiros
tivesse um ritmo consoante ao ritmo do tempo abstrato nascente, o tempo
abstrato pressupõe um universo mais vasto de fundamento da vida social,
um universo que não permite o fetichismo religioso senão quando
deslocado para a esfera privada, sem mais determinar a vida social e
subjetiva em profundidade.

Nesse sentido, mesmo com essa espécie de ``nicho'' representado pelos
mosteiros cistercianos, que de fato não o é, já que não há como a forma
mercantil se materializar de forma restrita, mesmo com esse quase
laboratório de produção isolado numa sociedade largamente fundada em
outras bases objetivas e subjetivas, Postone explica que é na cidade que
o tempo abstrato se desenvolve. Os sinos passam a assinalar a abertura e
o fechamento das feiras, o início e o fim do dia de trabalho, a hora
depois da qual não se podia mais vender álcool, etc. Embora a Igreja
fosse a favor de uma medida do tempo, ela tinha consciência de que a
mudança do antigo sistema de ``horas'' variáveis (\emph{horae}
\emph{canonicae}) abalaria o que representava seu domínio. Mas as
cidades não tinham interesse em manter esse sistema. (\versal{POSTONE}, 2009, p.
308).

Apesar disso, Postone se opõe à ideia de que foi o ritmo de trabalho nas
cidades que deslanchou o desenvolvimento do tempo abstrato enquanto o
campo continuava seu ritmo colado à ``natureza''. Segundo ele, até o
século \versal{XIV}, a cidade pouco se diferenciava do campo em seu ritmo de
``trabalho'' que era guiado do nascer ao pôr"-do"-sol pela hora canônica.
E Postone dá exemplo disso com a China, onde não só no campo, mas na
cidade, era a atividade que constituía o tempo e a ideia de
potencializar o tempo com a produtividade não existia. O que o autor
pretende defender, e subscrevemos sua opinião, é que a abstração do
tempo não é causada apenas pela técnica. Há que haver um ambiente
objetivo e subjetivo para a circulação mercantil, bem como uma produção
mercantil, mesmo que incipiente, para que o tempo abstrato possa fazer
sentido e passe a determinar paulatinamente a experiência concreta: e
isso só cabia nas cidades de então. Não é a cidade que diferencia o
tempo, mas o contrário. O fluxo comercial se dá nas cidades, a morada do
mercador são as cidades. É nesse terreno que vão se criar primeiramente
essas tensões rumo a uma concepção moderna de tempo. Para exemplificar
essa concepção moderna de tempo, Postone cita o historiador Thompson que
salienta o fato de que ``nas sociedades pré-industriais o tempo é
orientado pela tarefa''. Neste caso, o tempo é medido pela atividade
produtiva. No caso do capitalismo, que é o tipo de vida social onde esse
tempo faz sentido e se torna de fato cada vez mais totalizante, a
atividade produtiva é medida pelo tempo, por isso que toda atividade
produtiva se torna trabalho, uma atividade abstrata medida pelo tempo,
não importando se o tempo dessa atividade abstrata se materializou em
pão ou em canhão. (\emph{Idem}, p. 320).

Jacques Le Goff (1999a) denomina essa passagem como sendo a passagem do
tempo da Igreja para o tempo dos mercadores ou a passagem do tempo
medieval para o tempo moderno. Ele explica que os sinos de trabalho se
difundem a partir do século \versal{XIV} sobretudo nas cidades que fazem tecido.
A indústria medieval existia de alguma forma, mas privilegiava o mercado
local, sem ainda a ideia de expansão sem limites que caracteriza o
capitalismo. Mas a indústria metalúrgica e agora a têxtil produziam para
um mercado externo. (\versal{POSTONE}, 2009, p. 310).

Nas manufaturas têxteis, que não excluíam uma produção doméstica para
venda, os artesãos estavam diante de uma relação de fato entre
mercador e trabalhador, e não podiam mais vender diretamente aquilo que
produziam. São as primeiras relações entre capital e trabalho
assalariado numa escala reduzida da sociedade. E essa indústria
contribuiu para a monetarização crescente de alguns setores da sociedade
na Baixa Idade Média. É primeiro nela que a categoria da produtividade
passa a ter uma importância fundamental, visto que os mercadores davam a
lã e queriam que a diferença entre o salário que pagavam e a matéria
prima fosse cada vez maior. A produtividade do trabalho dependia,
portanto, do grau de disciplina e coordenação regular.

No final do século \versal{XIII} uma forte crise atinge o setor têxtil --- o que
engendra conflitos entre empregadores e trabalhadores. Na retomada
pós"-crise, são os trabalhadores que exigem prolongar o tempo de trabalho
para além do ``natural'' para recuperarem os salários perdidos com a
crise. Neste contexto, o relógio municipal foi introduzido marcando o
início e o fim da jornada de trabalho. As jornadas antes relacionadas à
natureza se modificam. Essa exigência por parte dos operários marca
justamente essa distância em relação aos laços com o tempo ``natural''.
(\emph{Idem}, p. 312). Essa luta em relação à jornada de trabalho,
explica Postone, não é somente ``a expressão mais direta da luta de
classes na economia capitalista'' -- embora o autor pareça exagerar ao
denominar luta de classes moderna uma luta por salários num contexto em
que a produção fabril ainda não é predominante --, ela tem a
consequência expressa de contribuir igualmente para a constituição
social do tempo como medida abstrata da atividade.

O relógio começa a expressar, assim, uma relação social em que o
rendimento da atividade é medido temporalmente. ``Em outros termos, com
a emergência das primeiras formas de relações sociais capitalistas nos
agrupamentos urbanos produtores de tecido da Europa Ocidental, aparece
uma forma de tempo que constitui uma medida e eventualmente uma norma
coercitiva da atividade.'' (\emph{Ibidem}, p. 313). Para Postone, a
emergência desse tempo abstrato como medida do trabalho está ligada ao
desenvolvimento da forma"-mercadoria nas relações sociais.

\begin{quote}
O ``progresso'' que representa o tempo abstrato enquanto forma dominante
de tempo está estreitamente ligado ao ``progresso'' que representa o
capitalismo enquanto forma de vida. À medida que a forma"-mercadoria se
impôs como forma estruturante da vida social no decurso dos séculos que
seguiram, o tempo abstrato se tornou cada vez mais corrente.
(\emph{Ibidem}, p. 315)
\end{quote}

Esse trecho parece esclarecedor do processo dialético. O tempo abstrato
vai se tornar a forma de temporalidade dominante conforme a vida social
mercantil vai subvertendo a vida social fundada no universo
simbólico"-religioso. Do mesmo passo, a subjetividade vai se habituando à
experiência dessa nova temporalidade. Apesar disso, durante séculos,
essas relações com o tempo ficaram ainda amalgamadas. O tempo abstrato
só se hegemoniza mais tarde. O campo continua seu ritmo e, nas cidades,
só os mercadores e os restritos operários são regidos por esse tempo.
Somente no século \versal{XVII} com a invenção do relógio de pêndulo feita por
Huyguens, o relógio mecânico se torna um instrumento de medida fiável e
a ideia de um tempo matemático foi explicitamente formulado.

Para Postone, ainda que socialmente constituído, o tempo no capitalismo
exerce uma forma de constrangimento (imposição) abstrato. Citando Aaron
Gourevitch:

\begin{quote}
A cidade se tornou dona de seu próprio tempo [\ldots{}] no sentido de que
esse tempo foi arrancado do controle da Igreja. Mas não é menos verdade
que é na cidade que o homem deixou de ser senhor do tempo, porque o
tempo, livre, desde então, para passar independente dos homens e dos
acontecimentos, estabeleceu sua tirania, à qual os homens devem se
submeter. \versal{GOUREVITCH}, Apud, \versal{POSTONE}, 2009, p. 317)
\end{quote}

Postone explica que, do mesmo modo que a atividade produtiva se
transforma, de uma ação individual, em um princípio geral alienado da
totalidade, ao qual os indivíduos estão submetidos, do mesmo modo o
dispêndio de tempo se transforma, de consequência da atividade, em
medida normativa para a atividade.

Fica claro, assim, que o capitalismo não teria se instalado sem uma
revolução no sentido do tempo e na relação da humanidade com o tempo,
uma vez que começava paulatinamente a se embrenhar na vida e a tensionar
com outras formas de reprodução social a frase com a qual o americano
Benjamin Franklin (1706-1790) resumiu o capitalismo: ``tempo é
dinheiro''.

\section{Sentido do tempo rumo à modernidade}

\begin{quote}
Um outro usurário muito rico, começando a lutar na agonia, pôs"-se a se
afligir, a sofrer, a implorar sua alma para não deixá-lo, já que lhe
tinha dado tudo, e lhe prometia ouro e dinheiro e as delícias desse
mundo se ela ficasse com ele. Mas que ela não lhe pedisse em seu favor
nem um vintém nem a menor esmola para os pobres. Vendo enfim que não
tinha como segurá-la, ficou encolerizado e, indignado, disse"-lhe: ``Eu
te preparei uma boa morada com abundância de riquezas, mas tu te
tornaste tão maluca e tão miserável que não queres encontrar repouso
nessa boa casa. Dá no pé! Eu te devoto a todos os demônios que estão no
inferno.'' Pouco depois, ele entregou seu espírito entre as mãos dos
demônios e foi enterrado no inferno.
\end{quote}

\emph{Exemplum} \emph{de usurário, citado por Le Goff}

Mesmo com o advento do tempo abstrato, o ``Tempo mensurável, mecanizado
até, como é o vivenciado pelo mercador, é também descontínuo, cortado
com paradas, momentos mortos, afetado por acelerações e movimentos de
lentidão.'' (\versal{LE} \versal{GOFF}, 1999a, p. 60).

Seria contraditório, pelo tom de nossa pesquisa, achar que o tempo
abstrato se impôs de imediato como o único. Há ainda muito terreno do
universo simbólico estabelecido a ser corroído. O próprio mercador, o
mais próximo desse tempo abstrato, não vai viver todos os momentos da
sua vida esse tempo -- ele próprio tem sua subjetividade imersa no seu
tempo histórico ao mesmo tempo que inconscientemente com ele tensiona.
``O tempo em que ele age profissionalmente não é o mesmo no qual ele
vive religiosamente''. (\emph{Idem}, p. 60). Temos aqui apenas um
primeiro protótipo de uma máscara de caráter, de um quase sujeito
esquizoide. Na perspectiva da salvação, ele vai aceitar as diretivas da
Igreja. Dos seus ganhos, uma parte vai para Deus, para a
``benevolência''. Dito de outro modo, ele é o ancestral próximo da ética
protestante: para ele, são ``diversamente legítimos os objetivos
perseguidos em perspectivas diferentes: o ganho e a salvação''.
(\emph{Idem}, p. 61). Mas como isso poderia se coadunar com o arcabouço
ideológico da Igreja?

O tempo medieval tradicionalmente se fundava na Escritura santa. O
tempo, portanto, começa com Deus e é dominado por Ele. Uma das razões
fundamentais para a Igreja condenar a usura, de início, estava ligada ao
tempo. Como poderia o mercador ou usurário querer cobrar juros daqueles
que precisam de um tempo para honrar o pagamento se aquele tempo não lhe
pertence? Era como se ele vendesse o tempo divino, o que coerentemente
seria um sacrilégio. Mas, como já dissemos, a relação da Igreja com os
mercadores é ambígua, ou vai se tornando sempre mais: ``Como poderia uma
religião que opõe tradicionalmente Deus e o dinheiro justificar a
riqueza, em todo caso a riqueza mal adquirida?'' \emph{(}\versal{LE} \versal{GOFF}, 1999a,
p. 1266). Momento digno de nota diz respeito à mudança na própria
prática confessional da Igreja após o século \versal{XIII}, que busca uma
``verdadeira justificação para a atividade do mercador'', ao mesmo tempo
que tenta atrelá-lo a uma regulamentação ``em que muito amiúde a
religião se deteriora em moralismo casuístico [\ldots{}]'' (\versal{LE}
\versal{GOFF}, 1999b, p. 63). Como historiciza Le Goff, esse século tem
como grandes problemas a usura e a monetarização da vida cotidiana que
ameaçam os valores cristãos. Nesse sentido, ele nota que será travada
uma luta cotidiana encarniçada, ``balizada por proibições repetidas, no
entrecruzamento de valores e mentalidades''. Isso para procurar
distinguir o lucro lícito do ilícito.

Mas a mudança do ato confessional tem consequências que vão na direção
de uma sutil valorização do indivíduo. Ora, até então, os pecados eram
julgados conforme os atos e não os atores. Mas, conforme Le Goff, a
visão da Igreja acerca do pecado e da penitência muda profundamente, ela
se espiritualiza, se interioriza. Desde então, para saber a gravidade do
pecado, a Igreja deve obter do pecador a \emph{confissão} para saber ser
ele agiu com boa ou má intenção. Assim, o penitente precisa se
interrogar ``sobre sua conduta e suas intenções, entregar"-se a um exame
de consciência. Um fronte pioneiro se abre: o da introspecção. Que vai
transformar lentamente os hábitos mentais e os comportamentos. São os
primórdios da modernidade psicológica.'' (\versal{LE} \versal{GOFF}, 1999a, p. 1267).

Trata"-se de um movimento sutil operado pela Igreja para que o usurário
ou mercador não precise mais fazer uma opção dolorosa entre a
\emph{bolsa} e a sua \emph{vida} eterna (\emph{Idem}, p 1269). Com diz
Le Goff, num mundo em que o dinheiro é Deus; ``onde o `dinheiro é
vitorioso, o dinheiro é rei, o dinheiro é soberano'''; onde a
\emph{avaritia}, a avareza -- ``pecado burguês de quem a usura é mais ou
menos a filha'' -- destrona a \emph{superbia}, soberba -- pecado feudal
-- ``o usurário, especialista em empréstimos a juros, torna"-se um homem
necessário e detestado, poderoso e frágil''. (\emph{Idem}, p. 1265).
Assim, a tradição rígida religiosa vai se tornando mais maleável. O novo
poder que ronda a sociedade a obriga a essa ginástica do casuísmo que
aparentemente não mina as bases da Igreja que, enquanto instituição,
soube acomodar o poder da nova riqueza advinda do acúmulo mercantil --
sem notar que a longo prazo significaria seu deslocamento enquanto
fundamento da vida social. Até mesmo preceitos como jejum, abstinência,
repouso dominical já não são considerados ao pé da letra, mas, diante
das necessidades profissionais, passam a ser recomendações ``segundo o
espírito''. (\versal{LE} \versal{GOFF}, 1999b, p. 64).

Eis então, como mais um elemento empurra o homem para a modernidade,
para o desabrochar de uma subjetividade moderna, não sem antes passar
pelas tensões preparatórias do humanismo renascentista, que a própria
escolástica ajudara a impulsionar: ``Liberto e tirano, o homem do
Renascimento -- aquele que ocupa uma suficiente posição econômica,
política ou intelectual -- pode, ao grado de sua Fortuna que ele utiliza
segundo as capacidades de sua \emph{virtù,} ir aonde quiser. Ele é
mestre de seu tempo, como do resto.'' (\emph{Ibidem}, p. 64). Pelo menos
em aparência, ou em oposição à vida social pouco dinâmica reinante no
medievo.

Esse amálgama subjetivo, portanto, sobreviverá ainda por muito tempo.
Além do mais, seria exagerado achar que de fato o homem renascentista
estaria liberto das amarras da tradição religiosa. Essa adaptação da
Igreja vai jogar um grande papel nessa convivência. Do mesmo modo, não
se pode achar que o \emph{petit peuple} aceitou tranquilamente a
imposição do tempo abstrato como regulador da vida. Houve tensionamentos
também nesse quesito. Pelo desrespeito dos operários nas cidades
produtoras de tecido, sobretudo nos Flandres, foram feitas leis contra
atrasos, com multas. E a mais pesada era exatamente contra aqueles que
usassem o sino para chamar uma revolta (\versal{LE} \versal{GOFF}, 1999a, p. 71).

\begin{quote}
Talvez fosse preciso buscar na própria ciência, ou seja, na escolástica
científica, o aparecimento de uma nova concepção do tempo, um tempo que
não é mais uma essência, mas uma forma conceitual a serviço do espírito
que dele faz uso segundo suas necessidades, que pode dividi"-lo, medi"-lo
-- um tempo descontínuo. [\ldots{}] Perder seu tempo se torna um
pecado grave, um escândalo espiritual. (\emph{Ibidem}, p. 77)
\end{quote}

O tabu segundo o qual ``O tempo é um dom de Deus e não pode ser
vendido'' é vencido na aurora do Renascimento. O tempo agora é
propriedade do homem, é sinônimo de liberdade, de individualidade.
Poderíamos apostar na positividade desse pensamento claramente
antropocêntrico, humanista, em oposição com a moral congelada da Igreja.
No entanto, essa descida do tempo dos céus à terra, ou sua
dessacralização para ser pretensamente possuído pelo homem na terra, vai
ser apenas um ato preparatório da dominação, em verdade, do homem pelo
tempo, que, ganhando um caráter abstrato, vai paulatinamente tornar a
vida uma tautologia travestida de mudanças contínuas, uma série de
momentos submetidos -- queiramos ou não -- à cadência do relógio que
martela a cadência da vida embalada pela mercadoria.

\section{O impulso para a marcha rumo à modernidade}

Numa das aventuras de Asterix e Obelix, \emph{Asterix e Companhia}
(1976), um jovem romano ambicioso sugere a César um plano astucioso para
minar a coesão social dos gauleses, a única possibilidade de vencê-los.
Esse plano consiste simplesmente em comprar indefinidamente menires
feitos por Obelix. Até então, os objetos de pedra eram feitos como uma
atividade normal de Obelix, dentro de um conjunto de várias outras
dentro da comunidade. Mas com a compra dos menires pelos romanos, Obelix
se viu com poderes oriundos do dinheiro que ganhava com essa exportação,
tendo que produzir sempre mais pela encomenda crescente. Não demorou
muito para começar a empregar seus amigos que o questionavam por ele não
ter mais tempo para nada. De repente, Obelix tinha esquecido seus ideais
de guerreador pela sua comunidade, ele tinha quebrado seus laços com os
outros por ter se tornado alguém poderoso, não mais por ser forte na
proteção da comunidade contra os inimigos invasores, mas graças aos bens
que tinha comprado com o dinheiro das exportações crescentes do objeto
inútil aos romanos. Assim, Obelix passa a ser invejado na comunidade por
seu poder oriundo do dinheiro. Para resumir, Obelix acaba acordando do
encanto provocado pelo dinheiro -- que, na história, leva Roma à
bancarrota por gastar tanto com um objeto que não teria qualquer
utilidade.

O que salta aos olhos nessa história em quadrinhos aqui apenas resumida
é o quanto a sede que o dinheiro provoca, tão logo deixa de ser simples
moeda mediadora pré-moderna para se tornar fim em si, é perturbadora dos
nexos sociais. E o quanto o ato consciente dos romanos contribuiu para
que o dinheiro assumisse o poder dilacerador das relações naquela
comunidade\footnote{Há também um episódio dos Smurfs chamado \emph{Os
  Smurfs e a árvore do dinheiro} em que acontece algo parecido. Às
  voltas com os Smurfs, Gargamel aceita a ideia da mãe de implantar o
  dinheiro no seio daquela comunidade como único meio de arruiná-los.
  Como se tratava de uma comunidade de partilha em que não havia
  dinheiro, o dinheiro cumpriria o papel de dilacerador dos laços. O
  mais interessante é que, diferentemente da história de Obelix com seus
  menires, o plano de Gargamel não conseguiu implantar no seio da
  comunidade o desejo da moeda feita de chocolate. Parece que a
  comunidade era bastante arraigada em sua vida social coletiva para ser
  corroída pelo desejo individual do dinheiro. Em vez de o personagem
  guloso ser invejado e bajulado pela sua riqueza, é ele mesmo que acaba
  se isolando do restante da comunidade. Isso até ele conseguir resistir
  ao seu \emph{lado mal --} que ficava lhe sussurrando ideias de acúmulo
  e egoísmo -- quebrar o encanto e entender que a verdadeira riqueza que
  poderia acumular era a partilha da vida com os amigos da comunidade.}.

Algo parecido aconteceu na aurora da modernidade, guardadas as devidas
proporções. E não foi devido ao advento da produção manufatureira de
tecido tão florescente. Mas sim ao aparecimento de um poder social
inaudito representado pelas armas de fogo. Como se sabe, a guerra fazia
parte da paisagem da Idade Média. Mas o caráter da guerra tinha mudado
pouco da Antiguidade até então. Como diz Kurz, ``Só as armas de fogo,
canhões e mosquetes trouxeram a nova qualidade de máquina destrutiva,
que transcendia a capacidade de combate imediata do homem e que, a par
com a guerra, haveria de revolucionar as bases sociais -- processo que
se arrastou ao longo de vários séculos.'' (\versal{KURZ}, 2014, p. 102). Antes
dessa mudança, qualquer um da comunidade podia fazer suas armas como
artesão. Além do mais, a classe dos guerreadores era tão importante,
como já vimos, como a dos rezadores e cultivadores de terra. Mesmo sendo
muito mortífera, a guerra continha certos preceitos que se
materializaram na moral cavaleiresca que se misturava com a moral da
própria sociedade como um todo. Tanto que o próprio Concílio de Latrão,
em 1129, proibira o uso de balestras, surgida em torno do século X, e
considerada uma arma desleal, uma arma que mata à distância, que evita o
enfrentamento e que prescinde de um valente guerreiro (\versal{NEF},1954a;
\versal{GIMPEL}, 1975; \versal{KURZ}, 2002). Ou seja, qualquer um, com uma habilidade
apenas técnica, podia derrubar um valente cavaleiro cuja vida era
devotada à guerra. Se a proibição foi respeitada ou não no campo de
combate, não é central aqui. Interessa"-nos mais ressaltar que se tratava
de uma relação social simbólica de onde se gestavam tantas mediações
que, mesmo em se tratando de disputa, era possível não aceitar qualquer
invenção somente porque poderia trazer uma vantagem maior. Ainda havia
possibilidade de limite. É nesse sentido que John Nef destaca que nem
mesmo as cruzadas, quando eram necessários esforços de guerra
consideráveis, significaram revolução das armas. Até mesmo a Igreja
proibia o uso contra os muçulmanos de armas cujo uso estivesse proscrito
entre católicos, como o uso da balestra. (1954, p. 46).

Mas os passos que a Idade Média vinha paulatinamente dando para fora de
si mesma criava o quadro social propício para a aceitação e implantação
do canhão e de outras armas. Embora a pólvora tenha sido inventada pelos
chineses no século \versal{IX}, parece que a Europa somente vai conhecer tal
invenção no século \versal{XIII}, segundo Braudel (1979). O que é fato histórico
é que o canhão vai surgir na Europa no século \versal{XIV}, sobretudo nas cidades
de grande desenvolvimento comercial e industrial, e que vai jogar,
juntamente com outras armas de fogo\footnote{Não podemos passar ao lado
  de um fato histórico interessante para nosso estudo: no período
  histórico chamado Isolacionismo japonês, após a expulsão dos
  portugueses e unificação do Japão, no século \versal{XVI}, foram destruídas
  armas de fogo, herança dos portugueses, em proveito do sabre,
  considerado mais civilizado (\versal{BRAUDEL}, 1979).}, um grande papel no
nascimento do Estado moderno, com o poder centralizado, diferentemente
da enorme descentralização medieval. Como diz Crosby: ``A Europa
Ocidental era um viveiro de jurisdições -- reinos, ducados, baronatos,
bispados, comunas, guildas, universidades e muito mais --, uma mistura
de controles e contrapesos.'' (1999, p. 61).

Já na metade do século \versal{XIV}, descobriu"-se que o procedimento de fundir os
sinos podia ser usado para fazer canhões. Mas o que resultou foi uma
arma pesada, de difícil manuseio, cara, que causava pouco estrago. Foi
preciso ainda mais de um século para que se pudesse aperfeiçoar a
invenção, uma demora que certamente não se deveu somente às condições
técnicas, mas também ao meio social ainda não aberto de todo ao uso
dessa arma de poder destrutivo sem precedentes, embora perante os
primeiros canhões até mesmo as catapultas antigas fossem mais eficazes.

Mas em 1490 os franceses conseguiram fazer um canhão que significava uma
inovação importante: ``Artesãos e sineiros franceses [\ldots{}] no
início da década de 1490 [\ldots{}] tinham desenvolvido um canhão
que era reconhecivelmente a mesma criatura que iria decidir as batalhas
e assédios pelos quase quatrocentos anos que estavam por vir.'' (\versal{DUFFY},
apud \versal{KEEGAN}, 2006, p. 407).

Aparentemente, essa revolução foi posta em prática em 1494 (\versal{NEF}, 1954a;
\versal{PARKER}, 1993; \versal{KEEGAN}, 2006), quando a França parte pela Itália para
conquista do reino de Nápoles com seus 40 canhões. Embora o rei Charles
\versal{VIII} tenha morrido logo depois de ser forçado a sair de Nápoles, ``sua
revolução de artilharia revelou"-se duradoura. Os novos canhões obtiveram
um efeito que os engenheiros de assédio tinham buscado durante milênios
sem êxito.'' (\versal{KEEGAN}, 2006, p. 409). Embora os canhões aperfeiçoados
pelos franceses fossem mais leves, o terror que a artilharia implantava
era ainda mais por causa do barulho que dispersava as tropas do que
pelos estragos realmente diretos. Seu estouro era uma novidade tão
grande que houve quem os condenasse como coisa do diabo no século \versal{XVI}
(\versal{NEF}, 1954a, p. 67).

Nesse sentido, o uso da pólvora como força propulsora de uma arma exigia
que a subjetividade dos indivíduos também avançasse, saísse das
crendices, ou seja, era preciso superar a ``desconfiança, a ansiedade e
o puro medo'' (\versal{KEEGAN}, 2006, p. 407) perante um elemento cujo poder de
destruição fugia ao controle do indivíduo. Um medo não infundado já que
muitos tiros explodiam no próprio canhão, sendo os próprios campos de
batalha um campo de experimentação. Apesar disso, na metade do século \versal{XV}
alguns soldados já experimentavam armas de fogo e, ``por volta de 1550,
elas já eram de uso comum.''(\emph{Idem}, p. 417).

Mas qual a consequência moderna das armas de fogo além de seu poder
destrutivo ainda não conhecido? Robert Kurz, em seu livro \emph{Dinheiro
sem valor} (2014), de certo modo, opõe a tese da revolução
militar como \emph{boom} da modernidade aos desenvolvimentos de Marx
sobre a chamada \emph{Acumulação primitiva.} Não é que a análise de Marx
estivesse equivocada, antes o contrário. Mas é que a análise de Marx se
baseara muito mais no caso original inglês, onde as transformações das
condições feudais desde muito, antes do que em outras partes, deixaram
uma massa de pessoas desenraizadas e forçadamente liberadas dos vínculos
tradicionais. Mas, para Kurz, esse desenraizamento não necessariamente
teria forçado à criação de uma massa pronta para ser explorada nas
fábricas. Assim, ele aposta muito mais na tese da revolução militar como
um detonador moderno, pela mudança qualitativa de caráter mais geral a
que ela conduz, diferentemente do caso das mudanças forçadas no campo
inglês que têm um caráter mais localizado, num primeiro momento, para só
depois servir de exemplo a outros países -- além de não darem
primordialmente o impulso à concorrência entre os Estados, como foi o
caso da armas de fogo.

Dito de outro modo, as armas foram \emph{a mercadoria} detonadora do
processo moderno, elas tiveram uma característica que nenhum outro
produto de manufatura conseguira adquirir, porque nenhuma teve o poder
de despertar no Estado uma sede fiscal que desequilibraria a vida social
e empurraria as subjetividades para um contato maior com o poder do
dinheiro. Para Kurz, não se assistia com a revolução militar a uma
simples ampliação da circulação do dinheiro, ou a um ``regresso das
relações monetárias da Antiguidade tardia'', mas a uma ``transformação
qualitativa do próprio dinheiro'' que, mesmo não sendo a forma universal
da riqueza até então, tinha as características requeridas para pôr em
aceleração o movimento rumo à modernidade. ``Na sequência da revolução
militar, os fluxos de recursos tinham que se movimentar sob a forma de
fluxos monetários.'' (2014, p. 107).

Para Nef, a concentração industrial não tem necessariamente relação com
a produção de armas, pois já havia grandes polos arsenais como o de
Veneza que data de 1104 -- que fabricava essencialmente navios de guerra
e mercantes. Mas essa concepção de Nef -- para quem o capitalismo é tão
antigo quanto a existência do homem na terra\footnote{Esse respeitado
  autor, embora diga que a característica da sociedade industrial é a
  multiplicação, defende a tese -- de espantar qualquer espírito
  preocupado com a dialética entre lógica e história -- de que ``os
  primeiros selvagens eram capitalistas quando construíam cabanas para
  morar e o capitalismo sempre existiu nas sociedade civilizadas.''
  (\versal{NEF}, 1954b, p 29.)} -- não distingue a diferença entre arsenais cujo
desafio é aperfeiçoar as técnicas de guerra que em essência continuavam
as mesmas, como era o caso do arsenal da Veneza capital dos mercadores,
e a produção moderna desse invento que injeta uma dinâmica nova na vida
social e subjetiva. Embora não se trate simplesmente do fato de as armas
de fogo terem engendrado um polo industrial de novo tipo, esse é também
um aspecto em meio a um conjunto de aspectos que formam o quadro tenso
no seio social ao alvorecer da modernidade. O próprio Nef afirma que as
armas de fogo fizeram multiplicar por cinco em poucos anos a produção de
minério de ferro. É ele também quem afirma que guerra e o industrialismo
moderno são também fundados na produção de prodigiosas quantidades de
metal, sobretudo ferro e aço. (1954a, p. 57). É ele mesmo quem diz que
se os canhões iam se tornar o principal meio de ganhar batalhas, era
preciso que fosse feito em abundância: ``A arte moderna da guerra não
depende menos das condições que facilitam o crescimento da produção
metalúrgica na véspera da Reforma do que da descoberta da pólvora e do
canhão.'' (\emph{Idem}, p. 59).

Além desse aspecto incentivador de um setor fabril de dinâmica nova,
incomparável pela sua essência com qualquer manufatura de tecido, as
armas engendraram uma monetarização tautológica no seio social, além de
serem um elemento fundamental na concorrência no sentido moderno, uma
concorrência que apressa a marcha para a modernidade capitalista. ``A
rivalidade entre a tecnologia dos canhões e as inovações na construção
de fortificações conduziu a uma escalada que se pode descrever
recorrendo a um conceito de concorrência, no plano da técnica militar''
[\ldots{}] (\versal{KURZ}, 2014, p. 104). Os Estados mantinham uma constante
rivalidade no sentido de uma corrida militar assim que uma nova técnica
parecia trazer vantagem. A história nunca conhecera esse critério
concorrencial. E o detalhe fundamental é que essa concorrência consumia
improdutivamente uma quantidade enorme de recursos, montantes com os
quais os Estados não estavam habituados\footnote{E o gasto militar só
  aumentaria. A França do início do século \versal{XVIII} aloca 75\% de suas
  receitas à guerra. A República inglesa nos anos 1650 chega a gastar
  90\% das receitas públicas com o exército e a marinha. (\versal{PARKER}, 1993,
  p.161).}. Segundo Parker (1993), ``a guerra tinha se tornado uma prova
de força financeira tanto quando de força militar.'' (p. 161). Nenhum
outro tipo de produção, nem a produção florescente de tecidos, nem mesmo
de antigas armas de guerra, tinham tido o poder de gerar um
desequilíbrio social devido à sede fiscal dos Estados. A revolução
militar teria assim conduzido a uma economia das armas de fogo, que
teria turbinado o sopro da marcha moderna em termos objetivos e
subjetivos que se pudera sentir desde o florescimento do comércio no
século \versal{XI}. Os mercadores não tiveram dificuldade, pela sua subjetividade
atinada com os novos tempos, em adentrar pela produção de armas. Muitos
enriqueceram prestando serviços às monarquias.

\begin{quote}
Os canhões deixavam de poder ser produzidos por ferreiros de aldeia ou
por fabricantes artesanais urbanos, uma vez que esse fabrico exige
grandes instalações. Devido à crescente procura de metais associada a
este desenvolvimento, as indústrias mineira e siderúrgica foram objeto
de um rápido crescimento forçado. Surgiu um complexo militar
proto"-industrial, até então desconhecido, que requeria uma logística de
grande envergadura, igualmente inusitada.'' (\versal{KURZ}, 2014, p. 105)
\end{quote}

E as armas de fogo foram talvez o primeiro exemplo de tecnologia, de
invento tecnológico moderno fundamental para a ``aceleração'' da marcha
rumo à modernidade. Já em 1512, 18 anos depois de ter aparecido
eficazmente no navio francês em Rapallo, os novos canhões já jogaram
``papel predominante numa batalha naval entre os navios à vela
ingleses\footnote{Como discorre Hobsbawn: ``Os navios e o comércio ultramarino eram,
  como todos sabiam, a seiva vital da Grã-Bretanha, e a Marinha, sua
  arma mais poderosa. Por volta de meados do século \versal{XVIII}, o país
  possuía talvez 6.000 navios mercantes, perfazendo talvez meio milhão
  de toneladas, varias vezes mais que a frota mercantil francesa, sua
  maior rival. Essa frota formava possivelmente um décimo de todos os
  investimentos em capital fixo em 1700, enquanto seus 100.000 marujos
  representavam quase o maior grupo de trabalhadores não agrícolas.''
  (2014, p.13). Era impossível impor"-se como potência moderna sem uma
  artilharia de guerra potente. Não é só por razões geográficas que a
  Inglaterra tinha supremacia nos mares, mas por questões econômicas que
  não carecem de detalhe aqui. A guerra, portanto, no sentido moderno
  ganha o peso de um investimento fundamental, um \emph{faux}
  \emph{frais} sem o qual a modernidade não tem como seguir em frente,
  já que pressupõe a concorrência. E é aqui que chegamos a um ponto
  fundamental da relação entre crescimento econômico, acumulação de
  capital, exportação de bens manufaturados e a indústria da guerra. A
  Inglaterra entre 1700 e 1750 aumentou sua produção voltada para o
  mercado interno em 7\% e para o mercado externo em 76\%; entre 1750 e
  1770, foram 7\% e 80\% respectivamente. Como seria isso possível sem
  uma artilharia nova de guerra para garantir o domínio perante a
  concorrência e os países submissos? (\emph{Idem}, p. 39). Além do
  mais, a guerra ``contribuía também, e mais diretamente ainda para a
  inovação tecnológica e para a industrialização.''. (\emph{Idem}, p.
  40).} e franceses.''. (\versal{NEF}, 1954a, p. 52). E a própria vitória
violenta sobre os povos das colônias -- bem como o aprisionamento de
escravos negros -- aconteceu de modo mais fácil pelo uso das armas de
fogo -- e não importa se elas somente assustavam ou se matavam. Nef cita
o caso dos indígenas mexicanos que, embora sendo milhares, foram
vencidos pela força das armas de fogo, entre 1519-21, por algumas
centenas de homens. (1954a, p. 50).

A vitória francesa em Nápoles, embora não duradoura, teve consequência
para a estratégia militar e os investimentos estatais, tanto em
armamentos como sobretudo num primeiro momento em fortificações: ``Os
castelos constituíam uma primeira linha de defesa para muitos,
principalmente para Estados da Europa renascentista, com sua construção
e manutenção absorvendo uma grande parte das receitas estatais.''
(\versal{KEEGAN}, p. 410).

Essa é a tese de Geoffrey Parker (1993) para o qual a principal
contribuição das armas de fogo para a monetarização não é pelo fato de
as fábricas de armas de fogo terem se espalhado por toda a Europa -- com
efeito, em princípio, a produção se concentra em canhões, porque ainda
vai ser preciso esperar mais de um século para se ter armas portáteis
eficazes\footnote{Como diz Parker, em 1490, a república Veneza substitui
  suas arbaletas por armas de fogo, mas as performances das primeiras
  armas de fogo deixavam a desejar: ``Um arqueiro bem treinado podia
  lançar dez flechas por minuto com uma precisão aceitável até o limite
  de 200 metros, enquanto a recarga de um arcabuz no início do século
  \versal{XVI} levava vários minutos e sua precisão só era aceitável até em torno
  de 80 metros.'' (1993, p.90).}. Sendo assim, o aspecto central para o
autor é que os castelos não serão mais imperscrutáveis com o
aparecimento do canhão, o que vai exigir uma revolução em termos de
engenharia de fortificação, o que significa um aumento considerável dos
custos do Estado: ``Por mais custoso que fosse lançar uma campanha
militar, em geral ela não era tanto quanto a construção ou a defesa de
fortificações modernas.'' (1993, p. 120). Mas não imaginemos uma
revolução imediata, mas como algo que vai amadurecendo no processo de
desenvolvimento dos Estados modernos pelos séculos \versal{XVI} ao \versal{XIX}. Se, na
Idade média, o assédio de um castelo não significava derrubar suas
muralhas, ``no fim do século \versal{XV}, as mais majestosas muralhas, tais como
as de Carcassonne, uma das glórias da arte medieval e o tipo mais
perfeito talvez de construção defensiva pelos fins do século \versal{XIII},
tornaram"-se obsoletos.'' (\versal{NEF}, 1954a, p. 79) A nova engenharia de
fortificação ensinava que o meio eficaz de defesa era a construção de
muralhas preparadas para receber canhões para o ataque, o que significa
duplo custo. Conforme Parker (1993, p. 83), ``a muralha de sete
quilômetros de Anvers, com seus nove bastiões e cinco portas
monumentais, custou um milhão de florins'', um montante que equivalia a
mais de 100 mil bois.

Já para o autor alemão Werner Sombart, citado tanto por John Nef (1954a)
quanto por Robert Kurz (2014), a guerra teria um aspecto positivo de
desenvolvimento econômico pela ascensão do capitalismo nos séculos \versal{XVI},
\versal{XVII} e \versal{XVIII}, ``pois o aumento do número de soldados armados e do
dispêndio necessário para sua manutenção tornou necessário o acúmulo de
grandes massas de capitais e conduziu ao desenvolvimento de empresas de
grande envergadura na indústria, no comércio e na finança.'' (\versal{NEF}, 1954,
p 99). Na França, a produção era dispersa, não só dos canhões, como da
pólvora, até por questões de fronteira -- uma dispersão que não impede
em nada a monetarização, embora não necessariamente incentive polos
industriais concentrados. Já na Inglaterra, ``a fabricação de pólvora
foi concentrada numa única empresa que possuía vários moinhos em
diferentes pontos dos arredores de Londres.'' (\versal{NEF}, 1954a, p. 103). No
começo do século \versal{XVII}, a única grande manufatura de armas era a da
cidade de Thüringerwald. De lá se exportava para vários países da Europa
incluindo França, Espanha, Rússia, Dinamarca. (\emph{Idem}, p. 104).

O que é importante reter aqui é que a introdução das armas de fogo,
sobretudo do canhão, nas guerras, foi transformando num processo
paulatino a forma de guerrear dos Estados -- pois somente os mais ricos
``são capazes de sustentar os custos fabulosos da guerra nova.''
(\versal{BRAUDEL}, 1979, p. 443) -- a vida econômica e, ao mesmo tempo, a ética
cavaleiresca. A introdução das armas de fogo não só tornava as antigas
armas ridículas, mas também significava uma mudança de ordem subjetiva
pela gestação de uma nova ``moral'' da guerra que tornava a honra, a
ética, as simbologias cavaleirescas ridículas, mediações ou floreados
inúteis: ``Depois do século \versal{XV}, o cavaleiro foi relegado ao mundo
misterioso da fantasia.'' (\versal{NEF}, 1954a, p. 53).

Portanto, seguindo Robert Kurz, não foi uma força criadora, como pensam
Sombart e Nef, mas uma força destrutiva que abriu o caminho para a
humanidade passar a soleira dos tensionamentos medievais e acertar o
passo rumo à modernidade.

Como nunca antes, uma parte enorme do produto social precisou ser usada
na máquina da guerra: ``os estados territoriais nascentes passaram a
rivalizar entre si na aquisição de armas de fogo, sobretudo canhões, e
também na construção de fortalezas cada vez mais sofisticadas.'' (\versal{JAPPE},
2006, p. 189). As armas, os mercenários e as grandes obras de
fortificação de novo tipo precisavam ser pagos em dinheiro. Segundo
Kurz, somente por obra da produção agrícola ou artesanal crescente não
se teria chegado ao poder anônimo do dinheiro. Foi a sede incessante de
dinheiro dos Estados\footnote{Como diz Quesnay lá no meio do século
  \versal{XVIII}: ``Um homem pobre que apenas tira da terra com seu trabalho
  gêneros de pouco valor, como batatas, trigo"-preto, castanhas, etc.,
  que se alimenta com eles, que não compra nada nem nada vende, apenas
  trabalha para si próprio: vive na miséria; ele e a terra que cultiva
  nada rendem ao Estado. (1758-1983, p. 334).} para manter a guerra
constante -- frente a qual as guerras medievais parecem mais rixas --
que elevou o dinheiro à categoria de poder anônimo, ao fazê-lo sair da
sua condição de meio limitado de troca (moeda) submetido a todo um
emaranhado de relações para se tornar um fim, agora com uma tendência a
não mais ser um ente estranho ao conjunto da sociedade, mas um ente que
se entranha cada vez mais no seio social com uma tendência em si de se
ampliar sempre mais.

\begin{quote}
Os caudilhos dos mercenários dos primórdios da Modernidade
(condottieri), assim como os seus subordinados, os simples canhoneiros e
mosqueteiros, foram os primeiros sujeitos completamente retirados do
contexto da reprodução natural agrária e que, assim sendo, tinham
perdido os seus vínculos sociais. Com isso, a sua forma de existência
constituiu o protótipo da própria forma do sujeito que só na Modernidade
deveria tornar"-se o princípio geral da sociedade sob a forma da
abstracção da actividade com relação às necessidades. (\versal{KURZ}, 2002, s.p.)
\end{quote}

Esse processo significou uma enorme fissura no fundamento que cimenta o
universo simbólico medieval. A subjetividade dos indivíduos passou a
entrar em relação com uma monetarização aumentada da vida social, o que
significava uma mudança subjetiva paulatina para poder lidar
cotidianamente com o dinheiro e as exigências de produtividade, porque,
mesmo ainda no quadro rural, ``todas as relações hierárquicas de
obrigação, contribuições, tributos, emolumento, `dádivas' dos mais
diversos tipos, são brutalmente `monetarizados' e, em simultâneo,
inflacionados.'' (\versal{KURZ}, 2014, p. 107).

É assim que começa a ter fim a figura do cavaleiro e de sua moral
cavaleiresca. Os cálculos com os lucros da guerra não levavam em conta
qualquer ideal de honra cavaleiresca ou de brio. Ao homem de
\emph{qualidade}, o cavaleiro, se seguia o \emph{literal homem sem
qualidades}, o mercenário, que luta pelo dinheiro, não pela sua
comunidade.

\begin{quote}
Como ideal de vida bela, a concepção cavaleiresca tem aspectos
peculiares. É um ideal essencialmente estético, feito de fantasias
coloridas e sentimentos elevados, que também almeja ser um ideal ético.
[\ldots{}] Nessa função ética, o ideal cavaleiresco fica sempre a
dever, estorvado por sua origem pecaminosa. Pois o cerne do ideal é
sempre o orgulho elevado à condição de beleza. [\ldots{}] Do
orgulho, estilizado e exaltado, nasce a honra, que é o centro da vida
nobre. (\versal{HUIZINGA}, 2010, 99).
\end{quote}

Não significa que o aparecimento das armas de fogo decretou o fim
imediato das cavalarias. Como o que nos guia aqui são os tensionamentos
rumo à modernidade, podemos dizer que foi um processo, mas um processo
diferente em que se pode notar a sociedade, ou determinados Estados,
caminhando para a modernidade num ritmo maior. O sinal disso é não só a
superação paulatina da ética tradicional, ensejada pela mudança
subjetiva em marcha, mas também a mudança no sentido do dinheiro, que
vai ao encontro do seu conceito a partir da monetarização aumentada pela
sede fiscal dos Estados dependentes das armas de fogo -- o que não
significa que essa sede fiscal não houvesse antes, mas havia alguns
limites sociais ainda não rompidos. Já no novo contexto, rapidamente os
impostos passaram a fugir de quaisquer limites e a guiar"-se pela
concorrência entre Estados no terreno da máquina de guerra.

Aspecto fundamental, como explica Jappe (2006), é que tanto o dízimo da
igreja, quanto os impostos, que eram na sua esmagadora maioria pagos em
\emph{natura}, passaram a ser cobrados em dinheiro. O que obrigava os
camponeses a trocar sua produção por dinheiro para poder pagar o devido
e o além do devido. O limite a ser pago também se tornava independente
do concreto, do limite natural da produção. Isso fez com que a vida
social passasse a ser mais mediada pelo dinheiro do que antes e a
exigência do pagamento de impostos em dinheiro expunha os camponeses e
produtores em geral a uma coerção que não dependia da natureza de sua
colheita ou de sua atividade. No fim das contas, a natureza da colheita
e da atividade sofriam uma abstração uma vez que se devia chegar a um
determinado volume de dinheiro, independentemente da concretude da
colheita.

O aparecimento das armas, portanto, foi um acaso que acelerou a marcha
rumo à modernidade. Tanto subjetivamente quanto objetivamente, foi um
transtorno social. Mas a sociedade ainda tinha de tal modo suas raízes
na profundidade da tradição que foram necessários ainda vários séculos
de superações contínuas do universo simbólico passado para que esse
processo se mostrasse vencedor. E não se deve imaginar que as
subjetividades que viviam essa experiência imaginavam que estava selada
a modernidade. Nós hoje podemos olhar retrospectivamente e ver o quanto
o ``desenvolvimento econômico'' e intelectual dos séculos \versal{XI} ao \versal{XIV} não
necessariamente desembocaria -- senão somente de forma muito mais lenta
-- na vida social moderna sem essa mudança qualitativa de grande
envergadura que representaram as armas de fogo.

\section{O Burguês fidalgo, O avaro, Preciosas ridículas: tensionamentos}

A subjetividade saída o renascimento filosófico e econômico, anterior ao
Renascimento, vai ser um entrecruzamento mais acabado entre a tradição e
o impulso para a modernidade. O dito burguês ou mercador, após o
Renascimento, embora seu prestígio não seja o mais elevado, não será tão
desprezado quanto era séculos anteriores. Até porque, o Renascimento foi
um ponto de encontro fundamental entre sua forma de vida e a forma de
vida nobre à qual ele aspira.\footnote{Alguém poderá objetar que não se
  dá aqui atenção ao camponês. Obviamente, estamos dando atenção àquele
  espaço onde a modernidade começou a engatinhar. Não foi da
  subjetividade camponesa que surgiu a subjetividade moderna. Ao que
  parece, tanto para a crítica tradicional do capitalismo de matriz
  marxista quanto a crítica burguesa colocam o camponês como um
  conservador, atrasado, arraigado em sua vida. Não é preciso citar o
  exemplo da Rússia, que repetira rapidamente o que o restante da Europa
  levara anos para transformar.}

O desenvolvimento subjetivo que se dá é o de um amálgama em que o ideal
ainda é o nobre. Mas o que se erige aos poucos é uma subjetividade
sincrética, ou, em termos dialéticos, fruto de sínteses anteriores. O
sujeito burguês é um projeto -- embora sem autor --, não nasceu de uma
vez por todas. O que se chama burguês na Idade Média e até no
Renascimento não é o que modernamente se entende pelo nome. Basta pensar
em Pico della Mirandola, que escreveu o que se poderia chamar de
manifesto renascentista, humanista, moderno. Seu \emph{Discurso sobre a
dignidade do homem} (1486-2006) é uma ode à liberdade, ao mesmo tempo em
que defende uma filosofia moral baseada na religião. É clara na sua
\emph{Oratio de hominis dignitate} a tensão entre um impulso moderno e
um finca"-pé no mundo divino\footnote{É caso inclusive de Descartes que
  precisa fazer uma ginástica intelectual para justificar com base em
  deus que o corpo é só matéria e não forma um só com o espírito. E isso
  quase 2 séculos depois de Pico. O que representa um aspecto notável do
  quanto o mundo objetivo e subjetivo era perpassado pela tensão entre
  simbólico religioso e aspectos ditos modernos. A ponto de ainda
  Descartes -- considerado pai da ciência moderna -- não considerar seu
  proceder científico contrário à fé, conforme se pode sentir em seu
  \emph{Discurso do método} (1966).}. Seu esforço é sincretizar no homem
um ser que se constrói no mundo pelo seu fazer, sem determinante \emph{a
priori}, um ser condenado a fazer escolhas no mundo por si e um ser ao
mesmo tempo temente a Deus, que nutre uma fé inabalável no além. É um
homem que pode degenerar"-se às bestas ou regenerar"-se aos anjos
(\versal{MIRANDOLA}, 2006, p. 53). Essa tensão, esse amálgama, atravessa o
Renascimento, que não deve ser considerado o ponto alto do sujeito
burguês, mas como o ponto alto da tensão entre a subjetividade burguesa
em formação e a subjetividade de então, que acumulava várias camadas de
sociabilidade anteriores. Porque no Renascimento, o poder que o dinheiro
dava a seus detentores, chamados burgueses, ainda estava imerso numa
rede de relações tradicionais, porque o próprio dinheiro ainda estava no
seu processo histórico de devir"-dinheiro"-moderno. O decisivo é o quanto
a liberdade do Renascimento, sem que a subjetividade da época disso
tivesse consciência, já estava minada pelo desenvolvimento subterrâneo
da liberdade moderna ligada aos atos de mercado. Portanto, não se trata
também de pensar que o Renascimento foi um ponto alto da cultura humana
-- depois das trevas medievais -- que retomava o esplendor da
Antiguidade e que depois tudo se degenerou. Se o Renascimento pôde
representar ao mesmo tempo um altíssimo nível cultural ao lado de um
grande desenvolvimento do que há de mais vil e vazio, é porque o
fundamento simbólico moderno lutava silenciosamente contra o fundamento
simbólico passado. Assim, o florescimento renascentista não deveria ser
visto como burguês, mas como contendo tensionamentos burgueses, porque a
vida social transcorria num universo simbólico que não se encaixava na
vida social burguesa que a superaria.

Demonstra isso o fato de que mesmo as cidades em que florescia o
comércio na Idade Média, ou a arquitetura renascentista, ou mesmo aquela
até certo ponto de desenvolvimento da modernidade, tinham marcas muito
diferentes em relação ao que se começou a ver no século \versal{XX} -- o século
do sujeito burguês, o século da vitória (não total!) da forma mercantil
de organização da existência. Seria impensável ver cidades funcionais
como as de hoje, onde reina o concreto imediato, liso, mesmo que
curvado, como Brasília, mesmo onde reinava o comércio\footnote{Poderia
  citar o exemplo da belíssima cidade de Ypres, nos Flandres, entre
  Bélgica e França, berço da produção e comércio de tecido no século
  \versal{XIII}.}. O que é um indício de que a forma"-sujeito moderna ainda não
estava desenvolvida, embora presente e tensionada com as formas
existentes. Como nos diz Le Goff, mesmo o mecenato dos
burgueses"-mercadores não significava obras que representavam aspirações
dessa ``classe''. Não se pode falar de fato de uma cultura burguesa no
sentido profundo e lato do termo senão no século \versal{XX} -- quando a vida
social burguesa dá sinais de amadurecimento. O entrecruzamento era entre
a ideologia da Igreja, os ideais nobres e a aspiração burguesa dos
mercadores. E não é que os mercadores quisessem simplesmente imitar os
nobres ou eram hipócritas do ponto de vista da religião. Não há razão
para que não acreditemos que os mercadores também fossem homens pios.
Sua subjetividade desabrochava numa mistura que ia encontrando
justificativa social e subjetiva.

\begin{quote}
No início, quando a falta de educação artística obrigava os novos ricos
a adotar o gosto das classes dominantes tradicionais, mais tarde, quando
os mercadores, como vimos, tiveram um desejo cada vez maior de entrar na
nobreza, apagar as distâncias entre a antiga aristocracia e a nova que
eles queriam constituir, as tendências artísticas burguesas não se
distinguiram das da nobreza e da Igreja. Para se tornar nobre, o melhor
meio era, primeiramente, adotar o ``tipo de vida'' nobre. (\versal{LE} \versal{GOFF},
2011, p. 118).
\end{quote}

Esse entrecruzamento estará muito ressaltado nas peças de Molière. Em
especial, como se fossem personagens"-máscara, em \emph{O burguês
fidalgo}, \emph{Preciosas ridículas} e \emph{O Avaro}.

O contexto das peças de Molière é o da nobreza de corte. Algo que nasce
com o processo civilizador de que fala Elias, um processo que transforma
os antigos guerreiros tornados inúteis pelas armas de fogo em cortesãos
-- a chamada curialização (1993, p. 215). A construção da corte se
produziu, segundo Elias, paulatinamente do século \versal{XI} e \versal{XII} ao século
\versal{XVII} e \versal{XVIII} no ocidente. O que importa aqui é que na corte vão se
desenvolver comportamentos que tenderão a se constituir como o aspecto
distintivo dessa camada chamada nobre. No mesmo movimento, é o Estado --
que está delineando seus contornos modernos -- que passa a ter maior
presença na vida das pessoas. A corte, assim, é uma microssociedade,
onde uma convivência exige determinados comportamentos, ou melhor, uma
padronização de comportamentos, o autocontrole, um abrandamento das
pulsões, e o controle da violência, que cada vez mais era monopolizada
pelo Estado, cada vez mais regulador das relações sociais. É na corte
onde chegarão os \emph{Tratados de comportamento,} notadamente após o
Renascimento, que jogarão grande papel no desenvolvimento da civilização
europeia: ``[\ldots{}] na Itália, na França, na Espanha e na
Inglaterra, aparecem numerosos tratados fornecendo modelos, valores,
regras a seguir na vida em sociedade''. (\versal{RICCI}, 2009, p. 10).

Talvez valesse a pena atentar o quanto esses tratados refletem também as
mudanças objetivas da própria sociedade. As tensões em termos subjetivos
entre o ideal nobre e o ideal burguês tendem cada vez mais a produzir
novas sínteses de subjetividade, e também de organização social. Por
exemplo, como explica Teresa Ricci, o modelo do \emph{cortesão}
prescrito por Baldassar Castiglione no livro \emph{O Cortesão} dirige"-se
ao nobre por nascimento, ao dito \emph{homem de qualidades,} por isso
diverge do homem \emph{discreto} ou \emph{universal} prescrito por
Baltasar Gracián:

\begin{quote}
Se o cortesão, nobre por nascimento, representa ainda um modelo para a
nobreza, em Gracián, a perfeição não tem mais nada a ver com o sangue.
Gracián não defende a causa da nobreza, ele dá preceitos para o
indivíduo que queira ter êxito, fazendo abstração de toda distinção
social. Ao cortesão nobre, ele opõe o homem prudente e desabusado, o
homem que age e adquire reputação e honra pela arte, a inteligência e a
cultura. (\versal{RICCI}, 2009, p. 77).
\end{quote}

Embora sem querer Gracián tenha dado um impulso a um momento da síntese
da subjetividade moderna, importa ressaltar que as características que
ele põe em cena como distintivas não têm nada a ver com a forma"-sujeito
burguesa. O ideal de Gracián serve, no máximo, a uma subjetividade
sincrética que tensiona com os ideais nobres pelo fato de romper com a
ideia de que o nascimento determina a posição social. Isso não
significa, como explica Ricci, que Gracián tenha um sentimento
democrático, ou que, mantendo"-se afastado do povo, tenha um sentimento
elitista simplesmente. ``O que ele critica é principalmente a
ignorância, porque o `homem é grande o quanto sabe; e quando sabe, pode
tudo. O homem que não sabe nada, vive no mundo tenebroso'".
(\emph{Idem}, 2009, p.77). Não podemos senão concordar com Gracián,
embora saibamos, como discutiremos no segundo capítulo, o quanto essa
saída das trevas também introduziu novas formas de ignorância -- que
nada têm a ver com o \emph{indivíduo} que ele desejava, que é diferente
do \emph{Indivíduo} que tomou posse do mundo. O indivíduo de Gracían é
aquele que deve cultivar as qualidades do espírito, é aquele que só se
distingue de outro pelo espírito, não por qualquer coisa exterior, como
ele mesmo diz: ``Qualquer distância que pareçam criar entre dois homens
o nascimento, a posição, os bens de fortuna, pois bem, esses dois homens
só diferem pelo espírito: aí está o fundamento que os faz mais ou menos
homens [\ldots{}] (\versal{GRACIÁN}, 1994, p. 9). O indivíduo de Gracián não
é ainda o \emph{Indivíduo} moderno, embora vá precisar sê-lo, já que
``apresenta em certo sentido os traços do homem resultante da
desagregação das antigas formas de sociedade, as sociedades
pré-capitalistas, onde cada um tinha a priori seu lugar no mundo, pois
ele está sozinho no mundo hostil e luta contra todos para sua própria
afirmação''. (\versal{RICCI}, 2009, p. 25).

De qualquer modo, esses modelos, antes somente dirigidos à elite da
sociedade, vão aos poucos se democratizando, como atesta Norbert Elias:

\begin{quote}
Devido à forma peculiar de interdependência em que vivia, a aristocracia
de corte não podia impedir -- através de seus contatos com o estrato
burguês rico, do qual ela necessitava por uma razão ou outra -- a
difusão das maneiras, costumes, gostos e linguagem por outras classes.
Inicialmente no século \versal{XVII}, essas maneiras foram adotadas por pequenos
grupos dirigentes da burguesia [\ldots{}] e , em seguida, no século
\versal{XVIII} por estratos burgueses mais amplos. (1993, p. 215)
\end{quote}

Nesse sentido, temos motivos para concordar com Menegat (2011, p. 4)
para quem esse processo \emph{civilizador} guarda relação com o processo
de entrada na vida social do fetichismo da mercadoria, portanto com o
desenvolvimento da forma burguesa de vida social e subjetiva. Não
podemos deixar de pensar num certo lado obscuro dessa sociedade de
corte, desse processo civilizador, onde reinariam a honra, as
qualidades, a dignidade, a elevação. E esse lado obscuro está na luta
pouco civilizada, pouco leal, pouco digna que é feita na violenta
ocupação das colônias e no aprisionamento de escravos africanos,
ocupações que não prescindiram do arsenal de guerra que foi sendo
acumulado desde que a forma de fazer guerra tinha mudado com as armas de
fogo. Talvez se justificasse pelo fato de tanto os povos das colônias
quanto os negros escravizados serem por eles considerados bárbaros,
não"-humanos, ou baixos, incivilizados -- frágil justificação,
evidentemente. Mesmo assim, sabe"-se que nesse processo
\emph{civilizador} determinadas qualidades se mantiveram, não por obra
do sujeito burguês, mas porque não se pode simplesmente destruir formas
de subjetividade por decreto, embora se possa por um processo paulatino
e contínuo. Autores como Pico della Mirandola e Gracián estão a léguas
de distância da forma"-sujeito mercantil. O que não significa aqui
elogiar a nobreza como portadora das virtudes ou incensar o conceito
abstrato de \emph{Indivíduo.}

\section{O burguês fidalgo: \emph{ridículo}}

Todo o desenvolvimento reflexivo que temos empreendido até aqui é uma
espécie de moldura do processo de amadurecimento da marcha moderna do
qual o teatro de Molière é também um quadro. Estamos no século \versal{XVII},
portanto, já há mais de seis séculos depois que um certo
\emph{desabrochar produtivo} -- seria errôneo usar a categoria moderna
de desenvolvimento econômico -- fez"-se sentir na história. Um
desabrochar que, mesmo impulsionado pelo \emph{boom} das armas de fogo,
faz com que se chegue a esse século sem que se tenha ainda uma
subjetividade delineada pelo universo simbólico fundado no movimento
mercantil. Por isso ainda se pode falar de um entecruzamento objetivo e
subjetivo acentuado.

Portanto, as peças que analisamos aqui estão dentro do quadro do
amadurecimento que temos visto em termos sempre de tensionamentos que se
desdobram, quase sempre no rumo moderno. Vislumbramos nas peças de
Molière em análise a tensão entre uma subjetividade voltada para o
passado -- notadamente nobre -- com seu emaranhado de relações
simbólicas, e uma subjetividade impulsionada para o futuro, sempre
entrecruzada de aspectos do passado. Do mesmo modo, na nossa visão, do
ponto de vista da objetividade social o contexto também é de
tensionamento e entrecruzamento. Ou seja, assim como ainda não temos uma
forma"-sujeito burguesa bem delineada como forma de subjetividade
dominante, também a forma"-social moderna, capitalista, não está bem
delineada, não é a forma dominante. Dito de outro modo, não se trata
simplesmente de formas dadas, já com seus aspectos determinantes, que
apenas estão num determinado grau de desenvolvimento que o desenrolar
histórico permite. As formas social e subjetiva burguesas simplesmente
não chegaram ao seu amadurecimento, um amadurecimento que significa
vencer o entrecruzamento com formas anteriores de vida social e
subjetiva.

Nesse sentido, o quadro do século \versal{XVII} de Molière ainda é o quadro das
manufaturas, distante ainda mais de um século de amadurecimento da
revolução industrial. É o quadro da Paris com ruas enlameadas -- nada
comparável às ruas das passagens de Benjamin no século \versal{XIX}. Portanto, o
burguês de suas peças está mais próximo do mercador e do usurário
medieval do que do burguês do século \versal{XIX}. Vejamos como descrevem o
começo do século \versal{XVII} Geoges Duby e Robert Mandrou:

\begin{quote}
A primeira parte do século \versal{XVII} anuncia o século \versal{XVIII} pela fecunda de
transbordante atividade da burguesia que constrói, tanto quanto ou mais
do que a nobreza em Paris, no entorno da praça Real e no bairro do
Marais que não é mais aristocrático do que burguês: o estilo dessa
criação arquitetural é o hotel particular, [\ldots{}] mercadores e
advogados julgam"-se merecedores de vidas mais afortunadas e de carreiras
exitosas, multiplicam os móveis, mesas, poltronas, e seu orgulho se
espalha por esses retratos pintados de corpo inteiro [\ldots{}]:
belas imagens cheias de dignidade, a peruca bem cuidada [\ldots{}]
Tudo respira riqueza e satisfação pessoal. Mercadores, financistas e
advogados, a quem a profissão proporciona lazeres e o gosto de ler, de
aprender, de receber amigos que também gostam de leitura e discussão
[\ldots{}] (1958, p. 15)
\end{quote}

Jules Michelet na sua \emph{História da França} descreve a França no
século de Molière como uma época de mutações sociais marcadas pelo
aumento dos gastos, pela quintuplicação em um século do preço dos
objetos fabricados, pela diminuição do preço do trigo e pela fome que
ronda o produtor de três em três anos -- embora o luxo ganhe destaque.
Além disso, do ponto de vista que podemos chamar de subjetivo, a época
de Molière é descrita pelo historiador como ``um século intermediário,
que vagueia entre duas almas, a alma antiga e a nova [\ldots{}]'' (1877,
p. 127). Não seria exagero entender a \emph{alma} de que fala Michelet
como a subjetividade entrecruzada da época, uma subjetividade imersa em
dois universos em luta tensa. Um luta tensa que faz o historiador
destacar o caráter corroído do universo simbólico"-religioso num mundo em
que a ideia religiosa vai se esmaecendo e só se mantém ao se adaptar, ao
abdicar, segundo ele, de sua influência moral: ``Ela só reina obedecendo
aos vícios públicos.'' (\emph{Idem}, p. 127). A esse contexto de
entrecruzamento subjetivo ele chama de \emph{flutuação} \emph{moral.}

Esse entrecruzamento objetivo e subjetivo da época se mostra por exemplo
na própria distinção que existe entre Luís \versal{XIV} -- nobre de nascimento --
e seu famoso secretário de finanças Colbert -- um plebeu da finança.
Como diz Michelet, por trás de um ministro que ``organiza laboriosamente
seu grande sistema comercial e industrial''-- com uma subjetividade
marcadamente voltada para o futuro na época -- , há um rei que, embora
não se oponha à ofensiva mercantil por ele empreendida -- já que seus
gastos precisam ser financiados --, tem seus caprichos e ideias
grandiosas que estão ``bem acima dessas baixas ideias mercantis''.
Conforme Michelet, Luís \versal{XIV} escrevera que ``se os ingleses se contentam
em ser os mercadores da terra e deixam"-no ser um conquistador, há meio
de se negociar. A França não teria problema em deixar ¾ do comércio
mundial com os ingleses e ficar só com ¼''. (\emph{Idem,} p. \versal{IX})

As próprias leis modernas de Colbert nem sempre vão ser implantadas sem
qualquer resistência muda da vida concreta para a qual a vida mercantil
ainda é estranha. Michelet sublinha que Colbert instituiu leis
protecionistas que incentivaram a indústria francesa -- deve"-se entender
manufatura de tecido, de seda, tapeçaria fina, não no sentido geral de
indústrias (p. 121) -- mas duvida do êxito dos planos de Colbert pelo
quadro subjetivo da época. Diz Michelet (p. 126): ``Ele cria o trabalho
aqui e ali pelo incentivo que a exclusão dos produtos estrangeiros
significa para tal indústria. Mas o gosto geral das gentes é pelo ócio e
pela vida improdutiva. [\ldots{}]. Colbert obtém, exige dos clérigos a
supressão de algumas festas da Igreja, mas as pessoas não deixam de se
fazerem presentes.''

Nesse sentido, o que mais nos parece digno de ressalte não é o fato de
Molière trazer à cena os defeitos da sociedade por seus personagens, a
hipocrisia, a vaidade, a propensão dos indivíduos a buscar a todo custo
seus fins até por meio da manipulação. Não é a denúncia moral pela
sátira que ele faz dos defeitos humanos de sua época através das
personagens. Embora o autor tenha a arte de pôr em cena esses elementos,
essas características são até mesmo atemporais.

O que nos chama mais a atenção aqui nessa análise é a tensão entre as
formas de subjetividade em cena. A crítica empreendida pelo autor ao
desejo do Senhor Jourdain de ser quem ele não é, o que o torna ridículo,
não pode nos obnubilar para o fato de que sua busca, embora farsesca,
portanto com trejeitos exagerados, é por ser um \emph{homem de
qualidades}, é buscar uma vaidade que rivaliza inclusive com a riqueza
material que ele tanto ama. Terá de nos desculpar o esteta por não
adentrarmos por uma análise literária imanente; por não traçarmos, por
exemplo, um paralelo entre o próprio Molière ``filho de um `Burguês
fidalgo' que tem mais de um traço em comum com o Senhor Jourdain'' (\versal{RAT},
1962, p. \versal{IX}) e seus personagens; por não adentrarmos por uma análise dos
aspectos da comédia, do riso em Molière; por não debulharmos a estrutura
do teatro de Molière, seu lado mágico (\versal{GUICHARNAUD}, 1963) a importância
dos ornamentos de música nessa peça em particular que culmina com o
\emph{balé das nações}.

Ora, o que nos chama a atenção nas peças aqui em discussão, e na peça
\emph{Bourgeois} \emph{Gentilhomme} (Burguês Fidalgo) em especial,
encenada em 1670, é a representação da tentativa de ``apoderamento'' das
maneiras nobres pela burguesia. É a idealização social da figura do
nobre e não da figura do homem endinheirado. De antemão, é digno de
destaque o contrassenso risível do título da peça: \emph{burguês}
\emph{fidalgo.} Mas à época, esse contrassenso poderia se realizar de
modo enviesado por meio do dinheiro. O Senhor Jourdain, o burguês
ridicularizado, quer fazer o que estiver ao alcance de sua riqueza para
tornar"-se um \emph{homem de qualidades}. Como explica Marie"-Claude
Canova (1999, p.132) em sua análise da peça, não surpreende ``Que o
estado nobiliário se tenha imposto como modelo ideal à sua imaginação
(do burguês) em busca de norte'', pois, ``como atestam os
contemporâneos, ele concentra tudo o que há de grande, de potente, de
respeitável no mundo.'' Por esses comentários, sem ler a peça, alguém
poderia até ver em Senhor Jourdain o indivíduo de Gracián que se faz por
si mesmo independente do sangue e do nascimento. Mas diferentemente do
indivíduo de Gracián, ele não tem espírito que o eleve. Antes o
contrário, até mesmo o espírito ele precisa arranjar com seus mestres,
sua elevação não passando de mera aparência de se travestir de uma
classe de prestígio.

Como relatam Duby e Mandrou (1958, p. 44), as crônicas da época abundam
``em testemunhos sobre as rivalidades que opõem a nobreza de sangue e as
classes emergentes, de toga e burguesia mercantil'' que protagonizam
brigas ``que as encenações de Molière sobre o mesmo ficam parecendo
coisa miúda.''

A burguesia inveja \emph{as qualidades nobres} e a nobreza inveja a
riqueza vil da burguesia, algo que na peça se expressa bem no personagem
Dorante, que se aproveita de sua posição de pretensamente nobre para --
``feito um Don Juan pilantra'' (\versal{MICHELET}, 1877, p. 79) -- arrancar
empréstimos de Jourdain, para quem é um prazer emprestar dinheiro a um
\emph{homem de qualidades}. Roland Bruyelle destaca assim o contexto:

\begin{quote}
Sob a regência de Ana da Áustria, e sob o reino de Luís \versal{XIV}\footnote{Que
  atravessou a segunda metade do século \versal{XVII} até o primeiro quarto do
  \versal{XVIII}.}, as pessoas de finanças tinham criado fortunas consideráveis.
Inspirados pelo exemplo e encantados pelos esplendores da Corte, eles
também desejam conhecer os luxos e as honras. A frequentação de
\emph{gentilhommes} fê-los perder a cabeça: pegaram de modo desajeitado
como modelo seus novos ``amigos'', de modo que nas ruas podiam"-se
encontrar muitos ``senhor Jourdain'' [\ldots{}] É, portanto, um
ridículo bem contemporâneo, e o burguês de Molière é uma caricatura
dessas pessoas. (1946, p.108)
\end{quote}

No início da peça, aparecem os preceptores de senhor Jourdain numa
discussão que não é anódina. Eles discutem uma questão que já diz
respeito à própria qualidade, uma questão de dignidade da atividade, e
também uma questão ética. O mestre de música se mostra por demais
satisfeito pelo dinheiro que ganha: ``É verdade: encontramos aqui um
homem como nos convém aos dois. Eis uma doce renda, esse senhor Jourdain
com suas visões de nobreza e galanteio que pôs na cabeça; e vossa dança
e minha música haveriam de desejar que todo mundo se lhe assemelhasse.''
(\versal{MOLIÈRE}, 1893, p. 5). Já o mestre de dança se sente enfastiado em ter
que produzir para um tolo. Não é mero desdém pela pessoa que ele dá
provas -- embora aspire à glória --, mas pela situação de ter que fingir
a possibilidade de o burguês conseguir alcançar as qualidades que
deseja. ``[\ldots{}] Sim, a recompensa mais agradável que se pode receber
das coisas que se faz é vê-las conhecidas, vê-las acariciadas pelos
aplausos que tanto vos honram. Não há nada, a meus olhos, que nos pague
melhor que isso por todas nossas fadigas; e são doçuras requintadas,
louvores esclarecidos'' (\emph{Idem}, p. 7). Pode parecer mera vaidade,
transparece o desprezo pelo burguês, mas parece também esse sentimento
uma mediação importante, comparado ao simples olhar instrumental sobre a
atividade que se executa, que só conta pela quantia recebida em troca,
que parece ser a opinião do mestre de música.

Prossegue o mestre de música:

\begin{quote}
Estou de acordo, sinto"-o como vós. [\ldots{}] É um homem, em
verdade, de pouco brilho, que fala pelos cotovelos sobre tudo, que
atravessa até quando aplaude; mas seu dinheiro eleva os juízos de seu
espírito; ele tem discernimento em sua bolsa, seus elogios são de ouro:
e esse burguês ignorante nos vale mais, como vê, do que o grande senhor
esclarecido que nos introduziu em seu meio. (\emph{Ibidem}, p. 7
[Grifos Nossos]).
\end{quote}

Eis o tom geral da peça. Mesmo estúpido, o Senhor Jourdain \emph{vale}
mais do que um senhor esclarecido. Segundo Defaux (1999), ``A tensão
essencial da peça não é entre graça e inépcia, mas entre real e
imaginário.'' (p. 105). Mas essas seriam tensões evidentes que não
empurram essa comédia para além de si mesma, para além de um retrato de
época, para além de uma mera comédia de costumes. A tensão fundamental,
a nosso ver, é entre mundo burguês tal qual se desenhava na época,
estereotipicamente colocado como ignorante e o mundo nobre onde
pretensamente reinaria o saber, a qualidade e a distinção. O que sabemos
não ser bem o caso. Tanto o burguês da época não era somente um amante
do dinheiro\footnote{A reflexão de Hannah sobre a cultura ajuda a
  fundamentar nossa ideia de entrecruzamento subjetivo tenso entre mundo
  tradicional e mundo burguês, ou seja, a ideia de que a subjetividade
  burguesa apenas vem realizando em processo amalgamado com outras
  formas mais cheias de mediação social. Mediação esta que a própria
  vida burguesa tende a achatar rumo à imediatidade. Para ela, as
  classes médias europeias socialmente inferiorizadas, assim que
  ``adquiriram riqueza e lazer suficientes'', viram"-se numa ``luta
  acirrada contra aristocracia e o desprezo desta pela vulgaridade do
  mero afã de ganhar dinheiro. Nessa luta por posição social, a cultura
  começou a desempenhar enorme papel como uma das armas, se não a mais
  apropriada, para progredir socialmente e `educar"-se',
  [\ldots{}].'' (2009, p. 254). Embora aqui já haja uma perda de
  \emph{qualidade} da cultura, como ela diz: os objetos culturais
  serviram de passe de entrada numa posição mais elevada na sociedade e
  ``Nesse processo os valores culturais eram tratados como outros
  valores quaisquer, eram aquilo que os valores sempre foram, valores de
  troca, e, ao passar de mão em mão, se desgastaram como moedas
  velhas.'' (2009, p. 256).}, quanto o nobre não era necessariamente a
fonte da virtude. Gérard Defaux parece ver Jourdain apenas como um
personagem sonhador, não vendo nele a tensão entre dois mundos reais,
duas formas de vida social em tensão, duas formas de subjetivação
tensionadas, quase que um personagem vivendo entre \emph{duas almas,}
para usar a expressão de Michelet.

Mas na peça, o ridículo da situação não deixa de transmitir certa
indulgência do autor em relação ao burguês, senhor Jourdain, que quer a
todo \emph{custo} tornar"-se um homem de qualidades, contratando, além
dos mestres de dança e música, um de esgrima e filosofia, além de um
alfaiate para vestir"-lhe como homem de qualidades: ``Meu alfaiate me
disse que as pessoas de qualidade vestiam"-se assim de manhã.''
(\emph{Ibidem}, p. 9). Portanto, apesar do ridículo do personagem --
ainda mais realçado pelo modo caricatural como seu quadro é pintado como
alguém sem talento para quaisquer artes --, a impressão que se tem é que
a simpatia expressa na peça é maior pelo burguês idiota, porém honesto,
do que por homens como Dorante que, embora da corte, é desonesto e o que
se poderia chamar de aproveitador barato.

A crítica que transparece implicitamente à postura de Dorante está
relacionada com uma crítica moral que tanto a corte apreciava, por
apresentar de certo modo o que não se deve esperar de um cortesão, que
deve manter sua distinção. Mas não como distinção externa ao ser
simplesmente, o tipo de riqueza prezado pela modernidade, mas como algo
que o ser já traz ``consigo'', ``de berço'' e que era preciso cultivar
-- embora as aparências fossem algo também fundamental, mas não bastava.
Obviamente, essas distinções congeladas não são algo necessariamente
louvável, pois também justificava uma perpetuação do poder fetichista de
considerar como vis aqueles que não vinham de tal berço. Mas o que
queremos destacar é mais a ideia de uma distinção que não se dê por
aspectos meramente exteriores, o que acaba por construir certas
qualidades nas relações sociais.

Esse peso moralizante que transparece na peça, ao colocar o personagem
que tenta enganar o burguês numa posição de reprovação pública, é índice
do quanto a época abundava em nobres que viviam de expedientes. Seguindo
a explicação de Bruyelle:

\begin{quote}
Mas ao mesmo tempo que a burguesia se tornava opulenta, a nobreza
empobrecia. [\ldots{}] Então, espetáculo estranho, viam"-se esses
aristocratas cheios de orgulho adulando \emph{pavenus}\footnote{O
  \emph{parvenu} seria grosseiramente o que se chama hoje novo rico.
  Aquele que eleva rapidamente sua condição, sem ter as maneiras
  condizentes com o ambiente a que chegou.}, desposando filhas de gente
do campo e indo buscar nas lojas o dote para retocar o dourado de seu
brasão. Se os Jourdains não são raros, os Dorantes pululam e pilham os
Jourdains. (1946, p.108).
\end{quote}

Molière não parece demonstrar qualquer simpatia pela atitude vil do
nobre, embora ele jogue papel importante na trama para ridicularizar o
burguês. O que parece demonstrar, ao mesmo tempo, que o burguês não
podia ser nobre de fato, mas também que há nobres cujo procedimento é
dos mais vis. E essa ridicularização do burguês acaba sendo ao mesmo
tempo uma ridicularização do nobre que não é nobre. Mas, em essência, o
nobre deve sempre ter gestos nobres. Se a realidade não condiz com essa
máxima, é porque há um problema na nobreza. E não parece forçado ver nas
entrelinhas dos diálogos que, ao burguês não estúpido, seria possível
adquirir algumas \emph{qualidades} não sendo vil como alguns nobres. Não
precisamos lançar mão da informação de que o próprio Molière era de
família burguesa, embora bem instalada na corte, para fundamentar a
intuição de que sua crítica moral tinha também a intenção de chamar a
atenção para os burgueses de \emph{qualidade} como pessoas mais elevadas
que nobres \emph{sem qualidades.}

Mas Jourdain é o protótipo caricato do tolo, e está enfeitiçado pela
palavra \emph{qualidade}. Quando o mestre de música oferece"-lhe aulas,
ele pergunta ``Os homens de qualidade também aprendem música?''
(\emph{Ibidem}, p. 14). E os mestres de música e dança o conduzem a uma
argumentação que somente os palermas conseguem acompanhar. Argumentam,
por exemplo, que não há Estado sem música, nada é mais necessário aos
homens que a dança, todas as desordens e guerras que se veem no mundo se
devem ao não aprendizado de música, todos os revezes políticos, todos os
infortúnios se devem ao fato de não se saber dançar. Se todos os homens
soubessem música, poderiam viver afinados, se os homens soubessem
dançar, não dariam passos errados em suas condutas. Ou seriam essas
argumentações verossímeis? Não cabe aqui julgar, mas até o mestre de
dança, que tinha demonstrado preocupações de outro nível agora se
entrega ao jogo de interesse com o burguês.

A busca de qualidades é incessante e obsessiva. Canova usa o conceito de
\emph{simulacro do eu} para explicar Jourdain: ``Senhor Jourdain renega
sua ascendência e seu passado burguês, reescrevendo a história e
reconstruindo a realidade presente de modo a confirmar sua ideia
quimérica de eu [\ldots{}]. (1999, p.132). Envolvido nesse seu
\emph{eu quimérico,} Jourdain não pode ouvir a palavra \emph{qualidade}
que logo se vê introduzido nesse mundo quimérico. E os mestres de música
afirmam que um homem magnífico como senhor Jourdain, ``que tem
inclinação para as belas coisas, precisa ter um concerto de música em
casa todas as quartas e quintas''. (\emph{Idem}, p. 23). Jourdain aceita
toda proposta, desde que seja algo também feito pelas \emph{pessoas de
qualidade}. Quando o alfaiate lhe propõe uma roupa, resiste, somente
cedendo quando sabe que é a mesma das \emph{pessoas de qualidade}. Mas
ele é ignorante, não sabe o que é latim, não sabe o que é prosa ou
verso. Vê-se cercado por mestres, preceptores que querem o seu dinheiro.
Não sabe o que é lógica, não sabe o que é a moral, tampouco a física.
Mas quer ``ter espírito, saber raciocinar coisas entre as pessoas de
bem'' (\emph{Ibidem}, p. 71), diz ele. Acaba escolhendo aprender a
ortografia, com o mestre de filosofia. A cena em que o mestre de
filosofia ensina"-lhe os sons, a fonética, mais parece alguém ensinando a
uma criança adulta a produzir os sons da língua, ou, para ser mais
imaginativo, um homem saindo da condição animal e aprendendo a produzir
sons humanos. O que torna o burguês ainda mais ridículo, como se nem
mesmo emitir os sons da fala soubesse, ou somente grunhidos. Essa forma
grotesca de apresentação do burguês ignorante não deixa de nos lembrar a
tendência dos sujeitos contemporâneos a compilarem um saber exterior,
que se acumula sem que tensione com sua própria existência -- ou que
apenas impulsiona sua própria existência a seguir seu curso. É o que
acontece com o burguês Jourdain, em vez de importar"-se com as
disciplinas que poderiam elevar"-lhe o espírito, as quais desconhece,
escolhe a ortografia, mais especificamente a parte da produção dos sons.

Sua obsessão em ser um homem de qualidade leva"-o ao ridículo de
sentir"-se como tal apenas por estar vestido pretensamente como tal. Não
estaria aqui, mesmo um tanto estereotipado, o modelo em germe do sujeito
burguês na contemporaneidade, que é capaz de depositar seu Eu em pura
aparência? Jourdain crê nos elogios que recebe do aprendiz de alfaiate e
paga por eles:

\begin{quote}
Aprendiz alfaiate: Meu \emph{gentilhomme}, por favor, dê uma gorjeta aos
rapazes.

[\ldots{}]

Sr. Jourdain: Meu \emph{gentilhomme}! Eis o que é postar"-se como pessoa
de qualidade. Vá ficar sempre vestido como burguês, nunca que lhe dirão:
meu \emph{gentilhomme}. Pegue, aqui está pelo ``Meu
\emph{gentilhomme''.} (\emph{Ibidem}, p. 59).
\end{quote}

Gradativamente, aumentam os títulos: monsenhor, Sua grandeza. Até que o
próprio Jourdain se dá conta de que até chegar a alteza ele terá ficado
sem sua bolsa. E seu passeio pela cidade, com seus lacaios, causa risos
até em sua empregada. Enquanto isso, sua mulher, que encarna a típica
burguesa da época, quer casar sua filha, preocupa"-se com o piso da casa
que pode ser quebrado pelo mestre de armas.

\begin{quote}
Senhora Jourdain: Estais louco, meu marido, com todas essas fantasias; e
isso sucedeu depois que passastes a rondar a nobreza.

Senhor Jourdain: Quando rondo a nobreza, faço meu juízo parecer tal; e
isso é mais belo do que rondar vossa burguesia. (\emph{Idem}, p. 76).
\end{quote}

Senhor Jourdain acha que de fato está conseguindo ser alguém de
qualidades, tanto que considera sua empregada e sua mulher duas
ignorantes e passa a lhes explicar a diferença entre prosa e verso, a
forma como se pronunciam as letras. Além de explicar como se faz
esgrima. Nesse caso, seus conhecimentos mostram"-se tão exteriores que
acha que somente com as regras conseguirá não ser tocado num combate. O
que se mostra falso já na sua tentativa com a empregada, Nicole.

A busca pelas qualidades só pode ser titanesca, visto que o nobre
canalha, personificado em Dorante, aproveita"-se da situação para enganar
Jourdain numa pequena trama amorosa que lhe possibilita extorquir o
burguês. Este, por sua vez, crê ser um gesto de altíssima importância
emprestar dinheiro para um fidalgo que, por sua fé de fidalgo, não
deixará de honrar seus compromissos. Dorante, o ``don Juan picareta'',
fingindo ajudar o burguês fidalgo a conquistar o amor de uma dama
\emph{de} \emph{qualidades}, dá todos os presentes que o Senhor Jourdain
envia à marquesa como se fossem seus. A obsessão em querer
\emph{possuir} inclusive uma mulher de qualidades não permite a senhor
Jourdain, sendo por essência sem qualidades, distinguir o quão sem
qualidades também é o nobre Dorante. ``Uma mulher de qualidades tem para
mim charmes encantadores; e é uma honra que eu compraria ao preço de
todas as coisas''. (\emph{Ibidem}, p. 93).

Do mesmo modo, senhor Jourdain não quer casar sua filha com um homem sem
qualidades. Cléonte, pretendente de Lucila, não é fidalgo,
\emph{gentilhomme}, por isso não é aceito por Jourdain. Mas Cléonte
explica que tem outras características que se aproximariam do ideal do
\emph{honnête} \emph{homme,} o homem de bem francês do século \versal{XVII}:

\begin{quote}
Senhor, a maioria das pessoas, sobre essa questão, não hesitam muito;
pronuncia"-se essa palavra claramente [\emph{gentilhomme}]. Esse nome
não causa escrúpulos em quem o usa, e o costume hoje parece autorizar
seu roubo. Para mim, confesso"-vos, tenho sentimentos, sobre esses modos,
um pouco mais delicados. Acho que toda impostura é indigna de um homem
de bem, e que há covardia em disfarçar o que o céu nos deu como
nascimento, em querer ornar"-se no mundo com um título despojado de
outrem, em querer parecer o que não se é. Nasci de pais que sem dúvida
tiveram cargos honrosos; [\ldots{}] mas com tudo isso, não quero
ornar"-me de um nome onde outros, em meu lugar, acreditariam poder
pretender; e vos direi francamente que não sou fidalgo. (\emph{Idem}, p.
117).
\end{quote}

Parece que Cléonte dirige essas palavras ao próprio Jourdain, pois a
reprovação se encaixaria perfeitamente à sua conduta. Cléonte é fruto já
de uma nova síntese de ideal subjetivo após a do cortesão. Para Teresa
Ricci, o \emph{honnête} \emph{homme}, o homem de bem:

\begin{quote}
[\ldots{}] continua o ``processo de civilização'' bem representado pelo
cortesão, que tinha por sua vez substituído o cavaleiro, ligado a outros
valores que não eram mais válidos na sociedade de corte. O ideal de
honestidade não diz respeito necessariamente a uma classe social como a
nobreza, mas a uma elite que se distingue por suas qualidades
intelectuais e morais. O homem honesto é o homem de honra, o homem que
se comporta de maneira honesta. A honestidade aqui é ligada a uma noção
fortemente ligada à moral. (\versal{RICCI}, 2009, p. 365)
\end{quote}

Mas senhor Jourdain, nem mesmo um \emph{honnête} \emph{homme} quer ser.
O que aumenta a impossibilidade. Ele quer ser um perfeito cortesão, que
para ele é o homem de qualidades.

Apesar da estupidez de Jourdain, é digno de nota o quanto o dinheiro
podia ser de certo modo relativizado em relação às ditas qualidades --
estamos de fato no entrecruzamento de subjetividades, de formas de viver
no mundo que convivem em tensão. Jourdain está disposto a despender seu
dinheiro pelas qualidades, já Harpagão, de \emph{O} \emph{avaro}, não
seria capaz de tal gesto. Ele é capaz de fazer \emph{cantigas de amor} a
seu dinheiro, e não à amada. Ele prefere seu dinheiro à mulher que ele
diz amar e sintetiza o ideal do mercador e da forma"-sujeito burguesa que
anda no mundo apenas a fazer cálculos. Harpagão é aquele que não aceita
outra divisa na vida senão a do mercador florentino do século \versal{XIV}: ``Tua
ajuda, tua defesa, tua honra, teu lucro é o dinheiro''. Harpagão é o
protótipo do \emph{homem sem qualidades} no sentido literal, sem tensão
com o mundo. É o protótipo do indivíduo que já deixou que a máscara de
caráter da vida mercantil, portanto, que a subjetividade mercantil, se
lhe aferrasse ao rosto sem resistência.

Além da diferença entre Jourdain e Harpagão, \emph{O} \emph{avaro}, o
entrecruzamento subjetivo é visível também nas diferenças entre Jourdain
e sua esposa. Jourdain pode ficar esfuziante ao crer na mentira de
Covielle de que seu pai era fidalgo em vez de mercador: ele apenas, por
sua estirpe, conhecia bem tecidos, ia escolhê-los em toda parte ``e os
dava a seus amigos por dinheiro''. (\versal{MOLIÈRE}, 1893a, p. 141). Já sua
esposa não tem qualquer problema em aceitar sua linhagem de burguesa sem
qualidades, alguém que sabe lidar com seu dinheiro, alguém de uma
subjetividade já mais voltada para o futuro, uma subjetividade
objetivada em valores novos, que cria uma resistência dura contra
qualquer desejo de imitação de sentimentos ditos nobres: ``Estou me
lixando para suas qualidades'' (\emph{Idem}, p. 139).

O ápice da comédia, em \emph{O Burguês Fidalgo}, só pode ser o momento
da mascarada com a qual querem turcamente transformar Jourdain em
\emph{mamamouche}\footnote{Segundo o dicionário francês de Émile Littré
  (1957, p. 1927), trata"-se de um termo inventado por Molière para dar
  um título honorífico turco e burlesco a esse personagem. Equivaleria a
  paladino, embora o termo não queira dizer nada em árabe.} e Cléonte no
filho do Grande turco que quer ser seu genro. Temos aqui outro oxímoro.
As características do paladino, que seria o \emph{mamamouchi}, estão
ligadas a um homem de qualidades, alguém honrado, cavalheiresco e
intrépido, com caráter irrepreensível, sempre disposto a proteger os
fracos e lutar por causas justas, tudo o que ele não pode fazer, visto
que ele só luta por seu bolso e pelas qualidades exteriores acidentais.

Como já vimos, esse desejo ardente de Jourdain, embora Molière o pinte
de forma grotesca, não se pode crer algo isolado para a época. Antes,
era esse o desejo acalentado pela burguesia em geral que comprava seu
nobilitamento, ou casava"-se com nobres em falência (\versal{GOUREVICHT}, 1989).
Mas isso não podia deixar de causar remorso numa nobreza que via seu
mundo de hierarquias fixas ruir. Canova resume bem o contexto:
``Jourdain é apenas o que o monarca [Luís \versal{XIV}] quer que ele seja,
isto é, mais um `desses inúmeros usurpadores sem nenhum título ou com
título adquirido à custa de dinheiro sem qualquer apanágio, dos quais
convém purgar a nobreza.'' (1999, p. 139).

\section{Preciosas ridículas: \emph{ridículas}}

Tanto em \emph{O Burguês Fidalgo} como em \emph{Preciosas Ridículas}
vemos, em cena, a busca do ideal nobre por parte de não nobres. Também
aqui, o autor expõe ao ridículo os anseios da burguesia de entrar no
mundo nobre. O diferencial nessa peça é que as duas \emph{preciosas}
\emph{ridículas} são provincianas, sem que deixem de ter em mente os
ideais nobres desenvolvidos e prezados nas cortes. O próprio adjetivo
\emph{preciosa} tem um significado próprio nesse século \versal{XVII}. A
\emph{preciosité} é praticamente um movimento, em consonância com a
corte, de busca de distinção nas maneiras e na linguagem. Como define o
\emph{Dicionário alfabético e analógico da língua francesa}, é um termo
usado no século \versal{XVII} para ``mulheres distintas, refinadas, que adotaram
uma atitude nova diante do amor (recusa do amor vulgar, concepção do
perfeito amor e da amizade amorosa, direitos da mulher) uma maneira de
falar original (recusa de uma linguagem comum, emprego de metáforas) e
afirmam a superioridade do gosto sobre as regras e o saber dos
pedantes.'' (\versal{ROBERT}, 1962, p. 541). Essa \emph{vida} \emph{preciosa} ``é
um aspecto da vida urbana, quase uma moda'', e sobretudo ``um fenômeno
parisiense'' (\versal{DUBY} \& \versal{LANDROU}, 1958, p. 17).

Assim como Molière talvez não quisesse criticar os burgueses com
pretensões nobres -- que era até o caso de sua família -- também não
necessariamente queria atacar as mulheres que assumiam a postura
\emph{preciosa}, mas a pura imitação grotesca desses modos
\emph{preciosos.} Como ele diz:

\begin{quote}
Eu queria mostrar que [\ldots{}] as mais excelentes coisas estão
sujeitas a serem copiadas por desagradáveis macaqueadores, que merecem
ser caçoados; que essas maliciosas imitações do que há de mais perfeito
foram em qualquer tempo matéria para a comédia; e que, por essa razão
[\ldots{}] também as verdadeiras preciosas cairiam em equívoco em se
chatear quando se encenam a ridículas que as imitam mal.'' (\versal{MOLIÈRE},
1962, p. 220).
\end{quote}

Nesse sentido, também nessa peça há um oxímoro cômico no título, já que
a preciosidade não casaria com o ridículo, pelo menos na época. Mas o
desejo da \emph{preciosidade} tem na peça a força do desejo dos ideais
nobres idealizados. Portanto, as \emph{Preciosas} estão na mesma fileira
de Jourdain: nos dois casos, está em marcha um verdadeiro processo
impossível de negação de \emph{si} em prol da tentativa de assumir um
\emph{outro} -- que se transforma em máscara mal colocada, já que remete
para uma forma de subjetividade voltada para um passado em processo de
superação. Para Bruyelle, ``[\ldots{}] as \emph{Preciosas} são a primeira
expressão forte da paixão de Molière pela verdade, pelo natural e o
bom"-senso: a peça abre a luta que ele empreendeu e perseguiu toda a vida
contra a mentira e a afetação nas maneiras, na linguagem, no pensamento,
no estilo [\ldots{}].'' (1946, p. 20). Por esse trecho, podemos ver
que não estava em absoluto em questão no teatro de Molière essas tensões
que agora se pode observar. O autor não necessariamente se dá conta de
que o entrecruzamento subjetivo que ele põe em cena são um momento no
caminho moderno de superação da forma subjetiva calcada no fetichismo
anterior ao capitalismo. Até mesmo porque é o passado que ainda cimenta
a marcha para o futuro. Mesmo sendo um tema historicamente determinado,
é digno de nota a forma como esse eu postiço, um eu de ilusão, ou
\emph{simulacro de eu} -- tema chegado ao paroxismo na contemporaneidade
-- faz"-se presente nas duas peças e como foi refletido no século \versal{XVII}
pelo moralista Pierre Nicole, citado por Marie"-Claude Canova:
``[\ldots{}] por meio dessa ilusão, o homem fica ausente de si mesmo
e presente para si mesmo; ele se olha continuamente e não se vê nunca
realmente, porque ele vê apenas, em lugar de si mesmo, o vão fantasma
que formou de si''. (p. 132)

Ora, aspecto que salta aos olhos é que as donzelas parecem conhecer esse
mundo por meio de romances de cavalaria, de modo, portanto, idealizado.
E é por esse viés ideal que elas verão o amor. Cathos diz achar o
casamento chocante: ``Como se pode sofrer com o pensamento de deitar com
um homem realmente nu?'' (\versal{MOLIÈRE}, 1893b, p. 264).

As duas donzelas dispensam de seus amores uma dupla de amigos por não
serem de qualidade. No entanto, elas mesmas são filha e sobrinha de
burguês. A trama da peça é exatamente a ação dos amigos burgueses que,
pela injúria sofrida, resolvem vingar"-se das duas com uma encenação na
qual o valete de La Grange, ``um extravagante que pôs na cabeça querer
ser um homem de condição'' (\versal{MOLIÈRE}, 1893b, p. 253), vai se passar por
alguém de elevado espírito, ``porque nada mais barato do que um belo
espírito agora'' (\emph{Idem}, p. 253). Aparece irônica e sutilmente,
nessa passagem da peça, a ideia de que os ideais da nobreza nesse
momento já podem estar ao alcance também dos não nobres, que não
necessariamente estariam à altura do belo espírito. Além disso, os dois
farsantes desempenham bem seu papel, o que ridiculariza as falsas
preciosas, mas também sutilmente os ideais nobres, já que poderiam ser
apreendidos por qualquer um, ao menos enquanto \emph{maneiras.}

Gorgibus, pai e tio das \emph{preciosas,} como burguês típico, já não
suporta sustentar as moças, preocupa"-se em casá-las, por não aguentar
mais gastar em pomadas para os lábios. As preciosas tentam lhe explicar
porque não deram atenção aos rapazes aos quais estavam prometidas com um
raciocínio interessante sobre o galanteio -- algo que não conheciam os
dois homens, segundo elas. Como explica Madelon, a filha, a ideia do
casamento só poderia vir de seu pai, o modelo burguês. Para ela, o
casamento não é um arranjo de interesse, mas o resultado de galanteios e
conquistas. Diz ela:

\begin{quote}
[\ldots{}] o casamento não deve acontecer antes de outras aventuras.
É preciso que um amante, para ser agradável, saiba dispensar seus belos
sentimentos [\ldots{}] Primeiramente, ele deve ver no templo ou no
passeio, ou em alguma cerimônia pública, a pessoa por quem se apaixona:
ou senão, ser conduzido fatalmente à sua casa por um amigo ou parente e
sair de lá melancólico e povoado de sonhos. Ele esconde por um tempo sua
paixão pelo objeto amado, mas faz"-lhe muitas visitas [\ldots{}] O
dia da declaração chega, e deve acontecer de ordinário numa ala de algum
jardim [\ldots{}] Depois disso, vêm as aventuras, os rivais que se
jogam no caminho de uma inclinação estabelecida, as perseguições dos
pais, os ciúmes concebidos sobre falsas aparências, as queixas, os
desesperos [\ldots{}] (\emph{Ibidem}, p. 259).
\end{quote}

Essas venturosas imaginações de Madelon são mediações que certamente
passariam por \emph{ridículas} no mundo contemporâneo. Madelon considera
que fazer amor somente depois de fazer o contrato de casamento não é
mais do que um procedimento mercantil. Elas vivem um mundo idealizado,
de romance de cavalaria, e a imposição do pai faz com que o vejam com
desprezo, tal como se poderia supor o desprezo que um nobre teria por um
burguês: ``É-me penoso aceitar que sou vossa filha, e creio que alguma
aventura um dia me virá desenvolver um nascimento mais ilustre''. (Idem,
p. 265). É a expressividade da palavra nascimento aqui, que ganha
sutilmente novo significado, algo a ser ressaltado. A preciosa pretende
que durante a vida possa desenvolver um \emph{nascimento} mais digno. Se
a obsessão de Jourdain é ser como um homem de qualidades, a obsessão das
donzelas aqui é também negar sua origem para se revestir de outra, a de
uma \emph{dama de qualidades}. É lamentar seu berço desafortunado. As
formas afetadas com que querem imitar a nobreza é disso um indício. E
não é acaso que elas sejam facilmente enganadas pela encenação do
marquês falso, que acham ter vindo ao seu encontro após ter delas ouvido
falar na corte.

Do mesmo modo que o burguês Jourdain aspirante à fidalguia se deixa
obnubilar por palavras que remetem a homens de qualidade, as duas
preciosas se derretem ao ver o marquês falso falar empoladamente de
todas as suas relações na corte. Embora elas demonstrem ter certo
conhecimento do mundo nobre, o que é ressaltado na peça é o fato de esse
conhecimento não ser mais do que exterior, tanto que elas se veem
enlevadas por qualquer coisa que diga o pretenso marquês, como se o fato
de ser nobre já trouxesse um selo apriorístico de qualidade, sem
necessariamente passar por uma prova de realidade. Isso é patente quando
elas elogiam o anódino \emph{Oh! Oh!} do improviso musical do marquês
como o ponto alto: ``preferiria ter feito esse Oh! Oh! a um poema
épico.'' (\emph{Ibidem}, p. 279). Quando Cathos pergunta ao marquês
falso, Mascarilho, se ele estudou música: ``As pessoas de qualidade
sabem tudo sem nunca terem estudado.'' (\emph{Ibidem}, p. 281). Essa
frase evidentemente não as choca, o mundo nobre é o da
\emph{naturalidade.} Ética, saber, elegância, intrepidez são tão
pretensamente naturais quanto o dom musical. As preciosas estão
de tal modo enlevadas que não percebem que Mascarilho desafina ao cantar
seu improviso e que tem modos um tanto estranhos, apesar de encenar bem
o marquês. Quando chega o visconde, Jodelet, que faz parte da encenação,
elas de nada desconfiam: ``[\ldots{}] começamos a ser conhecidas; eis o
belo mundo que se encaminha até nós.'' (\emph{Ibidem}, p. 290).

Ao final, para desolação das preciosas, elas caem no ridículo de terem
acreditado em homens de qualidades que não o eram senão em aparência.
Não há como não perceber novamente certa crítica velada dos anseios
burgueses. Mas aqui, as provincianas são tratadas sem qualquer
condescendência. As vestes e a fala alambicada é para elas o suficiente
para ser um homem de qualidade. Tendo certa riqueza, mas ainda não uma
posição na sociedade, os que somente têm na vida os estribos do dinheiro
precisam se acomodar numa sociedade onde o dinheiro é deus apenas de
forma velada, disputando com uma série de símbolos sociais. Nessa
acomodação, a chamada burguesia toma os ares nobres, mas sempre
corroendo as mediações que seu amadurecimento como forma"-sujeito vai
fazendo ver como atravanco à vida mercantil. A sua imitação fetichista
de uma \emph{essência} voltada para o passado, que ela mesma precisa
superar, já prefigura o desamparo essencial da forma"-sujeito burguesa em
seu mundo repleto de mercadorias, por isso sempre precisou se apoiar em
outros universos simbólicos como bengala existencial. Esse desamparo vai
se tornar mais visível quanto mais a vida mercantil transforma o mundo
em seu mundo, não deixando ao próprio sujeito mercantil escoras fora de
seu mundo. A crítica a esse procedimento vem na fala cômica e séria de
Mascarilho no final: ``[\ldots{}] vejo bem que aqui não se aprecia nada
além da vã aparência, e que não se considera a virtude quando
desnudada.'' (\emph{Ibidem}, p. 307).

\section*{Outros tensionamentos em \emph{O avaro}: também \emph{ridículo}}
\addcontentsline{toc}{section}{Outros tensionamentos em \emph{O avaro}: também \emph{ridículo}
\medskip}


Em \emph{O avaro,} interessa menos o fato de ser uma peça, quanto tantas
outras de Molière, feitas sob o modelo da comédia italiana (\versal{RAT}, 1962,
p. 968), e mais o tensionamento entre mundo nobre e mundo burguês que se
resolve avançando rumo ao ideal burguês puro e simples. Se nas outras
duas peças, o dinheiro era relativizado em proveito de qualidades mais
elevadas imitadas comicamente, burlescamente, satiricamente, aqui, a
pura riqueza mercantil é que é admirada comicamente, posta em alto
relevo, e é a ideia de qualquer outra \emph{qualidade} que é posta de
lado. Em \emph{O avaro}, chega"-se clara e caricaturalmente ao cúmulo do
amor do dinheiro, um dinheiro que começa a ganhar feições modernas,
embora ainda não se tenha constituído uma vida social mercantil, onde o
dinheiro se investe ``produtivamente'' na produção de mercadorias num
ciclo sem fim. Dito de outro modo, o \emph{avaro,} ``Em uma palavra, ama
dinheiro mais do que reputação, honra, mais do que virtude; e só em ver
alguém que venha lhe pedir algo já sente convulsões'' (\versal{MOLIÈRE}, 1893c,
p. 62), diz Valério. Ele sente convulsões só em ver de longe alguém que
poderá lhe pedir algo que não seja empréstimo com garantia. Na peça, não
há relação entre pai, filho e filha, tudo está dilacerado\footnote{Eckermann
  relata conversa com Goethe em 12 de maio de 1825: ``Molière, disse
  Goethe, é tão grande que sentimos sempre uma surpresa nova toda vez
  que o lemos. É um gênio franco e sem par; suas peças, por mais jocosas
  que sejam, alcançam os confins da tragédia. Seu \emph{Avaro,} onde o
  vício destrói todo sentimento de respeito entre pai e filho é
  particularmente grande e trágico, no sentido elevado do termo.
  [\ldots{}] Todos os anos, leio algumas peças de Molière, assim como
  contemplo de vez em quando gravuras de quadros de grandes mestres
  italianos. Porque nós, seres pequenos, não somos capazes de guardar em
  nós o sentimento de grandeza que produzem tais obras e somos obrigados
  a voltar sempre a elas vez por outra para refrescar nossas
  impressões.'' (\versal{ECKERMANN}, 1863, p. 216-217).}. Harpagão, o avaro, que
enterra dinheiro no jardim com medo dos próprios filhos, vê tudo pelo
viés do ganho, fazendo abstração de tudo e qualquer coisa. Ele
materializa o sentido primitivo de um \emph{a priori} mercantil do
pensamento. Toda e qualquer pessoa que esteja em sua casa é vista como
alguém que espreita sua riqueza. ``Não quero ter diante de mim sem
cessar um espião de meus negócios, um traidor cujos olhos malditos
pretendem sitiar todas as minhas ações, devorar o que possuo e curiar de
todos os lados se não há nada a roubar. (\emph{Idem}, 1893c, p. 17).''
Sua preocupação é sua riqueza material, pois que de outra ele não dá
provas. Revista quem sai de sua casa e crê que todos tramam furtar"-lhe o
dinheiro. Ele reprova o filho por suas vestimentas, porque deveria
aplicar o dinheiro e ganhar juros. Faz, inclusive, no mesmo instante as
contas dos juros para 12 meses se o dinheiro da roupa que o filho
esbanjador usa fosse aplicado na usura. (\emph{Ibidem}, p. 28-29).

Mas por que, mesmo com essa caracterização delineada de um sujeito que
vive para fazer seu dinheiro multiplicar, ainda falamos em
tensionamentos? Ora, mesmo tendendo claramente a resolver o
tensionamento assumindo uma subjetividade burguesa de modo tão explícito
que se torna caricatural, uma verdadeira máscara do \emph{avaro}, mesmo
tendo trocado seu coração no mercado por uma pedra tosca, é interessante
ver que Harpagão pertence à burguesia do século \versal{XVII}, e ``o rico burguês
de então não é em geral avaro: ele tem diante de si o exemplo de uma
nobreza que gasta sem contar. [\ldots{}] A moda tem tal peso que
Harpagão não consegue cortar todos os gastos da casa.'' (\versal{BRUYERE}, 1946,
p. 73), ele tem valete, tem empregados variados -- que acumulam funções
é verdade -- mas nem ele mesmo consegue se desvencilhar do peso da
tradição de sua época. Mas, nunca é demais repetir, ele está à frente de
seu tempo. Harpagão é parente muito mais próximo na linha dialética do
burguês industrial delineadamente capitalista que o Senhor Jourdain.
\emph{O burguês fidalgo} e \emph{As preciosas ridículas} querem resolver
a tensão rumo ao passado, sem se preocupar em rumar ao futuro. Harpagão
resolve a tensão rumo ao futuro e arrasta consigo apenas resquícios que
vão ficando com o tempo apenas poeira que, entre uma sacudida e outra,
fica pelo caminho.

Ele quer dar a filha em casamento a Anselmo, não porque seja um fidalgo,
ou um homem de bem, mas porque tem riquezas, não tem mais filhos com
quem gastar e, principalmente, porque abrirá mão do dote para casar"-se
com sua filha. Todas as argumentações tentadas por Valério -- que quer
secretamente casar com a filha de Harpagão, e que é o único a saber
conversar com ele por sempre evitar contradizê-lo -- quanto à diferença
de idade, de humor e de sentimento, quanto à precaução que se deva ter,
quanto ao fato de que poderia haver pais que prefeririam talvez poupar a
insatisfação da filha a poupar o dinheiro não gasto no dote, é
respondida com uma frase: ``sem dote!'', várias vezes repetida. Já ele
próprio, embora pretensamente apaixonado, não aceita casar sem ganhar
alguma coisa. Vejamos o que diz Valério a Elise. Eles querem se casar.
Valério é astuto e, em vez de opor"-se ao desejo de Harpagão de fazer o
casamento de Elise com Anselmo, vai apoiá-lo:

\begin{quote}
Sim, o dinheiro é mais precioso que todas as coisas do mundo, e deveis
render graças ao céu pelo homem de bem [honnête homme] que vos deu
como pai. Ele bem sabe o que é viver. Quando alguém se abre à ideia de
desposar uma filha \emph{sem} \emph{dote}, não é preciso seguir adiante.
Tudo está contido nisso; e esse \emph{sem dote} substitui muito bem
beleza, juventude, nascimento, honra, sabedoria e probidade.
(\emph{Ibidem}, p. 45)
\end{quote}

A essas palavras de Valério, ditas apenas para ganhar a confiança,
Harpagão responde: ``Bravo rapaz! Eis quem fala como um oráculo.''
(\emph{Idem}, p. 45). Vê-se já o que Harpagão, o avaro, aquele que usa
sem quaisquer tensões a máscara de caráter abstrata da sociedade
mercantil de sua época, entende por homem de bem. Ele inverte as
virtudes do que se entendia por \emph{honnête homme} em função dos
simples aspectos exteriores e compráveis. Se esses aspectos, esses
ideais de uma certa elite não eram necessariamente praticados, como já
esboçamos, o que salta aos olhos é que esses ideais ao menos apontavam
para características humanas que não se perdiam simplesmente na
exterioridade que o dinheiro traz como riqueza.

O personagem Flecha, valete de Cléante, filho de Harpagão, faz neste
trecho um resumo do caráter do avaro: ``O senhor Harpagão é, de todos os
humanos, o menos humano, o mortal mais duro e encerrado em si de todos
os mortais. [\ldots{}] A secura e a aridez de suas boas graças e
carícias não encontram paralelo no mundo; e \emph{dar} é uma palavra
pela qual sente tanta aversão que nunca diz: \emph{Quero vos dar}, mas
\emph{Quero vos emprestar um bom dia.}'' (\emph{Ibidem}, p. 61).

O caráter do avaro é descrito durante toda a peça como sendo também
ridículo. Se em \emph{O} \emph{Burguês} \emph{Fidalgo} o ridículo se dá
pela busca tresloucada de qualidades, estando o dinheiro em certo
sentido relativizado, em \emph{O avaro} o ridículo reside exatamente em
viver uma vida cuja baliza é o dinheiro. A ideia desse entrecruzamento
salta aos olhos. Ou seja, por mais que a sociedade empurre os indivíduos
para posturas mercantis, uma vida completamente mercantil seria
invivível, como mostra, ao extremo, Harpagão, que se vê desnorteado
quando lhe roubam a riqueza enterrada no jardim:

\begin{quote}
Meu espírito está conturbado, e ignoro onde estou, quem sou e o que faço
da vida. Que pena! Meu pobre dinheiro! Meu pobre dinheiro! Meu querido
amigo! Privaram"-me da tua companhia; posto que tu foste de mim tirado,
perdi meu suporte, minha consolação, minha alegria: tudo acabou para
mim, e não tenho mais o que fazer no mundo. Sem ti, é-me impossível
viver. Pronto; não aguento mais; eu me morro; estou morto; estou
enterrado. Haveria alguém que gostaria de me ressuscitar devolvendo"-me
meu querido dinheiro, ou informando"-me que mo pegou? [\ldots{}]
Quero enforcar todo mundo; e se não achar meu dinheiro, enforco"-me a
mim. (\emph{Ibidem}, p. 139-140).
\end{quote}

O \emph{Meu mundo caiu,} de Maysa, perde em dramaticidade e tragicidade
para essa expressão da nova paixão humana que vai aos poucos se
instalando. Essa espécie de \emph{cantiga mercantil} nada tem a ver com
as cantigas medievais que tinham motes mais elevados. Todo esse
vocabulário trágico caberia na boca da Fedra de Racine, contemporâneo de
Molière, cujo mundo desaba ao ter que encarar o esposo Teseu e seu filho
Hipólito de quem está apaixonada. Também não seria um vocabulário
estranho ao próprio Teseu, cheio de remorsos ao saber da morte desse
filho (\versal{RACINE}, 1969, p. 777-798).

Mas poderíamos perguntar: apesar de estar aqui por demais explícito --
por isso causa uma repulsa imediata --, não estaria esse investimento
psíquico no dinheiro também presente nos cotidianos mercantis de forma
menos explícita? A modernidade não teria sido o paulatino processo de
fundamentação da vida na mercadoria e no dinheiro, sem os quais a vida
também parece perder sentido, do mesmo modo que com eles também acaba
rodando em falso por falta de relações que transcendam o mundo simbólico
que eles criam? Quantos de nós, guardadas as proporções da caricatura na
peça exposta, também ``perdem seu suporte, sua consolação, sua alegria''
quando o mundo simbólico mercantil sofre alguma ranhura? Com toda
evidência, esse sujeito não é o que reina na época de Molière, por isso
pode ser motivo de riso nessa tragicomédia. Mas não se pode também
afirmar que ele seja uma mera invenção, ou que essa forma de
subjetivação de que dá mostras Harpagão não tenha tensionado com formas
de subjetivação baseadas em outras riquezas. O próprio fato de a
sociedade capitalista precisar que o dinheiro se valorize a qualquer
custo, e a todo momento, não traria as marcas, no subterrâneo social,
dessa paixão pelo dinheiro em movimento perpétuo que parecem tão absurdo
nesse personagem? Como acrescenta Bruyelle:

\begin{quote}
O avaro, privado de seu tesouro, é um apaixonado privado bruscamente do
objeto de sua paixão. O dinheiro lhe é mais caro que a própria vida. Não
pode viver sem ele sem desejar morrer. Para pintá-lo, Molière encontrou
palavras, expressões, que Racine empresta a seus amantes que sofreram
desilusão amorosa ou foram traídos. Molière se movimenta nas margens da
tragédia, mas quis ficar em seu gênero e logrou êxito em tal empreitada.
(1946, p. 62)
\end{quote}

O que se sobressai nessa peça não é o desejo de ser um homem de
qualidades, a obsessão é pela posse, de modo também caricatural -- isso
\emph{O avaro} tem de comum com \emph{O Burguês Fidalgo}. Mas esse modo
caricatural não significa que seja falso. Ele expressa um tensionamento
subjetivo da época que foi amadurecendo no seio social.

Mas poderíamos perguntar: como esse entrecruzamento de subjetividades
pré-moderna e moderna vai caminhando para uma vitória da subjetividade
enformada na forma"-sujeito moderna? Por que tantos séculos foram
necessários para esse esvaziamento sinalizar sua força? Por que a
burguesia quis imitar as qualidades nobres e não se impor de uma vez por
todas como a classe do dinheiro sem qualquer outra mediação possível?
Esse entrecruzamento subjetivo -- que foi sofrendo sínteses contínuas
rumo à modernidade -- talvez possa abrir caminho a uma resposta:

\begin{quote}
Para alguns mercadores, a aquisição de vastas propriedades de terra e
casamentos mistos aos quais se resignavam cavaleiros em ruína
financeira, desejosos de restabelecer sua vida material casando com a
filha de um rico negociante, abriam a via da ascensão ``rumo ao topo''.
Alguns ricos citadinos conseguiam adquirir a dignidade dos cavaleiros. A
aspiração de viver o luxo caracteriza os mercadores patrícios. Para
ganharem prestígio, e impressionarem a sociedade, constroem casas e
palácios coroados de torres. A aristocracia poderia ter sentido inveja
das moradas pós"-góticas do patriciado da Alemanha do Sul e dos
palácios"-Renascença dos mercadores italianos. (\versal{GOUREVICHT}, 1989, p.
288-289, Tradução Nossa)
\end{quote}

Esse entrecruzamento subjetivo talvez tenha sido o que evitou por muito
tempo que o capitalismo mostrasse toda a sua face destrutiva em todos os
campos, da arquitetura à poesia, da música à reflexão. Esse
entrecruzamento de subjetividades ainda vai estar presente em \emph{O
homem sem qualidades}, mas não é mais a subjetividade calcada na solidez
tradicional que é a mais predominante. Ela vai se tornando pálida
paisagem e ficando para trás da novidade que constantemente se apresenta
no alvorecer do século \versal{XX} que sacramentará de uma vez por todas o
reinado sem partilha da mercadoria e do abolidor de todas as diferenças:
o dinheiro.

\chapter*{O homem sem qualidades:\\ do amorfismo humano ao caráter \emph{blasé}}
\addcontentsline{toc}{chapter}{\large\versal{O HOMEM SEM QUALIDADES:\\ \small{DO AMORFISMO HUMANO AO CARÁTER \emph{BLASÉ}}}}
\hedramarkboth{O homem sem qualidades}{}

\begin{flushright}
\scriptsize{Essa crença no desastre em meio a um mundo satisfeito com seus
progressos era a mais forte de suas qualidades.

\emph{Musil}}
\end{flushright}

Mais do que delimitar o momento exato do nascimento da vida social
moderna, interessa"-nos contribuir para traçar um caminho
histórico"-lógico de realização na história real de uma forma"-social e de
uma forma"-sujeito abstratas. Dito de outro modo, é perscrutar como um
\emph{conceito} de sociedade e um \emph{conceito} de subjetividade se
hipostasiam subsumindo a sociedade e a subjetividade concretas.

Já que usamos conceitos como \emph{tensionamentos},
\emph{entrecruzamento objetivo e subjetivo}, nem caberia procurar na
origem da dinâmica moderna um único elemento. A partir do momento em que
as trocas mercantis ganham espaço na vida social, como vimos, vão
corroendo as formas reinantes de vida social e de subjetividade. Mas
essa corrosão se dá num ritmo tão lento que não se poderia adivinhar que
somente a ampliação das trocas mercantis teria desembocado no
capitalismo sem que a mudança de qualidade trazida para o seio social
com a invenção e uso das armas de fogo viesse a impulsionar com uma
qualidade nova a concorrência. Ou seja, não cabe questionar, a nosso
ver, se foram somente fatores geográficos, demográficos, naturais, ou se
foi o amadurecimento científico, ou a Reforma Protestante que fizeram a
sociedade mercantil nascer. Em verdade, ao que parece, esses elementos
não podem ser vistos como isolados das tensões mercantis que nascem numa
determinada época no seio social. Muitas vezes, nosso olhar
retrospectivo tende a olhar as mudanças como se tivessem lugar de forma
abrupta, quase como algo conscientemente decidido por um sujeito muito
disposto a implantar a modernidade mercantil. O que não é o caso. Mesmo
os sujeitos da classe burguesa que lutaram desde sempre por seus
interesses não tinham uma forma de subjetividade amadurecidamente
moderna, e sua luta era mais por garantir seus privilégios financeiros
do que para se constituir como uma forma de subjetividade que pretende
levar adiante a forma social com a qual se identifica -- foi preciso um
amadurecimento histórico para que a forma social moderna e a
forma"-sujeito fossem se desdobrando com mais clareza. Assim, se
analisarmos a própria revolução científica nos séculos \versal{XVI} e \versal{XVII} e a
Reforma Protestante, veremos que são fruto de séculos também de
amadurecimento no seio social e não uma decisão consciente de
implantação da forma de vida social moderna.

Nesse sentido, parece mais condizente com o espírito desse estudo ver o
conjunto de elementos como fazendo parte de um grande caldeirão cheio de
ingredientes em tensão, que maturam no seio social já eivado da dinâmica
da objetividade e da forma"-sujeito modernas. Essa maturação vai se dando
pela própria dinâmica interna, que foge ao controle do indivíduo
concreto. É de certo modo o que Hobsbawn chama de ``acumulação de
material explosivo''. Mesmo sendo notável que em todo seu livro sobre a
eclosão da Revolução industrial Hobsbawn não faça qualquer menção a
mudanças de ordem subjetiva que efervesceram paulatinamente no seio
social, esse ``material explosivo'' certamente é um componente da
própria dialética contínua da subjetividade, dialética expressa no
tensionamento entre elementos que remetem ao passado e os que remetem
para o futuro, como veremos adiante. Como ele diz, não haveria como esse
processo de ``acumulação de material explosivo'' antes da Revolução
industrial não produzir uma ``combustão expontânea'' (2014, p. 30) mais
cedo ou mais tarde.

E esse material explosivo demora para se acumular no seio social. Porque
a vida social e a subjetividade concreta são de fato mais lentas que a
dinâmica mercantil e sua forma"-sujeito. Os tensionamentos, os
entrecruzamentos objetivos e subjetivos, que criam resistência concretas
e mudas ao avançar da marcha moderna estão na base da resposta à
pergunta do por que tanto tempo foi preciso para se desembocar na
Revolução industrial depois do desabrochar mercantil da Baixa Idade
Média. E mesmo do por que, após a difusão do \emph{domestic system} no
século \versal{XVI}\footnote{Ainda no último quarto do século \versal{XVIII}, com o
  consumo de algodão bruto duplicando na Inglaterra, e a importação das
  colônias e a exportação do tecido manufaturado demonstrando seu
  potencial lucrativo, ainda reinava o \emph{domestic} \emph{system}.
  ``No caso da tecelagem, os atrativos são grandes: um exército
  extensivo de artesãos e de camponeses, possuidores ou não de seus
  teares, podem fornecer produtos de qualidade a preços interessantes e
  fáceis de serem comercializados''. (\versal{RIOUX}, 1971, p. 62).
  Evidentemente, esse sistema doméstico de fabrico já foi um transtorno
  nas relações sociais anteriores e faziam parte desse caminho até o
  trabalho industrial. Com as primeiras máquinas nesse fim de século
  \versal{XVIII}, fazendo a produtividade aumentar até de 120 vezes, sem outra
  força que não seja o braço do operário, as rodas de fiar vão sendo
  deslocadas nas oficinas familiares. Produzir mais significa um aumento
  do ganho monetário, e assim: ``o artesão"-camponês abandona sua
  atividade rural e serve exclusivamente à máquina. O trabalhador
  pregado à sua máquina faz nascer o proletário doméstico.''
  (\emph{Idem,} p. 63).} e das manufaturas, ainda tanto tempo de
amadurecimento foi preciso para se chegar a uma revolução sem
precedentes nas formas de produzir.

Da mesma forma que a descoberta das armas de fogo significaram um
impulso de qualidade nova às trocas mercantis e à monetarização da vida
social, também a revolução industrial significou uma espécie de nova
revolução na vida social e subjetiva -- embora autores como John Nef não
vejam nenhuma ruptura, mas somente \emph{aceleração} de um movimento
plurissecular de progresso técnico. (\versal{NEF}, 1954b). O que há de comum em
ambas as revoluções é o caráter do impulso. Por mais que a diferença de
tempo entre os dois acontecimentos seja de mais de três séculos, tanto
no caso das armas de fogo quanto no caso da maquinaria, trata"-se de uma
nova força que ganha lugar, um força que não mais depende do homem como
principal motor, mas ao contrário, foge ao controle dos homens,
impondo"-lhe o ritmo. Como vimos, uma arma de fogo, além de não depender
de qualquer mediação simbólica, apenas depende de um mero preparo
técnico instrumental para ser usada, e seu poder de destruição escapa
aos homens, eles nada contra ele podem fazer a não ser contar com o
mesmo poder de fogo. Do mesmo modo, a maquinaria, entendida como
processo de razão técnica, contêm em si o impulso de se tornar
paulatinamente uma força independente do homem, não no sentido de que
não precise dele, mas no sentido de que o submete ao seu ritmo numa
proporção tal que as antigas e seculares rodas de fiar -- por mais que
impusessem um certo ritmo -- não teriam tido como fazer, pois sua
energia era puramente humana, portanto, com limitações dentro da esfera
humana. Em contrapartida, tanto as armas de fogo quanto as máquinas
oriundas da Revolução industrial -- que não são meros instrumentos de
aperfeiçoamento de uma ferramenta já existente -- expandem os limites
para além do terreno humano.

E para se chegar a esse impulso que significou a Revolução industrial
foram necessários séculos de amadurecimento objetivo e subjetivo. Mesmo
com o chamado desenvolvimento mercantil ligado ao comércio de tecidos,
foram necessários séculos para que as rodas de fiar fossem superadas
pela \emph{spinning jenny} (1765), pela \emph{water"-frame} e finalmente
pela \emph{mule"-jenny} (1779) de Samuel Crompton, que só podia ser
movida a água ou vapor (\versal{RIOUX}, 1971). Se as duas primeiras ainda
mantinham um caráter manual, de dependência do artesão, sendo apenas um
aumento da possibilidade produtiva, a \emph{mule"-jenny} trazia consigo o
potencial de ser uma força produtiva independente. Isso não faz senão
corroborar a reflexão segundo a qual toda revolução bem sucedida, no
sentido da subversão de uma vida social, primeiro precisa passar por um
sussurro coletivo no subterrâneo social que a prepara. Ela nunca é uma
mera eclosão de momento. Ela precisa ser preparada, no caso da forma
social capitalista, a preparação foi em larga escala inconsciente. Até
hoje, somente o capitalismo foi capaz de fazer esse gênero de revolução,
inclusive sobre si mesmo, porque conseguiu se imiscuir como
\emph{apriori} da própria dialética, de modo a sempre aparecer como
dominante nas sínteses oriundas do processo de tensão. E foi das tensões
sociais, fruto de uma mentalidade não necessariamente já amadurecida
para acompanhar as exigências que o amadurecimento da objetividade
social exige, que se gestaram resistências concretas que fizeram o tear
mecânico penar para se impor devido à revolta dos tecelões que temiam
desemprego. Mesmo com seu potencial produtivo, o tear mecânico se impôs
lentamente, porque sofria a concorrência do tear braçal. Ainda em 1815,
são 200.000 teares manuais disseminados nas zonas industriais inglesas e
2.500 mecânicos (\versal{RIOUX}, 1971, p. 64).

Se às vezes caímos na tentação de imaginar um estouro do capitalismo na
Revolução industrial, é bom lembrarmos que ela aconteceu basicamente no
setor têxtil que era modernamente o mais demandado, e que produzia um
bem útil de uso muito antigo. Como diz Rioux, ¾ dos trabalhadores
industriais estavam nesse setor em 1860 (\versal{RIOUX}, 1971, p. 92). O que
demonstra que mesmo a época da Revolução industrial não pode ser
confundida com um capitalismo que se aproxima de seu conceito, mas como
um momento qualitativamente novo em que se chegava a um amadurecimento
do quadro social e subjetivo que permitia essa mudança de grandes
proporções. Mas esse amadurecimento, como se deve intuir, não se deu do
mesmo modo nos vários países, mesmo europeus, e não nos cabe nesse
estudo adentrar pelas peculiaridades de cada rincão, não porque não nos
interessem, ao contrário, essas peculiaridades muitas vezes dizem muito
da resistência muda, ou seja, da inadaptabilidade da subjetividade à
dinâmica moderna, mas porque seria longa e árdua tarefa.

Somente entre França e Inglaterra, as diferenças em termos de
mentalidade burguesa e moderna já são notáveis. A burguesia francesa
parece mais atrelada à renda, notadamente aquela oriunda da terra. Não à
toa os fisiocratas parecem de ascendência mais francesa. O modelo
francês é o Pai Grandet\footnote{Vemos no personagem Grandet, pai de
  Eugênia Grandet, de Balzac também um protótipo interessante do sujeito
  burguês. Não o sujeito burguês \emph{accompli}, mas um momento do
  processo de amadurecimento do sujeito burguês. Sua subjetividade ainda
  é a do puro rentista de província, de mercador, uma mistura de moderno
  e arcaico, uma mistura entre homem de negócios moderno e ao mesmo
  tempo um acumulador inveterado cujos pensamentos se centram apenas no
  desejo da multiplicação de sua riqueza por meio de rendas e terras,
  não por meio da aplicação produtiva. Grandet estaria muito próximo do
  caráter do \emph{Avaro} de Molière, não sentindo qualquer desejo de
  nobreza.}, que nutre um carinho especial pelo investimento garantido,
a renda do Estado, os bons pedaços de terra, e negligencia a indústria.

Os ingleses ao contrário, já desde cedo se colocam num movimento que
hoje podemos ver como moderno e de vanguarda, um movimento de preparação
mais decidido que em outras partes para o automovimento do capital. Como
diz a historiadora Ellen Wood:

\begin{quote}
A unificação do reino inglês começara bem cedo, no século \versal{XI}, quando os
conquistadores normandos se estabeleceram, através de uma grande coesão
militar e política, como classe dominante na ilha. E no século \versal{XVI}, a
Inglaterra já percorrera um longo caminho no sentido de eliminar a
fragmentação feudal do Estado e a soberania ``dividida'' herdada do
feudalismo. Os poderes autônomos detidos pelos nobres, corpos municipais
e outras entidades corporativas existentes nos outros Estados europeus
estavam na Inglaterra cada vez mais concentrados no Estado central.
(\versal{WOOD}, 2000, p.16)
\end{quote}

Enquanto a França ainda era um país de camponeses proprietários, o
camponês inglês sumia (\versal{HABAKKUK}, 1965), para se transformar em
proletário do campo, porque a terra se concentrava em muito menos mãos e
o número de sem"-terras só crescia. A agricultura da França não conhecera
a ideologia da \emph{lucratização} da terra, enquanto ``os fazendeiros
ingleses estavam respondendo aos imperativos da competição e dos
melhoramentos.'' (\versal{WOOD}, 2000, p. 26). Esse aspecto tem consequências
importantes do ponto de vista da modernidade francesa, em que o peso da
tradição atrelada ao campo vai significar uma subjetividade menos apta à
penetração das relações mercantis, o que torna o capitalismo algo mais
difícil e lento do que no caso inglês. (\versal{RIOUX}, 1971, p. 194.)

Até mesmo do ponto de vista da ideologia econômica há desnível entre os
dois países: quando os fisiocratas quiseram dar um impulso moderno à
França -- para reagir ao modelo mercantilista de Colbert -- sob o modelo
inglês, já estavam atrasados. A revolução agrária dos fisiocratas se
baseava na crença de que a riqueza vinha da terra, dos ``bens de raiz'',
de que somente o cultivo da terra gera a verdadeira riqueza (\versal{QUESNAY},
1983, p. 333), enquanto os ingleses já estavam à frente do ponto de
vista das subjetividade moderna ao anunciarem pela pena de Smith que
\emph{a riqueza das nações} em verdade é fruto do trabalho humano
(abstrato), que ele transforma em constante antropológica: ``o valor
real de cada coisa, para a pessoa que a adquiriu e deseja vendê-la ou
trocá-la por qualquer outra coisa, é o trabalho [\ldots{}]. Não foi
por ouro ou por prata, mas pelo trabalho, que foi originalmente comprada
toda a riqueza do mundo; [\ldots{}]'' (1983, p. 63).

Foi também da Inglaterra que veio enorme impulso para uma subjetividade
com traços modernos, uma individualidade abstrata mediada pela mão
invisível, marcante na abolição da servidão já no século \versal{XVI}, dois
séculos antes da França. ``Na Grã-Bretanha, o parlamentarismo se
afirmou, consagrando desde o fim do século \versal{XVII} a potência política dos
proprietários de terra e da burguesia, ativos em face do poder real. O
\emph{Habeas Corpus} proclama a liberdade jurídica do indivíduo em face
das velhas opressões'' (\versal{RIOUX}, 1971, p. 54). Já a França demora um pouco
para dar sua contribuição ao impulso moderno. É na França revolucionária
e imperial que uma burguesia nova, ``enriquecida pela compra de bens
nacionais e o abastecimento dos exércitos'' se cria. Com ela, seguindo o
espírito do tempo, embora com grande atraso em relação ao \emph{Habeas
Corpus} inglês, vêm a declaração dos ``Direitos humanos e o Código
civil, que libertam o indivíduo e sua propriedade, liberam a atividade
econômica dos velhos cabrestos reais, enquanto a lei Le Capelier de 1791
proíbe toda e qualquer reação operária coletiva [\ldots{}]''. Esses
códigos são difundidos em toda a Europa, em todas as zonas em que ``uma
burguesia os espera, aquela mesma que levará a cabo a revolução
industrial.'' (\versal{RIOUX}, 1971, p. 55).

Entre França e Inglaterra pesa ainda a diferença perante a Reforma.
Claro que a França muito católica também terá uma indústria moderna,
``mas é evidente que as igrejas reformadas, com seu senso do indivíduo,
do esforço solitário rumo à perfeição, do calor do trabalho e do êxito
sob as bênçãos de Deus, aduba entre os fieis o gosto pela iniciativa e
pela novidade econômica e social.'' (\emph{Idem}, p. 51).

Diante desse quadro, como estaria a Áustria de Musil, ou do \emph{Homem
sem qualidades}? Ela se tornara dupla monarquia austro"-húngara em 1867
e, embora paralisada com os problemas de nacionalidade que arruinam sua
coesão, tenta se juntar apressadamente às grandes potências industriais.
A indústria têxtil tradicional parece poder ter um crescimento de tipo
inglês na segunda metade do século \versal{XIX}. Mas é a região tcheca que ganha
a frente no movimento de modernização, estando mais em contado com a
Alemanha -- lá está mais de 1/3 das indústrias do reino. É notável uma
febre de investimento entre 1850-1873, o que acelera o processo: tecidos
de algodão, linho, siderurgia. As estradas de ferro são desenvolvidas
num momento em que a Áustria"-Hungria se torna cruzamento importante
entre a Europa, o Oriente e o Mediterrâneo, e isso deveria cimentar um
vasto mercado.

Mas a industrialização arrasta consigo largas áreas atrasadas. A
Áustria, com sua burguesia alemã ou tcheca com espírito dinâmico,
tendente a moderno, arrasta atrás de si uma Hungria de grandes
latifúndios, sem uma unificação do mercado interior. (\versal{RIOUX}, p. 117). O
império é uma zona de acolhida para capitais europeus.

\begin{quote}
Por volta de 1880, pode"-se dizer que a revolução industrial está feita
na região de Viena e em alguns países tchecos. Mas se reduz à órbita
germânica, sem se expandir para o conjunto do império. Por todo lado,
sustentada na administração e no exército, a nobreza fundiária se
recusou sistematicamente seja a investir capitais oriundos de seus
domínios na indústria, seja a se aglomerar com a frágil burguesia
capitalista das cidades. As velhas estruturas sociais resistem, freando
a industrialização da Europa danubiana. Portanto, nada de revolução
industrial completa, lá ou em qualquer outro canto, sem revolução
agrícola e transformação social''. (p. 117-118)
\end{quote}

Em função do que vimos defendendo até aqui em nosso estudo, ou seja, a
ideia de que a forma social moderna e a forma"-sujeito que ela exige vão
se desdobrando historicamente numa dialética com o concreto social e
objetivo, por ser historicamente essa forma social moderna a única a
conter uma dinâmica interna, deveremos entender, chegado esse momento de
nosso estudo, o conceito de modernidade sempre como \emph{modernização},
e o conceito de subjetividade moderna como \emph{subjetivação moderna
rumo à forma"-sujeito burguesa}. Tanto modernização como subjetivação
moderna expressam uma maior ideia de dinâmica, de algo em movimento.
Nesse caso, o movimento de modernização, como veremos, é o do domínio
sempre maior da racionalidade instrumental"-mercantil sobre a realidade
concreta, assim como o domínio sempre maior da forma de subjetivação
moderna, que chamamos de forma"-sujeito moderna, burguesa ou mercantil,
sobre a subjetividade concreta, ou a individualidade concreta.

O lapso grande de tempo entre a revolução comercial que expusemos e a
Revolução Industrial e Francesa no século \versal{XVIII} -- com a revolução de
pensamento correspondente -- demonstra o quanto a dinâmica social maior
iniciada no final da Idade Média ainda estava longe da dinâmica
tipicamente moderna, e não por acaso falamos de tensionamentos que
poderiam ter sido também abafados se não tivesse havido outros fatores.
Não havia ainda um sentido da história inexorável como há na
modernidade, um \emph{telos.} Havia muitas idas e vindas, muita tradição
e agressividade da natureza. A própria invenção da arma de fogo, o
\emph{boom} em termos de aceleração do processo de colocação do dinheiro
como fim em si, demorou alguns séculos para chegar a um fenômeno massivo
de fato, que começou a se dar quando se inventaram armas portáteis sem
pavios, mais ágeis, no século \versal{XVII}. De qualquer maneira, já na época de
Molière, na época do \emph{processo} \emph{de} \emph{civilização} e do
requinte das cortes invejadas por muitos burgueses, a França tinha mais
de 5000 canhões e a Inglaterra mais de 8000 (\versal{BRAUDEL}, 1979a, p. 445).

E tanto a Revolução Industrial quanto a Francesa não ocorreram
evidentemente porque a classe burguesa acordou, um belo dia, não mais
querendo a miscelânea de pré-modernidade e processo de modernização, mas
porque esses séculos foram de amadurecimento consciente e inconsciente,
no seio social, das transformações que a relação com o dinheiro tinha
trazido. Do mesmo modo, esses séculos foram o terreno de amadurecimento
da forma"-sujeito moderna burguesa, uma forma de subjetividade que tende
a pôr de lado a onidimensionalidade\footnote{É importante esclarecer o
  fato de que, ao se falar de onidimensionalidade, não estamos querendo
  fazer o elogio da forma de vida pré-moderna em detrimento da forma
  moderna. Mas não se pode deixar de reconhecer que, apesar da rigidez
  da dureza da vida social pré-moderna, a dominação pessoal e
  não"-disfarçada não necessariamente se apoderava dos recantos mais
  íntimos da subjetividade. Na modernidade, a adequação exigida do
  sujeito é completa, pois a dominação é impessoal e disfarçada de
  liberdade de livre"-iniciativa.} da subjetividade (da individualidade)
em proveito da unidimensionalidade (\versal{MARCUSE}, 1973) de uma subjetividade
afinada com o movimento de valorização do dinheiro, que pressupõe uma
produtividade incessante -- que exige um consumo incessante.

Nesse sentido, caberia falar em \emph{dialética da subjetividade}
intrinsecamente moderna. Ou seja, de uma tensão permanente entre
formas de reprodução social e formas de subjetividade que foi levando à
superação das formas atrasadas pelas formas tidas por avançadas e
progressistas. Dito de modo simples, a forma de subjetividade calcada na
frieza e no cálculo, na ascensão social, por meio do trabalho e do
dinheiro, no investimento econômico amadureceu no seio social como
\emph{antítese} à \emph{tese} representada pela forma subjetiva baseada
na tradição, no fundamento fetichista simbólico"-religioso. Mas dessa
tensão se poderia dizer que uma nova forma de subjetividade foi
paulatinamente germinando. As novas sínteses não aniquilavam as formas
anteriores por completo, mas sempre traziam consigo aspectos da tese e
da antítese que voltavam a tensionar no momento da nova elevação
dialética, mas sem algumas características já deixadas para trás. De
modo que a nova subjetividade oriunda desse tensionamento contínuo,
embora não totalmente calcada já nos preceitos modernos, mostrava"-se
menos enraizada na tradição pré-moderna. Por isso que um mercador, como
chama a atenção Le Goff (2011, p. 12), da Idade Média, podia ter
``horizontes mais largos do que muitos eruditos modernos que o
estudaram.'' Mas esse processo dialético deu"-se paulatinamente,
continuamente e a cada nova síntese subjetiva, era a forma"-sujeito
moderna que ganhava terreno não só sobre a forma de subjetividade do
passado que ia empalidecendo, mas sobre a própria individualidade.

\section{Razão moderna e forma"-sujeito moderna}

Será que a forma"-sujeito burguesa teria sua base como protótipo no herói
Ulisses da Odisseia? Primeiramente, poderíamos dizer que para este
estudo não. A objetivação da subjetividade de que dá mostras Ulisses, na
forma como desenvolveram Adorno e Horkheimer (1986), não são suficientes
para a constituição desse sujeito. O paralelo feito por esses dois
autores, portanto, seria difícil de ser aprofundado proficuamente para
nosso intento. Da obra \emph{Dialética} \emph{do} \emph{Esclarecimento},
problematizaremos aquilo que diga mais respeito à modernidade de fato.
Afinal, apesar de nessa obra os autores refletirem sobre a dominação da
natureza em termos muito antigos, a obra tem pés mais fincados no
período moderno, quando a dominação não mais precisa de justificação na
tradição, no mundo transcendente, ou em qualquer outra forma de poder
exterior, mas ganha o sentido de segunda natureza, em que tudo se dá de
tal modo autojustificado que a forma social na qual estamos aparece como
se funcionasse tal como na natureza.

\emph{A Dialética do Esclarecimento} foi escrita tendo como pano de
fundo a barbárie da Segunda Guerra, mas sem opor a democracia ocidental
ou a Razão como uma redenção. Os autores, antes do mais, expuseram que
se aquele tipo de barbárie aconteceu é porque havia problemas ainda não
refletidos na própria racionalidade. Como chama a atenção Menegat, é
fundamental para a reflexão crítica a apreensão do ``ilógico por dentro
da própria apoteose do \emph{logos} do princípio de troca.'' (2003, p.
63).

Mas nesse momento, a pergunta que se poderia fazer é: o que há de
subterrâneo, de problemático, no pensamento racional na modernidade?

No seu célebre texto \emph{Resposta à pergunta: ``Que é
Esclarecimento''}, Immanuel Kant expõe o seguinte: ``Se for feita então
a pergunta: `vivemos agora em uma época esclarecida
[\emph{aufgeklärten}]'?, a resposta será: `não, vivemos em uma época
de esclarecimento [Aufklärung]''' (\versal{KANT}, 2005, p. 69). À época,
havia um grande entusiasmo com o progresso da Razão, caminho para se
chegar ao espírito esclarecido, livre. Mas a essa resposta de Kant, que
aponta para um \emph{devir}-Esclarecimento, Adorno opõe --- com dois
séculos de vantagem --- uma objeção, não pessimista, mas crítica em seu
texto Educação e Emancipação que encerra o livro homônimo:

\begin{quote}
Se atualmente ainda podemos afirmar que vivemos uma época de
esclarecimento, isto tornou"-se muito questionável em face da pressão
inimaginável exercida sobre as pessoas, seja simplesmente pela própria
organização do mundo, seja num sentido mais amplo, pelo controle
planificado até mesmo de toda realidade interior pela indústria
cultural. Se não quisermos aplicar o termo ``emancipação'' num sentido
meramente retórico, [\ldots{}] vazio como o discurso dos compromissos
[\ldots{}] é preciso começar a ver efetivamente as enormes dificuldades
que se opõem à emancipação nesta organização do mundo [\ldots{}]. O
motivo é a contradição social; é que a organização social em que vivemos
continua sendo heterônoma (\versal{ADORNO}, 1995, p. 181).
\end{quote}

Não pode senão chocar ao espírito crente no progresso a afirmação de que
continuamos vivendo numa organização social heterônoma. Afinal, essa
modernidade esclarecida olhava com ares de superioridade para os
pejorativamente denominados primitivos, povo \emph{incivilizado} --- que
faziam cerimônias mágico"-miméticas para aplacar seus medos diante da
natureza, numa vida de \emph{imediatez}, ou de condição natural do
homem, bem como para a tradição religiosa que não permitia ao homem sair
de sua minoridade.

Mas a desumanização, a barbárie, para os autores da \emph{Dialética do
Esclarecimento}, encontra sua explicação no fato de a Razão formal,
instrumental, na realidade, não poder se opor à dinâmica da violência,
destrutividade e desumanidade, uma vez que tais elementos, enquanto
resultado, enquanto sentido, são"-lhe indiferentes, posto que a Razão não
acolhe em seu seio a crítica, a sensibilidade, a diferenciação:

\begin{quote}
Se o esclarecimento não acolhe dentro de si a reflexão sobre esse
elemento regressivo (que segundo os autores está em germe no pensamento
esclarecido) [Grifo Nosso], ele está selando seu próprio destino.
Abandonando a seus inimigos a reflexão sobre o caráter destrutivo do
progresso, o pensamento cegamente pragmatizado perde seu caráter
superador e por isso também sua relação com a verdade. (\versal{ADORNO} \&
\versal{HORKHEIMER}, 1986, p. 13).
\end{quote}

Esse pragmatismo cego que irrompe com uma força interna revestida de
progresso não reconhece qualquer qualidade. Essa indiferenciação
universalizante dos conceitos, Adorno e Horkheimer localizaram já nos
escritos de Platão e Aristóteles que ``refletiam com a mesma pureza das
leis da física a igualdade dos cidadãos plenos e a inferioridade das
mulheres, das crianças e dos escravos'' (\emph{Idem}, 1986, p. 35).
Embora esses autores tenham vivido numa época histórica cujo fundamento
nada tenha a ver com o moderno, não parece acaso que esse pensamento de
abstração tenha sido retomado no renascimento filosófico e econômico a
partir do século \versal{XII}. E esta mesma indiferenciação os autores localizam
na Razão kantiana:

\begin{quote}
Mas, segundo Kant, as forças éticas, perante a razão científica, são de
fato impulsos e comportamentos não menos neutros do que as forças
aéticas, nas quais se convertem, tão logo deixem de se orientar para
aquela possibilidade oculta, buscando a reconciliação com o poder. O
esclarecimento expulsa da teoria a diferença (\emph{Ibidem}, 1986, p.
85).
\end{quote}

Essa abstração entranhada na Razão é um dos aspectos mais candentes, que
tem relação intrínseca com a própria vida social sobre a qual plana uma
abstração apriorística representada concretamente pelo dinheiro, que não
conhece qualquer diferença. Essa abstração está relacionada também com
um dos conceitos que os autores utilizam em sua empreitada crítica: a
\emph{dominação da natureza}. Um tipo de dominação que na sociedade se
transforma em dominação \emph{sobre os homens}. Ora, ao rebaixar a
natureza a \emph{uma mera objetividade} e o objeto a um mero exemplar de
uma espécie sem mais outra especificidade, (portanto, a uma abstração!),
o próprio sujeito torna"-se \emph{mero possuir,} mera identidade abstrata
que anda no mundo a fazer cálculos para dominá-lo e submetê-lo (\versal{KURZ},
1997, s.p.). Essa dominação da natureza, na era anterior ao capitalismo,
não tem parâmetro de comparação com o que se dá na modernidade, quando
as ciências se consolidaram sob esse prisma, em que o procedimento é o
que vale, não aquela satisfação ``que para os homens se chama `verdade',
mas a \emph{operation}, o procedimento eficaz'' (\versal{ADORNO} \& \versal{HORKHEIMER},
1986, p. 20), indiferente à qualidade do resultado visto apenas como
consequência de um bom método:

\begin{quote}
O saber que é poder não conhece nenhuma barreira, nem na escravização da
criatura, nem na complacência em face dos senhores do mundo. [\ldots{}] A
técnica é a essência desse saber, que não visa conceitos ou imagens, nem
o prazer do discernimento, mas o método, a utilização do trabalho de
outros, o capital. (\emph{Idem}, 1986, p. 20).
\end{quote}

A conclusão a que chegam os autores é a de que os homens, em verdade,
nada querem senão conhecer a natureza e empregá-la a fim de dominá-la e,
assim, dominar os outros. E no percurso rumo à ciência moderna, ``os
homens renunciaram ao sentido'' (\emph{Ibidem}, p. 21). Toda essa
concepção racional abstrata faz parte de um processo de subjetivação
que, conscientemente, ou não, casa muito bem com o desenvolvimento de
uma racionalidade instrumental"-mercantil. Isso porque a ciência moderna
se desenvolveu a partir de um processo de domínio objetivo da natureza
levado a um nível que a história humana ainda não conhecera. Nesse
processo --- um verdadeiro ``programa de desencantamento do mundo''
(\emph{Ibidem}, p. 19) --- as qualidades do mundo são destruídas,
juntamente com os deuses, pois o pensamento ordenador decompõe tudo o
que é próprio e o que nos homens e nas coisas não se resolve na
investida objetivante. Consequentemente, ``o que não se submete ao
critério da calculabilidade e da utilidade torna"-se suspeito para o
Esclarecimento'' (\emph{Ibidem}, 1986, p. 21) que, por princípio, é
totalitário, na medida em que despe a sociedade de qualidades sensíveis
para poder submetê-la ao mero cálculo. Não é acaso que a lógica formal
passa a ser identificada como ``a grande escola de uniformização'', pois
na medida em que faz abstração dos conteúdos sensíveis, ela torna o
número o ``cânon do Esclarecimento'' (\emph{Ibidem}, p. 22).

Na lógica formal, o pensamento é indiferente a seus objetos, que eles
sejam mentais ou físicos, ``pertençam à sociedade ou à natureza,
tornam"-se sujeitos às mesmas leis gerais da organização, cálculo e
conclusão --- mas o fazem [\ldots{}] em abstração de sua `substância'
particular.'' (\versal{MARCUSE}, 1973, p. 136). Essa lógica formal, portanto, não
é uma lógica social, é a mesma que domina no reino das coisas de cuja
utilidade se faz abstração perante o princípio meramente formal da
forma"-valor.

Podemos, portanto, afirmar que, embora algumas características desse
formalismo possam ser localizadas desde o mito --- como objetivação ---,
algo defendido por Adorno e Horkheimer, somente se desdobrariam e se
tornariam universais na modernidade produtora de mercadorias. Ora, no
universo simbólico"-mítico"-religioso essas abstrações ficavam no terreno
transcendente. Já no universo simbólico"-mercantil, o formalismo encontra
terreno fértil na própria vida social, em cujo fundamento mesmo, que é a
mercadoria, o uso, portanto, o aspecto concreto, sensível e qualitativo,
apenas importa porque é portador do que é mais importante para a
sociedade: o valor, mera forma que faz abstração de todo conteúdo
sensível da riqueza (\versal{MARX}, 1985), de toda e qualquer substância
particular. Como explicita Menegat:

\begin{quote}
As relações da sociedade moderna são sustentadas por um processo social
de abstração que funciona como uma espécie de formalização imposta às
formas materiais necessárias à existência coletiva. Esta formalização
faz girar o eixo constitutivo deste processo em torno da dinâmica
autônoma adquirida pela mercadoria e o valor.'' (2011, p. 03)
\end{quote}

Eis um aspecto fundamental neste estudo que trata, pode"-se dizer, desse
embaçamento das qualidades concretas -- que aqui chamamos também
processo de esvaziamento da realidade objetiva e subjetiva. Poderíamos,
nesse contexto, retomar a ideia de que não foi simplesmente a técnica
que deu à luz a modernidade capitalista. Conforme expõe Anselm Jappe nas
suas \emph{Aventuras da mercadoria} (2006), vários acontecimentos
decisivos para o advento da modernidade capitalista já tinham acontecido
em outras épocas, como a invenção de técnicas capazes de aumentar a
produtividade. Mas não havia um quadro geral objetivo e subjetivo no
qual o pensamento dito científico da sociedade se fundamentasse numa
racionalidade abstrata e tecnológica, porque, como escreve Marcuse
(1973), a lógica formal é:

\begin{quote}
o primeiro passo na longa viagem para o pensamento científico --- apenas
o primeiro passo, porque ainda é necessário um grau muito mais elevado
de abstração e matematização para ajustar o modo de pensar à
racionalidade tecnológica. [\ldots{}] Muito antes de o homem tecnológico
e a natureza tecnológica terem surgido como objetos de controle e
cálculo racionais, a mente foi tornada susceptível de generalização
abstrata. (p. 137).
\end{quote}

A ideia de Marcuse dessa \emph{longa} \emph{viagem para o}
\emph{pensamento} \emph{científico} parece ir ao encontro de nossa
reflexão de que o quadro social e subjetivo favorável a esse pensamento
só pôde se constituir num longo processo de tensionamentos, mas que
passou a ter uma dinâmica e um desenvolvimento numa dada direção tão
logo entrou em cena a forma de produção em larga escala, que só pode ser
feita num quadro de generalização abstrata do que é produzido, bem como
dos efeitos nocivos na vida das técnicas produzidas. Como explicita
Anselm Jappe:

\begin{quote}
A gênese da ciência moderna e da concepção quantitativa da natureza no
século \versal{XVII} esteve estreitamente ligada ao irromper do valor abstrato
nas trocas materiais e do tempo abstrato na vida social [\ldots{}] A
mesma quantidade sem qualidade que se impunha no dinheiro informava
também a concepção galileica da natureza: tal como a lógica do valor
reduz tudo e qualquer objeto a uma quantidade de valor, também a partir
de Galileu todos os corpos se encontram reduzidos a sua mera extensão no
espaço. Com a física de Newton, passa a acreditar"-se que uma única
força, a gravitação, rege todo o universo, tal como nessa mesma época o
mundo começava a unificar"-se sob a governação de uma única força, o
dinheiro. (\emph{Idem}, p. 191).
\end{quote}

Diante disso, caberia a seguinte objeção: será que o extremo domínio da
natureza encarada como mero objeto manipulável pelo sujeito do
conhecimento ``teria apenas o objetivo de livrar os homens do medo e
investi"-los na posição de senhores''? (\versal{ADORNO} \& \versal{HORKHEIMER}, 1986, p.
19). Ou haveria algo como uma lógica tácita interna ao próprio
desenvolvimento da sociedade, uma ``célula germinal'' (\versal{MARX}, 1985),
específica da modernidade, que permaneceria por refletir no âmbito das
próprias ciências, posto que é considerada como um \emph{a priori}?

Na modernidade produtora de mercadorias, a natureza é objeto, assim como
os seres humanos, para além de uma diferenciação de classe, de um
movimento dinâmico de valorização do dinheiro que encara tudo quanto
existe exatamente como o Esclarecimento: objeto quantificável que deve
se adaptar à calculabilidade. É nesse ponto que intervém um conceito
fundamental para nossa empreitada crítica, o conceito de
\emph{fetichismo da mercadoria} desenvolvido por Marx. Partir desse
conceito nos possibilita exatamente buscar a dominação, a heteronomia,
num nível mais profundo de abordagem.

\section*{O fetichismo da mercadoria como~fundamento~moderno}
\addcontentsline{toc}{section}{O fetichismo da mercadoria como fundamento\\ moderno}

Mas por que, depois da empreitada do esclarecimento moderno, depois de
se julgar ter chegado ao nível mais elevado de civilização humana, ainda
se pode afirmar que vivemos ainda sob reinado de uma heteronomia? Se
vivemos sob uma dominação depois de a Razão moderna ter espalhado seu
feixes de luz por quase todos os recantos do planeta é porque essas
luzes não chegaram aos recônditos sagrados do fetichismo da mercadoria
(\versal{MARX}, 1985), o único encantamento não atingido pela empreitada de
desencantamento do mundo.

Tomar a crítica do fetichismo da mercadoria de Marx como ponto de
partida significa opor"-se à afirmação política do próprio Marx de que
``a história de toda a sociedade até aqui é a história da luta de
classes.'' Tomamos o caminho aqui que reflete a história em verdade como
uma história de relações fetichistas (\versal{KURZ}, 2010, 2014; \versal{JAPPE}, 2006), ou
seja, de inconsciência social -- por isso a denominação segunda
natureza. Mas reconhecemos que o fetichismo da mercadoria é especial,
ele não é simplesmente o mesmo de outras tantas sociedades que
historicamente já tiveram lugar, que também não dispunham
conscientemente de seus meios e que projetavam seu poder social em algo
extrassocial que passava a dominar. Com diz Kurz (2014), o fetiche
moderno, usando um termo de Kant, seria agora transcendental, e não mais
transcendente.

\begin{quote}
O pressuposto mudo e objetivado, aparentemente evidente, do
conhecimento, do pensamento e da ação no mundo é arrancado de seu
ancoradouro no além e posto como `intramundano', sem no entanto ser uma
parte ou um momento integrante desse mundo (natural e social), ou não
seria um `princípio' inconscientemente pressuposto a todas as coisas e
situações no mundo.'' (p. 68)
\end{quote}

Em termos da forma de fetiche anterior à modernidade, os seres supremos
ou o ser supremo ficava no universo transcendente, ou seja, em termos
sensíveis, não era tangível em lugar nenhum. Nesse contexto, `` as
correspondentes abstrações não assumem forma visível, com aparente
atividade autônoma ou em processo, permanecendo antes fenômenos
confinados ao espírito humano.'' Claro que o fetichismo moderno também
não pode ser apreendido como tal, e por isso também ``contém em si um
momento transcendente''. A diferença é que ele é uma abstração real, que
não fica só no espírito humano, mas surge de modo ``físico"-empírico na
forma do dinheiro'' (\emph{Idem}, p. 69), que não é mero meio de troca,
mas o representante fantasmagórico da riqueza social do conjunto da
sociedade.

Nesse sentido, o fetichismo da mercadoria é especial porque o
objeto"-fetiche dos sujeitos modernos esclarecidos é especial e muito
mais poderoso do que qualquer totem anterior à modernidade esclarecida
ou poder divino, uma vez que ele possui uma autodeterminação e uma
característica únicas na história: uma dinâmica incessante,
incontrolável e aprioristicamente justificável de valorização de sua
substância que é o valor. Embora a modernidade venda a ideia de que
nunca houve sociedade mais livre e consciente.

Nesse contexto crítico, a especificidade do fetiche na modernidade
esclarecida, sua dinâmica interna, leva"-nos a pensar que a intuição
desenvolvida por Benjamin (2006, p. 41) -- com base na frase de Michelet
segundo a qual ``Cada época sonha a seguinte'' -- é somente moderna.
Ora, a ideia de Benjamin de que ``À forma de um novo meio de produção,
que no início ainda é dominada por aquela do antigo (Marx), correspondem
na consciência coletiva imagens nas quais se interpenetram o novo e o
antigo'', com essas imagens sendo imagens do desejo em que o coletivo
busca superar e ``transfigurar as deficiências e imperfeições do produto
social, bem como as deficiências da ordem social de produção'', não só
deve ser relacionada apenas à dinâmica moderna como pode ser aplicada do
mesmo modo à dinâmica da subjetivação na modernidade, quando formas
sempre novas e adaptadas à vida social mercantil são exigidas das
individualidades concretas. Dito de outro modo, desviando Benjamin e
Michelet: cada forma subjetiva na modernidade sonha a seguinte.

Não precisamos ir para a Europa para vermos um exemplo concreto de
transformação das condições materiais e subjetivas da existência exigida
pela forma social capitalista e para sentirmos na prática os efeitos do
fetichismo da mercadoria. Na minha pequena cidade do interior do Ceará,
uma cidade serrana que servia de refúgio para o gado no período de seca
no sertão, uma cidade então totalmente agrária, aconteceu em poucos anos
o que levou décadas ou séculos na Europa anterior à modernidade. Nessa
cidade, até o fim da década de 1980, a produção agrária era basicamente
para o consumo local e venda de excedentes. Obviamente, havia alguns
grandes produtores, mas os pequenos produtores tinham uma produção para
si e para o mercado, quando sobrava. Quem tinha o menor pedaço de terra,
podia colher e torrar o próprio café ou cultivar outras coisas -- quem
não tinha, podia ainda arrendar uma pequena terra com pagamento \emph{in
natura}. Havia a cana, da qual vinha a rapadura. Havia a mandioca, da
qual vinha a farinha e a goma de tapioca. Em segundo lugar, vinham o
feijão, o milho e as hortaliças. Sem esquecer a criação de animais,
notadamente a galinha que dava ovos cotidianamente. Chamava a atenção o
fato de que tudo que era produzido servia primeiro ao produtor e à sua
família, para depois servir ao mercado. Se pudéssemos isolar essa cidade
da produção global, poderíamos dizer que a forma"-mercadoria ainda não é
a forma de que se reveste essa produção, mas apenas a circulação -- o
que significaria a inexistência de forma"-mercadoria. Do mesmo modo,
poderia se dizer que a subjetividade das gentes dessa cidade se mostrava
menos próxima de uma subjetividade moderna. Ainda há nas mentalidades o
reconhecimento de um aspecto de utilidade que se sobrepõe em larga
escala. Não porque as pessoas eram mais inteligentes, mas porque a vida
era assim, com a sociedade seguindo um ritmo quase natural. Com a
difusão da televisão e dos produtos industrializados, que chegaram aos
lugares mais afastados, com o uso da tecnologia na produção, com os
altos índices de produtividade que passaram a ser necessários para
qualquer produção que passava a visar ao mercado em primeiro lugar,
houve a decadência do que se costumava chamar de \emph{agricultura} em
proveito do \emph{agronegócio} (uma explícita perda de qualidades
concretas), quer seja de grande porte, ou familiar. Antes que o crítico
se levante para dizer que minha pequena cidade não é significativa do
ponto de vista do capital global e que esse exemplo pode descambar para
um individualismo metodológico funesto, esclareço que esse relato aqui
só tem um objetivo: pintar um pouco o quadro de um caso concreto para
notar as mudanças rápidas, e não erigir como exemplo de funcionamento da
sociedade.

O que mudou? Em primeiro lugar, aquela gama variada de produtos deu
lugar a alguns produtos prediletos do mercado que têm grande
produtividade. O café agora cai de maduro no chão e ninguém mais o colhe
com a justificativa de não ser lucrativo em pequena escala e de ser
muito mais barato comprar o café no supermercado. Cana não se planta
mais com a mesma justificativa. Mandioca também não. Plantam"-se agora
mercadorias que têm rentabilidade. Significa que a utilidade prática do
bem cultivado apagou"-se quase por completo frente à lógica mercantil.
Café, cana, mandioca, hortaliças, sendo agora mercadorias desde seu
nascedouro, só podem ser cultivados para a venda. Aqueles que têm
pequenas terras já não plantam porque não ``vale a pena''. Mas a terra
continua lá, os produtos da terra ainda nascem. O problema é que pelo
critério de ser para o próprio sustento, ou até para a partilha, não
vale a pena, valeria apenas se fosse para o mercado. Em vez de plantar,
é mais barato e mais simples comprar. Dito de modo mais cínico: plantar
alimento não vale à pena, mas mercadorias, sim. Evidentemente, não se
pode abstrair a dureza que é a lida no campo, que era a vida de quem
passava o dia em tal lida e ao voltar à casa tinha que pilar, torrar e
pisar o café para poder tomá-lo, ou o arroz. Não se pode negar que está
bastante presente no imaginário dessas pessoas o sentimento da dura lida
com a natureza, que se vê aparentemente superada por um supermercado
repleto de mercadorias alimentares cuja origem não se conhece e às quais
se pode ter acesso ``facilitado'' com dinheiro.

Mas além dessa mudança de mentalidade, que apagou a utilidade prática do
que se produzia, há um segundo aspecto: o ciclo social não mais se
fecha, os filhos e netos desses produtores já não são absorvidos pela
agricultura que, ou acabou, ou utiliza muita maquinaria -- sem contar
que ser agricultor é algo que se tornou ainda mais desonroso, coisa para
pessoas menores. O que acontece? Uma massa de jovens desempregados se
acumula pelos lugarejos a viver da aposentadoria dos pais ou avós.
Muitos são os que estão no terreno das drogas pesadas que assolam
cidades grandes e pequenas. Mas muitos também estão no terreno das
drogas suavizadas e legalizadas, embora pesadas, que são os alucinógenos
televisivos e internéticos. Não são poucos os casos dos jovens
desempregados, mas que seguem a última moda e andam motorizados (a moto
-- e o carro já começa a substituí-la -- substituiu o jegue, o burro e o
cavalo no interior) sobretudo graças ao empréstimo consignado de um
aposentado. Mas há também o caso dos muitos agricultores e filhos de
agricultores de outrora que agora trabalham na construção civil e se
veem obrigados a lutar para que os órgãos ambientais liberem as licenças
e o capital imobiliário construa mais e mais condomínios fechados na
serra, para que possam ter o seu emprego. Frente à destruição que isso
acarreta, a ideia keynesiana de cavar buracos para depois tapá-los só
para mover a economia parece muito mais racional. Uma prova de que os
sujeitos numa forma social em crise, por precisar seguir
irremediavelmente adiante, podem tanto entregar"-se à violência pura e
simples ou defender com unhas e dentes a destruição criadora.

Há quem defenda que estamos indo bem. Que a vida agora é melhor. Que a
vida nessa agricultura era pesada e de sujeição pessoal. Mas será que
devemos sempre aceitar a escolha apenas entre Caríbdis e Cila? Não se
trata de defender a dureza e a sujeição anteriores, nem tampouco a
pretensa suavidade contemporânea -- que é falsa.

Diante disso, podemos dizer que o que aconteceu foi um processo sempre
maior de interiorização dos elementos da vida social mercantil que se
embrenharam por lugares onde há pouco tempo não tinham entrado. A
televisão -- num primeiro momento, e agora a internet -- cumpriram um
papel fundamental de levar a cada lar o contato diário com a mercadoria
e os modelos de êxito e gozo. Assim como deus antes ditava o sucesso ou
não daquelas plantações, agora outro deus menos vistoso e mais presente
dita que produção deve ser feita. É impossível, assim, defender o
desenvolvimento na modernidade sem defender o valor"-mercantil. Pois foi
em nome do valor que esse processo se deu.

Nesse sentido, a aparente superação da dominação pessoal que se operou
na modernidade, e que foi sentida nesse interior do Brasil tardiamente,
não significou a superação da objetivação da natureza externa e interna
dos sujeitos, segundo Kurz (1997), mas sim, por meio do mercado, a
``dominação pessoal foi substituída por uma `dominação da reificação',
ou seja, não se superou a `injustiça social', que foi apenas objetivada
pela mediação universal da concorrência a um grau de abstração mais
elevado do que antes''. E no exemplo dado, a guerra concorrencial deixou
marcas bem profundas. Ora, a sociedade moderna viu surgir um poder muito
maior do que qualquer absolutismo: o poder objetivo das coisas (\versal{MARCUSE},
1978, p. 34-35). É a isso que Marx chama de ``Fetichismo da
mercadoria'', ou seja, uma dominação especial em que:

\begin{quote}
[\ldots{}] a forma mercadoria e a relação de valor dos produtos de
trabalho no qual ele [valor] se representa não tem que ver
absolutamente nada com sua natureza física e com as relações materiais
que daí se originam. Não é nada mais do que determinada relação social
entre os próprios homens que para eles aqui assume a forma
fantasmagórica de uma relação entre coisas. Por isso, para encontrar uma
analogia temos de nos deslocar à região nebulosa do mundo da religião.
Aqui, os produtos do cérebro humano parecem dotados de vida própria,
figuras autônomas, que mantém relações entre si e com os homens. Assim,
no mundo das mercadorias, acontece com os produtos da mão humana
[\ldots{}]'' (\versal{MARX}, 1985, p. 71)
\end{quote}

O fato de os produtos humanos serem dotados de vida própria, o fato de
as relações entre os homens assumirem a forma da relação entre coisas
não é simples estilismo literário, jogo de palavras, mas se dá na
realidade concreta. Do mesmo modo, o que Marx expõe em seu conceito de
\emph{fetiche} \emph{da} \emph{mercadoria} vai muito mais além de uma
ilusão, de uma mistificação, de uma falsa consciência, de um feitiço
criado após a produção que impediria àquele que compra a mercadoria ver
que há escondido ali relações de produção em que o proletariado tem sua
força de trabalho explorada, uma exploração que se expressa numa
mais"-valia não paga -- como interpreta o marxismo tradicional. A questão
principal no fetichismo moderno não é se as horas são pagas ou não, ou
se esse não pagamento se vê enevoado. Marx também dá margem para essa
compreensão. Mas para nós, em realidade, o mais fundamental é sua
crítica ao fato de que o objeto"-mercadoria em si é problemático por
conter uma substância, uma essência, se preferirmos, que sempre tende à
desmedida, uma essência que submete a vida social concreta -- tanto
daquele que rouba as horas, como daqueles que querem receber mais por
suas horas pagas sem questionar a forma de riqueza produzida. Ou seja,
tanto a burguesia quanto o proletariado, embora em posições bem
diferentes perante a riqueza mercantil, estão submetidos ao mesmo
processo irrefletido do sujeito automático que lhes impõe uma
\emph{máscara de caráter} (\versal{MARX}). E essa essência é, para dizê-lo sem
rodeios, a dinâmica desmedida de valorização do dinheiro. No seio de
cada mercadoria, agita"-se invisivelmente essa dinâmica.

Em suma, se antes da modernidade a troca de equivalentes era um fenômeno
limitado dentro da vida social e submetida a uma rede de relações
oriundas do universo simbólico reinante não"-mercantil -- o que fazia com
que essa equivalência não espelhasse suas luzes na própria sociedade --,
na modernidade, o dinheiro encontra seu conceito ao se transformar em
início, meio e fim de um processo que se pretende justificado por si
mesmo, um processo cujo objetivo não é mais do que fazer com que o
dinheiro se multiplique, estando o resultado, o sentido, como
secundário, ou senão como consequência óbvia. Essa multiplicação
compulsiva e obrigatória introduz uma dinâmica que Adorno bem apreendeu
(2008, p. 122): ``[\ldots{}] a constituição específica da sociedade em
que vivemos [\ldots{}] é governada por um princípio dinâmico. Ou seja,
simplesmente que, vista nos termos de um protótipo, a sociedade
capitalista [\ldots{}] só se conserva na medida em que se expande''.

Apesar desse trecho fundamental, a concepção de fetichismo de Adorno e
Benjamin não é a mesma de Marx. Este via o fetichismo na própria
produção, aqueles viam o fetichismo mais em termos da circulação, uma
espécie de aura que ganham as mercadorias no mercado que obnubila a
consciência. Essa obnubilação em termos de Marx se dá já na produção
primeiramente, quando a sociedade nem mesmo tem poder de decidir sobre o
que é produzido. Armas ou brinquedos, pão ou agrotóxicos passam pelo
mesmo critério da pura rentabilidade, porque a vida social encara como
normal o fato de que ``Um pão não é um pão e um par de calças não é um
par de calças'' e de que ``as duas coisas `valem' apenas como
representação de `trabalho abstrato' passado e, assim, da `riqueza
abstrata' na forma do dinheiro.'' (\versal{KURZ}, 2014, p. 80). Em termos de
Marx, portanto -- seguindo nossa interpretação e não a do marxismo
tradicional --, o fetichismo não é menor num quilo de batata do que numa
calça de marca, num quilo de tomate ou num \emph{iphone.} De modo que,
no universo simbólico fundado na mercadoria, os indivíduos vivem um
mundo em que ``Seu próprio movimento social possui para eles a forma de
um movimento de coisas, sob cujo controle se encontram, em vez de
controlá-las.'' (\versal{MARX}, 1985, p.73).

E esse movimento social reificado contém uma dinâmica que, ao
desdobrar"-se e para desdobrar"-se, vai pondo a mercadoria cada vez mais
como centro da vida social objetiva e subjetiva, o que Benjamin
expressou como sendo a \emph{entronização da mercadoria} (1997, p. 36).
Assim, a mercadoria como riqueza social passa a ser o hieroglifo social,
algo \emph{fisicamente} \emph{metafísico,} de modo que uma mesa como
mercadoria ``desenvolve de sua cabeça de madeira \emph{cismas} muito
mais \emph{estranhas} do que se ela começasse a dançar por sua própria
iniciativa.'' (\versal{MARX}, 1985, p. 70). Quiproquó, cisma, mistério, místico,
enigmático, sensível suprassensível, fisicamente metafísico, sutilezas
metafísicas, manhas teológicas, são algumas caracterizações que Marx faz
da mercadoria e do complexo fetiche que ela introduz na vida
social\footnote{A introdução à tradução francesa de \emph{O Capital}
  feita por Althusser, na qual ele aconselha não começar a leitura dessa
  obra pelo primeiro capítulo, porque impediria a compreensão do mais
  importante, a mais"-valia, foi incluída na nova edição do primeiro
  volume de \emph{O Capital} lançado agora pela Boitempo. Para nós, a
  riqueza da crítica ao capitalismo feita por Marx se inicia de fato no
  primeiro capítulo. Ali, há de modo profundo aspectos distintivos do
  capitalismo em relação a outras formas sociais.}.

As palavras abstratas para algo tido por simplesmente material,
concreto, trivial à primeira vista e o paradoxo dos conceitos por Marx
empregados só podem indicar que a mercadoria tem um quê de mistério, de
secreto, que precisaria ser desvelado por aqueles que pretendem colocar
em questão a sociedade capitalista. Do mesmo modo como a forma de
riqueza capitalista precisaria ser alvo de crítica e não alvo de luta
para ser distribuída, visto ser uma forma de riqueza que pressupõe a
forma dinâmica da expansão pelo processo especificamente moderno de
reacoplamento do dinheiro em si mesmo, um processo dentro do qual está
emaranhado o conceito de progresso moderno.

Dessa forma, podemos caracterizar o capitalismo como uma sociedade
críptica e crítica. O aspecto críptico diz respeito ao enigma que ronda
o fetichismo da mercadoria, a substância abstrata e dinâmica da
mercadoria muitas vezes tida por obviedade axiomática. Ou seja, no fim
das contas, ao adorarmos a mercadoria no culto cotidiano da compra, quer
seja de feijão ou de objetos supérfluos, estamos adorando mesmo sem
querer a valorização do dinheiro -- contida na mercadoria -- a que a
vida social está submetida, uma vez que o fetichismo apaga as
características específicas do objeto transformando"-o em mera mônada de
trabalho que se realizará em dinheiro. Obviamente, a consciência
individual cotidiana não apreende dessa forma, e entende o ato da compra
como o ato mais singelo de satisfação de suas necessidades.

O aspecto crítico diz respeito ao fato, que será desenvolvido no quarto
capítulo, de que o capitalismo tem uma dinâmica interna que não lhe
permite encontrar um equilíbrio, o que significa que a forma"-social
mercantil vive em crise permanente por precisar sempre romper seus
próprios limites. Ela só existe na medida em que se desenvolve e
desenvolver"-se significa desdobrar a lógica vazia da indistinção, da
falta de qualidades que lhe é inerente e que ela impõe quanto mais a
mercadoria é entronizada na vida social e subjetiva. Passemos à análise
da máscara de caráter que essa forma social exige.

\section{O \emph{indivíduo} com a máscara de caráter do~sujeito~burguês~--~o~\emph{Indivíduo}}

\begin{quote}
Os avarentos não creem numa vida futura, o presente é tudo para eles.
Essa reflexão lança uma luz horrível sobre a época atual, onde, mais que
em qualquer outro tempo, o dinheiro domina as leis, a política e os
costumes. Instituições, livros, homens e doutrinas, tudo conspira para
minar a crença numa vida futura, sobre a qual se apoia o edifício social
há 1.800 anos. Hoje em dia, o esquife é uma transição pouco temida. O
futuro, que nos esperava para além do réquiem, transportou"-se para o
presente. Chegar per fas et nefas [pelo lícito e pelo ilícito] ao
paraíso terrestre do luxo e dos prazeres vãos, petrificar o coração e
macerar o corpo em busca de posses passageiras, como outrora se sofria o
martírio da vida em busca de bens eternos, eis a ideia geral! Ideia
aliás inscrita por toda parte, até nas leis, que perguntam ao
legislador: ``Que pagas?'', ao invés de ``Que pensas?''. \emph{Quando
essa doutrina tiver passado da burguesia ao povo, que será do país?}
[grifos nossos]

\emph{Balzac}, Eugênia Grandet
\end{quote}

Não seria possível levar a cabo a objetivação da natureza sem a
objetivação da própria subjetividade, é o que defendem os autores da
\emph{Dialética} \emph{do} \emph{esclarecimento.} No excurso I sobre
Ulisses, Adorno e Horkheimer localizam nesse ser mítico o arquétipo do
sujeito burguês, abstrato e objetivante, ``que transforma o sacrifício
em subjetividade'', ``que desfere golpes contra si mesmo para se
conservar'' (\versal{ADORNO} \& \versal{HORKHEIMER}, 1986, p. 61), um herói que tem que
reprimir e dominar seus próprios instintos a fim de ser o sujeito da
dominação. Logo, um sujeito que não sucumbe face à natureza,
representada, por exemplo, pela sereia de canto envolvente ao qual não
se pode resistir ao ouvi"-lo. Os servos têm os ouvidos tapados com cera.
Já o dominante permite"-se ouvir o canto, estando antes atado ao mastro
da nau para não sucumbir.

Os frankfurteanos falam, então, que temos aí o ``arquétipo do sujeito
burguês''. Mas será que se poderia fazer uma projeção histórica para
colocar em paralelo o sujeito burguês moderno e o ``sujeito'' do início
da objetivação levada a cabo pelo Esclarecimento desde os primórdios?
Parece haver enorme diferença entre aquele indivíduo da antiguidade e o
sujeito moderno da concorrência, cuja subjetividade é objetificada na
racionalidade científica e no processo dinâmico de valorização do
dinheiro. Embora no conjunto da \emph{Dialética do Esclarecimento} haja
certa ambivalência no que respeita à historicidade de alguns conceitos
(\versal{TRENKLE}, 2002) --- como a troca que, às vezes, temos a impressão de ser
tida como trans"-histórica, como sempre tendo feito parte da
socialização, e não como histórica ---, o sentido de arquétipo deve ser
entendido muito mais num sentido metafórico do que conceitual. Porque
não se pode chamar de burguesa toda forma de objetivação da
subjetividade. O que caracteriza a subjetividade burguesa é muito mais o
caráter formal dessa objetivação, o que faz dela uma forma"-sujeito que
tende a se tornar generalizante e igualar as individualidades. Já a
objetivação no caso de Ulisses visa justamente a distinção. No
caso de Ulisses, a objetivação da subjetividade visa a distinção social
pelo que é próprio dessa subjetividade, pelo que ela expressa de
indivíduo dominante. No caso do burguês, a objetivação visa a
equalização da subjetividade e a distinção apenas no plano da posição
perante a riqueza que essa objetivação permite acumular. E essa
equalização se dá sempre numa dinâmica interna a essa mesma
forma"-sujeito moderna, o que faz com que a ilusão que se pudesse nutrir
com um mundo burguês liberal, democrático e cheio de cultura típico do
século \versal{XIX} seja mera miragem de um certo momento passageiro do
desdobrar"-se da forma"-sujeito ainda entrecruzada com a subjetividade do
passado\footnote{Nesse sentido, seguindo o raciocínio desenvolvido aqui,
  caberia muito mais falar em desdobramento e realização plena da
  sociedade burguesa do que de sua decadência, como faz, por exemplo,
  Hobsbawn: ``Mas, à medida que crescia a maré do fascismo com a Grande
  Depressão, tornava"-se cada vez mais claro que a na Era da Catástrofe
  não apenas a paz, a estabilidade social e a economia, como também as
  instituições políticas e os valores intelectuais da sociedade liberal
  burguesa do século \versal{XIX} entram em decadência ou colapso.'' (1995, p.
  112). Segue esse mesmo raciocínio autores como Dany"-Robert Dufour em
  seu \emph{A arte de reduzir as cabeças} (2005). E até mesmo
  Adorno em alguns pontos da \emph{Mimima Moralia} (1992).}.

Além disso, como tentaremos desenvolver adiante, a objetivação da
subjetividade moderna é bem diferente da que se processou no início do
percurso do Esclarecimento até desembocar na modernidade. A constituição
da subjetivação moderna é um verdadeiro movimento de apagamento da
\emph{individualidade} em proveito de uma \emph{Individualidade}
abstrata, identificado pelo nome de individualismo moderno ou burguês.
Isso é decisivo distinguir. O problema não é só a objetivação da
subjetividade, mas o fundamento dessa objetivação que ganha força com a
modernidade, depois de passar pelos tensionamentos que vimos surgir
desde o desenvolvimento das cidades medievais onde o mercador e seu
caráter se opunham de certo modo ao caráter daquela sociedade -- não que
esse caráter fosse necessariamente bom. O fato é que a forma como
aqueles mercadores eram vistos, como amantes do dinheiro e despregados
de laço comunitário, é um dos aspectos da forma"-sujeito, do Indivíduo
moderno, do que chamamos com Marx de \emph{máscara de caráter.}

Na aurora do século \versal{XX}, que é o quadro social de \emph{O homem sem
qualidades}, a tensão entre mundo nobre e mundo burguês pende para o
lado burguês. Mas não como o burguês acabado, mas o burguês que ainda
carrega consigo marcas de certas \emph{qualidades} concretas que o tempo
passado lhe deixou -- aquele burguês do século \versal{XIX}. Esse sujeito burguês
em si só vai se desdobrando aos poucos, em sínteses contínuas advindas
das tensões entre formas mais ``atrasadas'' e formas mais ``avançadas''.
Esse sujeito burguês é idealmente o caracterizado pela escritora alemã
Roswita Scholz (1996) como o Macho"-Branco"-Ocidental. Dito de outro modo,
a forma"-sujeito moderna"-burguesa é uma abstração, é uma máscara que os
indivíduos concretos precisam vestir, bem ou mal. Ser macho e branco não
significa que mulheres, negros ou homossexuais não devam vestir tal
máscara, pois ela é um conjunto de posturas que são exigidas do
indivíduo concreto e sensível para que a sociedade mediada pelo dinheiro
siga adiante.

Essa máscara que a forma"-social mercantil pretende aferrar em nosso
rosto, por ser uma abstração de sujeito, não é implantada
automaticamente, nem tampouco sem resistências conscientes e
inconscientes dos indivíduos. Assim como a ideia abstrata de produção
linear e incessante de ``riqueza'' encontra limites concretos na
realidade, como vemos de modo cristalino na contemporaneidade, a ideia
de um sujeito afinado com a lógica dessa produção e consumo lineares e
perpétuos encontra limites na subjetividade real dos indivíduos, que não
necessariamente vivem toda sua vida dentro das demandas mercantis. Por
mais esforço de uniformização, os indivíduos ainda se constituem
mantendo recônditos -- cada vez menores, é verdade -- não
necessariamente colonizados na sua totalidade pela vida mercantil. E é
essa concretude, esse não"-idêntico e sua mínima diversificação que
historicamente vem criando certas resistências a que a forma"-sujeito
simplesmente se imponha de uma vez por todas. Apesar dessas
resistências, a experiência vem mostrando que a máscara de caráter vem
ganhando terreno sobre os rostos dos indivíduos concretos,
enrijecendo"-lhes o \emph{eu.}

Entendemos por forma"-sujeito, portanto, uma forma de subjetividade
abstrata, ideal, que pretende se hispostasiar nas individualidades
concretas. Dito simplesmente, é a máscara de caráter que impõe aos
indivíduos agirem e modelarem seu cotidiano em nome da vida mercantil. O
que significa agir para a valorização do dinheiro e pelo gozo mercantil.
Trata"-se, portanto, de uma forma de subjetivação que rompe o princípio
básico da não"-identidade entre indivíduo e sociedade. Assim, a
forma"-sujeito pretende colonizar a individualidade, o que significa um
pretenso apaziguamento social. Não no sentido de que não haja
enfrentamentos sociais violentos. Mas no sentido de que eles se deem
cada vez mais dentro do invólucro mesmo da forma"-sujeito, sem arranhar a
forma social da forma"-sujeito, logo, trata"-se de tensionamentos
imanentes aos limites da própria segunda natureza. Em suma, estamos
falando de uma forma de subjetivação que encara a vida social mediada
pelas leis mercantis como naturais e defensáveis, de um sujeito cuja
vida está fundada na dinâmica social moderna. Seus atos cotidianos se
pautam pela busca de espaço nesse mundo, encarado como o mundo que,
embora defectível, pode ser melhorado. Evidentemente, a realização plena
dessa forma"-sujeito é a utopia capitalista, ainda não realizada. O que
não significa que não haja individualidades concretas com rostos mais
corroídos do que outros por essa máscara de caráter social. Nos termos
de Menegat, ``os átomos isolados da segunda natureza são também
governados desde a sua interioridade. A sua composição orgânica comporta
a presença crescente de um momento da estrutura que é impulsionado pela
intencionalidade do todo.'' (2003, p. 65).

Nesse sentido, argumentar sobre a existência de uma forma"-sujeito
significa dizer que o critério das classes para se refletir sobre a
subjetividade na modernidade perde em profundidade. Ora, nossa concepção
da forma"-sujeito burguesa vai muito além do conceito de classe burguesa.
Trata"-se de um fenômeno que subsume tendencialmente as características
específicas das próprias classes sociais. Em outros termos, quanto mais
a forma social mercantil se desdobra, quanto mais a forma"-sujeito se
desdobra, as diferenças subjetivas entre as classes sociais tendem a uma
unificação, independentemente de sua posição perante a riqueza
mercantil. Esse aspecto já notara Goldmann nos anos 60. E é a mercadoria
cada vez mais entronizada na vida social quem opera essa sutil
equivalência. Distinguir classe burguesa e classe proletária, ou mesmo
classe dos despossuídos, pobres, já não fornece elementos profundos para
a compreensão de uma subjetividade moderna. Se essa distinção ainda pode
servir para estudos sociológicos, do ponto de vista de uma psicologia
social crítica ela fica encabulada com o desdobrar"-se da forma"-social
mercantil e sua forma"-sujeito de \emph{eu} enrijecido.

Esse enrijecimento do \emph{eu} em termos modernos, após séculos de
maturação nos mercadores e usurários, encontrou sua primeira feição
inacabada no ascetismo protestante, que delimitou as características da
máscara de caráter da forma"-sujeito no advento da sociedade moderna. Não
que autores renascentistas não tenham contribuído para isso, mas a
fundamentação da ética protestante o faz de forma mais explícita. Weber,
embora não pretenda delimitar um conceito exclusivo do que chama de
\emph{espírito do capitalismo}, captou esse espírito num determinado
momento de seu desenvolvimento. Como ele não problematiza a dinâmica
interna a esse espírito, crendo"-o já maduro em sua época, não parece
suspeitar que ele pode evoluir em outras direções sem perder sua
essência central.

O espírito do capitalismo representado pelo protestantismo em sentido
lato, e Weber não se apoia apenas em Calvino e Lutero, está ligado à
``ideia do dever que tem o indivíduo de se interessar pelo aumento de
suas posses como um fim em si mesmo.'' (\versal{WEBER}, 1905-2013, p. 45).

Analisemos mais algumas características desse sujeito da ética
protestante e do espírito do capitalismo encarnado nessa forma"-sujeito.

Se, para uma visão religiosa tradicional, a busca do acúmulo, da
riqueza, pode constituir um perigo grave, algo insensato, apesar de uma
tentação incessante, para o protestantismo notadamente calvinista, a
riqueza não constituía qualquer obstáculo à eficácia dos pastores da
igreja, antes de tudo poderia permitir um cuidado maior para fazer
frutificar sua fortuna, ``com a única condição de evitarem o
escândalo.'' (\versal{WEBER}, 1905-2013, p. 143).

O espírito do capitalismo se materializa, portanto, na ética que prega:

\begin{quote}
ganhar dinheiro, e sempre mais dinheiro, no mais rigoroso resguardo de
todo gozo imediato do dinheiro ganho, algo tão completamente despido de
todos os pontos de vista eudemonistas ou mesmo hedonistas e pensado tão
exclusivamente como fim em si mesmo, que, em comparação com a
``felicidade'' do indivíduo ou sua ``utilidade'', aparece em todo caso
como inteiramente transcendente e simplesmente irracional. O ser humano
em função do ganho como finalidade da vida, não mais o ganho em função
do ser humano como meio destinado a satisfazer suas necessidades
materiais. Essa inversão da ordem, por assim dizer, ``natural'' das
coisas, totalmente sem sentido para a sensibilidade ingênua, é tão
manifestamente e sem reservas um \emph{Leitmotiv} do capitalismo, quanto
é estranha a quem não foi tocado por seu bafo. (\emph{Idem}, p. 46-47).
\end{quote}

Podemos notar que a objetivação da natureza interna aqui já não tem
apenas o objetivo de livrar os homens do medo da natureza, ela tem um
objetivo que vai além disso: moldar um caráter adaptado a essa forma
social em que o dinheiro tornou"-se o princípio e o fim de um processo
irrefletido. Do mesmo modo, essa ética só pode causar estranheza para
quem não tenha sido ``tocado por seu bafo''. Uma estranheza que vai
precisar ser superada no seio social para que esse espírito subjetivo
ideal se entranhe nas individualidades concretas.

Para essa ética, ou para esse espírito, a importância dada à profissão
sob o nome de vocação é algo que se relaciona com a glorificação do
trabalho moderno, cuja característica principal é medida pelo tempo.
Vejamos o trecho em que Weber analisa o puritano inglês Richard Baxter:
``A \emph{perda de tempo} é, assim o primeiro e em princípio, o mais
grave de todos os pecados. [\ldots{}] Perder tempo com
sociabilidade, com `conversa mole', com luxo, mesmo com o sono além do
necessário à saúde -- seis no máximo oito horas -- é absolutamente
condenável em termos morais.'' (\emph{Idem}, p. 143).

Com Weber, determinamos, em certa medida, qual espírito devia encarnar o
sujeito moderno, ou seja, qual máscara de caráter social ele devia
vestir, mas somente naquela época histórica. Somos, assim, levados pela
nossa análise a ver no sujeito desse protestantismo ainda um amálgama
subjetivo em que o ascetismo -- fundamental para fundar a modernidade --
parece um resquício subjetivo da vida social passada usado mais como
instrumento para o acúmulo moderno\footnote{Como veremos em capítulo
  posterior, esse ascetismo não pode combinar com uma sociedade que
  sempre precisa ir além de si mesma.} -- sendo uma das faces de uma
certa \emph{acumulação primitiva da subjetividade moderna.}

Para Rodrigo Duarte, o domínio da natureza interna é algo bastante
antigo na história: ``Platão e Aristóteles [\ldots{}] compreendem o
domínio da natureza antes de tudo, como domínio da natureza interna do
humano, já que aos gregos faltam os pressupostos teóricos de uma
concepção de domínio do mundo físico.'' (\versal{DUARTE}, 1993, p. 19). Apesar de
essa ideia de dominação da natureza interna ser algo que remonta ao
início da objetivação da natureza externa, como já está claro em nosso
intuito, privilegiamos a especificidade moderna inclusive no tocante à
objetivação da subjetividade, pois, também aí, parece haver uma
especificidade que já chamamos de \emph{máscara de caráter} (\versal{MARX}, 1985)
do sujeito moderno. Isto é, não se trata apenas do fato de que é
condição da civilização ``um processo violento de negação dos impulsos,
isto é, a abdicação pelos sujeitos da sua vitalidade mais originária''
(\versal{GAGNEBIN}, 1993, p. 71). Essa preocupação com a negação dos impulsos não
é nova historicamente. É preciso determinar que na modernidade esse
enrijecimento do \emph{eu} se deu para que se erigisse uma forma"-sujeito
com determinados comportamentos adequados à guerra concorrencial no
mercado que parece ser algo típico de uma forma social específica na
história. Esse enrijecimento não se dá simplesmente como um
\emph{mal"-estar na civilização}, com uma repressão das pulsões em prol
da socialização. Não estamos discutindo qualquer forma social na
história, mas uma forma de vida social cujo palco fundamental da vida é
aquele em que se desenrola a concorrência onde está em jogo a própria
vida.

Por isso, para nós, interessa discutir o domínio da natureza interna
numa socialização onde reina a liberdade de concorrência entre os
sujeitos livres, entre os quais tende apenas a haver laços abstratos
mediados por uma \emph{mão invisível}, portanto, interessa a dominação
da natureza interna como \emph{máscara de caráter} (\versal{MARX}, 1985) de uma
forma"-sujeito. E podemos afirmar que essa forma"-sujeito é característica
da modernidade, uma vez que ela não é uma simples exigência moral aos
indivíduos, não é como a negação explícita da individualidade como
acontecia no quadro medieval em proveito de um ser transcendente. O
indivíduo no mundo medieval está imerso num conjunto de relações que
fazem com que sua individualidade seja socialmente negada. Claramente,
sob a fetichismo religioso, o homem medieval pertence à comunidade, não
existe enquanto um ser individual -- o que não significa que não tenha
individualidade concreta em relação ao social, ela apenas não é digna de
nota, não é algo a ser reivindicado, refletido, como na modernidade,
quando é objeto de estudo teórico. Não à toa, o orgulho, a vaidade, a
\emph{superbia}, é colocada como o primeiro dos sete pecados capitais
por Tomás de Aquino na questão 84 de sua \emph{Summa} \emph{Theologica
--} como anterior à própria avareza, \emph{avaritia}. Portanto,
pretender ser um indivíduo era confundido com a soberba, que traz
consigo, segundo Tomás de Aquino, um ``certo desprezo de Deus''. (1855,
p. 7)\footnote{Como destaca Tomás de Aquino: ``A apostasia relativa a
  Deus forma o começo do orgulho, em razão do fato de que o homem se
  distancia de seu autor. Tão logo o homem deixa de querer se submeter a
  Deus, ele busca de forma desregrada sua própria proeminência nas
  coisas temporais; o orgulho, portanto, [\ldots{}] não é um pecado
  especial, mas a condição geral de todo pecado, ou seja, é o
  distanciamento de todo bem imutável. [\ldots{}] Ou se pode dizer
  que a apostasia é o começo do orgulho, porque ela é a primeira espécie
  de orgulho. É próprio do orgulho não querer submeter"-se ao jugo do
  superior, principalmente não querer submeter"-se ao jugo a Deus; (1855,
  p. 9).}. Le Goff ressalta que durante ``muito tempo o indivíduo
medieval não existe na sua singularidade física. Nem na literatura nem
na arte os personagens são descritos ou pintados com suas
particularidades. Cada um se reduz ao tipo físico de seu rol, à sua
categoria social.'' (2008, p. 258).

Sendo assim, ``O homem medieval não tem nenhum senso de liberdade
segundo a concepção moderna'' e sua liberdade é um estatuto garantido
segundo sua inserção na sociedade, ou seja, ``Não existe liberdade sem
comunidade''. E essa liberdade só pode consistir na dependência, em que
o superior garante ``ao subordinado o respeito de seus direitos. O homem
livre é aquele que tem um protetor poderoso.'' (\emph{Idem}, p. 259).

Apesar disso, a individualidade mais concreta e cotidiana se mantinha
nos atos sociais sem que necessariamente fosse vista como tal. O fato de
o universo religioso exigir a menoridade do indivíduo não significa a
criação de uma subjetividade formal, como uma máscara de caráter que
esses indivíduos precisam vestir, como no caso moderno. A própria
diferença em termos de classe social significava uma individualidade, o
que fazia com que um camponês fosse tão diferente de cavaleiro. Nesse
sentido, ``apenas se pode falar em máscara de caráter quando as relações
humanas são determinadas por um princípio autonomizado'' (\versal{KURZ}, 2014, p.
72), um princípio que exige das individualidades concretas uma forma de
agir unitária, em consonância com o movimento da própria sociedade, dito
de outro modo, uma identificação entre indivíduo e sociedade. Isso é
apenas moderno.

Deste modo, a hipótese sobre a qual refletimos aqui é que a
forma"-sujeito tende a consumir o indivíduo. E parece ir nessa direção a
reflexão de Arendt quando ela afirma existir uma ``importante diferença
entre os primeiros estágios da sociedade e da sociedade de massas com
respeito à situação do indivíduo'' (2009, p. 252). Mas que diferença é
essa? Diz respeito ao fato de que algo foi corroendo as tramas de
autonomia (relativa, posto que socialmente mediada) do indivíduo em
proveito de uma pretensa identidade entre o particular e o universal,
entre o indivíduo e o social. Este, por mais que medeie a constituição
daquele, não pode pretender que ele lhe seja idêntico, sob pena de a
própria sociedade transformar"-se puramente numa natureza -- ou seja, em
não"-sociedade. Para Arendt:

\begin{quote}
Enquanto a sociedade propriamente dita se restringia a determinadas
classes da população, as probabilidades de que o indivíduo subsistisse
as suas pressões eram bem grandes; elas se baseavam na presença
simultânea, dentro da população, de outros estratos além da sociedade
para os quais o indivíduo poderia escapar e um dos motivos pelos quais
tais indivíduos tão amiúde aderiram a partidos revolucionários era que
descobriam, nos que não eram admitidos à sociedade, certos traços de
humanidade que se haviam extinguido na sociedade. (2009, p. 252).
\end{quote}

E se era possível encontrar esses traços de humanidade nos inadmitidos
da sociedade, era porque a diferença entre o indivíduo e o Indivíduo era
mais sensível. E essa distinção é fundamental: para nós o
\emph{indivíduo} se diferencia do \emph{Indivíduo}, a forma"-sujeito
moderno. Este é, no dizer do antropólogo Louis Dumont, ``o ser moral,
independente, autônomo e assim (essencialmente) não social, tal como se
encontra, antes de tudo, na nossa ideologia moderna do homem e da
sociedade.'' (\versal{DUMONT}, 2000, p. 20), que se distingue daquele, que é ``o
sujeito empírico da palavra, do pensamento, da vontade, mostra
representativa da espécie humana, tal como é encontrado em todas as
sociedades;'' (\emph{Idem,} 2000, p. 20). Seguindo Dumont, portanto, o
individualismo como ideologia diz respeito ao \emph{Indivíduo} e não ao
\emph{indivíduo}. Este normalmente se constitui muito mais diversamente.

Para Robert Kurz (2010b), o Esclarecimento moderno denuncia com espanto
as sociedades de fetiche pré-modernas como ainda inclinadas à natureza,
como uma sociedade com estrutura de rebanho onde não teria surgido
qualquer individualidade. Para esse autor, cuja crítica neste tocante
compartilhamos, essa caricatura pretende desviar"-se do fato de que a
modernidade também tem uma estrutura fetichista que impõe aos indivíduos
``uma forma unitária: o `uniforme' do sujeito do trabalho, do dinheiro e
da concorrência'' (\emph{Idem}, p. 85) e acrescentaríamos, o
``uniforme'' do sujeito do consumo.

Segundo esse autor, corroborando Dumont, todas as sociedades históricas
contaram com uma individualidade, uma vez que com a primeira objetivação
da natureza estabelece"-se uma relação do ser humano particular com a
forma social, uma relação que coincide com o próprio tornar"-se humano.
Portanto, havia uma individualidade, por mais que ela se manifestasse de
forma distinta dependendo das formações sociais. Mas sempre havia uma
margem de manobra para esse indivíduo e sua tensão com a sociedade
aparecia.

\begin{quote}
O próprio conceito de indivíduo origina"-se com efeito na Antiguidade
[\ldots{}] do mesmo modo, nas civilizações agrárias da assim chamada
Idade Média, o conceito de criatura humana individual
(\emph{individuitas}) mostra"-se de maneira variegada. [\ldots{}] O que a
ideologia do Esclarecimento faz valer como conceito único de indivíduo
[\ldots{}] é sem dúvida o ``eu'' abstrato, isto é, a forma
especificamente moderna da individualidade abstrata. [\ldots{}] Neste
sentido, ``indivíduo'' significa, já de si, a forma na qual os seres
humanos particulares são pensados como imediatamente idênticos à relação
social coercitiva (\emph{Ibidem}, p. 86).
\end{quote}

Para Kurz, a forma"-sujeito seria a forma que a moderna sociedade
mercantil impõe aos indivíduos. Já a individualidade advém da tensão
entre seres humanos reais, individuais e sensíveis e a forma social,
portanto, a individualidade estaria ligada à ```lacuna' penosamente
vivida, à retenção das necessidades e sensações no interior de tal
invólucro coercitivo.'' (\emph{Ibidem}, p. 89).

Desse modo, o estatuto de transhistórico não caberia ao sujeito, mas ao
conceito de indivíduo. Não como uma essência que permanece imutável,
essência ontológica, mas sempre na dialética com o social, uma vez que a
individualidade nunca existe para si, mas sempre em relação a uma forma
social. Fica claro, assim, que quando falamos de
\emph{dessubstancialização do indivíduo} não pretendemos defender uma
essência ontológica do indivíduo que estaria se perdendo. A
dessubstancialização do indivíduo se dá quando em processo: quanto mais
a forma"-sujeito consome os espaços de não"-identidade que o indivíduo
mantém em relação ao social, mais esse indivíduo se identifica com essa
máscara de caráter. E a realização dessa forma"-sujeito significa a
concretização do vazio no próprio social, uma vez que essa forma"-sujeito
é a forma subjetiva correspondente da forma"-mercadoria. A realização
plena da forma"-mercadoria na vida social coincide com sua
dessubstancialização, porque essa realização só é possível pelo
aniquilamento da própria vida social concreta em sentido lato. Do mesmo
modo, a realização do \emph{Indivíduo} ou da forma"-sujeito coincide com
sua dessubstancialização, porque essa realização só é possível pelo
aniquilamento dos recantos de individualidade, dos recantos de
não"-identidade.

Essa nossa argumentação no sentido de distinguir a forma"-sujeito, o
\emph{Indivíduo} moderno, e a individualidade concreta não deve nos
levar a imaginar uma fronteira clara que os separe. Assim como o
operário não desenvolve o trabalho concreto de manhã e o abstrato à
tarde, nem produz numa hora o valor de uso e noutra o valor de troca, ou
seja, assim como o bem útil não tem outra materialidade distinta da
mercadoria, não se pode superficialmente delimitar no indivíduo os
momentos em que ele é um indivíduo concreto com posturas não"-coadunadas
com a vida social mercantil, e os momentos em que ele veste a máscara de
caráter da forma"-sujeito. Não há como haver uma separação clara na
subjetividade. São formas que sempre se entrecruzam no mesmo indivíduo e
que avançam de modo dialético dentro desse mesmo indivíduo. Seríamos
tentados a quase ver na tensão entre indivíduo concreto e forma"-sujeito
burguesa uma forma esquizofrênica. Evidentemente, não no sentido como
veem Deleuze e Guatari a relação entre \emph{capitalismo e
esquizofrenia.} Eles colocam a esquizofrenia, não no sentido comumente
entendido como patológico, como um limite ao capitalismo. Segundo esses
autores, ``o capitalismo não tem e, ao mesmo tempo, tem um limite
exterior: a esquizofrenia'', que é entendida como ``a descodificação
absoluta dos fluxos'', que o capitalismo precisa ``repelir e esconjurar
para poder funcionar.'' (\versal{DELEUZE} \& \versal{GUATARI}, 2004, p. 261). E esses
fluxos descodificados ``precipitam"-se numa produção desejante''
(\emph{Idem,} p. 143) em que o esquizofrênico ``mistura todos os
códigos, e traz em si os fluxos descodificados do desejo.''
(\emph{Idem}, p. 39). Os autores pretendem superar a dicotomia entre as
classes no capitalismo, mas não no sentido de analisar criticamente a
tendência da dinâmica moderna à subsunção das classes pela forma"-sujeito
mercantil. Pelo contrário, para eles a verdadeira oposição teórica não é
entre as classes, mas entre:

\begin{quote}
os fluxos descodificados, tal como entram numa axiomática de classe
sobre o corpo pleno do capital, e os fluxos descodificados que se
libertam não só desta axiomática como também do significante despótico,
que atravessam o muro \emph{e} o muro do muro, e se põem a correr sobre
o corpo pleno sem órgãos; a oposição está entre a classe e os
fora"-da"-classe, entre os servos da máquina e os que a fazem saltar ou
estoiram com as suas peças, entre o regime da máquina social e o das
máquinas desejantes, entre os limites interiores relativos e o limite
exterior absoluto. Ou se quiser, entre os capitalistas e os esquizos, na
sua intimidade fundamental ao nível da descodificação e \emph{na sua
hostilidade fundamental ao nível da axiomática} (donde a semelhança que
existe entre o proletariado, tal como é retratado pelos socialistas do
século \versal{XIX}, e um perfeito esquizo.). [Grifos Nossos] (\emph{Idem},
2004, p. 266).
\end{quote}

Mesmo argumentando que ``a esquizofrenia não é a identidade do
capitalismo mas, pelo contrário, a sua diferença, o seu desvio e a sua
morte.'' (\emph{Idem}, p. 256), ou seja, embora coloquem a esquizofrenia
como o não"-idêntico, uma vez que fugiria à triangulação freudiana, os
autores reconhecem estranhamente que não há ``nenhuma cadeia molecular
que não intercepte e reproduza blocos inteiros de código ou de
axiomática molares, nem blocos desses que não contenham ou não encerrem
fragmentos da cadeia molecular.'' (\emph{Idem}, p. 356).

Para nós, não há como apontar uma forma subjetiva positiva oriunda da
vida social capitalista -- gestada pelo próprio capitalismo e escapando
ao seu controle. As \emph{máquinas desejantes} e o \emph{corpo sem
órgãos}, com seus fluxos múltiplos, qualificados, descodificados, como
veremos à frente, parecem muito mais com a forma"-sujeito burguesa em seu
estágio contemporâneo de desdobramento, ou seja, muito mais próxima de
uma subjetividade mercantil do que de uma alternativa a ela.

A análise de Adorno em \emph{Minima Moralia} parece ir mais na direção
do que estamos chamando aqui de certa esquizofrenia na relação entre
indivíduo concreto e forma"-sujeito burguesa. Adorno fala de uma
``composição orgânica do ser humano'', em comparação com a composição
orgânica do capital. Adorno aqui ainda está preso a uma concepção da
subjetividade como ainda atravessada principalmente pela produção, mas o
que interessa é o termo sugestivo: ``Aquilo através do qual os sujeitos
são neles mesmos determinados como meios de produção e não como
finalidades vivas cresce como a proporção das máquinas em relação ao
capital variável.''(1992, p. 201). E ele sublinha o caráter
esquizofrênico gestado por esse processo em que o indivíduo, na
conservação de si, perde seu si; em que suas qualidades ``sobrevivem
apenas como envoltórios ligeiros, rígidos e vazios dos impulsos
[\ldots{}] sem qualquer traço pessoal.'' (\emph{Idem}, p. 202).
Nessa ``patogênese social da esquizofrenia'', há um processo de
separação entre as características do indivíduo e sua base pulsional, e
depois entre estas ``e a ipseidade que as comanda lá onde antes apenas
as mantinha juntas'', o que leva o homem a uma crescente desintegração
interna.

Mas se é possível falar em certa esquizofrenia na relação entre
indivíduo concreto e forma"-sujeito é somente de modo parcial. E ainda
não no sentido comumente entendido como patológico, embora não deixe de
ser uma dissociação das funções psíquicas, em que o indivíduo concreto
precisa acompanhar a todo custo a dinâmica da forma"-sujeito, e não deixe
de causar transtornos no indivíduo concreto no qual essa dissociação
psíquica se dá. Mas se usamos o termo esquizofrenia para falar dessa
relação tensa é sobretudo no sentido de haver uma espécie realmente de
dissociação, de clivagem do espírito (\versal{LAPLANCHE} \& \versal{PONTALIS}, p. 158,
2010). Todavia, essa própria forma esquizofrênica só se mantém enquanto
as paralelas representadas pela individualidade concreta e as exigências
do social, entendidas como a forma"-sujeito, não se aproximam ou se
entrecruzam na mesma direção. Dito de outro modo, o processo de
desdobrar"-se da forma"-sujeito é a superação paulatina também dessa
dissociação do espírito ou, pelo menos, a sua suavização, no sentido de
um amainar das tensões, as distinções possíveis.

Mas, para Kurz, a oposição que a modernidade esclarecida costuma fazer
entre o sujeito objetivante e o objeto ainda não abarcaria a
especificidade moderna. Isto porque o sujeito se colocaria em oposição
ao objeto apenas enquanto representa a forma de atividade imposta pela
objetividade, a um só tempo consciente e inconsciente, do processo
moderno de objetivação do mundo representado pela lógica mercantil.
Significa dizer que o sujeito não é sujeito de fato, porque há uma
lacuna de inconsciência social entre ele e o objeto. Há uma espécie de
aura que age entre ele e o objeto e que torna o objeto autodeterminado.
Isso também explica porque a máxima materialista precisaria ser
revisitada.

Ou seja, a Razão mercantil embutida na Razão Instrumental instala"-se
como um \emph{a priori} tácito (\versal{KURZ}, 2008) nas relações sociais. Ora,
as mercadorias não vão sozinhas ao mercado, nem tampouco saem sozinhas
das prateleiras e vitrines. Tampouco a obsessão pelo crescimento
econômico age por si só. Há sujeitos que agem. Por mais que as relações
sejam dadas de antemão, é preciso um agente para levar adiante a
sociedade, e quem leva adiante a lógica mercantil cotidianamente é a
forma"-sujeito. Ao mesmo tempo que leva adiante essa lógica, sua
subjetivação adaptada ao caráter exigido pelo social tensiona com
momentos não"-mercantis da vida, que são cada vez menores em função da
aceleração da marcha dos momentos mercantis sobre os terrenos até então
tidos por não"-mercantis. Mas não se pode esperar outra coisa de uma
sociedade sem limites, uma sociedade que pretende passar cada momento da
vida pelo buraco da agulha da vida mercantil.

Dito de outro modo, a relação entre sujeito e objeto deveria ser
colocada numa relação em que o sujeito é ao mesmo tempo sujeito"-objeto.
Ou seja, o sujeito é ``um ser que já se encontra cegamente numa forma
determinada e que, enquanto tal, não é refletivo [\ldots{}] e, por outro
lado, é um portador consciente de ações no interior dessa mesma forma e
que se vê obrigado a executar suas `leis'.'' (\versal{KURZ}, 2010b, p. 91-92).
Assim, o sujeito age dentro de uma forma social em que ele é forjado
pelo objeto, mas um objeto especial, como já vimos. Em certa medida,
Marx já tinha chamado a atenção para essa questão ao afirmar que ``A
produção não produz apenas um objeto para o sujeito, mas um sujeito para
o objeto'' (\versal{MARX}, 1972, p. 157). Esse sujeito produzido para o objeto é
a forma"-sujeito moderna que tensiona com os indivíduos concretos. E
Adorno se dava conta desse processo:

\begin{quote}
A ordem econômica e, seguindo seu modelo, em grande parte também a
organização econômica, continua obrigando a maioria das pessoas a
depender de situações dadas em relação às quais são impotentes, bem como
a se manter numa situação de não"-emancipação. Se as pessoas querem
viver, nada lhes resta senão se adaptar à situação existente, se
conformar (\versal{ADORNO}, 1995, p. 43).
\end{quote}

Esse con-\emph{fôrmar}-se deve ser entendido ao pé da letra. Não como
realizado, mas como um processo em que paulatinamente a forma social
pretende pôr a individualidade concreta em \emph{fôrmas} abstratas.
Podemos entender a partir dessa análise que entre a forma"-sujeito e a
sociedade não há tensão, mas tendência à identidade, à mímese, portanto,
à pretensa identificação entre primeira e segunda natureza: fim da
história.

\begin{quote}
A dissolução das mediações, que acontecem no terreno da história,
representa também o abandono da memória e dos laços com o passado. Elas
tornam o que veio a ser e o que foi reprimido, assim como o que ambos
eram antes, \emph{num doloroso processo de abandono das potencialidades
de ser outro} -- o ``ainda não'' --, ao fim do qual o que se forma ganha
a marca de um ser racional, sem a medida de outra razão do que esta que
se afirmou como a navalha que deixou para trás tudo o que não lhe era
adequado. É neste endurecimento do ser que a razão dialética desaparece,
[\ldots{}] (\versal{MENEGAT}, 2033, p. 67). [Grifos Nossos].
\end{quote}

As potencialidades de ser outro só podem advir da tensão com a
sociedade, algo que diria respeito ao indivíduo, que se constrói numa
trama mais ampla do que o sujeito, pois o indivíduo sentiria exatamente
o mal"-estar nascido da tensão entre o que \emph{é} bem ou mal
multifacetado e o que a forma social dele exige enquanto máscara de
caráter abstrata e universal que serve ao andamento irrefletido do todo.
Somente o indivíduo pode sentir o corte dessa navalha -- bem ou mal. A
forma"-sujeito moderna é o que vai tomando corpo após o corte. É o
indivíduo que pode viver o dinamismo da recusa do existente expressa
numa dialética negativa, cujos momentos são a contradição e a
resistência. Essa dialética negativa defendida por Adorno pressupõe uma
lógica da não"-identidade, ou seja, uma inadequação entre realidade e
conceito (\versal{MAAR}, 1994, p. 63). Essa inadequação entre realidade e
conceito está relacionada com o entrecruzamento subjetivo e as tensões
sobre as quais estamos refletindo. Uma inadequação que vai diminuindo
tendencialmente pela força do \emph{conceito} que vai se desdobrando na
\emph{realidade}. Uma realidade que vai, aos poucos, eliminando o
não"-idêntico, e fazendo com que os indivíduos concretos, cada vez mais,
reduzam sua existência à autoconservação dentro do invólucro da segunda
natureza: ``A redução da vida social ao movimento incessante da
autoconservação revela o caráter pré-histórico desta forma de
sociabilidade [\ldots{}]'' (\versal{MENEGAT}, 2003, p.66). O não"-idêntico, a
contradição, está naquilo que não é considerado pela troca, pelo valor,
pelo gozo mercantil, e que foge, assim, à forma"-sujeito. O conceito de
massa, tão caro ao século \versal{XX}, tanto à direita como à esquerda, vai ao
encontro dessa homogeneidade que joga para o canto o momento da
contradição.

Ademais, para Adorno, o que precisa se conservar na personalidade é ``a
força do indivíduo, o potencial para não confiar"-se ao que cegamente se
lhe impõe, para não identificar"-se cegamente com isso.'' (\versal{ADORNO}, 1995,
p. 68). Essa reflexão é fundamental para não cairmos numa conceituação
transhistórica de indivíduo e sociedade, como faz Norbert Elias na obra
\emph{A sociedade dos indivíduos} (1994). Ele resolve a querela entre
subjetivistas e objetivistas com uma formulação interessante, na qual
coloca indivíduo e sociedade como relacionados, imbricados, mas deixando
de ver a especificidade dessa relação como refletimos acima:

\begin{quote}
Considerados num nível mais profundo, tanto os indivíduos quanto a
sociedade conjuntamente formada por eles são igualmente desprovidos de
objetivo. Nenhum dos dois existe sem o outro. Antes de mais nada, na
verdade, eles simplesmente existem -- o indivíduo na companhia de
outros, a sociedade como uma sociedade de indivíduos -- de um modo tão
desprovido de objetivo quanto as estrelas que, juntas, formam um sistema
solar, ou os sistemas solares que formam a Via"-Láctea. (p. 18)
\end{quote}

Elias pretende com isso responder se ``a sociedade é o objetivo final e
o indivíduo o meio'' ou o contrário, se a sociedade é o meio para a
felicidade do indivíduo. Não deixa de ser interessante a quebra da
dicotomia estanque que ele opera, mas a forma não"-finalista como ele
coloca a relação entre indivíduo e sociedade deixa transparecer a não
problematização da especificidade de tal relação na modernidade, quando
de fato há uma finalidade, um \emph{telos,} envolvido nessa relação. De
modo que a questão de Elias de saber como é possível que a existência
das pessoas em sua vida comum crie ``algo que nenhum dos indivíduos,
considerado isoladamente, tencionou ou promoveu'' (p. 19) precisaria ser
problematizada de modo peculiar na sociedade moderna. Pois nesta forma
social, além do indivíduo e da sociedade -- com suas especificidades --
há ainda uma forma"-social e uma forma"-sujeito que envolvem essa relação,
que tensionam com essa relação que já é tensa. Elias, nessa obra, não
parece problematizar o que ele chama de ``leis sociais'' ou
``regularidades sociais'', as ``leis autônomas das relações entre as
pessoas individualmente consideradas'' (p. 23). Talvez porque considera
indivíduo e sociedade como entidades abstratas, simplesmente existentes
no decurso da história, e não como realmente existentes numa dada
sociedade.

É certo que sua refutação da ideia de um indivíduo como mônada fechada
em si, e da sociedade como uma concha que se deposita no indivíduo é
fundamental para uma visão dialética da vida social. Ele se opõe
corretamente, portanto, à ideia de uma individualidade singular,
extra"-social, numa sociedade vista como externa e alheia. Para ele, ``o
que é moldado pela sociedade também molda'' (p. 52). Portanto, a própria
individualidade é mediada socialmente: ``a pessoa que cresce fora do
convívio humano não atinge essa `individualidade', como acontece com os
animais.'' (p. 55). Não está, porém, em Elias, essa problematização da
forma"-sujeito moderna, que não é idêntica ao indivíduo concreto que ele
problematiza. Ou, simplesmente, o \emph{indivíduo} que ele problematiza
não é o \emph{Indivíduo} como se configurou na modernidade. Esse
indivíduo, mesmo socialmente constituído, não se constitui, a todo
momento, sob a base dos ditames da forma"-sujeito que o acossa
cotidianamente.

Diante dessa reflexão, parece claro que não podemos simplesmente
idealizar as individualidades para se opor à forma"-sujeito. As
individualidades não são simplesmente um núcleo de humanidade que ficou
intacto pela racionalidade do cálculo e da frieza. Um recanto onde
moraria a parte boa do ser humano que bastaria despertar de seu sono
forçado para se alcançar a emancipação. Essas individualidades são
também um construto social e, como argumentamos, formam um só corpo com
a forma"-sujeito. Não é por se formarem numa trama mais multifacetada do
que a forma"-sujeito exigida pela vida capitalista que não trazem consigo
as riquezas e as baixezas de que o ser humano pode dar provas em
qualquer sociedade. Inclusive, as individualidades podem simplesmente
ser usadas como um lado humanizado, um lado obscuro, que contrabalança a
frieza da forma"-sujeito, ou seja, as qualidades humanas podem ser usadas
simplesmente como ``lubrificante para o andamento macio da maquinaria''
(\versal{ADORNO}, 2003, p. 57). O que pretendemos ressaltar é o fato de que é
nessa individualidade -- até pela sua imprevisibilidade, plasticidade,
portanto, pelas possibilidades que encerra -- que pode nascer um núcleo
de fato impulsionador da saída do ser humano da segunda natureza, no
caminho de outra humanização.

Por conseguinte, a forma"-sujeito não constitui o agente de emancipação
exatamente porque ele é a forma de consciência que pretende agrilhoar o
indivíduo. Entender a si mesmo como sujeito significa vestir a máscara
subjetiva do agente da marcha mercantil. Assim, o núcleo impulsionador
da resistência, no dizer de Adorno (1995), é a condição de sobrevivência
do indivíduo. (p. 154). O que não significa que superar a forma"-sujeito
seja aniquilar o corpo subjetivo que o carrega. Assim como a superação
do trabalho não é a eliminação das atividades concretas e múltiplas dos
indivíduos, assim como a superação da forma"-mercadoria não é o
aniquilamento de todo objeto útil no qual ela se materialize, também a
superação da forma"-sujeito mercantil não é o aniquilamento das pessoas.

É preciso apostar na força do indivíduo que pode opor"-se à identificação
numa sociedade que impõe, por via de uma \emph{mão invisível}, a
concorrência, onde a derrota dos mais frágeis não significa o fracasso
como num jogo depois do qual pode haver uma revanche. A derrota na
corrida social típica do sujeito concorrencial pode significar a derrota
na conservação da própria vida.

Essa reflexão sobre a especificidade do sujeito na modernidade
capitalista nos ajuda a não entrever um rompimento entre o sujeito do
início do sistema produtor de mercadorias e o sujeito contemporâneo que
aparentemente teria se afastado desse sujeito de eu enrijecido para se
constituir como eu multifacetado, flexível, \emph{descolado} e avesso a
um comportamento ascético. Como refletiremos mais adiante, o sujeito
contemporâneo é o desdobrar das tensões entre formas diversas de
individualidade arraigadas em outras bases subjetivas e a forma"-sujeito
moderna que vai aos poucos vencendo a batalha. O pretenso eu
multifacetado e hedonista é a forma mais contemporânea do desdobramento
da forma"-sujeito moderna. Se ele é menos reflexivo do que o sujeito
burguês clássico do século \versal{XIX} ou começo do \versal{XX} não é porque ele é um
novo sujeito, mas porque aquele que era tido por sujeito burguês
clássico ainda era um sujeito que tinha muitas camadas de subjetivação
não"-coadunada com a dinâmica sem limites da forma moderna.

E na obra \emph{O homem sem qualidades,} Musil pinta um amplo painel
desses tensionamentos. Desde personagens em sintonia com o tempo que
passa a dominar, como personagens apegados a um passado em trapos, como
também personagens de subjetividade à deriva.

\section{O \emph{homem sem qualidades} e o \emph{amorfismo humano} como ponto de fuga sem saída}

\bigskip

\begin{flushright}
\scriptsize{Acabei de me encontrar um belíssimo nome: senhor\\
dissecador [\ldots{}] Senhor dissecador: sou eu! Minha vida:\\
as aventuras e a odisseia de um dissecador da alma\\
no início do século \versal{XX}!\\

\emph{Musil}, \emph{Folhetos do noturnal do} monsieur vivisecteur.}
\end{flushright}

Musil não parece ele próprio um autor com qualidades definidas. Sua obra
\emph{O homem sem qualidades} parece mais um canteiro de obras de
caracteres vastos. Suas ideias ao longo de seus cadernos de diários,
ensaios e romances vão nesse sentido exploratório das possibilidades,
não são uma doutrina filosófica. Como diz Bouveresse (2001, P. 14),
Musil não tem uma filosofia, o que significaria o encerramento de uma
multiplicidade de possibilidades num sistema acabado. E a postura de
Musil é a de ver aquele momento histórico como possibilidade: ``desde
minha mais tenra idade, sempre pensei a estética como uma ética''.
(1981b, p. 291). Ou o terreno da estética como palco de experimentações
no terreno da ética. Para Cometti (2001), esse é o verdadeiro nervo
central da obra de Musil, de modo que esclarece a dimensão filosófica de
seus escritos (p. 79). É um conjunto reflexivo nem sempre fácil de
abordar: ``\emph{O homem sem qualidades} é construído sobre uma
experiência de pensamento.'' (\emph{Idem}, p. 64). Ou, como diz David
Dawlianidse (1981, p. 202), no romance, a ação romanesca é absorvida por
uma multidão de reflexões. A própria reflexão dos personagens é a ação
romanesca -- já que aquilo que serve em aparência de impulso narrativo,
a saber, a Ação Paralela, roda em falso. E, nessa experiência, um dos
aspectos mais interessantes, complexos, e ambíguos nos escritos de
Robert Musil, e que ecoa fortemente em sua obra principal, é sua
\emph{teoria do amorfismo humano,} que está mais formulada num ensaio
intitulado \emph{O alemão como sintoma}, mas que perpassa muito da sua
obra. Por essa teoria, o ser humano não teria caracteres próprios,
ontológicos, mas seria fruto do social:

\begin{quote}
Se tentamos abstrair de nós mesmos o que não é senão convenção inerente
à época, resta algo de realmente amorfo; porque mesmo o que temos de
mais pessoal se liga, sob forma de desvio, ao sistema do mundo que nos
rodeia. \emph{O homem só existe nas formas que lhe são fornecidas pelo
exterior.} [\ldots{}]É a organização social que dá ao indivíduo sua forma
de expressão, e não é senão através dessa expressão que nos tornamos
homem. [\ldots{}] Pode"-se assim medir a gravidade do erro desses
bons espíritos que creem mais urgente hoje mudar o homem do que suas
formas de organização. (1984, p. 349) [Grifos nossos]
\end{quote}

Mas Musil alerta, apesar de toda a clareza do trecho, que não se deve
interpretar essa ideia como ``uma teoria exclusiva do meio''.
(\emph{Idem}, p. 352). Porque no fundo se trata uma combinação
alternativa ``absolutamente inextrincável de ações e reações entre o
homem e seu redor, o essencial e o contingente, o impulso e a amarra, o
eu e sua expressão, que produziram diferentes fisionomias
características de cada época.'' (\emph{Idem}, p. 353). Dessa observação
que reconhece uma forte dialética na construção da individualidade,
Musil volta novamente na frase seguinte a falar da informidade da
natureza do homem ``que o obriga a se agarrar com formas, a adotar
caracteres, costumes, uma moral, um estilo de vida e todo um aparelho de
organização.'' E novamente, na mesma página, volta a expor uma relação
dialética entre subjetividade e sociedade: ``A extrema crueza de nossa
forma de organização econômica e política, sentida pelo indivíduo como
uma verdadeira agressão, não é inelutável senão porque esse indivíduo
não teria, sem ela, nem consistência, nem possibilidade de expressão.''
Ora, esse imbricamento entre indivíduo e sociedade, essa relação
dialética já pressupõe que não são idênticos, que, portanto, o ser
humano não é mero depositário amorfo das formas de organização social,
embora possa em maior ou menor medida, dependendo do nível de
individualidade que desenvolver na dialética com o social, sofrer dos
reflexos da forma social -- como parece ser o caso moderno, em que a
forma"-sujeito burguesa tende a consumir a margem de individualidade para
se desenvolver como reflexo do social. E é essa forma"-sujeito que não
entra na reflexão de Musil, o que dá à sua análise um quê de
subjetivismo.

Musil defende nessa \emph{teoria do amorfismo humano} a ideia de que o
homem é uma substância moralmente coloidal, cuja forma se constitui pelo
mundo exterior. Por essa ideia questionável e ambígua de Musil que
coloca o homem como ser moldado pelas circunstâncias sociais, o homem
seria ``moralmente capaz de tudo dependendo das circunstâncias nas quais
ele se encontre.'' (\versal{COMETTI}, 2001, p. 82). Essa reflexão acerca da
substância moralmente coloidal que possuiria o homem -- que se encontra
num texto de seus cadernos de \emph{Diários --} é esclarecedora. Musil
desenvolve também aí o que seria sua filosofia da relação, pode"-se
dizer, entre indivíduo e sociedade. Ele critica a crença num
``auto"-desabrochar do espírito''. Para ele, talvez os homens sejam, no
que diz respeito a todas as questões morais, ``uma massa
extraordinariamente indolente.'' (1981b, p. 23). A ética na Europa
ocidental, segundo ele, não seria diferente daquela dos mares do sul,
apenas estaria submetida a um contexto mais racionalizado. E não é mais
rica que a da Antiguidade, pela simples razão das ``monstruosas
perversões da guerra''. O homem seria, para ele, um ser moralmente sem
forma, ``uma substância coloidal capaz de pensar, mas não de criar,
formas. [\ldots{}] o homem não tira de si sua forma de vida. Existem
dois processos paralelos de tempo diferente, e defasado. O encadeamento
causal da evolução humana e o da forma de vida são diferentes.'' (p.
24). Musil, portanto, oscila entre uma visão mais perpassada pela
dialética e outra mais assertiva quanto ao peso do social.

É digno de nota que essa reflexão de Musil tem relação com a experiência
da Primeira Guerra da qual ele mesmo participou. Em ensaio intitulado
\emph{A Europa desamparada,} Musil escreve que depois de 1914 o homem se
mostrou como uma massa modelável, ``surpreendentemente mais maleável do
que se admitia geralmente.'' (1984, p. 140). O que iria de encontro ``à
importância acordada em nossos sistemas morais ao `caráter', ou seja, à
exigência de poder tratar o homem como uma constante, quando uma
matemática moral mais complexa é não somente possível, mas provavelmente
necessária.'' (\emph{Idem}, p. 140). Para ele, a guerra teria confirmado
por meio de uma experiência de massa que o homem pode oscilar sem
maiores problemas entre os extremos sem mudar sua substância -- mas ele
não diz qual é essa substância primeira imutável. ``A criatura humana é
capaz tanto de praticar a antropofagia quanto de escrever a
\emph{Crítica da Razão pura}.'' (\emph{Idem}, p. 141), o determinante
seria o meio social. Musil, tentando medir o peso das circunstâncias na
modificação dos dados, no caso, os dados da subjetividade, ``viu na
Primeira Guerra mundial uma ilustração tão cruel quanto inaudita do
caráter moralmente coloidal da substância humana. A absorção do eu e da
vontade na massa, essa perda da posse do eu [\ldots{}].'' (\versal{COMMETTI},
2001, p. 107). Sendo assim, o que constitui o ser humano estaria ligado
às ``relações que ele estabelece com o que faz parte de sua
exterioridade e das fontes de que dispõe para encarar essa
exterioridade.'' (\emph{Idem}, p. 83).

Se, como diz Hanke (2004), há na obra maior de Musil uma uma espécie de
\emph{falta de qualidades de cunho psicanalítico,} ``que identifica `por
baixo' das qualidades e disposições de uma pessoa um ser pulsional de
difícil domínio, que priva as qualidades de sua substancialidade'' (p.
138), o que se materializa no interesse de Ulrich e outros personagens
pela figura do criminoso sexual Moosbrugger, essa falta de qualidades,
essa falta de um caráter definido, a presença desse terreno movediço das
pulsões, em verdade estaria em contradição com a própria \emph{teoria do
amorfismo humano} de Musil. Ou seja, o ser humano não seria tão amorfo
assim ao entrar em contato com o social. Se assim o fosse, não haveria
uma tensão histórica entre indivíduo e sociedade. E obviamente Musil
esteve em contato com a psicanálise, quando os cafés de Viena recebiam a
nata da intelectualidade vienense, e justamente o personagem Moosbrugger
é tão metafórico quanto representativo dessa ideia das pulsões que
deixam o \emph{eu} em terreno movediço.

\emph{O homem sem qualidades} de Musil, portanto, não é uma denominação
necessariamente negativa da subjetividade moderna, como é a nossa
interpretação da obra quase cem anos depois de sua escrita, numa leitura
a contrapelo. Em verdade, parece ser a pretensão de Musil explorar, num
emaranhado que nem sempre é discernível claramente, esse terreno das
possibilidades da individualidade em oposição a uma fixação do caráter
em moldes congelados. Nesse sentido, a falta de caráter definido
representado no romance por Ulrich seria um vasto terreno de
experimentações da subjetividade, em termos de Musil. Vai nesse sentido
uma reflexão de Musil em seus \emph{Diários}, na qual ele exprime como
alternativa ``a falta de linha diretivas claras, de valores estáveis,
seguros.'' E acrescenta, embora esteja entre aspas no diário: ``Viver
sem valores ou criar seus próprios valores e suas hierarquias, talvez o
homem ainda não seja capaz.'' E numa frase sem coesão com as demais ele
exprime o que talvez fosse o problema de seu tempo na sua visão: a
``incapacidade de encarar nossa época de transição e trabalhar pelo
futuro.'' Uma ideia em sintonia com o que ele pensava da linguagem
literária: ela ``não descreve uma realidade, ela cria uma idealidade;
seu objetivo está no além. (1984, p. 331)

É assim que para Marie"-Louise Roth, Ulrich de fato se apresenta como um
personagem crítico, uma resposta a uma sociedade ``de qualidades'', a
personagens ``de qualidades'' que têm seu \emph{eu} sufocado pelos
atributos, as qualidades, ``que lhes colam na pele e que lhe são
impostas por uma educação, uma rotina, uma tradição, desejadas por
regimes políticos e sociais.'' (1981, p. 22). Mas a utopia de Musil era
fugir às ``qualidades'' da tradição não para criar um átomo flutuante
livre de qualquer amarra como tende a ser o sujeito contemporâneo --
nesse sentido, dois níveis de leitura se abrem com duas interpretações
distintas: uma em relação ao tempo da obra, e outra em relação ao tempo
presente, numa leitura cem anos depois. Para Musil, o homem não pode
viver sem um princípio intelectual fundamental e sem uma ética. Mas como
resposta ao ``homem de qualidades'', à ``sociedade de qualidades'', ele
opõe uma espécie de moral dinâmica, que ganha contornos no contato com o
social, que tanto a forma quanto é formada por ela, uma moral dinâmica
que permitiria experiências novas, o ``enriquecimento contínuo do
psiquismo e o desenvolvimento da inteligência, mas graças ao contato com
o vivo que é o tecido das realidades enredadas, complexas e
inacabadas.'' (\emph{Idem,} p. 22).

Essa moral dinâmica evidentemente não leva em conta a relação dinâmica
existente entre forma"-sujeito e indivíduos concretos e acaba apostando
numa atitude individual perante a realidade para modificá-la -- o que
era o espírito do tempo de Musil. Nesse sentido, o ideal do \emph{homem
sem qualidades} estaria bastante próximo das vanguardas artísticas do
início do século \versal{XX} no que elas têm de questionamento da tradição, com
seus aspectos impositivos, rijos, enquanto pretensão de ser fundo
substancial para a constituição subjetiva. Mas é preciso levar em conta
que esse tipo de questionamento só encontra terreno enquanto a tradição
ainda tem alicerce, enquanto o substancialismo tradicional ainda tem
terreno social e subjetivo, enquanto ainda não é corroído pelo
substancialismo dinâmico da vida social capitalista.

Seguindo essa interpretação, Ulrich simbolizaria o ``espírito lúcido e
moderno'' na obra. Para Marie"-Louise Roth, Ulrich seria ``o indivíduo
potencial motivado que tenta compreender a novidade de sua época, livre
de toda e qualquer moral rígida e esclerosada, aberto, acessível
sobretudo à regeneração ética[\ldots{}]'' (1981, p. 23). Assim, o ``outro
estado'' buscado por Ulrich seria uma espécie de terreno da
experimentação psíquica, onde o amorfismo humano inventaria formas
novas. Seria um mundo próprio que foge ao imediatamente social. Não
seria exagero comparar esse ``outro estado'' com o procedimento
surrealista da exploração do terreno do inconsciente. De todo modo,
Musil empreende uma exploração vanguardista de um terreno não levado em
conta pela sociedade tradicional, de \emph{qualidades} da época. Como
todas as vanguardas artísticas, embora o contexto austríaco tenha sua
peculiaridade, Musil estava perante uma sociedade que negava o sonho, o
desejo, o desabrochar das possibilidades do indivíduo, numa mistura de
tradição pré-moderna e impulso rumo ao moderno. Tratava"-se, pode"-se
dizer, de um capitalismo ainda neurótico.

Pela teoria do \emph{amorfismo humano} de Musil, é possível fazer uma
aproximação com o existencialismo. Não se trata de colocar Musil como
existencialista, isso não é o importante. O que importa é que Musil
parece se aproximar da ideia de que a existência se sobrepõe à essência,
ou seja, não há uma essência a priori. O problema é que a crítica ao
essencialismo, ao substancialismo, deixa passar ilesa a substância
moderna que é supraindividual, abstrata e subreptícia -- por ser
transcendental. Não há uma substância a priori que constitui o sujeito
de modo automático, como depositário de sua essência, mas há uma
substância social, universal, apriorística que pretende determinar tanto
a marcha do mundo quanto a relação do indivíduo com o mundo. De modo que
somente em termos de um individualismo metodológico incompatível com uma
psicologia social de cunho mais radical poderíamos ver a existência como
em si e para si, como livre ontologicamente. Evidentemente, é na
dialética entre indivíduo e sociedade -- se é dialética é porque o
indivíduo traz consigo aspectos que não são meramente reflexos do social
-- que se constitui a subjetividade. Mas isso não é tudo, a menos que
nos fiemos na aparência. É preciso entender como esse mundo não é um
simples amontoado de particularidades, de individualidades que formam o
todo, mas um todo que pretende jogar seu manto na constituição subjetiva
dos indivíduos como forma"-sujeito. É esse terceiro elemento na relação
dialética entre indivíduo e sociedade que precisa ser levado em conta.

Assim, somente se poderia falar em existencialismo em Musil nesse
aspecto da perspectiva de não haver uma essência primordial. Porque em
Musil não há um dualismo entre consciência e ser\footnote{Walter Sokel
  (1981, p.191) desenvolve uma argumentação na qual estabelece uma
  relação entre Musil e Sartre, entre Musil e o existencialismo francês.}.
Assim, é possível pensar uma relação com o existencialismo pelo menos no
que diz respeito à afirmação de que não é a essência que determina a
existência, mas o contrário.

Nesse sentido, a possibilidade de um \emph{homem sem qualidades} poderia
ser respondida num nível da crítica da metafísica como crítica do
essencialismo da ontologia substancialista (\versal{HANKE}, p. 131). Haveria
portanto uma crítica do sujeito moderno, ou do chamado \emph{eu},
enquanto um portador firme de qualidades essenciais, aspecto central da
ontologia clássica. A ideia central do romance de Musil seria a de que
``a crítica do conhecimento de Mach e sua doutrina elementar, junto com
a dissolução da ontologia substancialista, representam uma dissolução do
conceito tradicional do Eu, cuja realidade é garantida através da sua
substancialidade.'' (\versal{RENTSCH}, \emph{apud} \versal{HANKE}, 1990, p. 53).

Segundo Cometti (2001), a construção de \emph{O homem sem qualidades}
está ligada ao momento de questionamento do \emph{eu} na tradição
empirista. Para ele, Ernst Mach jogou papel importante nesse sentido,
porque Musil teria lhe tomado de empréstimo ``a ideia central segundo a
qual o eu não passa de uma ficção econômica, um `complexo de sensações',
como sugeria a \emph{Análise das sensações''.} (p. 72). Vai nesse mesmo
sentido a análise de Bouveresse, para o qual o conceito musiliano de
\emph{Eigenschaftslosigkeit} pode ser relacionado:

\begin{quote}
com a crítica nietzscheana e machiana do pensamento substancialista, que
vai desembocar no fenomenismo como teoria do conhecimento e no
perspectivismo como atitude ética e estética. Recusando a dissimetria
que existe, do ponto de vista da lógica da inerência tradicional, entre
o substrato e suas propriedades, o fenomenismo de Mach reduz ao estado
de simples ficção cômoda a ideia de um Eu detentor de qualidades.''
(2001, p. 109).
\end{quote}

A teoria do \emph{amorfismo humano} deve ser entendida como uma resposta
ao substancialismo clássico, à ontologia clássica. Cabia naquele momento
histórico, mas agora podemos ter a medida de que há outro
substancialismo, uma ontologia moderna, negativa -- representada pelas
categorias basais da vida social capitalista, que se revestem de um
fundo apriorístico pretensamente não"-substancial. Mas não é por serem
abstratas que não são substanciais. O substancialismo moderno se
embrenha ardilosamente na realidade social e subjetiva sem se deixar ver
como tal. É isso que é preciso apreender em termos modernos e
contemporâneos.

Segundo Jean"-Pierre Cometti (2001), Musil se divertiria em sua obra,
tanto quanto seu ``herói'' Ulrich, ``com os terrores -- exagerados ou
não -- que a ciência e o mundo moderno inspiraram a um grande número de
seus contemporâneos.'' Veríamos na obra um quadro ``normalmente
engraçado dos comportamentos que os avanços da ciência, o recuo das
tradições, a `racionalização' do mundo, para falar como Max Weber,
geralmente deram origem.'' (p. 76). Mas para Cometti, a atitude de Musil
não é a dos extremos. Nem é uma atitude de recusa, que engendra uma
``sátira negativa'', cujo exemplo flagrante é Karl Krauss. Nem é uma
atitude dos defensores incondicionais da modernidade, dos herdeiros das
Luzes ou dos herdeiros do positivismo. Sua atitude consiste em pensar
``que algumas experiências precisam ser aprofundadas, o que constitui o
único meio de vislumbrar um horizonte de possibilidades que não se
imaginava inicialmente, mas que oferece uma alternativa aos impasses
perante os quais uma época se encontra.'' (p. 76).

Evidentemente, seguindo o estudo Cometti, a conclusão é que Musil não
parece um crítico incondicional da racionalidade instrumental na forma
como se desenvolverá mais tarde na Teoria Crítica, mas também não é um
defensor incondicional. O entusiasmo de certo modo que sentia Musil pela
ciência e pelo positivismo também é fruto de uma época em que as pessoas
pensavam poder fazer grandes coisas, em que não se sentiam dominadas
pelas realizações do passado, apesar de conhecê-las, em que o futuro
apesar de tudo podia se apresentar como radiante. Esse parece um terreno
de experimentação de Musil, que seria ``muito mais respeitoso e até
muito mais amante da ciência do que se imagina alguém do meio literário
e filosófico, e muito mais racionalista do que se pode entender ou
aceitar da parte de um poeta.'' (\versal{BOUVERESSE}, 2001, p. 14).

Mesmo assim, como veremos, a obra é também uma grande reflexão crítica
sobre a racionalidade moderna -- sobretudo por meio do narrador --,
embora Musil não pareça pretender uma superação dessa racionalidade, mas
muito mais uma síntese com esferas da vida não determinadas por essa
racionalidade. É nesse sentido que ele separa dois mundos, dois
terrenos, que estariam separados por níveis. A literatura, para ele, faz
parte desse terreno que ele denomina com um neologismo de
\emph{não"-racioide} em oposição ao mundo \emph{racioide.} Este seria o
mundo de ``tudo que pode entrar num sistema científico, tudo que pode
ser resumido em leis e regras, portanto, antes de tudo: a natureza
física [\ldots{}] Esse terreno é caracterizado por uma certa
monotonia dos fatos, a predominância da repetição, uma relativa
independência dos fatos uns em relação aos outros.'' (\versal{MUSIL}, 1984, p.
81). Seria o terreno da solidez, da certeza lógica, das leis da física.
Para Musil, o grande problema de seu tempo era que a chamada moral --
para ele, a moral burguesa -- estava submetida a essa solidez, a um
processo de cofragem de concreto armado, em que se empurram na sua
profundeza indeterminada ``vigas de conceitos em torno das quais se
estabelece toda uma rede de leis, de regras e de fórmulas. O caráter, o
direito, a norma, o bem, o imperativo, o sólido em todos os sentidos''
pretendem ser um lastro para todas as ``pequenas decisões morais
exigidas por cada dia de vida.'' (\emph{Idem}, p. 82). Mas Musil via o
homem numa ``estufa de mil movimento secretos'' (\emph{Ibidem}, p. 82)
que não cabem nessa fôrma de concreto armado da moral, ou da
subjetividade, se quisermos. É nesse sentido que sua reflexão não é
necessariamente clara. Se a sociedade é assim de tal modo uma fôrma,
como é que o ser humano poderia trazer consigo esses ``mil movimentos
secretos? Ao mesmo tempo que parece colocar a própria sociedade como uma
espécie de ontologia negativa em relação ao indivíduo, que se via
penetrado por essa sociedade, também reconhece os recantos dos
``movimentos secretos.'' Evidentemente, seria um tanto forçado ver no
autor uma sutil intuição da forma"-sujeito burguesa. Mas talvez seja
exatamente a não identificação desse elemento que escapa à relação entre
indivíduo e sociedade que coloca em maus lençóis a ideia do ser humano
como determinado pela sociedade e ao mesmo tempo rico de possibilidades.

Se o domínio \emph{racioide} é mundo ``da regra com exceções'', Musil
parece exatamente querer explorar esse terreno \emph{não"-racioide} onde
``as exceções vencem as regras'' (\emph{Ibidem}, p. 82), um terreno que
exige uma ``mudança completa da posição do sujeito do conhecimento'',
porque nesse terreno:

\begin{quote}
os fatos não são dóceis, as leis são alvos, os acontecimentos não se
repetem, são infinitamente variáveis e individuais. Não teria como
caracterizar melhor senão precisando que é o terreno das reações do
indivíduo em relação ao mundo e ao outro, o terreno dos valores e
avaliações, das relações éticas e estéticas, do domínio da ideia.
[\ldots{}] Os fatos, nesse terreno, e consequentemente suas relações, são
infinitas e incalculáveis''. (\emph{Ibidem}, p. 83)
\end{quote}

Mas como seria possível surgir esse terreno se subsistisse de fato o
caráter amórfico do ser humano? Porque não pareceria plausível que até
mesmo esses recantos de individualidade fossem mero reflexo do social. E
é esse terreno, Para Musil, a pátria do escritor, onde sua razão --
\emph{não"-racioide} -- seria suserana. Nesse terreno, a tarefa é sempre
encontrar novas soluções, novas variáveis, é criar ``imagens sedutoras
das possibilidades de ser um homem, de \emph{inventar} o homem
interior.''(\emph{Ibidem}, p. 83). A sua aventura é de explorar esse
mundo interior, essa subjetividade que não deveria se submeter ao
terreno que ele chama de \emph{racioide.} E ele chama a atenção que não
se trata de interpretação psicológica, porque para ele, ainda encarando
a psicologia como psicologia experimental, a psicologia não sabe o que
fazer com os \emph{temas} da alma, que são de uma diversidade
incalculável, e extravasariam a limitação da psicologia. (\emph{Ibidem},
p. 83)

Visto por outra perspectiva, esse raciocínio musiliano não deixa de ser
ambíguo, já que o próprio Musil escreve em outro ensaio que ``o
pensamento artístico e o pensamento científico, entre nós alemães, ainda
não entraram em contato. Os problemas da zona intermediária ainda
esperam uma solução.'' (\emph{Ibidem}, p. 73). Ele parece querer erigir
essa zona intermediária, uma espécie de síntese entre os dois terrenos.
Ele parece querer explorar as possibilidades interiores do homem
inclusive usando o universo da técnica. Segundo Cometti (2001), Musil
considerava, como muitos outros, que a particularidade de sua época era
o ``desabrochar de possibilidades de expressão favoráveis à
individualidade''. (p. 105). E a racionalidade moderna poderia
contribuir para isso, embora contraditoriamente fosse ao mesmo tempo a
causa da perda do \emph{principium} \emph{individuationis.}

Musil distingue os dois mundos, mas não os opõe. Embora diga que é
preciso uma mudança de posição do sujeito do conhecimento em cada
terreno, diz que a diferença entre ambos é de nível. Torna"-se mais clara
talvez a síntese que ele pretende com o trecho de um ensaio,
\emph{Confissão política de um jovem}:

\begin{quote}
E num período, o nosso, em que cada sentimento espreita em duas
direções, em que tudo vagueia, em que nada mais tem sustentação, em que
nada mais é associável a nada, deveríamos conseguir testar e reinventar
todas as nossas possibilidades interiores, transferir enfim dos
laboratórios de física à moral as vantagens de uma técnica de
experimentação sem preconceitos. Que isso nos ajude a sair da lenta
evolução que conduziu, por meio de muitos fracassos, do homem das
cavernas ao do presente, para entrar numa nova era, continuo a crer
nisso hoje. (\versal{MUSIL}, 1984, p. 63).
\end{quote}

Embora, como diz Cometti (2001), Musil não acreditasse ``na ciência no
sentido de que teria reconhecido nela a capacidade de responder a todas
as questões que nós nos colocamos ou a todas as nossas preocupações.''
(p.75), transparece um movimento meio aporético em que a racionalidade
das sociedades modernas criaria uma estrutura coloidal do homem
``segundo formas padronizadas que não deixam mais lugar para a
individualidade.'' (2001, p. 106), ao mesmo tempo que abriria inúmeras
possibilidades. Para Cometti, em Musil a racionalização das sociedades
modernas -- as relações objetivas, o \emph{racioide}, a civilização -- é
a realidade que dá a espinha dorsal do homem, ao mesmo tempo que ataca
sua individualidade. Mas Musil parece convencido de que ``se a vida
humana é sempre uma resposta ao `amorfismo humano', e se uma tendência
natural consiste para ela em se alojar em formas de organização que
tendem a se tornar exclusivas e a se perenizar, o amorfismo se abre
sempre para outras possibilidades, sempre disponíveis e possíveis.'' (p.
83).

Em Musil, parece que esse amorfismo é o caminho que permite pensar as
saídas, porque o homem sendo ``mais maleável do que se pensa'', as
possibilidades se tornam infinitas. É nesse sentido que ele explora o
universo que ele chama de ``outro estado'' como contraponto ao mundo
racioide. Num tópico intitulado \emph{Os adversários dos fatos}, que faz
parte do ensaio \emph{O alemão como sintoma} (1984), Musil condena o
antirracionalismo\footnote{O que Musil entende por antirracionalismo é o
  ataque puro e simples à razão. Nele, não há a distinção fundamental
  entre a razão como capacidade de elaboração intelectual, tipicamente
  humana, e a forma de razão específicamente moderna que é representada
  pela Razão instrumental e mercantil.} como resposta ao mundo racioide.
Ele avança como resposta ao positivismo o que ele denomina de
\emph{outro} \emph{estado.} Um terreno meio místico, cuja definição pelo
próprio Musil é complexa e não desprovida de ambiguidades. Para ele,
``existe um estado humano que possui uma diferença fundamental que se
opõe ao estado do conhecimento, do cálculo, da perseguição de objetivos,
da avaliação, da coação, da ambição e dos vis temores. Ele é difícil de
ser caracterizado.'' (1984, p. 374). Bondade, amor, irracionalismo,
religiosidade seria apenas parte desse \emph{estado outro}.

O essencial que sobressai na argumentação musiliana acerca desse
\emph{outro estado} é um certo subjetivismo, subjetivismo que ele
próprio rechaça. Mas é ele próprio quem diz existir ``uma tensão e uma
coloração que mudam no mundo exterior segundo as disposições interiores
do observador.'' (\emph{Idem,} p. 374). Nesse outro estado, não haveria
uma distinção rígida, marcada, entre \emph{eu} e \emph{não"-eu}, mas um
certo transtorno nessa relação. Se no estado ordinário, racioide, o eu
se apossa do mundo, ``no outro estado, o mundo aflui para o eu ou se
confunde com ele ou o carrega consigo. [\ldots{}] Nesse estado, o
ato de compreensão não é impessoal (objetivo), mas se manifesta de modo
pessoal como um acordo perfeito entre sujeito e objeto.'' (\emph{Idem,}
p. 375). É assim que ele destaca um papel maior da subjetividade em que
o \emph{eu} em si mesmo é algo que ``se vive de modo indefinível e muito
cambiante.'' Um \emph{eu} que, nas relações objetivas, seria eliminado.

Assim, Musil pretende com esse estado superar certas dicotomias. O homem
bom nesse sentido outro, o homem que ama, não é bom por oposição ao
egoísta. Em seu mundo o egoísmo não existe: ``Trata"-se de uma outra
forma de avaliar. A oposição altruísta"-egoísta perde seu sentido; bem
como aquela entre bom e mau.'' (\emph{Idem,} p. 376). Em vez de
distinção, uma avaliação em termos de aumento ou diminuição. Nesse outro
estado, estado dito \emph{contemplativo}, também não haveria desprazer,
ou só haveria no afastar"-se desse estado. Cólera, raiva, inveja,
vergonha também não teriam morada: daí ``a estreita proximidade
histórica da contemplação e da beatitude.'' (\emph{Idem,} p. 379). Mas,
ao mesmo tempo, diz Musil, não se pode confundir esse outro estado como
sendo ``exclusivamente um estado de prazer.'' Trata"-se de um estado de
beatitude, de bem"-aventurança, de felicidade plena, em que teríamos
tanto o amor beato, a cólera beata, a vergonha beata, como a angústia
beata. Como ele próprio reconhece ser impossível sentimentos como
inveja, raiva, avareza beata ou contemplativas, ele lança mão do par
moral ``sentimentos grandes e mesquinhos'' que diriam respeito à
oposição \emph{estado contemplativo} e \emph{estado normal}, ordinário.

É nesse outro estado, nesse estado contemplativo, que se mostra um Musil
com uma teoria da subjetividade um tanto mística, em que esse
comportamento contemplativo se mostra ``amalgamado com a hipótese da
existência de Deus'', embora seja independente deste. ``Mesmo o estado
terrestre, o estado erótico de amor pode levar inúmeras pessoas, que não
creem de modo algum no além, muito longe nesse reino contemplativo.''
(\emph{Idem}, p. 381). Esse outro estado seria um desvio do mundo, uma
elevação acima dele, uma transcendência, uma renúncia ao combate para
metaforicamente unir"-se ao mundo divino. Mas Musil deixa entender que
esse mundo contemplativo em verdade é um mundo inacessível, irreal
quando pergunta se esse outro estado não seria ``um impulso, uma
necessidade, uma metade do homem que, reprimida, nunca parou de tentar
vir a lume?'' (\emph{Idem}, p. 381).

A busca de Ulrich pelo ``outro estado'', pela abolição do real estaria
nesse terreno da experimentação fora das \emph{qualidades definidas}.
Esse ``outro estado'' seria uma resposta não"-racioide ao mundo. Não
seria uma superação transcendente ao racioide, mas uma forma de ser uma
alternativa de contrapeso. Para Musil, o amor seria exatamente a forma
eclodida do \emph{outro} \emph{estado}. (1981, p. 130). Não à toa, a
relação entre Ulrich e sua irmã Ágata é vista como um ápice desse
\emph{outro estado}. Assim, a relação incestuosa com a irmã, em vez de
ser vista como uma impossibilidade de distinção de um homem sem
atributos, de um homem contaminado pelo espírito matemático e abstrato,
seria interpretado como mais um elemento da busca sem limites de novas
possibilidades na existência utópica assumida por Ulrich, seria o apogeu
da busca pelo \emph{outro} \emph{estado.} O mundo à parte que cria com a
irmã seria ``a fuga de um mundo criado por outros'', a busca de ``uma
aventura impossível'', e a irmã representaria ``o que ainda há de
inacabado nele e que pode crescer indefinidamente.'' (\versal{ROTH}, 1981, p.
23).

Em contrapartida, ao mesmo tempo, segundo Cometti (2001), não só o
interesse de Musil ``é a experiência de um enfraquecimento das
faculdades cognitivas e a ruptura do \emph{principium}
\emph{individuationis.}'' (p. 82), como a ausência de caracteres
próprios está relacionada com a crise da identidade que no romance não
atinge apenas Ulrich. É quase um impasse, e a obra vai refletir também
esse tensionamento.

Seria Ulrich um personagem positivo naquele entremeado de personagens? O
único que \emph{viveria por hipótese} essa busca fora dos caracteres
definidos, seria quem foge da cimentação da moral tradicional? Seria o
indivíduo que pode experimentar, ser ``observador e criador da própria
vida, propondo uma nova estética e uma nova ética'' (\versal{LESSA}, 2006, p.
21)? Sendo assim, a falta de caracteres próprios, o
\emph{Eingenschaftslosigkeit,} o homem sem qualidades, ganharia um
sentido positivo, de descoberta, de exploração de terrenos inexplorados
que poderiam dar nova forma ao \emph{amorfismo humano,} e Ulrich seria
uma resposta, seria o homem que flutua em meio a subjetividades
atreladas a uma ontologia. Dito de outro modo, o \emph{homem sem
qualidades} seria uma crítica ao \emph{homem de qualidades} das
sociedades anteriores à modernidade. Poderíamos até cair na tentação de
ver em Ulrich o homem de Gracián, que não haure suas qualidades em
nenhum princípio externo, mas a partir de sua ação no mundo. Mas Ulrich
parece em verdade ser um homem bem moderno, não está ligado a nada, não
é em relação a posição, a família, a riqueza que sua subjetividade se
constrói, mas isso não significa que ele seja um sujeito que se oponha à
sociedade, mas antes um átomo flutuante que acompanha a sociedade. Ele
se faz por si mesmo tentando haurir sua subjetividade da experimentação
ilimitada da \emph{sociedade sem qualidades} que estava corroendo a
anterior.

Nesse sentido, o romance de Musil aparece como uma reflexão crítica
acerca de uma crise num determinado momento do desenvolvimento da
modernidade, ao mesmo tempo que aparece como uma tentativa de superação
pela via do questionamento do que seria uma tradição ontológica que
cimenta as possibilidades do homem -- uma espécie de amarra que era um
dos elementos daquela crise. Ele quer implodir a cimentação dessa
subjetividade, dessa moral como ele chama, abrindo o horizonte para
infinitas possibilidades. Assim, a falta de qualidades definidas, de
caracteres definidos, pretende ser em Musil uma resposta à cofragem, à
\emph{fôrmação} da subjetividade advinda de um substancialismo
ontológico. No entanto, esse mundo de infinitas possibilidades explorado
por Musil como resposta ao \emph{racioide} acaba por desembocar, com o
desenvolver"-se da vida social moderna, em nivelamento e cruzamento entre
\emph{racioide} e \emph{não"-racioide}. Dito de outro modo, sua tentativa
de pôr em contato os campos \emph{racioide} e \emph{não}-\emph{racioide}
pode ser posta em paralelo com a tentativa habermasiana de de pôr em
\emph{comunicação} mundo da vida e mundo sistêmico. Nos dois casos,
escapa o caráter totalitário do mundo racioide ou do mundo sistêmico.
Assim, como no caso de Habermas, a síntese só é possível sob a condição
de que o mundo essencial por excelência, o mundo sistêmico, seja menos
determinado do que determina o mundo não"-racioide ou o mundo da vida.
Portanto, parece hoje uma postura conformada apostar positivamente nessa
síntese não superadora, mais parecida com uma dialética não"-negativa.

Enquanto Ulrich se veria aberto a todas as possibilidades trazidas pela
perda da totalidade tradicional, outros personagens da narrativa sofrem
exatamente por levarem uma vida desprovida de centro. A chamada
``anarquia dos átomos'' analisada por Nietzsche como símbolo de uma
decadência do estilo ganha em Musil um contorno criador, positivamente
crítico. Esses átomos transtornariam a hierarquia da vida e a hierarquia
do discurso que deveria organizar a vida lhe dando ordem e sentido. É o
que Claudio Magris expõe como ``desordem libertadora, no que ela
desarticula o rigor dominador de uma vontade que tende a impor uma
hierarquia unitária e unificante em cada detalhe rebelde.'' (1981, p.
139).

Essa anarquia dos átomos restabeleceria a liberdade do indivíduo, a
vibração e a exuberância da vida nua e selvagem. Por isso Magris se
entusiasma na sua interpretação do romance onde encontramos, segundo
ele, uma multiplicidade ``de sujeitos livres, sem laços e sem eixo
[\ldots{}] encontramos essa proliferação centrífuga das
singularidades que afirmam em cada domínio -- literatura, ideologia,
política -- sua selvagem autonomia''. (1981, p. 145).

E é assim que o \emph{homem sem qualidades} seria um \emph{homo
potencialis} e não a forma"-sujeito burguesa num determinado momento de
seu desenvolvimento na história da dinâmica mercantil. Seria o
personagem do incomensurável, criado pelo romancista como ``resistência
obstinada às forças que querem transformar os homens em seres
unidimensionais'', e seria o primeiro passo na luta ``contra a
iniquidade social''. (\versal{DAWLIANIDSE}, 1981, p. 205).

Mas como temos tentado refletir, Musil pode ser ambíguo. Sua exploração
das múltiplas possibilidades não significava ter clareza de todos os
elementos. E Bouveresse nos traz essa importante ambiguidade quanto à
ideia de um mundo de unidade e totalidade. Segundo Dieter Hornig, há uma
contradição profunda em Musil que se apresenta ``na avaliação
ambivalente que sua crítica da civilização faz da `dissolução', da
desintegração: uma vez é vista como perda da unidade e dos laços, um
sintoma de decadência, que deve ser estancado por um `sentido' novo, por
um significado último, para desembocar num novo universalismo; outra vez
ela aparece como uma uma chance, como um ganho de liberdade e de novas
possibilidades de dar forma à história.'' (\versal{HORNIG}, apud \versal{BOUVERESSE},
2001, p. 30).

Mas essa ambivalência em Musil é rapidamente jogada para baixo do tapete
por uma leitura de feição pós"-moderna que tenta colocar Musil como
dinamitador da totalidade pura e simplesmente. Para autores como David
Dawlianidse, Musil

\begin{quote}
foi um dos raros escritores burgueses a sentir que a realidade
totalmente nova que tinha sido engendrada pelos transtornos
sócio"-políticos e pelo desenvolvimento fulminante das ciências e da
técnica, constituía um desafio que a literatura devia encarar. Ou a
literatura devia renunciar a pretender representar a totalidade do
mundo, e sobretudo exercer toda a influência ativa sobre a realidade ou
devia tentar encontrar uma linguagem nova adaptada à época moderna''
(1981, p. 203).
\end{quote}

O problema é que Musil não tenta buscar uma simples linguagem, no
sentido formal, para entender a época moderna. É exatamente a ação
romanesca transformada em reflexão que tenta apreender essa totalidade
do mundo, não mais no sentido tradicional de uma ordem una estabelecida
por uma tradição fixa, mas uma totalidade dinâmica e em fragmentos.
Musil apreende exatamente essa efusão de mudanças relacionadas à
modernidade, à técnica, à ciência, ao império do racioide.

O que salta aos olhos é que a própria marcha moderna foi corroendo ``as
qualidades'', o substancialismo tradicional. De modo que a aposta na
superação das qualidades, dos atributos, estaria mais em consonância com
a vida social mercantil no seu estágio contemporâneo do que com uma
postura crítica e transcendente em relação a esse mundo. Ironicamente,
cinicamente, o próprio capitalismo realizou a utopia do \emph{homem sem
qualidades}, do mesmo modo que muitas das ideias das vanguardas
artísticas foram acomodadas pela sociedade mercantil no seu
desdobramento, um desdobramento que demonstrou o quanto aqueles
questionamentos vanguardistas a ajudaram involuntariamente a realizar
seu conceito na realidade.

Claro que a ideia de Musil de levar em conta o imponderável e o complexo
que é o vivo como fonte de possibilidade e desenvolvimentos infinitos é
algo que se mantém atual: mas somente se estiver relacionado a um
conteúdo crítico determinado. Do contrário, recai num existencialismo de
matiz abstrato que até mesmo a propaganda pode utilizar.

De todo modo, é o próprio Musil quem dá a chave para se entender o seu
tempo e o quanto sua teoria do \emph{amorfismo humano} e sua busca do
\emph{outro estado} só podem ser relacionados ao seu tempo histórico.
Não que não seja urgente a busca de um \emph{outro estado} nos tempos
atuais. Ao contrário, urge a construção de um estágio transcendente à
vida social regida pelo racioide"-mercantil, não para amainá-lo, torná-lo
menos totalitário, mas para que organização da vida social se submeta à
consciência crítica dos indivíduos concretos e não se lhes apresente
como movimento socialmente natural. O próprio Musil era consciente dos
perigos inerentes à tentativa de realização de um pensamento
\emph{ideal}. Como ele mesmo lucidamente diz:

\begin{quote}
Os ideais têm também, com efeito, essa propriedade estranha de
desembocar, quando plenamente realizados, num contrassenso. [\ldots{}]o
que chamamos no sentido próprio de ideais, as grandes representações
motoras da vida individual e social, comportam também uma desmedida na
exigência que conduziria fatalmente à ruína se não houvesse de antemão a
recusa de considerá-las ao pé da letra.'' (1984, p. 383)
\end{quote}

Significaria quase dizer que pôr abstrações no plano da realidade é como
destruir a realidade. E o avançar da marcha moderna e da Razão
instrumental, que ele quer explorar no terreno vasto chamado de moral,
vai demonstrar que a falta de qualidades próprias vai se tornar
cinicamente um terreno dos mais férteis para que a forma"-sujeito
burguesa \emph{ideal} se realize, sempre implodindo qualquer cimentação
do que pode caracterizar um \emph{caráter definido}. O \emph{amorfismo}
\emph{humano}, de terreno de oposição ao caráter fixo, congelado e
tradicional que fundamenta as esferas de poder na sociedade, tornou"-se
um terreno de exploração de múltiplas possibilidades para o próprio
capitalismo quanto mais se aproxima do seu conceito -- que é em verdade
uma tendência ao amorfismo social.

Se na época de Musil o \emph{homem sem qualidades} é precisamente o
homem da ação não qualificada, que não é nem catalogada nem sancionada
[\ldots{}], como diz Bouveresse (2001, p.100) e sua antítese pode
ser vista como o ``homem de ação de tipo realista, cujas relações com o
mundo se fazem por meio da clareza e da eficácia'', cem anos depois, tal
postura já não causa danos à organização social estabelecida. O mundo de
possibilidades de Ulrich, um mundo que transbordava aquele seu tempo,
hoje já cabe com margens de sobra no universo simbólico mercantil da
vida pretensamente descolada e multifacetada, rica de possibilidades
compráveis.

E aqui notamos toda a diferença entre o Ulrich da época de Musil e o
Ulrich lido na contemporaneidade, uma diferença que a crítica de Musil
não parece querer ver. O próprio Bouveresse parece intuir, mas sem
explorar essa rica diferença: ``A diferença entre Ulrich e a maioria de
seus contemporâneos é que ele apreende como um fenômeno de imaturidade
cheio de promessas o que outros interpretam como uma tendência à
degenerescência e ao caos.'' (2001, p. 113). Ora, diferentemente do
tempo de Musil, na época atual, a vida social não apresenta imaturidade,
ao contrário, sua dinâmica a levou para além da maturidade. Os desígnios
mercantis já estão quase realizados na terra, não se trata de julgamento
moral quanto a uma decadência. No caso contemporâneo, os que veem
degenerescência e caos no ocaso da vida social mercantil é que podem
gestar novas promessas, porque o entusiasmo positivista da época de
Musil já não cabe nos tempos atuais. E essa é uma mudança que precisa
ser levada em conta, sob pena de lermos Musil com os olhos de 100 anos
atrás, o que impediria a reflexão sobre o significado de um século de
desdobrar"-se de uma vida social e subjetiva sem qualidades.

\section{O \emph{homem sem qualidades} e o entrecruzamemento rumo ao futuro}

\begin{quote}
Eu sou em realidade, e isso depois que empreendi o H.s.q., tão pobre, e
por natureza tão inapto a ganhar dinheiro, que vivo somente das rendas
de meus livros, mais exatamente dos adiantamentos que meu editor me
consente com a esperança de que essa renda acabe talvez por aumentar.
Chegou a me acontecer inúmeras vezes, enquanto trabalhava no primeiro
volume, de ficar do dia para a noite tão privado de qualquer recurso que
não via mais como dar conta de minha subsistência durante os quinze dias
seguintes, e não ser salvo senão pela intervenção de alguém, geralmente
no 14º dia. Consequentemente, se meus livros são de difícil abordagem e
não disputam os favores dos leitores, não se deve ver nisso arrogância
da parte de quem passa muito bem sem lê-los. Trata"-se mais,
aparentemente, de um elemento fatal, de uma fatalidade que me é própria;
e essa injustiça da vida que devo evocar hoje deve ser relacionada
estritamente ao trabalho que resolvi fazer.

\emph{Musil}, \emph{Testamento \versal{II}}, 1932.
\end{quote}

É o que se pode chamar de obra de uma vida. Seu autor nunca gozou de
condições estáveis para levar a cabo seu longo empreendimento estético,
crítico. Exilado após a anexação da Áustria por Hitler, acabou morrendo
sem concluir suas experimentações de ``dissecador da alma no início do
século \versal{XX}''. O autor vivia o ambiente de uma Viena submetida a rápidas
mudanças, uma espécie de rápido remodelamento da vida social e
subjetiva. O clima da obra é o que o Edmonde Roux chama de ``romance de
uma ordem ameaçada.'' (1981, p. 179).

As mudanças na cidade, que chamavam a atenção do observador sensível,
tinham o sentido de ``um grande pulsar rítmico e do eterno desencontro e
dissonância de todos os ritmos, como uma bolha fervente pousada num
recipiente feito da substância duradoura das casas, leis, ordens e
tradições históricas.''(\versal{MUSIL}, 2006, p. 28). E essa ``substância
duradoura'' não tinha como resistir ao inexorável fervilhar dessas
bolhas ferventes da marcha moderna. Como diz Kraus no seu escrito
intitulado \emph{A literatura demolida,} ``Viena está sendo demolida
numa metrópole moderna. Com suas velhas casas, desabam os pilares de
nossas lembranças e em breve uma picareta sem respeito algum terá feito
tábua rasa do honorável café Griensteidl. [\ldots{}] Nossa
literatura não terá mais teto, e os fios da produção poética serão
cruelmente cortados.'' (1993, p. 43).

\emph{O homem sem qualidades} é uma obra no cruzamento de dois séculos,
que representam duas mentalidades, duas formas de sociedade e de
subjetividade em tensão. No caso da Áustria, o entrecruzamento pendia
mais para o passado. Musil, num ensaio intitulado \emph{A política na
Áustria} reflete sobre o caráter do sentido das forças em tensão naquela
mudança de época. E nesse sentido ele identifica grande diferença entre
a Alemanha e a Áustria em termos modernos.

\begin{quote}
A ferramenta da social"-democracia ainda não está bem afiada, e não
dispomos dos contrastes violentos como o que existe na Alemanha entre o
impulso intelectual de alguns estraga"-festas, vivendo como magníficos
parasitas sobre os destroços do Estado mercantil, e a legitimidade --
dois pés na Bíblia e dois pés na gleba -- dos proprietários fundiários.
A estrutura social apresenta, até um altíssimo nível, uma mistura
coerente de burguês e fidalgo.'' (1984, p. 44).
\end{quote}

E Musil relata que a grande antítese conceitual entre burguesia e
aristocracia não existe na Áustria. O que dá ao país uma especificidade
nessa marcha moderna, com os pés ainda fincados no passado, mas tendo
que correr para o futuro inexoravelmente. E segue sua reflexão
relacionando o desenvolvimento de uma burguesia nacional e o
desenvolvimento da cultura, algo que ele vê como um fenômeno novo e
paradoxal. Em Estados com um mercado desenvolvido, que seriam um
``terreno tormentoso'', embora estranho ao espírito, a cultura prospera
melhor do que nos terrenos calmos, ``nas superfícies lisas que lhe
convinham melhor outrora.'' (\emph{Idem,} p. 68). E segue sua reflexão:

\begin{quote}
O que, culturalmente, distingue nossa época das precedentes é a
dissolução efetuada pelo grande número, a solidão e o anonimato do
indivíduo no seio de uma massa que cresce sem parar, impondo ao espírito
uma atitude nova de consequências ainda imprevisíveis. O pouco de obras
de arte dignas desse nome que temos hoje são o exemplo mais claro: o
fato de elas não poderem ser ao mesmo tempo boas e acessíveis ao grande
público constitui realmente um fato novo e provavelmente, para muito
além das querelas estéticas, o aparecimento de uma nova função da
arte.'' (\emph{Idem,} p.p. 44-45).
\end{quote}

Para Musil, essa mudança qualitativa na essência mesma da arte, ou seja,
o que uma cultura nesses padrões pressupõe é a existência da burguesia
-- no sentido entendido por ele de classe. Mas ele capta a dinâmica
interna à lógica burguesa, que escapa ao conceito de classe ao refletir
que é próprio da burguesia ``não poder produzir famílias que não se
desvencilhem rapidamente de tradição, ideal hereditário, moral estável,
que são coisas úteis para quem aprende a andar, mas que pesam para quem
corre.'' (\emph{Idem}, p. 45). Assim, diz ele, os negócios da burguesia
fazem com que ela não possa ela mesma se ocupar da cultura, por isso que
a financia. Mas ele mesmo diz que essa burguesia na Áustria não existe.
O que cria um impasse. Ao mesmo tempo que seria essa lógica burguesa em
movimento a operar a mudança na arte, ela mesma não existe na Áustria,
mas ao mesmo tempo o país não deixa de sentir os ares da mudança, assim
como o artista sente a tensão entre os dois mundos. Dito de outro modo,
essa tensão entre tradição e marcha moderna existe na Áustria, apenas
ainda num nível de amadurecimento menor -- e o exemplo dos outros países
também cria uma pressão. Ulrich, o \emph{homem sem qualidades}, é tão
fruto desse entrecruzamento quanto os personagens de \emph{qualidades.}
Jean Gyory fala do \emph{homo austriacus} do qual ``\emph{O homem sem
qualidades} é na verdade o catálogo completo, a análise mais precisa
desse ser estranho [\ldots{}] é a descrição de sua história, a
análise de seu desenvolvimento e a tragédia de seu desaparecimento.''
(1981, p. 188).

A obra, pode"-se dizer, é uma larga reflexão não só sobre a
racionalização do mundo, causada pela Razão instrumental"-mercantil, como
também sobre uma tensão entre um \emph{mundo de qualidades} e um
\emph{mundo sem qualidades}, que é par da tensão entre \emph{seres de
qualidades} e \emph{seres sem qualidades}. Mas seria erro pensar que o
autor coloca o \emph{mundo sem qualidades} como mundo de decadência.
Como vimos, ele é muito mais ambíguo quanto à questão da marcha da Razão
moderna, que ele não distingue da própria capacidade intelectual. Mas
entre o \emph{homem sem qualidades} e os \emph{homens de qualidades,} há
o narrador. É ele quem muitas vezes aponta uma reflexão para além dessa
dicotomia. É ele quem primeiro bem expressa o que seria a modernidade de
fato, uma era em que a ciência e a técnica dominam não para simplesmente
livrar os homens do medo, mas para levar a humanidade ao progresso
inexorável:

\begin{quote}
Ganhou"-se em realidade, perdeu"-se em sonho. Não nos deitamos mais sob a
árvore, espiando o céu entre o dedo grande do pé e o dedo médio, mas
trabalhamos; também não devemos passar fome nem sonhar demais, se
quisermos ser eficientes, mas comer bifes e fazer exercício. \emph{É
exatamente como se a velha humanidade ineficiente tivesse adormecido
sobre um formigueiro; quando despertou a humanidade nova, as formigas
tinham entrado em seu sangue, e desde então ela precisa fazer movimentos
incessantes, sem conseguir se livrar desse chatíssimo ímpeto de
fanatismo pelo trabalho.} Realmente não é preciso falar muito a
respeito; a maioria das pessoas sabe perfeitamente, hoje, que a
matemática entrou em todos os campos de nossa vida, como um demônio.
Talvez nem todas as pessoas acreditem na história do Diabo a quem se
pode vender a alma; mas todas as pessoas que entendem alguma coisa de
alma, por serem sacerdotes, historiadores e artistas, e tirarem boas
vantagens disso, testemunham que foi a matemática que arruinou a alma,
que a matemática é fonte de uma inteligência perversa que faz do homem
senhor da terra, mas escravo da máquina. (\versal{MUSIL}, 2006, p, 58 [Grifos
Nossos])
\end{quote}

Esse trecho da obra é também ambíguo. Pode"-se ver tanto uma crítica
mordaz e poética da marcha moderna pela voz do narrador, como talvez
também de modo irônico um otimismo com essa humanidade nova, desperta de
um sono ineficiente, uma humanidade que tem algo que lhe formiga no
espírito e a obriga a movimentos incessantes. Mas a crítica do narrador
à obsessão moderna pelo trabalho não é menos digna de nota -- como já
dissemos, ele é uma voz muitas vezes mais crítica, que está para além da
tensão entre dois mundos que é pano de fundo na obra. O fato mesmo de o
narrador expressar com essa bela imagem do formigueiro fervilhando no
corpo moderno a especificidade da dinâmica moderna já salta aos olhos.
Se àquela época essa imagem poderia se confundir com a superação de
formas de sociedade atrasadas, ``ineficientes'', hoje, quando podemos
ter a medida do que representa esse movimento incessante, de superações
contínuas de formas de vida sempre vistas como atrasadas, não podemos
mais ver nesse movimento de formigas formigando por nosso corpo e nos
forçando a movimentos inexoráveis senão como alusão de uma segunda
natureza. O homem pretendia assumir o lugar de senhor do mundo, mas
acabou servo de outro domínio: o da Razão instrumental"-mercantil. Não é
à toa que das três tentativas de se tornar alguém importante -- ``Ulrich
parecia ter nascido com esse desejo'' (p. 53), ``O fatal era apenas que
ele não sabia como a gente se torna importante, nem o que é um homem
importante'' (p. 54) -- a mais importante foi a empreitada matemática. A
primeira foi a cavalaria, a segunda, a técnica, que já prenunciava a
terceira e mais importante. E o narrador observa que a mudança não foi
anódina: ``[\ldots{}] ao passar da cavalaria para a técnica, Ulrich
apenas trocou de cavalo; o novo tinha membros de aço, e corria dez vezes
mais depressa.'' (p. 55).

O narrador nos expõe a mudança subjetiva, a adaptação dos sujeitos ao
novo que cada vez menos traz marcas do passado, um novo que é uma
síntese de tensões com formas anteriores de subjetividade, mas que
carrega mais marcas do novo do que do passado, considerado um entrave ao
desenvolvimento: ``No mundo de Goethe, o ruído dos teares ainda
perturbava, mas no tempo de Ulrich, começava"-se a descobrir a canção das
salas de máquinas, martelos de arrebite e sirenes de fábrica'' (p. 55).

\begin{quote}
No momento em que iniciou no estudo da mecânica, Ulrich sentiu um
entusiasmo febril. Para que se precisa do \emph{Apolo} \emph{del}
\emph{Belvedere}, se temos diante dos olhos novas formas de um
turbo"-dínamo ou jogo de pistões de uma máquina a vapor? Quem se
encantaria com a milenar conversa sobre o bem e o mal depois de
constatar que não são ``constantes'', mas ``valores funcionais'', de
forma que o valor das obras depende das circunstâncias históricas, e o
valor das pessoas depende da habilidade psicotécnica com que avaliamos
suas qualidades? (p. 55-56)
\end{quote}

Eis expresso nesse trecho um rompimento de mentalidade de importância
digna de nota. A técnica relativiza e põe em seu nível de equivalência
-- portanto, destrói suas especificidades, suas qualidades -- dois
campos que eram tidos até a modernidade como diferenciadores das
sociedades: a arte e a reflexão. Algo imponderável tinha se perdido, diz
o narrador, como ``quando fios de novelo se desmancham'', ou como quando
``uma orquestra começa a desafinar''. Não era algo ruim, apenas ``havia
ruindade demais misturada ao que era bom, engano demais na verdade,
flexibilidade demais nos significados'' (p. 77). Se o mundo é dinâmico,
as noções de bem e mal, que fundamentam a ética do mundo \emph{de
qualidades} a ser destronado, também não têm um sentido sólido, mas se
modificam e se relativizam, vão paulatinamente se esvaziando -- já que
não podem se enquadrar nas constantes como leis da física. O viés da
técnica tende a ser medida para tudo. O que funciona é bom.
Característica inerente da Razão instrumental moderna criticada
pioneiramente por Adorno e Horkheimer (1986).

\begin{quote}
E assim, já no tempo em que Ulrich se tornou matemático, havia pessoas
que profetizavam a derrocada da cultura europeia, porque nenhuma crença,
nenhum amor, nenhuma candura restavam do ser humano; e
significativamente todos foram maus matemáticos na juventude e nos anos
escolares. Isso provou para eles, mais tarde, que a matemática, mãe da
ciência natural exata, avó da técnica, também é mãe ancestral daquele
espírito do qual finalmente brotaram os gazes venosos e os pilotos de
guerra. Só os próprios matemáticos e seus discípulos, os cientistas
naturais, que sentiam em suas almas tão pouco disso tudo quanto os
corredores de bicicleta, que pisam no pedal e nada veem do mundo senão a
roda traseira do concorrente diante deles, viviam na ignorância desses
perigos. Ulrich, porém, com certeza amava a matemática, por causa das
pessoas que não a suportavam. Era menos um cientista do que alguém
humanamente apaixonado pela ciência. (\versal{MUSIL}, 2006, p. 59)
\end{quote}

A crítica empreendida pelo narrador nos deixaria entrever um paralelo
com a crítica à razão formal elaborada pela Teoria Crítica. O trecho,
como o narrador do romance, é denso. Mas Musil parece mais irônico e
ambíguo, parecendo até escarnecer de quem vê relação entre o formalismo
matemático e a barbárie. Ele deixa entender que as pessoas de sua época
que veem na matemática ``a mãe ancestral daquele espírito do qual
finalmente brotaram os gases venenosos e os pilotos de guerra'' (p. 59),
em verdade devem ter sido maus matemáticos. Como vimos, a intenção
estética e ética de Musil é exatamente explorar essa marcha num sentido
transcendente. Tanto Adorno e Horkheimer, ao identificar na própria
Razão moderna a barbárie moderna, quanto Musil, com seu narrador, veem o
desenvolvimento da razão técnica como processo de indiferenciação, de
abstração das consequências concretas. A diferença é que Musil e Ulrich
veem possibilidades positivas de exploração nesse terreno, desde que
haja uma síntese com um terreno de contraponto: o terreno que ele chama
de não"-racioide.

Mas seria preciso apreender o problema contido no seio da Razão moderna,
de cujo espírito podem brotar tanto técnicas consideradas
``construtivas'' quanto as mais destrutivas e desumanas. E na
modernidade capitalista, construtivo e destrutivo são muito
relativizados e se distinguem muito dificilmente, assim como para Ulrich
bom e mau são relativos. A comparação do cientista com o corredor de
bicicleta expressa bem a unidimensionalidade do desenvolvimento
científico, do progresso que cada vez mais se torna um \emph{seguir
adiante sem que se permita perguntar ``para quê}?''. Karl Kraus parece
mais crítico nesse sentido: ``eu vi o progresso em ação na pessoa no
engenheiro. Graças a ele, estamos indo mais rápido. Mas para onde vamos?
No que me diz respeito, minha necessidade mais iminente é me encontrar
comigo mesmo.'' (1993, p. 142). Kraus não era otimista com essa
velocidade trazida pela técnica como um seguir adiante inexorável. Essa
velocidade era vista de tal modo como criadora de um afastamento social
-- ao mesmo tempo que em aparência tudo aproxima -- que ``para a alma
moderna, que atravessa brincando oceanos e continentes, nada é tão
impossível quanto encontrar a ligação com as almas que moram na outra
esquina.'' (\versal{MUSIL}, 2006, p. 246), diz o narrador elogiando os
sentimentos da empregada de Diotima, Raquel, pelo assassino sexual
Moosbrugger, do qual a sociedade da Kakânia não quer ouvir falar.

Ulrich é um personagem, como diz Musil em seus \emph{Diários} que quer
se tornar um grande homem. O método é não ter qualidades, não se pregar
a nada que diga respeito àquela sociedade tradicional, substituir os
``conceitos antiquados de gênio e grandeza humana'' pelo esporte e a
objetividade (\versal{MUSIL}, 2006, p. 64). Talvez pudéssemos crer simplesmente
que seu modo de pensar, ou sua vida à deriva se justificaria no ``forte
componente autista'' de seu espírito, como o próprio Musil (1981b, p.
90) o caracteriza numa anotação esparsa, quando ele ainda chamava seu
personagem principal de Anders. Segundo Bouveresse (2001), poderíamos
ver em Ulrich um homem superior que, não descobrindo nenhuma forma de
engajamento digno de suas capacidades, ``escolhe o repouso da alma ou a
inação, ficando numa espécie de preparação indefinida a uma ação
hipotética.'' (p. 100).

Musil parece pensar Ulrich como alguém que se levanta contra a sociedade
de qualidades fixas, aos moldes de um pré-moldado ao qual as pessoas
precisam se encaixar ao entrar no social, uma espécie de fôrma que se
impõe às pessoas. Mas não pensemos que ele está falando da forma"-sujeito
burguesa. Embora esses modeladores da alma também digam respeito, para
ele, a uma moral burguesa, ele via essa coerção que pretende fixar as
subjetividades em caracteres fixos como fruto da sociedade passada. O
mundo de infinitas possibilidades como era o moderno poderia romper com
essa ``cofragem'':

\begin{quote}
No fundo, poucos sabem, no meio da sua vida, como se tornaram aquilo que
são, com seus prazeres; sua visão de mundo, sua esposa, seu caráter,
profissão e realizações, mas têm a sensação de que já não se poderá
mudar lá muita coisa. Até se poderia afirmar que foram traídas, pois não
se encontra em lugar algum uma razão suficientemente forte para tudo ter
sido como é; poderia ter sido diferente; os acontecimentos raramente
dependeram delas [\ldots{}] Assim, na juventude ainda jazia à frente
delas algo como uma manhã inesgotável, cheia de possibilidades e de
vazio por todos os lados; mas já ao meio"-dia aparece de repente algo que
pode pretender ser a vida delas. (Musil, 2006, p 153)
\end{quote}

Portanto, o \emph{homem sem qualidades} pretende um personagem para
explorar essa ``manhã inesgotável, cheia de possibilidades e vazio por
todos os lados.'' Mas, cem anos depois, a questão que se poderia
levantar a Ulrich é se essa bela manhã cheia possibilidades não foi
aprisionada pela lógica mercantil. Nesse caso, as múltiplas
possibilidades de Ulrich acabariam por deixá-lo como uma forma de
subjetividade \emph{blasé}, que ganha terreno quanto mais a lógica da
mercadoria é entronizada na vida social. Assim, em vez de ser a
alternativa, um ponto de fuga, vira normalidade social burguesa, a ponto
de seu pai, cheio de \emph{qualidades} de um tempo passado, ganhar razão
quase cem anos depois com uma frase que àquela época era coisa de um
mundo tradicional decadente: ``aquele a quem permitem fazer tudo o que
deseja em breve não sabe mais o que desejar.'' (\versal{MUSIL}, 2006, p. 39).

Ulrich não é o burguês típico de sua época, não tem o espírito de
iniciativa, de investimento, o caráter de cálculo e ao mesmo tempo meio
temperado de nobreza como Arnheim -- que tampouco é o sujeito burguês
desenvolvido. O próprio Arnheim ainda traz caracteres que demonstram uma
subjetividade também amalgamada, cheia de \emph{qualidades.} Embora
burguês, ele não tem uma subjetividade totalmente \emph{enfôrmada} pela
vida mercantil. Parece mais aquele burguês típico do século \versal{XIX} com o
qual há quem se iluda como o ápice do sujeito burguês. Em verdade, ele
também está no meio do caminho entre dois mundos, mas dá mostras de
estar decidido a seguir o que a síntese nova de sua subjetividade lhe
mostra, ou seja, ele está preparado para seguir adiante. Arnheim podia
deixar seus poemas de juventude para segundo plano depois de descobrir
``que a vida de trabalho e organização é um poema maior que aqueles que
os poetas inventaram em suas bibliotecas'' (\versal{MUSIL}, 2006, p. 418). E
dedica"-se a construir esse poema da vida dos novos tempos, sem esquecer,
no entanto, as belas emoções da arte, partilhando com a classe média
espiritualizada o gosto pelo consumo de arte. As obras que escreve
parecem mostrar a tentativa de conjugar os mundos mais distintos numa
síntese equalizadora, pois em seus textos ``falava"-se de séries
algébricas e anéis de benzol, da concepção materialista de história e da
universalista, de suportes de pontes, da evolução da música, do espírito
do automóvel [\ldots{}]'' (\emph{Idem}, p. 418).

Já Ulrich não pretende ficar preso entre dois mundos, o tradicional e o
moderno, entrecruzamento que é em verdade o tecido social do romance.
Ele quer explorar esse entrecruzamento no sentido de uma vida que fuja
daquele padrão tradicional de existência das pessoas que ``adotam o
homem que apareceu nelas'' e suas ``experiências agora lhes parecem a
expressão das próprias qualidades, e seu destino parece ser seu próprio
por mérito ou desgraça. (\emph{Ibidem}, p. 153). Embora pretenda ser uma
alternativa ao mundo dito calcificado, sua consonância com a nova marcha
do mundo se dá inclusive em seu espírito ordenador, mas que ainda tateia
num mundo que lhe aparece como despersonalizado, sem nexos firmes. Com
diz Bouveresse, mesmo sendo \emph{sem} \emph{qualidades} definidas,
Ulrich vive em busca de ``uma identificação ou de uma individuação
\emph{interiores}, obtidas pela reapropriação \emph{moral} de suas
qualidades. A ausência de qualidades traduz a liberdade profunda do
homem em relação a suas determinações e a recusa de se deixar constituir
desde fora pelo que existe.'' (2001, p. 110). Mas essa escolha de viver
por hipótese, de libertar"-se das determinações externas pode também
desembocar numa postura estética perante a vida que beira à
indeterminação da própria subjetividade, o que significa um indivíduo
feito átomo flutuante num emaranhado de forças sociais abstratas que ele
não capta e com as quais não tensiona: ```A gente pode fazer o que
quiser', disse o homem sem qualidades para si mesmo, dando de ombros,
`que isso não tem a menor importância nesse emaranhado de forças!'"
(\emph{Ibidem}, p. 31).

A citação acima, do homem sem qualidades, poderia até ter sido dita por
qualquer personagem das peças de Beckett. Dito de modo contemporâneo,
com Beckett, teríamos: ``Nada a fazer'', que é como começa
\emph{Esperando} \emph{Godot}. A diferença é que o momento histórico de
Ulrich -- em que a forma social mercantil ainda estava para chegar em
seu estado maduro -- dava margem para ver possibilidades onde seus
contemporâneos \emph{de qualidades} viam decadência, o que significa que
a vida social permitia seu terreno de experimentações. Já as personagens
beckettianas têm o claro sentimento de estar caminhando inexoravelmente
para o ocaso, e caminham porque é preciso, porque é assim, porque uma
força que lhes é estranha os coloca na posição do ``nada a fazer''. E
esse sentido linear da história, como se cada etapa justificasse a
seguinte, independentemente do conteúdo dela, é tipicamente moderno. Mas
em Beckett as marcas da destruição são mais patentes, sobretudo
subjetivamente.

Seria errôneo dizer que Ulrich é vazio, que não reflete e não pensa.
\emph{O} \emph{homem sem qualidades}, descreve Walter, amigo de
infância:

\begin{quote}
É talentoso, cheio de vontade, despreconceituoso, corajoso, resistente,
destemido, prudente. Não quero examinar isso em detalhes, acho que ele
tem todas essas qualidades. Mas também não as tem! Elas fizeram dele
aquilo que ele é, e determinaram seu caminho, mas não lhe pertencem.
[\ldots{}] Qualquer má ação lhe parecerá boa em algum aspecto. É um
possível contexto que vai determinar o que ele pensa de um assunto. Para
ele, nada é sólido. (\emph{Ibidem}, p. 84)
\end{quote}

Apesar do despeito de Walter, a descrição casa bem com Ulrich de modo
caricatural, casa com essa subjetividade à deriva, aberto às
experimentações que o mundo moderno lhe parecia proporcionar. O problema
da subjetividade moderna que se expressa em Ulrich está também nas
grelhas de compreensão e reflexão, na falta de distinção de conteúdos,
na entrega apaixonada à destruição de todas as qualidades do passado, na
adesão otimista à marcha de seu tempo:

\begin{quote}
Estou convencido de que estamos galopando! Ainda estamos longe dos
objetivos, não nos aproximamos, nem os vemos, vamos nos perder ainda
muitas vezes nessa cavalgada e ter de trocar de cavalos; mas um dia --
depois de amanhã ou em dois mil anos -- o horizonte vai começar a
disparar ao nosso encontro, com um grande bramido! (\emph{Ibidem}, p.
242)
\end{quote}

A essa teleologia positiva e irrefletida da história, exposta por
Ulrich, Walter opõe a pergunta se ele quer que renunciemos a qualquer
objetivo de vida. Mas ``Ulrich perguntou"-lhe para que precisava de
sentido. Assim também se podia viver, na sua opinião.'' (\emph{Ibidem},
p. 242). A adesão à marcha moderna coloca a experimentação e o
formalismo do \emph{homem sem qualidades} em sintonia com uma espécie de
realização do Espírito hegeliano, que não procura senão realizar"-se na
história, na forma de uma teleologia positiva segundo a qual existe no
homem ``uma atitude para a mudança, e mais precisamente, uma atitude
para tornar"-se melhor, mais perfeito: um impulso na direção da
perfectibilidade.'' [Tradução Nossa] (\versal{HEGEL}, 2007, p. 177)\footnote{É
  essa mesma visão teleológica positiva que Kant expressará na sua
  \emph{Resposta à pergunta ``O que é Esclarecimento?}''.}.

Na discussão entre Ulrich e Walter, vem à tona a ideia da perda de
sentido da vida depois que tudo estiver racionalizado. Diz Walter a
Ulrich:

\begin{quote}
Você tem razão ao dizer que hoje não há mais nada de sério, racional ou
pelo menos compreensível. Mas por que não quer entender que essa
crescente racionalidade que invade todas as coisas é culpada disso? Em
todos os cérebros instalou"-se o desejo de ser cada vez mais racional, de
racionalizar a vida mais do que nunca, torná-la mais especializada; e,
ao mesmo tempo, a incapacidade de imaginar o que será de nós quando
tivermos tudo entendido, analisado, classificado, transformado em
máquinas e normas. As coisas não podem continuar desse jeito. (\versal{MUSIL},
2006, p. 245)
\end{quote}

Como entender a contraposição entre ``não há mais nada de sério,
\emph{racional}'' e ``essa crescente \emph{racionalidade} é culpada
disso''? Como a racionalidade pode acabar com o racional? Isso só é
possível se distinguirmos a Razão moderna da razão como capacidade de
reflexão, para ver que a Razão moderna é uma forma específica que ganhou
na história a capacidade de raciocínio do ser humano. Se não entendermos
assim, cairemos num aporia que nos remete novamente à problemática que
diz respeito ao lado obscuro da Razão: seu lado mercantil, que corrói
paulatinamente a capacidade do pensamento transcendente. A observação
interessante de Walter no trecho acima faz dele, no painel de Musil,
mais uma subjetividade tensa, mas aqui a tensão se dá em forma crítica,
talvez conservadora e cheia de \emph{qualidades}, portanto contrária à
experimentação de uma vida sem qualidades. De todo modo, Walter coloca
na discussão a própria questão do sentido que parece não considerado por
essa racionalização que pretende resolver todos os problemas humanos,
inclusive de ordem relacional. Alguém poderia objetar que não caminhamos
para essa racionalização da vida, mesmo com uma abertura subjetiva cada
vez maior para acolher as quinquilharias que a razão técnica nos
proporciona. Mas quase que ouvimos Walter perguntar ``de que adiantaria
a tecnologia descobrir formas inauditas de prolongamento da vida se a
própria vida vem se tornando um peso sem sentido para muitas pessoas?''.

Musil poderia ter escolhido para centro de seu romance o par Diotima e
Arnheim, o que não teria trazido talvez nada de inovador em termos de
romance, apesar de seus conflitos belamente descritos. Poderia ainda ter
centrado sua trama mais no casal de empregados, os que mais sofrem com
amarras concretas de uma dominação pessoal. Esse casal que tem uma bela
relação no romance, o casal que passa ao largo de todas as empreitadas
patrióticas e racionalizações dos ``grandes homens'', e que, mesmo
temendo a repreensão de seus senhores, entregam"-se um ao outro, mesmo
com os esforços de resistirem em função do medo: ``[\ldots{}] ela acabou
se descontrolando, e Solimão se esqueceu de sentir vergonha por ser
desajeitado, e a louca tempestade do amor varreu a escuridão.''
(\emph{Idem}, p. 642). Mas não. Não é o conflito de amor de Diotima, nem
sua ilusão com uma grandeza finda; não é o desejo de Bonadeia de
transpor os limites de uma vida regrada, enquadrada e de aparência; não
é o enquadramento rígido de Fischel ou Clementina; não é a estupidez
caricatural do general; não é o conflito entre Walter e Clarissa. Não é
o jovem representante do movimento antissemita. Poderíamos dizer que
essas são temáticas de certo modo não estranhas à literatura, quando
tratadas isoladamente. Nem mesmo pode"-se dizer que a experimentação do
\emph{outro estado} que atinge o ápice no amor incestual ente Ulrich e
Ágatha -- experimentação que no fundo pode ser vista também como um
sintoma do apagamento das distinções quando a sociedade mercantil se
mostra ela mesma \emph{sociedade sem qualidades --} seria uma temática
que teria o mesmo peso sem todo o conjunto da obra que narra a rachadura
do mundo tradicional.

Musil conseguiu algo de mais elevado. Conseguiu apreender, agarrar, no
dizer de Jorge Coelho, ``um desses momentos especiais [o alvorecer do
século \versal{XX}] onde tudo parecia perigosamente possível, tempo de
transição, no início, somente percebido por poucos. [\ldots{}] Uma
atmosfera completamente nova, prenhe de possibilidades,
irreversivelmente tão sedutora quanto assustadora [\ldots{}]'' (2013, p.
19). Se pudermos compará-lo a um grande pintor, arriscaremos dizer que
ele criou um políptico crítico -- realçando crise e crítica
especificamente no seu tempo. Mas um políptico cuja unidade não é a
narrativa em si, em termos de acontecimentos, mas a narrativa da crise
de uma forma de sociedade e de uma forma de subjetividade tradicional
que sofre rachaduras em contato com o movimento dinâmico moderno -- que
é o que fica em relevo. Nesse sentido, já a experimentação do
\emph{outro estado} e da \emph{vida por hipótese} de Ulrich ganha
sentido apenas no emaranhado desse políptico, ou seja, no próprio meio
social em que passado e futuro se entrecruzam num presente que consome
as subjetividades tradicionais para seguir adiante sobre novas bases.

A marcha moderna vai consumindo as subjetividades mais fincadas na
tradição cheia de \emph{qualidades}, também aquelas dispostas a se
despojar de suas \emph{qualidades} para seguir a marcha, e até mesmo
aquela que representa a tentativa de criar, inventar, alcançar uma
transcendência em relação a formas sociais que sufocam naquele tempo o
desabrochar da múltiplas possibilidades do indivíduo -- que é a
tentativa malfadada de Ulrich. E essa crise, ironicamente o narrador
relaciona com um desgaste ``da coesão que apoiava o contentamento
artificial das almas'' (\versal{MUSIL}, 2006, p. 562), e esse contentamento
artificial ele diz ser uma espécie de ``capital de giro do espírito''.
Seria mero acaso a comparação com o mundo da finança?

A \emph{K}a\emph{K}ânia (a cidade ao mesmo tempo \emph{Kaiserlich} e
\emph{Königlich,}ou seja, imperial e real) estaria em situação crítica
porque já não tinha esse \emph{capital} \emph{de} \emph{giro de vida,}
tinha perdido ``o prazer de viver, a crença em si mesmo e a capacidade
de todos os estados civilizados de divulgarem a útil ilusão de que têm
uma tarefa a cumprir.'' (\emph{Idem}, p. 563). Era um país
inteligente, com pessoas cultas que faziam as atividades banais e, como
as outras pessoas:

\begin{quote}
[\ldots{}] liam e ouviam diariamente algumas dúzias de notícias que as
deixavam de cabelo em pé e estavam dispostas a se irritar com elas, até
interferir, mas não chegavam a fazê-lo, porque alguns minutos depois
aquela excitação fora substituída por outras mais recentes. Como todas
as demais, essas pessoas se sentiam envolvidas por assassinatos,
homicídios, paixões, sacrifícios, grandeza, que corriam dentro de um
novelo que se formara ao seu redor, mas não chegavam a participar dessas
aventuras, porque estavam prisioneiras de algum escritório ou qualquer
profissão; e quando se viam livres, à noite, aquela tensão com a qual
não sabiam o que fazer explodia em diversões que não as divertiam em
nada. (\emph{Ibidem}, p, 563).
\end{quote}

Talvez não estranhássemos se esse trecho que expressa o ambiente da
crise da Kakânia estivesse nas reflexões de Adorno sobre a
\emph{Indústria} \emph{Cultural} (1986) ou nas reflexões de Debord
(1997) sobre o \emph{espetáculo}. A crise que invade a Kakânia -- que,
mesmo tendo relação com um momento da história da Áustria, é uma
cidade"-alegoria, uma cidade"-metáfora das tensões da vida moderna --
deve"-se ao ambiente criado pela encruzilhada em que se encontra. Ela
está presa a meio caminho da modernidade. Ela não é o modelo de cidade
superamericana, ``onde todo mundo corre ou para com um cronômetro na
mão.'' Onde ``Céu e terra formam um formigueiro varado pelos diversos
andares de ruas sobrepostas.'' Onde ``salta"-se de um meio locomotor a
outro nos pontos de junção, sem pensar, sugado e arrebatado pelo ritmo
dos veículos [\ldots{}]'' (\versal{MUSIL}, p. 49). Kakânia ainda não é uma cidade
em que ``Perguntas e respostas articulam"-se como peças de máquina'',
onde cada pessoa ``tem apenas tarefas determinadas, as profissões estão
agrupadas em lugares certos, [onde] come"-se em pleno movimento
[\ldots{}]'' (\emph{Idem}, p. 49). O ritmo da Kakânia não é esse descrito
pelo narrador em relação à ideia de uma cidade superamericana. Ela está
num estágio menos avançado da marcha moderna, tem ainda pretensões mais
lentas, apreciando o luxo sem necessariamente ser como Paris, não tendo
ambições de ser potência econômica mundial, gastando muito com o
exército, mas apenas para ``continuar sendo a penúltima das grandes
potências (\emph{Ibidem}, p. 51). Apesar de a Kakânia não seguir aquele
modelo, pressente ter que, senão segui"-lo à risca, tomá-lo como
horizonte a ser seguido.

\begin{quote}
A modernidade vienense não é um modernismo triunfante e seguro de si
mesmo. Conserva sempre presente o sentimento extremamente forte de uma
perda, de uma decadência contra a qual se devia tentar reagir, de um
mundo que desmorona e um futuro ainda vago. Os modernos vienenses se
embrenham na via da modernidade com a consciência de uma
\emph{necessidade que sentem quase como uma fatalidade.} Raramente
pensam em reduzir este mal"-estar a um problema político e consideram a
modernidade primeiramente em termos estéticos, éticos, psicológicos,
filosóficos e sempre individualistas. (\versal{LE} \versal{RIDER}, 1993, p. 41[Grifos
Nossos])
\end{quote}

O \emph{homem sem qualidades} está imerso nesse meio do caminho em que
uma pedra precisa ser inexoravelmente tirada para que o movimento do
progresso possa passar, embora essa pedra signifique para ele os restos
de sentido da vida tradicional que precisam ser superados. Para o
\emph{homem sem qualidades}, o problema de seu mundo não é tanto a
racionalidade, mas as amarras do mundo tradicional -- nesse sentido, sua
preocupação é muito moderna. Claro que ele não deixa de ter razão ao
escarnecer da risonha vontade dos sujeitos de se agarrar ao passado.
Como diz Edmonde Roux: ``A obra constrói um meticuloso inventário dos
particularismo cacanianos, coisas engraçadas do meio militar,
singularidades da administração'', os modos e gestos velhos de um peso
desmesurado do passado. ``O próprio Ulrich é nomeado como Secretário de
honra e colaborador voluntário de Sua Alteza o conde Leinsdorf, o que é
engraçadíssimo''. (1981, p. 182). Mas Ulrich quer responder aos impasses
daquela Kakânia lançando"-se de braços abertos na marcha moderna para
explorar todas as possibilidades que o passado pretendia encerrar. Como
já vimos, não deixa de ser uma postura de vanguarda na época.

Conforme explica Jacques Le Rider, o \emph{homem sem qualidades} da
Kakânia está imerso numa modernidade vienense. Embora atrasada em
relação a outras potências, no tempo histórico de \emph{O homem sem
qualidades,} a Áustria estava passando por uma modernização tão rápida a
partir dos anos 1870 que ``teria finalmente atingido o mesmo nível que
os outros grandes países da Europa em 1913.'' (\versal{LE} \versal{RIDER}, 1993, p. 31).
Coincidentemente, é um dia do mesmo ano de 1913 que é descrito no começo
do romance com sua descrição das condições meteorológicas e uma ``Mancha
escura de transeuntes [que] formavam fios nevoentos'' (\versal{MUSIL}, 2006,
p. 27): as massas modernas. Mas se a Áustria conseguiu esse
desenvolvimento condensado, não foi por graças de uma ``elite
protestante portadora da modernização alemã'', mas por ação de um
``capitalismo de Estado'' que ``usufrui de incontestável popularidade'',
como explica Le Rider (1993, p. 33).

É a essa marcha que a chamada ação patriótica, de matiz conservadora,
pretende se opor. Nesse sentido, é uma ação tão quixotesca, tão
\emph{ridícula} -- para retomar a situação dos personagens de Molière --
quanto ineficaz frente à força da modernidade, do processo modernizador
para o qual a estrutura pesadamente passada da Kakânia era um entrave. E
ela é ridícula não pela oposição em si, mas pela denegação total do
quadro geral da sociedade que tem em seu subterrâneo a chegada de um
novo tempo, um novo espírito, algo que a ``corte'' deslocada no tempo a
discutir a \emph{ação patriótica} não consegue ou não quer enxergar: ``O
mundo estava sendo abalado por uma série de acontecimentos, e quem, pelo
fim do ano de 1913, estivesse bem informado, teria a imagem de um vulcão
em ebulição, embora a pacífica atividade por toda parte sugerisse que
não haveria novas erupções'' (\versal{MUSIL}, 2006, p. 412). Segundo Le Rider,

\begin{quote}
A maior parte dos ``modernos'' vienenses compartilha do sentimento
melancólico de Hofmannsthal, que declara em 1906: ``A característica de
nossa época é a ambiguidade e a indeterminação. Não pode apoiar"-se senão
em bases que afundam, sem perder a consciência de que tudo afunda ali
onde as gerações anteriores acreditavam possuir embasamentos sólidos''
(1993. p. 41).
\end{quote}

Musil coloca Ulrich como oposição a esse espírito que vê decadência na
rachadura desses embasamentos sólidos, nessas qualidades. O atributo
\emph{homem sem atributos} de Ulrich deve ser entendido no sentido lato.
Ele pretende não estar pregado a nada. Pretende estar livre para aposta
no futuro, por ver nos críticos da marcha moderna os partidários da
tradição \emph{enfôrmadora}. E muitos seguiram esse caminho da aposta no
futuro, vendo na sociedade que atacavam como símbolo do arcaico, do
passado. Lendo as aventuras do \emph{homem sem qualidades} cem anos
depois, podemos ver que cada geração que o sucedeu na vida capitalista
foi tendo esse sentimento de estar apoiada em bases movediças, as mesmas
bases que outrora forneciam possibilidades sólidas de subjetivação. A
diferença é que a melancolia criadora que poderia surgir é em verdade
diluída numa \emph{euforia} \emph{perpétua} (\versal{BRUCKNER}, 2002), num
pretenso dever de felicidade e gozo da vida, como veremos no último
capítulo.

De qualquer modo, o \emph{homem sem qualidades} do contexto histórico de
Ulrich não teve como atravessar cem anos de dinâmica capitalista e de
racionalidade sem desembocar numa indeterminação. Nesse sentido, o
\emph{sem qualidades} já não tem o potencial transcendente daquele
momento. Pode tanto designar sem atributos nenhuns, como sem atributos
construídos numa troca rica com o social, que já não possibilita essa
troca senão em esferas cada vez mais marginais. A aposta de Musil com
seu personagem Ulrich não se realizou. A experimentação no terreno do
\emph{outro estado} não tornou a vida mais rica de verdadeiras
possibilidades. A tentativa de Ulrich de viver por hipótese, de abrir"-se
a várias possibilidades de enriquecimento só era possível porque ele
mesmo carregava em sua subjetividade camadas da \emph{sociedade de
qualidades} que ele pretendia superar, da qual pretendia se despojar.
Como diz Marthe Robert, ``Ulrich ainda é abundantemente qualificável,
ainda que seja sem qualidades. Tem um nome, uma família, um país, uma
posição, relações, uma papel pessoal a desempenhar''. (1981, p. 195).
Dito de outro modo, tinha um mundo \emph{de qualidades} a rachar. Quando
é a própria sociedade mercantil que corrói paulatinamente todas as
qualidades, o \emph{homem sem qualidades} perde sua base de atrito com a
realidade, base de seu tensionamento com o social. Sem um mundo \emph{de
qualidades} a atacar, ele vira átomo flutuante e se desintegra, pois
contra que \emph{qualidades} fixas do mundo ele vai reagir? Assim, cem
anos depois, sua subjetividade flutua num terreno movediço, que o faz
tatear em todo lugar em busca de algo que lhe dê uma substância, uma
qualidade essencial. E é o universo mercantil, o mundo encantado das
mercadorias que se abre como mundo de exploração das possibilidades. Um
mundo em que o Ulrich contemporâneo pode se realizar plenamente como
\emph{homem sem qualidades,} sem atributos, porque aqueles que a
mercadorias lhe promete são acidentais e intercambiáveis, indefiníveis
-- que dependem de indivíduos com uma estrutura subjetiva ``coloidal''.
Ulrich, em certo sentido, parece ser o adulto em que se transformou
\emph{O Jovem} \emph{Törless}, personagem do primeiro romance de Musil,
que apresenta jovens vivendo confusamente seus sentimentos e seu estar
no mundo num internato. Törless se sentia dilacerado entre dois mundos,
o burguês, sólido, ``no qual tudo acontecia de modo sensato e regrado,
como estava habituado em casa'', e um outro ``aventuroso, sombrio,
misterioso, sanguíneo'' (\versal{MUSIL}, 2003, p. 45).

Ao mesmo tempo, o contexto de \emph{o homem sem qualidades} é herdeiro
da \emph{entronização} da mercadoria operada no século \versal{XIX} e visível nas
passagens parisienses e nas exposições universais, segundo Benjamin. É o
tempo em que a mercadoria desenvolve um \emph{sex"-appeal} como se fosse
orgânica, viva. A mercadoria deseja ser adorada. (\versal{BENJAMIN}, 2006, p.
45)\footnote{Evidentemente, essa visão de Benjamin sobre o fetichismo se
  distingue da que desenvolvemos com base em Marx -- que via o
  fetichismo na própria produção. Benjamin aqui desenvolve as
  \emph{manhas} \emph{teológicas} da mercadoria como algo no nível da
  circulação, mas sabemos que esse inorgânico já defende seus direitos
  de cadáver na própria produção.}. Assim, esse contexto de crise da
aurora do século \versal{XX} na Kakânia advém também do desenvolvimento dessa
entronização na vida social. A Razão instrumental e mercantil -- ao
mesmo tempo que criava ilusórias esperanças de progresso humano --
causava tal transtorno na base social e subjetiva que ``Ninguém sabia
exatamente o que acontecia; ninguém podia dizer se seria uma nova arte,
um novo homem, uma nova moral, ou talvez uma alteração das camadas
sociais''. É esse o contexto que o narrador descreve como
\emph{revolução no espírito} que se expressa também no fato de que:
``[\ldots{}] por toda parte pessoas erguiam"-se para combater as coisas
antigas.'' (\versal{MUSIL}, 2006 p. 76).

Por fim, a figura do narrador de Musil é central no romance. Ele é
também a porta reflexiva, ele critica sutilmente a denegação da
realidade de que dão mostras os personagens iludidos com grandes
empreitadas embaçadas, num contexto de avançar da modernidade com seu
aspecto destruidor e às vésperas de um acontecimento que é o primeiro de
dois atos destrutivos que vão fundar a modernidade do século \versal{XX}: a
Primeira Guerra. Eis outro aspecto que torna a \emph{ação patriótica}
ridícula. Já não era mais possível afirmar uma identidade em trapos -- o
próprio antissemitismo que vemos sendo adubado no romance era reflexo da
corrosão do passado pela marcha moderna, cujas consequências sociais,
fruto de uma forma abstrata de organização da vida, não são normalmente
vistas pelos indivíduos como consequência dessa abstração danosa e real,
só podendo ser apreendidas pela projeção em determinados grupos sociais.

Assim, é preciso ver \emph{O homem sem qualidades} como um romance
também construído sob o esteio de um momento catastrófico,

\begin{quote}
vivido na inconsciência e na irresponsabilidade, um ano antes da entrada
da Áustria na guerra, e portanto, do fim de uma época, a começar pela do
Império: 1913-1914. E é também a situação política, social, intelectual
da Áustria, sua aparente capacidade para superar as diferenças e
disparidades, sua consciência de si mesma e do que a separa da Alemanha,
sua alma[\ldots{}] (\versal{COMETTI}, 2001, p. 74).
\end{quote}

Em suma, no romance, vemos ``inconsequências de uma época em relação à
qual a Áustria não passava de um sintoma'' (\emph{Idem}, p. 74). Nesse
contexto de ``inconsciência e irresponsabilidade'', a chamada ``Ação
paralela'', a invenção do romancista, não passa de uma tentativa
``risível e patética'' para substituir, por meio de uma ação narrativa
de improviso, a improbabilidade ``de toda e qualquer ação real, com a
convicção ou pelo menos a esperança de que as ideias -- a `grande ideia'
-- darão à história, chegado o momento, o impulso que lhe falta, ao
mesmo tempo que condena os indivíduos, tanto quanto os Estados ou as
nações, a uma impotência sem remédio''. (\versal{COMETTI}, 2001, p. 67) No
romance, a chamada ação heróica ou Paralela é sintoma de uma ausência de
objetivo que preside a narrativa e serve de pano de fundo para os
esforços de manter o mínimo de ``experiência vivida sem a qual a noção
mesma de indivíduo perderia seu sentido (2001, p. 103).

No antepenúltimo capítulo, é pelo personagem Lindner que o narrador traz
à tona a temática do progresso. Em meio a dados acerca dos progressos
materiais que o século \versal{XIX} trouxera, o narrador põe em cena o caráter
``pessimista'' de Lindner frente às promessas de que o progresso
material, industrial e comercial trariam um alargamento do espírito:

\begin{quote}
O estudante Lindner, pálido e nada abundante, que chegava mesmo a sofrer
fisicamente com o próprio crescimento, possuía uma repulsa instintiva
por aquela confiança cega, aquela mania de prosperidade, aquele
liberalismo fatidicamente alegre. [\ldots{}] mas tanto maior a
lucidez com que contemplava o outro lado da evolução, o apodrecimento de
uma ideologia que, em nome da liberdade espiritual, colocara o
livre"-comércio à frente das atividades humanas, para abandonar então o
livre espírito ao comércio livre; \emph{e Lindner farejava a catástrofe
espiritual que, de fato veio a acontecer. Essa crença no desastre em
meio a um mundo satisfeito com seus progressos era a mais forte de suas
qualidades.} (p. 1250, [Grifos nossos])
\end{quote}

O trecho em destaque é quase a divisa empunhada pela Teoria Crítica.
Como não ver nesse trecho a explícita crítica da usura do pensamento
crítico? O que significaria colocar o livre"-comércio à frente das
atividades humanas e deixar o livre espírito entregue ao comércio livre?
Poderíamos também refazer a pergunta no mesmo tom: o que significa
colocar de modo paulatino, mas contínuo, um ente vazio -- o dinheiro em
movimento, a mercadoria -- como mediador das relações sociais senão
abrir caminho para que as próprias relações sociais se tornem
paulatinamente leves ou líquidas e sofram os efeitos da indistinção,
desse vazio mediador? Parece que é essa qualidade de ver o
\emph{desastre} exatamente no momento em que o mundo se vê satisfeito
com seus progressos que precisa ser reabilitado. Ou seja, a profundidade
crítica não pode ficar esmaecida perante a aparência de uma marcha
triunfal da modernidade capitalista e sua abundância de mercadorias --
abundância de vazio.

Na obra, as reflexões do narrador merecem sempre atento olhar: crítico,
irônico, mordaz, com tonalidades satíricas, ele expõe mais um espírito
do tempo nos vários personagens do que um personagem"-tipo simplesmente.
O narrador parece muitas vezes um crítico entre o entusiasmo racional do
\emph{homem sem qualidades} e o pessimismo dos sujeitos voltados para as
qualidades do passado. Mas nem Ulrich tem sucesso em sua empreitada de
viver a vida por hipótese, nem as subjetividades calcadas nas
\emph{qualidades} fixas do passado conseguem resistir à força da marcha
moderna. Nesse insucesso, Ulrich acaba se tornando \emph{blasé} tão logo
a própria marcha moderna corrói as qualidades que ele mesmo pretendia
superar.

Esse sentido de \emph{blasé,} o \emph{blasement,} como já dissemos, não
tem relação com o sentido corriqueiro de apatia ou inação. O sujeito
\emph{blasé} pode ser muito ativo. É um estado de espírito ligado ao
dilaceramento da capacidade distintiva, seja em relação ao gosto, seja
em relação à vida. O \emph{blasement} está sempre ligado ao excesso de
excitações -- por exemplo, muitos temperos aleatórios em uma comida não
permitem distinguir mais nenhum --, ao abuso de excitações que vão
construindo uma subjetividade indiferente às emoções vivas, indiferente
ao gosto. Tem relação com o sentido latino de \emph{lanx satura}, prato
excessivamente cheio de frutos que satura. Mas não satura a ponto de
deixar o sujeito sem ação ou reação física. A sua inação se dá no
terreno do pensamento mais exigente. Em suma, esse \emph{blasement} é
parente próximo da letargia.

Obviamente não caberia aqui relacionar esse estado subjetivo que
encontramos em Ulrich diretamente com um caráter \emph{blasé}. Parece
ter ficado claro que lendo \emph{O homem sem qualidades} em relação com
sua época, trata"-se de uma experimentação de ruptura com o mundo
tradicional identificado com qualidades, caracteres coercitivos. Mas
lendo \emph{O homem sem qualidades} hoje, com Ulrich cem anos mais
velho, é possível colocar sua postura sem atributos essenciais como em
sintonia com o mundo daquele frente ao qual nada pode manter suas
características próprias: o dinheiro. Evidentemente, a experimentação de
Ulrich não tinha qualquer relação com o mundo mercantil, e também não
estávamos ainda numa sociedade em que o consumo se revestia de aspectos
de gozo e obturação subjetiva. Mas hoje, essa experimentação de um
\emph{mundo sem qualidades} leva a uma subjetividade \emph{blasé.}
Segundo Simmel, aquele que considera possível obter todos os bens
bastante diferentes entre si por meio de uma quantia que os igualha,
deve se tornar \emph{blasé --} embora precise ser proativo para angariar
riqueza vencendo os concorrentes à espreita. Isso pode ser dito também
de qualquer conteúdo que pretenda operar uma abstração totalizante, como
a ciência de que tanto gostava Ulrich. Vejamos o que diz Simmel (2009)
do \emph{blasé}:

\begin{quote}
Enquanto o cínico [\ldots{}] ainda tem capacidade de reação, embora
com a perversidade de encontrar no movimento descendente dos valores um
certo atrativo à vida, o \emph{blasé}, conforme a esse conceito -- que
certamente nunca está realizado -- é de fato incapaz de sentir
diferenças de valores, para ele, todas as coisas estão banhadas de uma
tonalidade uniformemente morna e acinzentada; [\ldots{}] O que é
decisivo nessa atitude, \emph{não é a desvalorização das coisas em
geral, mas a indiferenciação frente a suas variações específicas}, uma
vez que é delas que rebenta precisamente toda a vivacidade do sentir e
do querer, que é recusada ao \emph{blasé}. (p. 308, [Tradução
Nossa][Grifos Nossos])
\end{quote}

Poderíamos talvez dizer que o homem \emph{blasé} é um outro nome para o
homem sem qualidades tipicamente moderno, parente mais próximo do
restolho de individualidade que algumas peças de Beckett trazem à cena.
Mas quem é o homem sem qualidades do século \versal{XXI}? Como se poderia
escrever um \emph{homem sem qualidades} na contemporaneidade? Com as
tensões próprias à contemporaneidade? A subjetividade contemporânea
estaria mais próxima da forma"-sujeito burguesa ideal? São perguntas que
tentaremos enfrentar mais à frente, junto de Beckett, porque a
exploração das possibilidades na forma da escrita do romance de Musil,
como grande canteiro de obras, parece vinculada a um momento histórico
que, embora na catástrofe da guerra, ainda imagina uma aurora com novas
possibilidades. Talvez Musil vivo, cem anos depois, desse contornos bem
distintos ao \emph{outro estado} e à \emph{vida por hipótese} de Ulrich.
Sendo os tensionamentos do \emph{homem sem qualidades} do século \versal{XXI} de
outra ordem, não seria impossível imaginar o ``dissecador da alma
humana'' escrevendo um romance em que a decadência não se dá mais pelo
apego a um passado em ruína, mas pelo apego obsessivo em seguir em
frente rumo a um futuro que já se apresenta como ruína. A decadência
talvez fosse vista por ele como excesso de modernidade que já consome o
futuro para manter a vida social mercantil. Quem sabe ele veria que a
marcha moderna da vida social mercantil, ao se desdobrar, expressa na
realidade uma crise do potencial crítico que a Razão
instrumental"-mercantil só aprofunda. Quem sabe também veria em ato a
experiência do perecimento da cultura: ``Mas uma cultura não pode
perecer devido ao esgotamento de seu teor de espírito?''. (1981a, p.
492). E assim, daria possivelmente razão da Kraus e sua sátira mordaz:
``O progresso pode empreender de tudo, mas acho que não se revelará
muito mais eficaz, quando tiverem lugar catástrofes do espírito, do que
um geólogo num terremoto.'' (1993, p.145). Se a razão não se despregar
da Razão instrumental e mercantil, não se poderá acusar de apocalíptico
quem vir se desenhar no horizonte uma mistura de catástrofes ambientais,
sociais e do espírito.

Nesse sentido, a forma da exploração das possibilidades de Musil se
distingue da forma do encerramento das possibilidades que é a forma
estética da tautologia de Beckett. Mas antes tentemos refletir como a
forma"-sujeito burguesa aportou por nossas terras.

\chapter*{Forma-sujeito burguesa no Brasil: entre \emph{natureza} e \emph{cultura}}
\addcontentsline{toc}{chapter}{\large\versal{FORMA-SUJEITO BURGUESA NO BRASIL:\\ \small{ENTRE \emph{NATUREZA} E \emph{CULTURA}}}}
\hedramarkboth{Forma-sujeito burguesa no Brasil}{}

\begin{flushright}
\scriptsize{Contudo, do ponto de vista de sua estrutura econômica, o Brasil da
metade do século \versal{XIX} não diferia muito do que fora nos três séculos
anteriores.

\emph{Celso Furtado}

\medskip

Quem percorre o Brasil de hoje fica muitas vezes surpreendido com
aspectos que se imagina existirem nos nossos dias unicamente em livros
de história; e se atentar um pouco para eles, verá que traduzem fatos
profundos e não apenas reminiscências anacrônicas.

\emph{Caio Prado Júnior }}
\end{flushright}

Como desenvolver um estudo em que se dá tamanho peso às vicissitudes da
vida social concreta sem fazer algumas considerações sobre o solo
concreto em se vive? Se tais vicissitudes se constituem muitas vezes em
oposição muda ao desenvolvimento de uma vida social e subjetiva moderna?
Parece digno, portanto, de interesse fazer uma reflexão sobre como a
forma"-sujeito burguesa aterrou por essas plagas, sobretudo porque por
aqui tais vicissitudes se mostram tanto mais emaranhadas num complexo de
relações quanto mais inibidoras do florescer da vida social e subjetiva
moderna. Adentrar por uma reflexão desse nível poderia nos levar a
caminhos tortuosos e até nos desviar do leito que escolhemos para nossa
empreitada, porque fazer considerações sobre a forma"-sujeito burguesa em
solo brasileiro implica também adentrar pelo terreno complexo da própria
forma"-social moderna por esses lados tropicais.

Começo com uma reminiscência de uma viagem ao sertão que talvez espante
espíritos acostumados há tempos com a ideia de uma marcha moderna no
Brasil. Partiríamos à meia noite para aquelas terras que só através dos
ralos relatos, principalmente de meu acabrunhado pai, desenhavam"-se em
minhas lembranças. Era ele quem me falava das viagens que fizera em
criança com meu avô entre a serra e o sertão para negócios que muitas
vezes não envolviam somas monetárias, embora fossem trocas de
equivalentes. Saía com frutas, rapadura, farinha, e voltava com
carneiro, vaca, queijo, dizia ele.

Ali no pau"-de"-arara, iríamos sentar ou deitar entre bananas e verduras
serranas levadas para o sertão que parecia ainda mais distante naquele
meio de transporte moderno e ao mesmo tempo de colorido tão arcaico, que
faria as menos de 30 léguas que nos separavam parecerem mais esticadas.
Mas o importante é que chegamos por volta das sete horas da manhã
daquele julho em que o verde da estação chuvosa do início do ano já se
tornava acinzentado. A paisagem era triste e só foi encontrar paralelo
quando tempos depois me embrenhei pelos \emph{Sertões} de Euclides da
Cunha descrito como ``meio ingrato'', onde o sol ``é o inimigo que é
forçoso evitar, iludir ou combater'' (\versal{CUNHA}, 2009, p. 117). Do mesmo
modo, os parentes tios mais velhos que conhecia ali não pareciam se
afastar daquela descrição feita por Euclides de um sertanejo que esconde
sua força e sua resistência numa ``postura normalmente abatida'',
fazendo dele um ``Hércules"-Quasímodo'' (\emph{Idem}, p. 207).

A chegada de cerca de 40 pessoas a um local de grandes extensões de
terras e de poucas casas era um acontecimento. As pessoas acorriam de
vários lados para ver, para perguntar, só para olhar, para espreitar
aquela gente da cidade -- que nem éramos. Enquanto isso, as conversas
dentro das duas humildes e avarandadas casas, onde a maioria ficou,
instalada em redes por dentro e por fora, corriam soltas. Os mais velhos
relembravam histórias de travessias de gado do sertão à serra, que
enchiam de certa curiosidade quem disso nunca tinha ouvido falar. Outros
combinavam pescarias, banhos no açude. Depois de um dia de reencontros,
alegrias e combinações, chegara a noite. E o que fazer naquele pedaço de
mundo cortado de todo o resto pela falta de eletricidade? As conversas e
anedotas por entre as redes já se enchiam de intervalos quando alguém
fala de um forró na casa de uma velha prima de nossa avó. Fomos para lá.

Dentre aquelas paredes de taipa, nada mais que uma dezena de casais
dançavam ao som de uma radiola a pilha, enquanto a galhofa ocupava outro
tanto. Não demorou e as pilhas não aguentaram o peso da sanfona de Luiz
Gonzaga e acabaram a festa. Mas só até que alguém se propusesse a passar
o açude de canoa e comprar na bodega as raiovaques novinhas para a festa
poder retomar. Para nós, adolescentes que morávamos em Fortaleza ou
mesmo na pequena cidade da serra, aquilo já aparecia como uma bela
lembrança, mas pitoresca. Não passava pela cabeça de ninguém viver
aquela rude vida -- que muitos de nós até ridicularizávamos. Mas antes
que nos lembrem que não se trata aqui de um ensaio literário, resumamos
a história para que apareça nosso objetivo.

Vamos à segunda viagem três anos depois. Nossa chegada foi uma festa.
Pescarias, banhos, passeios de açude eram uma festa. Mas algo tinha
mudado. Em grande parte das casas em que andamos: uma televisão, um
aparelho de \versal{DVD}, e um monte de gente esparramada em frente a artistas de
todos os gostos. Na época, aquilo aparecia, para nós que tínhamos talvez
também o mesmo hábito, como descaso com os parentes e conhecidos vindos
de longe, como falta de atenção. Mais três anos depois dessa segunda
viagem, misturado àqueles que tinham ainda atividades ligadas ao campo,
via"-se um séquito de adolescentes e adultos -- para quem as rudes
atividades de agricultura já não consistiam uma verdadeira atividade --
vivendo em casas onde a aposentadoria rural fazia verdadeiros milagres.
Inclusive o de dotar aqueles jovens de aparelhos impensáveis há bem
pouco tempo num ambiente pouco monetarizado. A primeira viagem foi em
1997. Não participei de mais viagens além dessas três. Escuto os relatos
dos meus tios que vão todos os anos e dizem entusiasmados que moto e
celular é o que não falta. O que não impede que ainda haja certa
atividade agropecuária.

Esse relato não visa partir de uma forma de vida aparentemente
particular como forma de compreender o geral que certamente se dá sobre
outras bases; também não visa uma crítica moral do consumismo daqueles
que têm pouco acesso à riqueza mercantil, muito menos uma crítica aos
moldes daqueles que criticam programas governamentais de divisão da
riqueza mercantil. Nosso objetivo é muito mais colocar a situação nesses
termos: se o atraso pode se constituir em vergonha, o progresso pode se
constituir em desgraça, para parafrasear Robert Schwarz. E mais do que
isso, o relato é motivado e foi ativado na lembrança pela reflexão de
Caio Prado Júnior, numa nota da introdução da sua \emph{Formação do
Brasil Contemporâneo}, na qual ele afirma ser uma viagem pelo Brasil
muitas vezes ``uma incursão pela história de um século e mais para trás.
Disse"-me certa vez um professor estrangeiro que invejava os
historiadores brasileiros que podiam assistir pessoalmente às cenas mais
vivas de seu passado.'' (2011, p. 11). Do mesmo modo, como teremos a
oportunidade de ver, o desenvolvimento tardio de uma vida social
mercantil no Brasil, e ainda sempre partindo dos centros nacionais mais
adiantados oriundos da estrutura colonial, possibilitou à pessoa que
percorresse o Brasil até bem pouco tempo -- antes da difusão dos meios
de comunicação de massificação mercantil -- uma surpresa comparável à
que sentiu Caio Prado em seu tempo perante aspectos que se poderiam
imaginar ``existirem nos nossos dias unicamente em livros de história;''
embora atentando melhor se pudesse ver que traduziam ``fatos profundos e
não apenas reminiscências anacrônicas.'' (\emph{Idem}, p.10).

A forma"-sujeito burguesa no Brasil não nasceu como na Europa, isso
parece evidente. Se na Europa ela foi fruto de uma série lenta, secular,
de tensionamentos tanto do ponto de vista da vida social concreta quanto
do ponto de vista da elaboração de uma visão de mundo, de uma
subjetividade; se ela foi herdeira da forma de subjetividade amadurecida
com os mercadores; e se um sistema mercantil ganhou dinâmica interna com
o \emph{boom} da modernidade que representaram as armas de fogo, no
Brasil, a forma de subjetividade moderna passou por tensionamentos mais
lentos ainda. Mas não se trata de tensionamentos nascidos do
amadurecimento de tensões internas, mas de tensionamentos vindos do
mundo externo e que se chocavam com a estrutura social e subjetiva
interna muito arraigada no modelo colonial.

Desde o início, foi a implantação desde fora que deu o tom. Quando da
chegada dos colonizadores, não havia por aqui uma sociedade com a qual a
mentalidade mercantil dos exploradores da colônia pudesse tensionar. Os
indígenas não constituíam essa forma de organização social que pudesse
ser uma antítese para fazer avançar o empreendimento mercantil. Pelo
contrário, eram em essência a sua própria negação, objetiva e
subjetivamente, não por um alto padrão de elaboração crítica, mas porque
a própria vida prática no geral se erigia como oposta àquela implantada
de chofre nessas terras. O que havia aqui era uma relação muito próxima
com a natureza. Os indígenas dessas terras, fiando"-se nas pesquisas de
Gilberto Freire, partilhavam uma vida social rudimentar. Inclusive se
comparados com povos originários de outras paragens da América Latina,
os indígenas dessas terras constituíam as ``populações mais rasteiras do
continente'' (\versal{FREYRE}, 2003, p. 157).

E o choque que houve foi entre uma forma exploratória acachapante que
tudo devora -- um aspecto da acumulação primitiva representante da
\emph{Cultura} europeia -- e a forma de vida de forte ligação com a
\emph{Natureza} e seus ritmos da terra. Dito de outro modo, o fetiche do
indígena ainda era outro se comparado com o fetiche que movia os
colonizadores e mercadores aventureiros. Desde o início, portanto, a
forma de vida social e subjetiva desses povos foi simplesmente vencida,
seja pela ideologia ambígua dos Jesuítas, seja pela força bruta dos que
por aqui vinham buscar riqueza rápida, seja ainda por aqueles que
queriam fundar sua riqueza na venda de indígenas, como os bandeirantes.
Os íncolas não tinham como fazer frente ao movimento violento aportado
da Europa. Sendo assim, ou foram escravizados ou foram mortos, ou
incorporados, mas sempre sem se constituir como uma força que pudesse
ter contribuído para o amadurecer de uma dialética de tensionamentos
tendencialmente modernos. Porque tal dialética, por ser positiva, só tem
como se mostrar como tal, dinâmica, se tanto \emph{tese} quanto
\emph{antítese} contiverem, além dos elementos opostos, algo em comum
que se conserve no momento da \emph{síntese}, no momento da elevação. E
entre os povos que aqui existiam e os que chegaram de súbito, quase nada
em termos de organização social ou de subjetividade tinha traços comuns.

Significa dizer que o que veio a se chamar Brasil não foi um
amadurecimento das relações existentes anteriormente nessas terras, um
amadurecimento oriundo de tensões internas. Ao contrário, o conceito de
Brasil está intimamente relacionado desde o começo à vitória violenta
sobre os chamados povos originários que aqui forjaram sua vida social
atrelada à natureza. Ou seja, a ideia de Brasil é desde o começo o de
uma sociedade desterrada, implantada desde fora. As formas de
organização social -- a objetividade moderna -- e as formas de vida
subjetiva sempre chegaram aqui como padrões desde fora, impostos pela
marcha do mundo externo sempre em patamar elevado se comparado com o
ritmo interno. O Brasil enquanto país e forma"-social capitalista não se
forjou a partir de sua força interna, a partir de um amadurecimento de
sua própria dinâmica -- sempre o determinante foram as exigências do
mundo externo. E isso tanto na época colonial quanto na época do
desenvolvimento de relações sociais capitalistas.

Antônio Cândido, em sua célebre \emph{Formação da Literatura
brasileira}, vislumbra na fase entre 1750 e 1880 os ``momentos
decisivos'' para a formação da Literatura brasileira. Para ele, não
havia uma \emph{Literatura} no Brasil, porque faltava um sistema
literário antes disso, ``um sistema de obras ligadas por denominadores
comuns, que permitem reconhecer as notas dominantes duma fase''
(\versal{CÂNDIDO}, 2000. p. 23). O sistema literário brasileiro teria nascido por
essa época ainda como \emph{momentos} \emph{decisivos}, porque antes
poderia até ter havido \emph{manifestações literárias}, mas não um
sistema de transmissão de obras e pensamentos calcado no tripé
\emph{autor"-obra"-leitor}.

Talvez um paralelo possa ser traçado entre essa análise e a formação de
uma vida social enquanto sistema mercantil dentro do Brasil. Assim, o
período colonial no máximo poderia ser caracterizado como
\emph{manifestações capitalistas} num país não"-capitalista. E os
\emph{momentos decisivos} para a dinâmica de uma vida social mercantil
internamente podem ser colocados também nesse ínterim de nascimento de
uma consciência nacional, que amadurece no transcorrer do século \versal{XIX},
até ao momento da passagem ao assalariamento -- cujas consequências para
uma vida social mediada pelo dinheiro serão \emph{decisivas.} Se não se
pode, e os argumentos têm alicerce conceitual sólido, pintar o quadro de
uma sociedade com resquícios feudais a serem superados (\versal{JÚNIOR}, 2014),
tampouco se pode falar -- sob pena de recair numa generalização
imprópria à compreensão da dinâmica moderna, portanto, imprópria ao rumo
dialético de nossa tese -- de um capitalismo no Brasil desde a colônia
-- entendido como vida social. Se o período que tomamos de empréstimo de
Antônio Cândido são momentos que chamamos somente de decisivos, pelo
amadurecimento lento de uma mentalidade nacional, ou seja, se ainda nem
se pode falar de capitalismo como fenômeno social, o que se poderá dizer
do que precedeu esses momentos decisivos? O fato de considerar a não
existência lógica do feudalismo no Brasil não nos obriga a encampar
encarniçadamente a tese da existência do capitalismo desde o princípio.
Nesse sentido, parece haver outros aspectos e entrecruzamentos, não os
mesmos que analisamos em relação à Europa, mas específicos à realidade
brasileira. Significa dizer que nossa reflexão acerca do Brasil parte do
mesmo pressuposto crítico que nos norteia desde o princípio: a
forma"-social capitalista tem uma lógica que se desdobra na história rumo
a sua realização. Ou seja, não se pode simplesmente dizer que cada
momento histórico é o nível de desenvolvimento possível para o
capitalismo, com suas possíveis imperfeições. É preciso entender que o
capitalismo tende a se desenvolver numa certa direção e sempre
desenvolvendo seus pressupostos lógicos. Se é possível ele se
desenvolver, como no caso do Brasil, com alguns desses pressupostos
meios mancos, sem que isso impeça que o país participe do movimento
geral, não significa simplesmente que o capitalismo já tenha
desenvolvido esses pressupostos a seu modo dentro das possibilidades. É
preciso entender que muitas dessas ditas imperfeições do capitalismo,
muitas das \emph{ideias fora do lugar,} se dão pelas vicissitudes da
vida concreta que impõem certas resistências mudas. Que não
necessariamente impedem a marcha do todo, mas que deixam uma certa
ranhura.

O paralelismo entre a falta de sistema literário até o período indicado
por Antônio Cândido e a falta de um sistema mercantil dentro do Brasil
do qual pudesse germinar uma vida social e subjetiva burguesa não deve
levar a pensar um paralelismo estritamente temporal. Até porque,
literariamente, as influências europeias eram de grande monta nos
literatos dessas terras -- fossem eles brasileiros ou portugueses.
Assim, as influências ideológicas dos países avançados em termos
modernos se faziam sentir de modo muito mais explícito e rápido, até
como moda. Já as influências em termos de uma estrutura de vida burguesa
não fincavam raízes por essas terras e só se manifestavam no momento das
trocas mercantis com o exterior, que não modificavam as estruturas
arcaicas internas. É por isso que o espírito nacional baseado nas ideias
avançadas e liberais da Europa vão sempre sofrer de um descompasso em
relação à vida social interna, vão ficar meio que deslocadas do solo a
que se referem, como que \emph{fora do lugar.} (\versal{SCHWARZ}, 2012). Dito
isso, tal paralelismo entre formas sociais e formas estéticas não
significa que a forma social é aqui um reflexo das expressões
literárias. Tampouco o contrário. Parece ser de outro modo. Talvez a
tese de Cândido da falta de sistema literário tenha lastro devido ao
fato de não haver um sistema de fato na colônia, no sentido de uma
organização definida, em termos de dinâmica moderna. E mesmo os momentos
decisivos em termos literários vão dizer respeito de início a lugares
onde há uma certa vida social em torno da riqueza exportada e uma
revolta com a exploração fiscal da Coroa. No entanto, há um ponto em que
o paralelismo não vai ao encontro da tese aqui defendida. É porque
Antônio Cândido não afirma ``a inexistência de literatura no Brasil''
(\versal{CÂNDIDO}. 2000, p. 15) antes do período por ele considerado. Já no nosso
caso, defendemos que não houve capitalismo no Brasil, porque as
\emph{manifestações capitalistas} a que nos referimos diziam respeito ao
momento da relação mercantil com o mundo externo, não a
\emph{manifestações} no seio da vida social interna. Porque para algo se
manifestar, é preciso que exista. E se não existe, não pode se
manifestar. Dito de outro modo, o que faz as manifestações capitalistas
existirem no Brasil é a relação com o mundo mercantil exterior, não a
dinâmica interna que, subjugada ao circuito mercantil externo, não vai
ativar uma vida social mercantil internamente.

Voltaremos mais adiante a esse aspecto que tem relação com o fato de a
própria forma"-sujeito burguesa penar para se implantar de início no
Brasil. Como vimos, essa forma de subjetividade não nasceu de chofre,
mas como entrecruzamento entre formas de subjetividades anteriores ao
início da dinâmica moderna e formas que vão numa dialética contínua da
subjetividade se distanciando do passado e apostando tudo na superação
de qualquer limite à acumulação de riqueza mercantil e à constituição de
uma comunidade mercantil, para a qual a forma"-sujeito burguesa é a forma
de subjetividade \emph{ideal.} E não foi por essa dialética que se
formou uma forma"-sujeito burguesa no Brasil, que se gestou no
entrecruzamento entre formas de subjetividade ligadas à exploração
econômica em todos os níveis, notadamente a do latifúndio voltada para a
constituição da vida mercantil no exterior, e aquelas formas de
subjetividade que somente a duras penas veríamos como tendentes a
modernas, como ligadas ao ritmo da cidade e sua dinâmica.

Nesse sentido, há sempre um descompasso entre a forma"-sujeito burguesa
nos lugares onde a lógica da mercadoria se estabelece como tendência à
constituição de relações sociais, ou seja, como tendência a englobar o
conjunto da sociedade e a tensionar no sentido de novas mentalidades, e
a forma"-sujeito burguesa em lugares onde o que é chamado de mercadoria
não são senão produtos valorizados no mercado internacional, cuja
produção não engendra uma dinâmica da qual possa germinar uma forma
social ou subjetiva mercantil -- que é o caso do Brasil. Mesmo a classe
dos dirigentes da Economia de exportação só epidermicamente poderia
aparentar uma mentalidade burguesa. Embora se relacionassem com o mundo
exterior, a base de sua vida social e subjetiva -- que lhe possibilitava
se relacionar com o mundo exterior -- não tinha nada das características
do mundo exterior ao qual servia com sua produção. Nesse sentido, há um
aspecto aparentemente anódino considerado por Sérgio Buarque de Holanda,
mas que ilustra a ideia de uma certa peculiaridade da forma"-sujeito
burguesa no Brasil. Ao analisar o momento da vinda da Corte para o
Brasil, ele afirma que grande parte das pessoas que assumiram postos no
serviço público provinham do microcosmo dos engenhos, e traziam a
mentalidade rural para a cidade. Ou seja, o que seria a chamada
burguesia urbana tinha uma mentalidade de Casa"-grande. Para dizer sem
meias"-palavras, trata"-se de uma burguesia urbana de improviso, o que é
rico em consequências: ``Estereotipada por longos anos de vida rural, a
mentalidade de Casa"-grande invadiu assim as cidades e conquistou todas
as profissões, sem exclusão dos mais humildes.'' (\versal{HOLANDA}, 2014, p. 103)
E cita o conhecido caso do oficial de carpintaria que se vestia feito
fidalgo, com sapatos de fivela e se recusava a portar as ferramentas,
que entregava a um negro. O que salta aos olhos é a adaptação que
precisa sofrer o aspecto interno em função do movimento externo, não do
amadurecimento de tensões internas. Assim, não só a vida social objetiva
teve sempre que se recuperar aos saltos diante do avançar da marcha
moderna, também do ponto de vista da subjetividade foi preciso sempre um
movimento de recuperação do ``atraso'' aos saltos. O que não deixa de
criar essas incongruências. Esses saltos de amadurecimento forçado e
sempre epidérmicos também fazem parte dos ingredientes do caldo que faz
a especificidade da forma"-social e da forma"-sujeito burguesas
brasileiras num primeiro momento, antes do século \versal{XX} e até depois.

Se o processo de desenvolvimento das relações sociais burguesas e da
forma de subjetividade burguesa coincide normalmente na Europa com a
primazia das cidades sobre o campo, já que, em tempos modernos, os
produtores, mercadores, a burguesia mercantil, morava nas cidades, no
Brasil era o campo que dominava sobre a cidade e o que ela traz de
moderno, objetiva e subjetivamente. Mesmo quando a vida urbana ganha
certa dinâmica, a vida na cidade comparada com a do campo parecia ainda
frágil. Sérgio Buarque cita o caso do Rio de Janeiro que é descrito em
1767 como ``só habitado de oficiais mecânicos, pescadores, marinheiros,
mulatos, pretos, boçais e nus, e alguns homens de negócios, dos quais
muito pouco podem ter esse nome.'' (\emph{Idem}, p. 108). Foram
necessárias várias gerações até que se chegasse a uma espécie de
burguesia mais moderna -- algo vislumbrável apenas no século \versal{XX} --,
citadina, embora ainda não totalmente despojada de muitos hábitos da
Casa"-grande.

O que caracterizaria a formação brasileira seria, assim, a falta de
bases estáveis para a constituição de uma vida social em termos
mercantis. Ora, da implantação da empresa de exploração colonial ao seu
fim, quatro séculos depois, não foi o amadurecimento tenso das relações
internas que levou o país a se posicionar perante a marcha do mundo.
Sempre as condições de mudança foram amadurecidas desde fora e
implantadas aqui dentro. Esse desterramento a que nos referimos deitou
raízes durante muito tempo e é ressaltado por Sérgio Buarque já na
primeira frase de suas \emph{Raízes} \emph{do} \emph{Brasil}:

\begin{quote}
A tentantiva de implantação da cultura europeia em extenso território,
dotado de condições naturais, senão adversas, largamente estranhas à sua
tradição milenar, é, nas origens da sociedade brasileira, o fato
dominante e mais rico em consequências. Trazendo de países distantes
nossas formas de convívio, nossas instituições, nossas ideias, e
timbrando em manter tudo isso em ambiente muitas vezes desfavorável e
hostil, somos ainda hoje uns desterrados em nossa terra. (\emph{Ibidem},
p. 35)
\end{quote}

Numa terra em que colonos vêm para enriquecer, escravos negros são
trazidos à força para servir de pura energia física, desemboca"-se desde
o princípio numa cisão entre o mundo voltado para fora -- o fora que
compra essa riqueza -- e outro para dentro. O voltado para fora é aquele
das relações comerciais, do ciclo monetário, do lucro adquirido com os
produtos produzidos ou extraídos da terra. O voltado para dentro é o
ritmado pelas relações arcaicas e desmonetarizadas, pela ``atonia
econômica'' no dizer de Caio Prado. Uma cisão entre uma exploração
mercantil voltada para um capitalismo europeu em desenvolvimento --
principalmente no caso inglês -- e as relações sociais no plano interno
que nada ou pouco acompanhavam a marcha mundial moderna -- embora sempre
tenham podido assim mesmo participar dela a seu modo. E as mudanças
dentro do Brasil durante mais de três séculos foram fruto de
tensionamentos importados por esse mundo voltado para fora, o que
significa dizer que eram mudanças relacionadas à manutenção da estrutura
da chamada \emph{verdadeira} \emph{Economia}, a voltada para fora, não
eram mudanças amadurecidas no próprio seio social -- como é próprio em
sociedades que contêm germes modernos que desabrocham. Não deverá
transparecer nessa abordagem qualquer lamento por essa falta de dinâmica
interna como é próprio dos \emph{intérpretes do Brasil --} fato já
abordado por Marildo Menegat (2011). O que interessa não é fazer uma
análise crítica dessa falta de dinâmica interna para fundamentar a
possibilidade de resolução desse estado de coisas, para resolver as
imperfeições do capitalismo brasileiro, como foi a \emph{Teoria do
subdesenvolvimento} representada no Brasil sobretudo por Celso Furtado.
Antes de tudo, trata"-se de perseguir a reflexão de que essa ``atonia
econômica'' interna foi preponderante para o não surgimento de uma
dinâmica realmente mercantil internamente. Como esse não"-surgimento não
foi uma escolha de sujeitos conscientes muito desejosos de uma
organização social superior, mas uma circunstância histórica
inconsciente com consequências igualmente funestas, também não cabe
regozijo.

Nesse sentido, a vida social dentro da Colônia e até mesmo durante o
século \versal{XIX} segue o ritmo do atraso, da dureza arcaica, misturado ao
ritmo da dureza moderna que vai aos poucos criando raízes internamente.
Só para fora aparece a relação mercantil, para dentro a relação social
que deixa marcas sobre a vida social e a subjetividade é aquela da
subsistência, da falta de dinâmica e pensamento modernos. E aqui podemos
citar o próprio Caio Prado acerca da importância que tiveram as
atividades econômicas subsidiárias ou de subsistência, que proviam a
Colônia em bens de primeira necessidade, como inibidoras de um processo
de desenvolvimento de relações mercantis dentro do país, já que, para
ele, tanto a grande lavoura quanto a mineração ``são atividades que se
desenvolvem à margem das necessidades próprias da sociedade
brasileira.'' (\versal{JÚNIOR}, 2011, p. 177). Ou seja, o grau de desenvolvimento
das relações mercantis no Brasil era de tal modo ainda arcaico que a
grande lavoura ou a mineração eram algo estranho, estrangeiro, cujas
repercussões em termos mercantis se faziam sentir somente fora do país,
notadamente na Metrópole"-Inglaterra, e entre as mãos dos produtores que
despendiam grande parte importando bens de consumo. Portanto, as somas
advindas desse lucrativo mercado exportador não repercutiam
necessariamente dentro do país criando sistema social monetarizado.
Primeiramente, porque não havia pequenos produtores que pudessem vender
pequenas produções para o mercado exterior e ir criando uma circulação
monetária ao encontro de uma dinâmica moderna. Em segundo lugar, a
população estava, na sua maioria, em estruturas sociais rígidas
completamente destoantes da marcha liberal da modernidade. Completamente
destoantes de qualquer conceito de forma"-sujeito burguesa, uma forma de
subjetividade de que nem mesmo os grandes produtores e investidores eram
embebidos. Sérgio Buarque descreve de modo interessante essa
subjetividade do explorador da Colônia, esse português de espírito
marcado por um ``gosto desordenado e imprevidente da pecúnia'', mas em
cuja mentalidade não se pode discernir qualquer germe do espírito
capitalista. Isso porque, para o autor

\begin{quote}
A simples ganância, o amor às riquezas acumuladas à custa de outrem,
principalmente de estranhos, pertence, em verdade, a todas as épocas e
não caracteriza a mentalidade capitalista se desacompanhada de certas
virtudes econômicas que tendam a contribuir decisivamente para a
racionalização dos negócios.'' (\versal{HOLANDA}, 2014, p. 161).
\end{quote}

Apesar de certo tom naturalizador do ``amor às riquezas acumuladas à
custa de outrem'', o trecho vai direto ao ponto do que em verdade
constituía a subjetividade dos colonos em geral vindos para o Brasil. O
próprio Schwarz cita um trecho de obra de Fernando Henrique que vai de
encontro à ideia de um germe de subjetividade burguesa entre a classe
dos dirigentes da empresa mercantil colonial, os latifundiários. Ele diz
que não há o cuidado tipicamente moderno de economizar tempo. Ao
contrário, ``É preciso espichá-lo, a fim de encher e disciplinar o dia
do escravo.'' É assim que se pode dizer que as ditas elites não tinham
uma consciência burguesa desenvolvida, ao contrário do que se poderia
pensar, e ao contrário do que diz o próprio Schwarz, para o qual ``era
inevitável, por exemplo, a presença entre nós do raciocínio econômico
burguês -- a prioridade do lucro e seus corolários sociais.'' (\versal{SCHWARZ},
2012, p. 13). Como já dissemos em outra parte, o que caracteriza uma
consciência burguesa ou um raciocínio de um sujeito burguês não é
somente a ânsia pelo lucro que é algo bem antigo. É preciso que essa
forma de subjetividade venha acompanhada de um contexto de ``corolários
sociais'' que impulsionem, em vez de inibir, o desenvolvimento paulatino
e contínuo de uma vida social mercantil. Os senhores da Economia
brasileira, até o século \versal{XIX}, pouco tinham de reflexo burguês além da
sede de lucro. Ao que parece, o lucro era o que os movia, o que lhes
dava poder e riqueza, explorando o máximo possível com vistas ao mercado
externo, sem que seus raciocínios passassem além disso. Quanto à
mentalidade, eram extremamente ``tradicionais'', nada neles apontava
para uma dialética da subjetividade do tipo burguesa. Tratava"-se de uma
forma de subjetividade que encarava a terra como lugar de pilhagem de
riqueza, uma forma de pensar que se manteve arraigada por muito tempo --
não só entre as classes abocanhadoras da riqueza mercantil --, sem que
ainda hoje se possa dizer superado.

E a formação do Brasil sempre foi permeada pela extração do que quer que
fosse possível para vender e render dividendos à Coroa e ao colono, mas
nunca construir o que pudesse dar certa estabilidade às relações, é a
constatação frequente entre os historiadores. O sentimento sempre foi o
de que ninguém está a salvo, de que é melhor apoiar"-se em quem tem
poder, em algum padrinho. E que, se tiver a chance de derrubar alguém,
derrube"-o antes que ele o faça. As relações aqui nascidas sempre foram a
de desconfiança, portanto, em que os sujeitos estavam sempre à espreita,
atentos ao que pudesse ser um bote fatal, fosse da estrutura de
organização, fosse dos próprios habitantes. Essa instabilidade das
relações fez com que essa terra sempre fosse tudo menos algo a ser
construído, mas sempre uma terra a ser explorada. Um aspecto que
certamente contribuiu para que as relações mercantis enquanto sistema
não fincassem suas bases. O espírito do português, ``do colono
recém"-vindo'', conforme citação do Marquês de Lavradio feita por Caio
Prado, ``mesmo quando lavrador no reino'', é de não pensar ``em outra
coisa senão em mercancia''. (2011, p. 92). Evidentemente a mercancia que
povoa o pensamento desse português em busca de riqueza fácil é um objeto
de comércio e não se confunde com o aspecto conceitual moderno de
mercadoria -- pelo menos em termos de relações internas. Porque entre os
aspectos que explicam a demora da superação das \emph{manifestações
capitalistas} na Colônia em proveito de uma forma"-social capitalista ou
moderna estão tanto a forma da colonização quanto o próprio nível de
desenvolvimento da forma"-social capitalista na Metrópole. Uma das razões
portanto que ajudam a entender o desenvolvimento mercantil endógeno, no
caso do Brasil, está em certa medida em Portugal mesma -- onde tampouco
tais relações eram bastante desenvolvidas. Sem desenvolver sua
manufatura devido à sua relação com a Inglaterra, Portugal estava
totalmente desarmada na Revolução Industrial, sem uma base que pudesse
sofrer uma \emph{revolução}, pois nem os rudimentos de uma indústria
nacional existiam. Seguindo Furtado: ``A inexistência desse núcleo
manufatureiro, na etapa em que se transformam as técnicas de produção
[\ldots{}] é que valeu a Portugal a dependência agrícola da Inglaterra.''
(\versal{FURTADO}, 2007, p. 129).

O Brasil, portanto, não passou pelo processo tipicamente europeu da
implantação paulatina e sub"-reptícia da lógica mercantil nas relações
sociais. As relações sociais rudimentarmente capitalistas foram bem ou
mal \emph{transplantadas} para cá a cada estágio de desenvolvimento, não
são fruto de um amadurecimento tipicamente moderno das relações cuja
dinâmica interna faz com que logo sejam corroídas por novas que se
forjaram no próprio seio das velhas.

Essa cisão ou o descompasso entre um mundo voltado para fora, para as
relações mercantis com o exterior, e o mundo interno, em que o aspecto
mercantil não toca senão ínfima parte da sociedade e pouquíssima gama de
produtos, vai atravessar mais de três séculos e amadurecer quase no fim
do século \versal{XIX} como uma sociedade em que as \emph{ideias civilizadas} do
mundo com o qual o lado exterior da cisão interna se relacionava
apareciam como deslocadas, \emph{fora do lugar.} Trata"-se de um Brasil
agrário, escravocrata, fundado no favor, no pessoalismo e ligado
economicamente ao mercado externo para o qual as ideias liberais eram o
espírito do tempo.

É a Robert Schwarz que devemos esse conceito que compreende o que
aparece em vários autores como uma incongruência entre dois mundos.
Partindo da análise da obra machadiana e de um romance de Alencar, ele
vai encontrar importantes fissuras entre o Brasil dito moderno, virado
para fora, e os vários fundamentos arcaicos e não"-mercantis do Brasil
interno, sobre os quais o Brasil de casca moderna se sustentava
secularmente. Ou seja, ele descreve o Brasil como um terra onde as
grandes abstrações burguesas permaneciam apenas abstrações (\versal{SCHWARZ},
2012, p. 12).

Ainda no século \versal{XIX}, quando vários países europeus já vinham passando
por séculos de tensionamento entre as formas de vida social moderna --
fundadas no movimento do dinheiro -- e as formas mais arcaicas ligadas à
vida social e subjetiva fundadas no caráter simbólico"-religioso, o
Brasil vivia uma mescla muito mais profunda entre aspectos arcaicos e
modernos. Porque aqui não havia contraposição. Enquanto o mundo virado
para fora pôde sobreviver secularmente estabelecendo relações mercantis
sobre a base de uma economia interna não"-mercantil, não parecia haver
incongruência. Mas quando o capitalismo mundial da segunda metade do
século \versal{XIX} alcança uma dinâmica sem precedentes e aumenta o ritmo de sua
marcha sobre o mundo, após passar pela Revolução Industrial, a estrutura
de seu fornecedor de produtos primários aparece como incompatível:
tornara"-se impossível manter a cisão entre os dois mundos econômicos que
tinham se cindido na Colônia e inibido o surgimento de uma dinâmica
interna das relações mercantis. Assim, embora a impropriedade do
pensamento liberal no Brasil tenha se mostrado no 2º Reinado, como diz
Schwarz (2012, p.12), não era fruto, como ele afirma, de uma
``dependência global'' que nos obrigava ``a pensar em categorias
impróprias'' (2012, p. 36), mas de uma secular história interna cindida
da verdadeira Economia e à margem da marcha moderna em termos objetivos
e subjetivos. Era o indício de que a estrutura social não poderia mais
funcionar sobre a base de uma cisão entre uma \emph{Economia} importante
em termos mercantis, e uma \emph{economia} subsidiária na qual não
repercute economicamente aquela Economia no sentido de criar as bases de
uma dinâmica interna. Somente amalgamando as duas esferas, num esforço
de acabar com essa cisão, seria possível fazer germinar uma vida social
minimamente moderna: e o fim da escravidão e o desenvolvimento de uma
força de trabalho formalmente livre eram o ponto nodal, sem o qual não
há como haver dinâmica mercantil dentro do país.

É assim que somente a partir do século \versal{XIX} se pode falar num
tensionamento entre as formas econômicas e de subjetividade aqui
gestadas e as formas novas que as forma"-social moderna passava a exigir.
Importa esclarecer que não estamos aqui afirmando que a subjetividade no
Brasil nasceu somente por essa época, o que seria medonho. Mas sim que a
forma de subjetividade moderna -- a forma"-sujeito burguesa --, aquela
que age e se submete ao mesmo tempo aos processos dinâmicos de adaptação
à forma de vida mercantil, só veio a germinar tempos depois de já dar
mostras de amadurecimento na Europa -- por isso aqui aparecia como
\emph{ideia fora do lugar}. Se era algo até disparatado para a forma de
subjetividade daqueles que viviam o lado da cisão voltada para fora,
imaginemos para a forma de subjetividade da grande maioria da população
que vivia o ritmo interno da economia subsidiária precária. Não à toa,
como afirma Schwarz, as ideias liberais aqui beiravam à desfaçatez.

Mas as \emph{ideias} \emph{fora} \emph{do} \emph{lugar} não são apenas
uma desfaçatez de uma classe no Brasil. Assim como os atrasos que
Schwarz viu convivendo com o moderno no ensaio \emph{A carroça, o bonde
e o poeta modernista}, trata"-se de fato de sintomas de um país que
sempre cresceu para fora, sem que uma forma"-social capitalista aqui se
desenvolvesse desde cedo. O país sempre foi arrastando esses arcaísmos
misturados com um timbre moderno. Porque a falta de um sistema, ou seja,
o fato de sempre ter sido uma elite a viver a \emph{Economia --}
enquanto escravos ou homens livres viviam no entorno dessa economia --
não era propício à germinação de uma dinâmica social moderna. Por isso,
é por demais geral dizer que ``tanto a eternidade das relações sociais
de base quanto a lepidez ideológica das `elites''' (\versal{SCHWARZ}, 2012, p.
25) faziam parte do quadro do capitalismo. Como dissemos acima, não se
pode resolver a questão apenas argumentando que o capitalismo é assim
mesmo, tem essas disfunções, e a cada momento histórico aparece com a
face possível. É preciso adentrar pela forma como a vida social interna
se posicionava perante o capitalismo e não somente a \emph{Economia}
voltada para fora. A questão não era só de dependência, mas de
descompasso em relação ao fundamento lógico da vida moderna, objetiva e
subjetivamente. Dito com outras palavras, a questão é de descompasso em
relação à forma"-social e a forma"-sujeito burguesas. Se há descompasso, é
porque há um \emph{ideal} para essas formas.

O momento das \emph{ideias fora do lugar,} a segunda metade do século
\versal{XIX}, foi o momento em que o fundamento da \emph{Economia} até ali, a
produção de bens coloniais com trabalho escravo, precisava deixar de ser
o fundamento, sob pena de uma vida social interna em termos mercantis
não ganhar impulso. Assim, diferentemente do que pensam autores como
Caio Prado e Celso Furtado, a chave para entender o Brasil contemporâneo
não está no Brasil colônia -- a não ser que se queira reduzir o quadro
aos resquícios escravocratas e ao latifúndio. Porque, como veremos, ali
não havia uma organicidade interna minimamente mercantil da qual pudesse
amadurecer uma dinâmica mercantil que daria a chave para a compreensão
de sua evolução séculos depois. Porque o processo de entrada do Brasil
na marcha moderna, cujo aspecto marcante é o fim da escravidão e o
advento de uma classe formalmente livre, não foi um mero desabrochar
daquilo que estava nas estruturas dos séculos da colonização. Foi
literalmente uma revolução, pois o que era o fundamento do
empreendimento colonial -- notadamente o latifúndio, o poder senhorial a
ele atrelado e a escravidão -- passou a ser uma característica acessória
e culturalmente importante, mas de outro fundamento: o mercantil, sem
cuja dinâmica não era possível criar uma vida social capitalista
enquanto sistema interno.

Mas essa troca de fundamentos não foi como uma troca de governos. Também
foi paulatina e encontrou resistências nas subjetividades do tempo. Como
relata Celso Furtado, o sistema escravista era caracterizado pela grande
estabilidade estrutural, algo secular. Sua abolição era vista como uma
hecatombe social, já que ``O escravo era uma riqueza e que a abolição da
escravatura acarretaria o empobrecimento do setor da população que era
responsável pela criação da riqueza do país.'' (\versal{FURTADO}, 2007, p. 199).

Ou seja, as \emph{ideias fora do lugar} de certo modo não evitaram que o
Brasil seguisse com seu espaço no mercado mundial e só lentamente
amadurecesse relações sociais capitalistas. O que faz pensar que mesmo
as observações de Schwarz sobre as \emph{ideias fora do lugar} parecem
se referir às zonas urbanas mais desenvolvidas do país. Ou seja, as
impropriedades da forma de vida burguesa em solo brasileiro só apareciam
numa parte minúscula se comparada com o restante do país onde tal
incongruência nem mesmo se mostrava. Significa com isso dizer que a
maioria da população na época vivia uma forma de vida social em que essa
tensão palpável não aparecia. Essa tensão só poderia aparecer nos locais
onde as tensões modernas bem ou mal germinavam, e essas tensões só
poderiam germinar em locais onde houve bem ou mal um desenvolvimento
forte em termos econômicos, inclusive em termos de comércio, e de uma
intelectualidade preocupada com a fundação nacional.

De todo modo, a situação artificial continuava: era como se o Brasil
fosse um país moderno por vender ``mercadorias'', mas ao mesmo tempo um
país de outro tempo por sua base produtiva calcada na escravidão e suas
relações sociais baseadas na pessoalidade e no favor, ``nossa mediação
quase universal'' (\versal{SCHWARZ}, 2012, p. 18), e não na impessoalidade das
relações modernas mediadas pela mão invisível. Nesse sentido,
seguindo Schwarz, o Brasil com essas características se colocava fora do
mundo dito \emph{esclarecido --} embora ao mesmo tempo estivesse dentro.

\section{A dinâmica da história brasileira e~os~germes~mercantis}

É muito comum ao se ler os autores que se dedicam à era do Brasil
colonial a denúncia da forma econômica da colônia como exploratória e
centrada somente no mercado exterior, assentada no trabalho escravo do
negro e na escravização e dizimação dos índios. Caio Prado -- mas também
Celso Furtado e Sérgio Buarque -- insiste no fato de que havia um
objetivo explícito na colônia e não se podia fugir desse objetivo
exploratório de enviar à Metrópole toda e qualquer riqueza produzida ou
extraída. O próprio povoamento, a subsistência e a organização
jurídico"-política da colônia deviam se submeter ao objetivo maior.
Ninguém dirá ser incorreta tal abordagem crítica. No entanto, salta aos
olhos o fato de os três autores citados não abordarem minimamente como
poderia ter sido a colonização senão com exemplos da colonização de
povoação da América do Norte, como se fosse o exemplo correto de
colonização. Dito de outro modo, os autores em suas abordagens não
transmitem nem nas entrelinhas o tensionamento citado por Schwarz que
era comum na literatura russa: ``o progresso é uma desgraça e o atraso é
uma vergonha'' (2012, p. 28). Não. Sobressai"-se muito mais nas
principais obras dos três autores a ideia de que o atraso combinado com
surtos de progresso são uma vergonha, mas que a civilização moderna é
quem pode corrigir tal impropriedade. Evidentemente, tendo em vista a
filiação desses autores com um certo marxismo de feições tradicionais e
seu contexto histórico, é possível compreender. Mas isso não os imuniza
da crítica. Não está implícito, sobretudo em Caio Prado e Celso Furtado,
somente que a forma de colonização dos \versal{EUA} ou em outras colônias de
povoamento é a mais correta, mas também uma adesão à marcha moderna,
criticada sobretudo no que ela tem de extorsiva e desigual, e não em
seus próprios fundamentos. Nesse sentido, perguntará o leitor porque
tais autores servem de base para um estudo que pretende seguir um
caminho tão distinto de suas abordagens. A questão é que tais autores,
nas suas pesquisas históricas, expõem elementos que possibilitam fazer
uma reflexão para além da reflexão que eles mesmos fazem. Se, para eles,
o atraso brasileiro é fruto da época colonial e algo a ser superado
desenvolvendo aquilo em relação a qual o atraso se faz sentir, para nós,
esse atraso é signo das vicissitudes típicas da colonização que
dificultam o desenvolvimento das relações mercantis internamente,
enquanto sistema, e não necessariamente algo a ser superado por meio do
desenvolvimento das próprias relações mercantis, encaradas
aprioristicamente como positivas se bem direcionadas no rumo do
progresso.

Dito isso, não é preciso fazer longas incursões para discutir os
\emph{sentidos da colonização} fundada na grande propriedade, no
trabalho escravo e na monocultura, temática que se encontra discutida
longamente não só na \emph{Formação do Brasil contemporâneo,} como na
\emph{Formação econômica do Brasil.} Basta termos em vista que:

\begin{quote}
Se vamos à essência de nossa formação, veremos que na realidade nos
constituímos para fornecer açúcar, tabaco, alguns outros gêneros; mais
tarde ouro e diamantes; depois, algodão, e em seguida café, para o
comércio europeu. Nada mais que isso. É com tal objetivo, objetivo
exterior, voltado para fora do país e sem atenção a considerações que
não fossem o interesse daquele comércio, que se organizarão a sociedade
e a economia brasileiras\ldots{} Virá o branco europeu para especular,
realizar um negócio [\ldots{}] (\versal{JÚNIOR}, 2011, p. 29)
\end{quote}

O que importa reter do trecho é o aspecto do ``objetivo exterior'' em
detrimento de qualquer objetivo interno. Resumindo, portanto, nos termos
de Caio Prado, o caráter da economia brasileira colonial é de ``empresa
mercantil exploradora dos trópicos e voltada inteiramente para o
comércio internacional.'' (\emph{Idem}, 2011, p. 248). Caracterizar a
empreitada colonizadora como ``empresa mercantil exploradora dos
trópicos'' não autoriza a interpretar a forma de organização interna nem
como sistema mercantil, nem como capitalismo, muito menos pintar os
colonos para aqui vindos como protótipos de sujeitos burgueses.

Ora, o capitalismo deve ser apreendido como uma forma de vida social que
tende à totalidade, não como trocas econômicas com o objetivo de lucro
que, como já vimos, não é o parâmetro distintivo do capitalismo em
relação a outras formas de sociedade. Somente o fato de considerá-lo
como forma de vida social já exige uma abordagem que não encare a
economia como um fenômeno à parte, com esfera separada da vida social.
Ao contrário, a economia na modernidade não é uma esfera separada, mas a
esfera que se imiscui nas outras criando um complexo de fenômenos
sociais e subjetivos que ganham necessariamente viço no seio social tão
logo encontrem o germe da dinâmica moderna. Ao entender o capitalismo
como uma forma"-social dinâmica, que parte de um \emph{a priori} abstrato
-- ou seja, moldar a vida social e subjetiva à imagem e semelhança das
relações mercantis para que estas transcorram como por \emph{natureza}
-- que busca realizar"-se na história, é preciso refletir como essa
dinâmica se desdobra no terreno histórico, seus embates com a vida
social concreta, que oferece resistências mudas, de circunstância, mais
do que escolhidas conscientemente. Ou seja, é preciso levar em conta as
especificidades dessa forma"-social quando não se apresenta com suas
categorias germinadas ou plenamente maduras, não para encampar um
individualismo metodológico, não para tentar refletir sobre o geral a
partir do particular, quando em termos modernos é o contrário mais
fértil proceder. Mas porque, ao que parece, uma análise que parte do
caráter geral, da linha mestra para entender o particular, que seria
índice desse geral, precisa também abrir os olhos para ver em que medida
esse caráter geral que está em marcha no mundo se manifesta nas
entranhas do caso particular -- e não somente como determinante dele
desde fora. Do contrário, podemos confiar demasiado na afirmação
abstrata e generalizante de que o particular é sempre índice do geral na
modernidade e esquecer a riqueza das particularidades no embate mudo com
o geral. No caso do Brasil, especificamente até o século \versal{XIX}, o geral
entendido como marcha inexorável rumo ao moderno não parece dizer muito
sobre o particular em termos das relações concretas e cotidianas. Não é
que no Brasil não repercutem as consequências de um capitalismo mundial
em desenvolvimento, já que a evidência óbvia é o próprio trabalho
escravo dos negros, que se opõe conceitualmente ao capitalismo, mas que
serviu bem à sua própria marcha num estágio de seu desenvolvimento. E a
escravização de indígenas e negros -- consequência direta da submissão
da Colônia ao comércio ampliado de produtos no mercado mundial e à
implantação da marcha do capital nos países de vanguarda -- não é pouca
coisa, ou algo a ser relativizado. Ou seja, há uma série de repercussões
na vida sobretudo dos mais desamparados, como por exemplo a carestia e a
fome que reinavam quando do aumento dos preços dos produtos de
exportação que exigiam mais terras de cultivo. Houve até mesmo leis nos
séculos \versal{XVII} e \versal{XVIII} para obrigar que parte da lavoura abrigasse roças
de víveres, principalmente de mandioca. Inclusive os traficantes de
negros deveriam ter plantações para dar mandioca a suas mercadorias. Nem
sempre a lei era seguida, ainda mais quando os preços das exportações
exigiam que se concentrasse na produção do que ia ser exportado. Há
ainda o caso do Maranhão, onde em certa época pelo menos arroz abundava,
somente porque era exportável, o que demonstra com efeito uma submissão
de qualquer produção na Colônia à economia de exportação.

Mas o que nos interessa discutir é se essa grande \emph{Economia}
repercutiu internamente no sentido de fazer germinarem relações
mercantis com dinâmica própria. Isso não pode ser confundido com uma
discussão sobre o atraso em relação ao centro, uma obviedade já
refletida pelo intérpretes do Brasil. Não se trata simplesmente de
atraso, porque atraso pressupõe que haja uma partilha das mesmas bases
e, no caso do Brasil colônia, parece inexistir o germe mesmo que pudesse
apontar para uma dinâmica posterior em termos capitalistas. Ao que
parece, a exploração da grande lavoura no Brasil, ou as outras empresas
de exploração dos trópicos, não podem nem mesmo ser comparadas aos
sistemas de exploração da mão"-de"-obra nas primeiras manufaturas de
tecido nos Flandres no século \versal{XIII}, citadas por Moishe Postone (2009)
como fundamentais para a mudança de mentalidade na Europa. Embora aqui
como ali o investidor investisse uma soma de dinheiro num produto --
aqui a cana"-de"-açúcar, ali o tecido -- sobre o qual queria ganhar uma
soma maior no fim -- o colono sobre a terra e a força de trabalho do
escravo, o mercador sobre a atividade manufatureira dos operários --, as
consequências são distintas. Nem um dos dois processos estavam incluídos
necessariamente dentro de um contexto capitalista generalizado. No caso
europeu, embora o desenvolvimento das manufaturas de tecido naquela
região tenha trazido mudanças significativas na temporalidade que
passava de concreta e guiada pelas horas canônicas, para abstratas e
guiadas pelo relógio da produção, na época a generalidade social não se
guiava por tais relações, que acabavam ficando restritas a determinadas
partes ou cidades. Mas há uma diferença fundamental em relação à nossa
Colônia. Ali, tanto os mercadores que encomendavam seus tecidos tinham
seus lucros, quanto os operários recebiam uma remuneração monetária.
Aqui, o colono dirigente da produção para o mercado externo embolsará
seus lucros, mas nenhuma parte monetária desse processo vai alimentar
qualquer germe de uma dinâmica mercantil no interior da Colônia. Nem
mesmo em pequenos nichos tal relação salarial se estabeleceu
consideravelmente. Ou seja, embora nem uma das duas experiências façam
parte de um sistema mercantil mundializado -- lembremos que não menos
que dois séculos separam as duas experiências em exemplo --, na região
produtora de tecido, vai nascer o germe de uma dinâmica interna de onde
vai germinar uma subjetividade burguesa, o germe de relações mercantis
que tenderão a tensionar paulatinamente com as formas tidas por
arcaicas.

O fundamental portanto aqui é perseguir a questão de saber em que medida
esse quadro brasileiro em termos das relações sociais internas
desenvolviam uma forma"-social e subjetiva capitalista ou moderna, ou
mesmo continham um embrião do qual pudesse surgir uma dinâmica moderna.
Dito de outro modo, eis a pergunta que nos ocupa nesse momento de nossa
reflexão: não haveria outras relações internas, distintas daquelas
chamadas capitalistas que se creem muitas vezes ver aqui desenhadas
desde o berço da colonização?

Caio Prado Júnior é intelectual avisado. Sua análise dos aspectos
fundamentais que caracterizam a colonização brasileira, -- que vai ao
encontro das análises de outros autores como Celso Furtado em sua
\emph{Formação econômica do Brasil --} bem como a polêmica por ele
aberta acerca da inexistência de feudalismo no Brasil parecem demonstrar
profundidade e ao mesmo tempo lucidez para não se deixar ofuscar por
categorias histórico"-conceituais que podem se apresentar em determinados
momentos históricos e em determinados locais, mas que não
necessariamente se apresentam em outros. Mas essa profundidade e essa
lucidez do grande intelectual brasileiro não o deixam imune a uma
espécie de teleologia da história brasileira cujo começo ele encontra
nos primeiros tempos da colonização e que prossegue seu curso pelo menos
até o tempo que lhe foi contemporâneo. Uma tal teleologia, somente
possível -- embora nem sempre garantida devido às vicissitudes da vida
concreta -- em sociedades que contêm internamente ao menos o germe de
uma dinâmica interna, mostra em verdade um certo ofuscamento em relação
às categorias capitalistas. O interessante é que ele traça tal
teleologia sem necessariamente afirmar que na Colônia já havia um
sistema econômico interno orgânico. O que não impede Caio Prado de ver
na \emph{formação do Brasil contemporâneo} do início do século \versal{XIX} o
momento ``em que os elementos constitutivos de nossa nacionalidade --
instituições fundamentais e energias --, organizados e acumulados desde
o início da colonização, desabrocham e se completam.'' (\versal{JÚNIOR}, 2011, p.
7-8). Ora, para que os elementos da nacionalidade ``desabrochem'' e se
``completem'', é necessário que já estejam em germe durante os séculos
de colonização. Evidentemente, esses elementos da nacionalidade a que se
refere o autor não dizem respeito a questões de ordem estritamente
culturais, mas ao sentido mesmo de uma sociedade brasileira em sentido
bastante amplo e à sua posição perante o mundo moderno. Poderíamos,
portanto, entender pelas considerações de Caio Prado que o germe da
forma como a sociedade brasileira se apresenta e se posiciona na marcha
do mundo moderno já estava no início da colonização e de seus objetivos.
O teor dessa afirmação não deixa de aparecer como no mínimo ambígua, o
que pode levar a entender que: 1) o Brasil desde o início da colonização
teve instituições arcaicas de organização e exploração sempre misturadas
ao capitalismo e à forma"-sujeito burguesa ou a seus germes e que essa
mistura cimentou a formação do Brasil até os dias atuais; e que 2)
somente o Brasil contemporâneo -- entendido como o do século \versal{XIX} --
começou a desenvolver relações capitalistas, tendentes a modernas, uma
forma"-sujeito burguesa, mas sempre com fundamentos arcaicos coloniais
que sustentavam esse impulso moderno. O autor assume grosso modo a
primeira via, o que não o impede de, em alguns momentos e
involuntariamente, parecer assumir a segunda posição, sobretudo com sua
insistência na ``atonia econômica'' e na inorganicidade de uma economia
interna, sem cuja superação qualquer outro problema do país não seria
solucionado, segundo ele. De todo modo, ele busca a formação do Brasil
na primeira via da ambiguidade, traçando uma linha evolutiva desde o
Brasil Colônia até o seu tempo, como se o futuro do desenvolvimento do
capitalismo no Brasil -- porque é disso que se trata -- já estivesse
contido no primeiro impulso do que ele próprio chama de ``simples
empresas comerciais.'' (\emph{Idem}, 2011, p. 17).

Quando Caio Prado, no início da obra, numa discussão acerca dos aspectos
metodológicos de sua rica pesquisa, considera que ``Todo povo tem na sua
evolução, vista à distância, um certo `sentido'.'' (\emph{Ibidem}, 2011,
p.15), pode"-se notar um tom transhistoricista que se coaduna com o
movimento geral de sua \emph{Formação do Brasil contemporâneo,} onde não
há uma análise acerca da especificidade do advento da modernidade, o que
faz com que a afirmação acima deixe a entender que a evolução de todas
as sociedades no fundo tem um sentido e que a sociedade colonial
brasileira também tem um sentido em sua evolução, um sentido que ``se
percebe não nos pormenores de sua história, mas no conjunto dos fatos e
acontecimentos essenciais que a constituem num largo período de tempo.''
(\emph{Ibidem,} p.15). Mas e se, no caso de uma análise do
desenvolvimento de relações mercantis e de uma forma de subjetividade
burguesa dentro do Brasil, e não somente no caso de uma análise acerca
da existência de uma \emph{Economia} virada para o mundo exterior, ``os
pormenores da história'' tiverem um sentido dissonante em relação ao
``conjunto dos fatos e acontecimentos essenciais''? Ou: e se os
``pormenores da história'' e os ``acontecimentos essenciais'' formarem
duas esferas tão imbricadas quanto cindidas? Imbricadas, já que a vida
social interna era submetida ao ritmo da produção em larga escala para o
mercado externo, e cindidas, quando se trata da vida social interna que
parece muito mais guiada pelo ritmo de uma ``atonia econômica'' -- que
não deve ser entendida como simples atraso em relação ao centro
capitalista, como pensa Prado Júnior, mas como falta de tônus mercantil
enquanto um sistema de relações com uma dinâmica interna. Estamos nos
referindo à ideia aqui defendida da existência de uma ``Economia
verdadeira'', voltada para fora, e que seria o dito ``acontecimento
essencial'', mas que só superficialmente tem repercussões sobre os
``pormenores da história'' interna, sobre as relações sociais internas
que não cavalgam no sentido da constituição de relações mercantis ou
modernas -- embora sirvam de lado obscuro de sustentação das relações
mercantis estabelecidas com o exterior.

Nesse sentido, a afirmação de que a evolução de um povo, tirando os
incidentes secundários, em conjunto se forma de ``uma linha mestra e
ininterrupta de acontecimentos que se sucedem em ordem rigorosa sempre
numa determinada orientação.'' (\emph{Ibidem}, p. 15) acaba por se
revelar uma projeção da dinâmica objetiva e subjetiva especificamente
moderna ao conjunto da história senão mundial ao conjunto da história do
Brasil, que tampouco parece comportar tamanha filosofia da história
somente possível de se vislumbrar tão logo as categorias econômicas e
sociais capitalistas alcancem um nível de maturação que lhe garanta a
submissão da vida social e subjetiva a seu movimento -- que pode ser
avassalador ou paulatino.

Para dizê-lo de outro modo, neste nosso estudo, ``os pormenores e
incidentes mais ou menos complexos'' que constituem a história de um
povo, no caso brasileiro, mais do que ``nublar o que verdadeiramente
forma a linha mestra'' (\emph{Ibidem}, p. 16) definidora dessa história,
poderão abrir os céus para novas interpretações que, em vez de ser um
embrenhar"-se equivocado no específico para entender o geral, vai tentar
ver nesse específico de incidentes o que no fundo é o geral das relações
internamente, de forma cindida das relações estabelecidas para fora.
Porque nesse caso específico, não é a nosso ver a ``linha mestra'' que
sozinha pode possibilitar a compreensão ou a explicação da evolução da
sociedade brasileira desde a colonização, já que os ``incidentes''
subsidiários à linha mestra ditaram o ritmo das relações internamente
durante bem mais três séculos, enquanto a ``linha mestra'' dessa
evolução, calcada nos objetivos da colonização, apresentava"-se com uma
feição para fora e outra para dentro. Ou seja, em vez de simples
submissa e subsidiária da Economia que tinha importância, em vez de ser
unicamente destinada ``a amparar e tornar possível'' (\emph{Idem}, p.
123) aquela \emph{Economia} essencial, a \emph{economia} interna ou de
subsistência tinha importância considerável no ritmo da vida social
interna. Porque, como diz Celso Furtado, ``havendo abundância de terras,
o sistema de subsistência tende naturalmente a crescer'', o que implica
``a redução na importância relativa da faixa monetária.'' (2007, p.
178). Ao mesmo tempo que sustentava a verdadeira Economia, servindo"-lhe
de lado obscuro fundamental, dava o caráter minimamente orgânico às
relações dentro da colônia que não seguiam o ritmo daquela
\emph{Economia} que somente para fora se mostrava como tal. Enfim, foram
esses ``pormenores'' e ``incidentes'' que ditaram o ritmo internamente
durante um longo período, foi em relação a eles que se desenvolveu a
subjetividade em termos gerais, sem que a ``verdadeira linha mestra''
desenvolvesse uma dinâmica interna em termos minimamente modernos,
objetiva e subjetivamente, uma dinâmica que dissesse algo sobre a
evolução brasileira fundada no embrenhar"-se das relações sociais
capitalistas -- que parecem ser outro tipo de linha mestra perante a
qual a linha mestra colonial passará a ser viga auxiliar.

Não é por se poder afirmar que todas as outras atividades da Colônia
existiam para manter o sistema de exportação que se pode entrar pelo
atalho da compreensão de que existiria aqui um sistema ou um germe de
capitalismo. Tampouco é o aspecto essencial que nega a existência de
feudalismo ou ``restos feudais''. A argumentação contra a existência de
feudalismo no Brasil é bastante fundamentada no próprio conceito, que
não cabe na formação brasileira de fato. No entanto, não significa que
não tenham existido relações arcaicas até bem pouco tempo. Um arcaísmo
de relações que, evidentemente, foi desaparecendo em função do que a
partir sobretudo do século \versal{XX} aparece como ``acontecimento essencial'',
como ``linha mestra'': o desenvolvimento de relações mercantis dentro do
próprio país com uma dinâmica própria. É por isso que esse lado obscuro
da economia deve ser bastante refletido para entendermos os entraves
para o desenvolvimento de relações sociais mercantis internamente.

A reflexão metodológica de Prado Júnior sobre a relação entre o caráter
particular e geral da história, bem como sobre seu caráter pretensamente
inexorável, com um fim determinado não pode dizer respeito senão a
sociedades modernas. E como o capitalismo não surgiu no Brasil, como não
houve aqui um verdadeiro \emph{boom} de modernidade, nem as trocas
mercantis exteriores serviram para criar uma dinâmica interna, trata"-se
obviamente de um país que vive a reboque dos níveis de desenvolvimento
dos países mais avançados em termos capitalistas -- e esse viver a
reboque não é anódino nem para a sociedade nem para a subjetividade.
Talvez o aspecto geral que ele defende entre em contradição com os
vários aspectos específicos e concretos que cimentaram a vida social no
terreno social e subjetivo. Essa contradição não significa que as
relações internas se opunham ao caráter exploratório de empresa
comercial nos trópicos -- já sabemos que a sustentava -- mas sim que
eram um entrave mudo a que as relações mercantis a que servia tal
exploração comercial não encontrassem terreno fértil, não tivessem uma
dinâmica internamente, já que as relações sociais internas seguiam um
ritmo distinto daquele que as trocas comerciais poderiam ditar com o
mundo exterior -- sobretudo em termos dos países mais adiantados em
termos capitalistas, notadamente a Inglaterra.

\section{O caráter orgânico e inorgânico da vida social e subjetiva burguesas no Brasil}

\begin{quote}
É assim que se formou e sempre funcionou a economia brasileira: a
repetição no tempo e no espaço de pequenas e curtas empresas de maior ou
menor sucesso. Algumas foram fulgurantes, mas pouco ou nada sobrou
delas. No conjunto, a colônia não terá nunca uma organização econômica
que mereça esse nome, e alcançará seu termo sem conseguir equilibrar
estavelmente a sua vida.

\emph{Caio Prado Júnior}
\end{quote}

O aspecto inorgânico das relações internas -- pelo menos em termos
mercantis --, o isolamento devido às grandes distâncias e às precárias
vias de comunicação, são aspectos que saltam aos olhos no processo de
formação do Brasil. Caio Prado cita as observações de um desembargador
chamado Luís Tomás Navarro quando de sua viagem em 1808 do litoral
baiano ao Rio para implantar uma linha de correios, nas quais se pode
ler que os núcleos de povoamento nessa faixa litorânea vegetavam, que
era gente de vida precária que se sustentava ``com a produção de alguns
gêneros que se exportam para a Bahia e Rio de Janeiro: farinha, peixe,
algum café.'' (\versal{JÚNIOR}, 2011, p. 48).

Para falar da falta de um caráter orgânico ou sistêmico das relações
mercantis no Brasil Colônia até o século \versal{XIX}, ou da falta de uma
dinâmica interna, basta levarmos em conta aqui três aspectos: a
organização da sociedade em termos político"-jurídicos, o papel
desempenhado pela chamada economia subsidiária, a implantação de uma
forte tradição fundada no terreno simbólico"-religioso. Três aspectos que
criavam uma espécie de organicidade inorgânica, sendo a inversão dos
termos também possível. Tratava"-se de uma organicidade, porque a
sociedade bem ou mal existiu e assim viveu por vários séculos e
progrediu, mas de uma organicidade sem o que na modernidade caracteriza
uma vida social orgânica, um fundamento econômico do qual germina uma
dinâmica interna que leva ao chamado desenvolvimento econômico.

Um primeiro aspecto que salta aos olhos no Brasil ainda no século \versal{XIX} é
a enorme mobilidade populacional, é a existência de uma larga parcela da
população migrando, procurando se acomodar nas terras, como demonstra a
expansão de Minas Gerais após a mineração, graças ao algodão, à
agricultura e à pecuária. Na região do triângulo mineiro que pertencia a
Goiás, por onde transitava o caminho de São Paulo à capital goiana, os
únicos sinais de vida humana estavam nessas vias de comunicação, ``salvo
algumas tribos indígenas mestiçadas e semicivilizadas.'' (\emph{Idem},
2011, p. 83). O indivíduo não se tinha ajustado bem ao meio,
compreendendo"-o e dominando"-o, e assim, ``Todo mundo imaginava sempre
que havia um ponto qualquer em que se estaria melhor do que no
presente.''(\emph{Ibidem}, p. 75).

Uma das razões para tal instabilidade, para tal inorganicidade das
relações internas, está nos próprios objetivos do povoamento da Colônia
que não compreendem a constituição de ``uma base econômica sólida e
orgânica, isto é, a exploração racional e coerente dos recursos do
território para a satisfação das necessidades materiais da população que
nela habita'' (\emph{Ibidem}, p. 75). Dito de outro modo, a falta
de organicidade das relações nos termos de Caio Prado está ligada à
ausência mesma de uma economia mercantil voltada para dentro da Colônia,
com sua dinâmica inerente. Não esqueçamos que o autor só denomina
\emph{Economia} a produção e extração de bens coloniais para as relações
de trocas com o exterior, o que não era capaz de gerar um
``desenvolvimento de autopropulsão'' (\versal{FURTADO}, 2007, p. 90). A produção
da subsistência, ele encara como mero apêndice da verdadeira
\emph{Economia}.

O próprio Caio Prado opõe dois tipos de sociedade existentes na Colônia:
uma orgânica e outra inorgânica. A orgânica é aquela que dá o sentido da
colônia, composta por senhores e escravos, os dirigentes da colonização
e o braço de trabalho. A inorgânica, formada pelos lugares da Colônia
onde não florescera uma \emph{Economia} baseada em produtos de
exportação. Ao que parece, esse caráter orgânico de que se revestem as
regiões onde a economia de exportação finca pés se deve em larga medida
à rigidez das posições sociais naquela estrutura que se manteve sem
dinâmica interna. Esse caráter orgânico não tem que ver com uma dinâmica
interna, mas muito mais com a falta dela. Assim, tanto a parte orgânica
da Colônia em torno da chamada Economia, quanto aquela parte inorgânica
chamada de ``vegetativa'' eram inibidoras do germinar de uma dinâmica
mercantil em termos objetivos e subjetivos.

Além disso, na sociedade colonial dita orgânica, entre suas duas classes
orgânicas garantidoras da manutenção secular da estrutura, há outra que
vai aumentando a cada momento de declínio de uma empresa mercantil: a
dos ``desclassificados, dos inúteis e inadaptados'', um elemento
indefinido socialmente cujo número aumenta, porque as causas continuam.
Trata"-se da grande maioria da população livre:

\begin{quote}
são pretos e mulatos forros ou fugidos da escravidão. Índios destacados
de seu habitat nativo, mas ainda mal ajustados na nova sociedade em que
os englobaram; mestiços de todos os matizes e categorias que, não sendo
escravos nem podendo ser senhores, se veem repelidos de qualquer
situação estável, ou pelo preconceito ou pela falta de posições
disponíveis. Até brancos, brancos puros [\ldots{}] (\versal{JÚNIOR}, 2011, p.
299)
\end{quote}

Essa população que não se enquadra ressalta não só o limbo que se
formava entre as duas classes da parte orgânica da sociedade, mas a
própria falta de organicidade mínima do todo no fim das contas, a falta
de um sistema pelo menos tendente a moderno dentro do Brasil. Esses
indivíduos que não cabem na \emph{Economia} orgânica tomam normalmente
dois caminhos: o da vadiagem e o do \emph{favor} (\versal{SCHWARZ}, 2012). Os que
entram pela vadiagem, como sempre, tendem a uma certa margem social, são
eles que implantarão o terror em algumas partes no campo e nas cidades,
fazendo com que em certo momento o Rio de Janeiro se tornasse perigoso
para quem andasse ``só e desarmado em lugares ermos, até em pleno dia.''
(\versal{JÚNIOR}, 2011, p. 302). Mas a maioria desses ``elementos indefinidos
socialmente'' vai entrar na instituição do \emph{favor}, da dependência
pessoal. Se entre o latifundiário e o escravo a relação era clara, o
``homem livre'' tinha sua entrada na vida e seu acesso aos bens
submetido ao favor, indireto ou direto de um grande: ``O favor é o
mecanismo através do qual se reproduz uma das grandes classes da
sociedade.'' (\versal{SCHWARZ}, 2012, p.17). O favor como mediação universal,
pelo menos em teoria contradiz a pretensão de uma dinâmica liberal.
Embora ainda seja uma característica marcante da forma"-sujeito burguesa
no Brasil. Não para a ele se opor, mas para que funcione melhor numa
sociedade mais fluída.

Uma fluidez que pena para germinar até mesmo no século \versal{XIX}. A própria
noção de ``vida frouxa'' (\versal{JÚNIOR}, 2011, p. 251) de que fala Caio Prado,
a enorme precariedade das vias de comunicação terrestres ou aquáticas,
faz pensar num desenvolvimento no mínimo moroso de relações tendentes a
modernas, para o qual é fundamental uma comunicação otimizada entre os
mercados existentes. O próprio ensaio de criação tendencial de um
sistema de vias de comunicação internas, aspecto fundamental para uma
organicidade, vai ser desbancado pela introdução da força a vapor -- que
leva o transporte de cargas para a água. Para Caio Prado, isso se
constitui num verdadeiro golpe na obra de cimentação do interior. E ele
lamenta que pequenas rodovias tenham sido deixadas vegetando ``numa vida
obscura'', ao mesmo tempo que lamenta uma contradição do progresso no
Brasil: ``Não chegou ainda a hora de reatarmos uma tradição e uma
tendência de tão grande alcance para a nacionalidade, e que o progresso
veio tão paradoxalmente abafar.''(\emph{Idem}, p. 281). A consideração é
interessante, sobretudo pelo caráter paradoxal como a seus olhos aparece
o progresso nessa época do Brasil. Mas numa terra onde sempre reinou a
extração de riqueza sem qualquer outra preocupação, uma mentalidade
arraigada e longe de ter sido superada, o progresso aparece sem aquela
feição que lhe é comum em sociedades que atravessam séculos de
tensionamentos, de desdobramentos de uma dinâmica mercantil interna. Em
sociedades sempre levadas a reboque pela marcha do mundo -- e a ânsia da
obtenção de riquezas valiosas no mercado exterior nada mais é do que ser
levado a reboque pelo mercado mundial --, onde não amadurecem a partir
do próprio seio, a partir de tensões internas, as condições para o
desenvolvimento de um sistema mercantil, de relações mercantis e de uma
subjetividade mercantil entrecruzada com a subjetividade voltada para o
passado, menos adaptada ao liberalismo mercantil, o progresso aparece
sem disfarce: sempre como destruição criadora do novo revestido de
progresso, sem os anos de amadurecimento que amainam por vezes o caráter
destrutivo do avançar da sociedade moderna. A violência que significou a
introdução da modernidade capitalista não se dá no desdobrar"-se das
tensões internas, mas como imperativo devido à marcha do mundo exterior
mais avançado. Assim, acontece de forma mais condensada. Quando
acontece, porque no caso brasileiro há um descompasso considerável em
relação a essa marcha do mundo exterior patente inclusive na
precariedade de suas vias terrestres, fluviais ou marítimas que
dificulta qualquer organicidade tendente a moderna. E nem falemos nos
ataques dos indígenas nas margens de estradas e rios -- porque nem todos
eram passivos.

Um outro aspecto que ressalta a inorganicidade das relações no Brasil em
termos mercantis, até o período chamado de Brasil contemporâneo por Caio
Prado, é o fato de não haver noções claras de funções e poderes do
Estado (poderes executivo, legislativo, judiciário), nem níveis de
atividades estatais locais, regionais, gerais, provinciais. Não à toa, o
clã patriarcal, em torno do qual uma grande população se agrupa, ocupará
também as funções de Estado. Se a autoridade pública é fraca, então, é o
senhor de terras a lei. Nos velhos centros tradicionais do Nordeste,
como Pernambuco e Bahia, há uma sedimentação forte e prolongada desse
tipo de poder que em nada contribui para o germinar de relações sociais
modernas.

Nesse mesmo sentido, apenas engatinhava a noção da separação tipicamente
moderna no indivíduo entre um ser jurídico, o cidadão, e um ser
moral"-religioso. Essas noções se misturavam (\emph{Ibidem}, p. 317). O
que demonstra o caráter da vida social na colônia, seu descompasso, em
vez de atraso, em relação aos centros modernos, e ao mesmo tempo a
presença de uma forte tradição com traços específicos de que falaremos
adiante. O próprio direito é disperso, desconexo e cheio de casuísmos.
Não há ou são poucas as normas gerais, trata"-se no mais das vezes de
determinações que vão se acumulando. Ao direito moderno, parece um
cipoal confuso, de modo que não se pode procurar na administração
colonial ``a harmonia arquitetônica das instituições que observamos na
administração moderna[\ldots{}]'' (\emph{Ibidem}, p. 320). O próprio
personalismo que grassa nas relações já se mostra nos governadores das
províncias que impingiam sua imagem pessoal à administração, pois não
havia papeis administrativos delimitados. Como afirma Antônio Cândido, a
unidade política que o país vai conquistar depois, e que se preservará
segundo ele ``por circunstâncias quase miraculosas'', quase esconde o
fato de que foi a diversidade que presidiu ``a formação e o
desenvolvimento de nossa cultura''. E que ``a colonização se processou
em núcleos separados, praticamente isolados entre si'', tendo um
desenvolvimento econômico e uma evolução social bastante heterogêneos
dependendo da região. (2000b, p. 267). Só no regime fiscal a
administração procurava sair da rotina. As instituições são cópias das
portuguesas em geral. Somente com muito esforço se poderá ver
internamente germes de relações realmente capitalistas. Nesse quesito, o
cipoal parece ainda mais confuso.

\section{A economia interna: o~lado~obscuro~da~\emph{Economia}}

Além desses aspectos relacionados à organicidade em termos da
organização interna e até populacionais, há o aspecto ligado à economia
interna, cujo ritmo fará a maior parte da população internamente seguir
um ritmo de vida cíclico, em contraste com o arremedo de ritmo
tautológico que a produção para a exportação pode deixar transparecer. O
aspecto da vida mercantil aparece como epidérmica e não dita o ritmo da
vida senão de modo exterior. A própria escravidão é o aspecto mais
explícito disso. Enquanto força de trabalho, é fundamental para
materializar o produto que simboliza um cordão umbilical com a marcha do
sistema mercantil mundial. Enquanto componente da vida social interna,
embora seja horrenda e desumana, joga um papel inibidor do
desenvolvimento de uma dinâmica mercantil internamente -- mesmo os
escravos sendo talvez a ``mercadoria'' mais valiosa durante quase 4
séculos no Brasil.

Nesse contexto, cabe ainda ressaltar que a maioria da população vivia
relações distintas daquelas que aos poucos se desenhavam nos grandes
centros como tendência a um sistema mercantil, sobretudo no dobrar do
século \versal{XIX}. Sem contar a distinção entre as áreas de exploração da
grande lavoura no Nordeste e as terras do Sul, incluindo o litoral sul
da Bahia e Espírito Santo. Ali, o nível de desenvolvimento econômico é
baixo e a população é considerada por Caio Prado como uma ``subcategoria
vegetativa e inorgânica'' (\versal{JÚNIOR}, 2011, p. 309). E não parece exagero
afirmar que muitos vilarejos por essas terras brasileiras se
constituíram e se mantiveram quase sem dinâmica interna por muito tempo
e com esta feição: ``Em torno da igreja paroquial e de um pequeno
comércio, a venda e a loja.'' (\emph{Idem,} p. 310). E há ainda o caso
da Amazônia, que ficou em geral naquilo que era no começo: na extração
de gêneros espontâneos da floresta. E com os cerca de 100 mil
ribeirinhos do Amazonas no início do século \versal{XIX} dispersos numa ocupação
linear e rala, ao longo de milhares de quilômetros de cursos de rios.
(\emph{Ibidem}, p. 224).

É a instabilidade, portanto, que caracteriza sempre a economia e a
produção brasileira e não lhes permite assentar numa base sólida e
permanente. Sempre não se teve em vista senão ``a oportunidade
momentânea que se apresentava.'' Para isso, mobiliza"-se momentaneamente
o que é preciso, povoa"-se certa área, com empresários e dirigentes
brancos e, depois de tudo arruinado, por alguma queda de preços ou pela
dizimação do solo ou fim do que se extraía, ``o que fica são restos,
farrapos de uma pequena parcela de humanidade em decomposição.''
(\emph{Ibidem}, p. 133). São sempre evoluções por arrancos, por ciclos,
e a cada declínio, desfaz"-se um pedaço de estrutura colonial,
desagrega"-se parte da sociedade em crise. Mais notável ainda se torna
essa desagregação no começo do \versal{XIX}, sobretudo nos distritos de
mineração. Ali, encontram"-se muitos indivíduos desamparados, sem a
certeza do dia de amanhã, ``sem ocupação normal fixa [\ldots{}] ou
desocupados inteiramente,'' indivíduos que alternam ``o recurso à
caridade com o crime. O vadio na sua expressão mais pura.''
(\emph{Ibidem}, p. 303). Como diz Furtado, não tendo sido criadas
atividades econômicas ao lado da mineira, mas somente em função dela,
era natural que com o declínio da produção, houvesse uma decadência
generalizada: ``E todo o sistema se ia assim atrofiando, perdendo
vitalidade, para finalmente desagregar"-se numa economia de
subsistência.'' (2007, p. 132).

A tese aqui defendida da falta de dinâmica ou de organicidade das
relações e da subjetividade mercantis dentro no Brasil até o século \versal{XIX}
encontra fundamento também nas considerações de Caio Prado, embora ele
não faça uma distinção clara entre a exploração capitalista de alguns
produtos de exportação e o desenvolvimento de relações mercantis dentro
do Brasil. Mas, em várias passagens de sua \emph{Formação}, ele próprio
considera que ``a colônia não terá nunca uma organização econômica que
mereça esse nome, e alcançará seu termo sem conseguir equilibrar
estavelmente a sua vida.'' (\versal{JÚNIOR}, 2011, p. 133).

Já dissemos que a \emph{Economia} que importava não era capaz de
impulsionar qualquer lampejo de relações mercantis e modernas dentro da
colônia. Do mesmo modo, internamente, não amadureciam relações com
tendência a uma organicidade mercantil -- e não esqueçamos que é disso
que se trata aqui: refletir o quanto a marcha do todo não se refletia em
termos das relações sociais internamente, o que é diferente de um mero
atraso ou da imposição externa de relações, por exemplo, do tipo
escravista como aspecto do próprio capitalismo da época. Também não é
por termos relações escravistas, fundamentais ao capitalismo da época,
que se pode chamar de capitalismo o que havia na colônia.

Internamente, a pequena economia -- aquela precária, subsidiária, mero
``apêndice'' que fazia a colônia viver internamente -- não teria como
amadurecer até uma dinâmica interna em termos modernos, porque nela não
tinha o embrião da tautologia moderna. O ritmo interno era cindido
daquele das trocas mercantis, era um ritmo muito mais cíclico, que
sustentava a produção do produto de exportação. O aspecto mercantil era
muito mais de acumulação arcaica de riqueza monetária voltada para fora
do que de acúmulo tautológico em vistas de reinvestimento para novo
acúmulo -- o próprio aspecto secular do investimento o denuncia. Embora
isso pudesse acontecer em casos concretos de subjetividades mais em
harmonia com o espírito moderno, não era o que determinava as relações
internas a ponto de criar uma revolução contínua das relações.

Outro aspecto fundamental que contradiz qualquer pretensão de que no
Brasil até o século \versal{XIX} tivéssemos vida social capitalista se esvai
quando nosso senso crítico nos lembra que o que, antes de tudo o mais,
``caracteriza a sociedade brasileira de princípios do século \versal{XIX} é a
escravidão'' (\emph{Idem}, p. 285). Com essa afirmação, Caio Prado,
contra sua própria intenção, já dá o tom do que caracteriza o Brasil do
início do século \versal{XIX}: tudo menos um país capitalista ou moderno. Como
poderia um país capitalista ou em desenvolvimento capitalista basear"-se
na escravidão? E não se trata de meras \emph{ideias fora do lugar} que
atrasariam o país, uma impropriedade que não atrapalharia a marcha do
sistema capitalista no país. Trata"-se do nervo central de qualquer
desenvolvimento moderno: uma mão"-de"-obra formalmente livre que
impulsiona a monetarização da vida social. Em verdade, no máximo se
poderia falar de um país que estabelece relações capitalistas com outros
países, sem que as consequências inerentes ao processo de vida social
capitalista pudesse se fazer sentir internamente. Já falamos dessa
cisão. A análise de Schwarz sobre essas \emph{ideias fora do lugar}
expressam bem essa incongruência -- embora para ele essa incongruência
não deixasse o país menos capitalista -- e ao mesmo tempo o quanto não
havia uma relação social capitalista no Brasil ainda no século \versal{XIX}, a
não ser que queiramos entender a forma"-social capitalista como uma forma
de sociedade em que um grupo de pessoas faz produzir ``mercadorias'' que
vendem a preços lucrativos no exterior. A não ser que também queiramos
entender a forma"-sujeito burguesa como sendo uma forma de subjetividade
de grupos que se dedicam a obter lucros sobre produtos coloniais no
exterior, sobre produtos produzidos com mão"-de"-obra que não pagam
monetariamente e que se constituem como uma classe de privilegiados
sociais e econômicos.

Sendo a escravidão o sangue que corre nas veias da organização social, o
aspecto que dá a tonalidade da vida social e econômica, não há como essa
terrível instituição não inibir o desenvolvimento de relações mercantis
internamente e não participar do caráter dito ``atrasado'' do país.
Porque a riqueza se mantém por demais concentrada, quase entesourada,
sem uma classe assalariada -- que é o mínimo de fundamental para
qualquer desenvolvimento de relações mercantis.

Em certo sentido, concordando com Caio Prado, o comércio interno dá a
chave para entender a dinâmica interna. Não em termos de produção de
\emph{valor}, mas em termos de monetarização da economia. O comércio do
Brasil durante muito tempo se divide naquela parte de exportação e
naquela que serve a essa produção -- e nunca se trata de um sistema
minimamente amplo de trocas comerciais, mas da venda de determinados
produtos rentáveis. O comércio de escravos em primeiro lugar, depois o
de bens para a subsistência da população das cidades -- quando não era
produzido na propriedade -- e em último lugar o comércio de coisas que
poderiam deixar a vida dos senhores menos pesada nos trópicos. Como diz
Furtado: ``Não existindo na colônia sequer uma classe comerciante de
importância -- o grande comércio era monopólio da metrópole -- resultava
que a única classe com expressão era a dos grandes senhores agrícolas.
(\versal{FURTADO}, 2007, p. 144), dos quais dificilmente germinaria, como foi o
caso, tensões para uma forma subjetiva burguesa. Mesmo um produto
extremamente comerciado como o próprio charque surgiu para substituir o
comércio de boi vivo, comerciado no litoral, pois o interior tinha sua
própria carne em geral, ou não tinha e não podia pagar monetariamente.
No mais, o comércio diz respeito ao abastecimento dos grandes centros
urbanos -- quando começam a ser grandes de fato -- ``pois as populações
e estabelecimentos rurais são em regra, a este respeito, autônomos.''
(\versal{JÚNIOR}, 2011, p. 247). As pessoas das aglomerações do campo procuram as
cidades só em busca do pouco que não têm, produtos importados: sal,
ferro, manufaturas. Caio Prado confirma por suas considerações que o
comércio no interior não pode nem ser levado em consideração pelo seu
tamanho, porque é só o comércio com as zonas urbanas de produtos de
subsistência que aparece com ``vulto ponderável''. Ao mencionar comércio
no interior, eu me refiro ao comércio entre as próprias pessoas, entre
arraiais e freguesias, em locais onde ``O atrofiamento da economia se
acentua à medida que aumentam as distâncias do litoral'' (\versal{FURTADO}, 2007,
p. 104). E não às aglomerações urbanas normalmente do litoral onde desde
sempre houve certo fluxo de dinheiro, embora o fluxo considerável fosse
em relação à Metrópole e sempre em poucas mãos.

De certo modo, a circulação monetária parece passar sempre nas mesmas
mãos em quantidades dignas de consideração, e sempre naquelas mãos
conservadoras cujo pensamento é se dar bem na colônia e enganar qualquer
lei que possa vir a limitá-lo. Entre a maioria da população, para não
falar nos escravos, a circulação monetária é algo raro. Grande parte das
somas oriundas das exportações eram gastas em gêneros ditos de luxo,
como azeite, vinho e manufaturados de ferro e até sal num primeiro
momento. Sem contar o maior negócio de importação do reino: o escravo
que, nos fins do século \versal{XVIII} e começo do \versal{XIX}, representava cerca de ¼
de toda a importação.

Além de todos esses elementos que vão ao encontro da inexistência de uma
dinâmica interna em termos mercantis, faltava à Colônia, além do
elemento mão"-de"-obra formalmente livre, qualquer arremedo de indústria,
nem que fosse de ínfima importância. A vida econômica no conjunto
nacional está em outras atividades ou semiatividades. Qualquer progresso
em terrenos particulares anulou"-se,

\begin{quote}
sobretudo num momento em que o progresso industrial do mundo marcha a
passos de gigante, e a ciência põe a serviço dela, cada dia, mais um
invento, uma técnica, uma possibilidade. Se tomadas as devidas
proporções, a nossa minúscula indústria colonial não representaria para
sua época mais que esses simulacros de atividades manufatureiras que
tivemos no século \versal{XIX}. (\versal{JÚNIOR}, 2011, p. 240)
\end{quote}

Essa chamada indústria, e houve leis para impedir seu desenvolvimento,
significa mais a pequena indústria doméstica -- fiação, tecelagem e
costura -- que tem seu papel, menos pelo aspecto comercial e mais pela
complementação da autonomia dos grandes domínios rurais.

Podemos citar outro fator que alimenta a \emph{inorganicidade da
Economia} interna ou a \emph{organicidade de uma economia interna
desmonetarizada:} aquele que dirige todos os serviços da fazenda de gado
é normalmente pago ``com o próprio produto dela, uma quarta parte das
crias.'' (\emph{Idem}, p. 202), embora os chamados fábricas que
auxiliavam os vaqueiros e que faziam as roças recebessem pagamento
monetário. Pode parecer de pouca monta, mas o sistema de pagamento dos
quartos de crias contribuiu fortemente não só para o devassar"-se dos
sertões do Nordeste, para o estabelecimento de povoamentos longínquos
mergulhados em tradições próprias, mas também para a perpetuação dessa
economia bastante desmonetarizada, não totalmente voltada para a venda.
Com suas rezes, o vaqueiro podia montar sua fazenda. Assim, até pessoas
modestas podiam ter a sua, arrendada de uma sesmaria ou comprada. Mas,
evidentemente, o que prevalece é o dono de várias fazendas que mora no
litoral e só recebe seus rendimentos. Há uma concentração de terras
enorme. Cita Caio Prado que o sertão da Bahia pertencia a duas famílias
e que no Piauí um só proprietário tinha 40 fazendas.

A própria indústria mineradora contraditoriamente contribuiu para a
inorganicidade das relações mercantis na Colônia, já que seu caráter
continuou acompanhando o espírito geral e nunca deixou de ser uma
aventura passageira -- embora fundamental para a Metrópole e a
Inglaterra. Com métodos arcaicos, quando cessava uma produção aqui,
ia"-se a outro canto, deixando sempre destruição. Evidentemente, deve ser
levado em conta nesse caráter inorgânico das relações mercantis dentro
da Colônia a incapacidade do reino no reinado de D. João V, que só se
interessava em esbanjar com sua corte de parasitas, que só se preocupava
com sua parte, o quinto. (\emph{Ibidem}, p. 181). O que demonstra, por
sua vez, o quanto a forma de subjetividade da chamada classe dirigente
de uma sociedade conta para o impulso ou para o entrave ao
desenvolvimento de relações sociais modernas. Porque uma forma de
subjetividade em consonância com o espírito moderno -- que não era ao
que parece o caso português -- teria encarado a riqueza do ouro de forma
distinta.

E poderíamos citar uma série de outros aspectos que não fazem senão
comprovar que o país não tinha os germes de uma forma social e subjetiva
moderna ou mercantil ou até mesmo tendente a moderna até o século \versal{XIX}. A
falta até do mais rudimentar sistema de ensino; o arcaismo no
beneficiamento de um bem de valor no mercado internacional, o algodão,
num momento em que o mundo já usava técnicas mais avançadas; a própria
subutilização de forças como a hidráulica em proveito de duas forças
mais rudimentares, a animal e a humana, num momento em que nos locais
para onde iam os valorizados produtos coloniais, na Europa, vivia"-se
plena revolução industrial, já uma revolução sobre modos de produção
avançados na Europa, se comparados com os da colônia que nem manufatura
tinha.

Salta aos olhos, quando lemos os estudos sobre Brasil Colônia, o aspecto
pobre das casas das cidades: nos três primeiros séculos da colonização,
simplicidade e pobreza é o que caracteriza as moradas que eram habitadas
por pessoas de poucos recursos, ``e visitada ocasionalmente por
proprietários de sítios e fazendas que necessitavam apenas de um abrigo
para estadias passageiras.''. (\versal{ALGRANTI}, 1997, p. 90). Os próprios
sobrados e vivendas erigidos pela elite econômica só surgirão mais
tarde, ``em consequência da diversificação da economia e do crescimento
urbano.'' O que está em relação estreita com a germinação de um mínimo
de sistema econômico interno, de uma organicidade social interna em
termos mais amplos, o que não existia. O próprio fato de a chamada
\emph{Economia} ser voltada para o exterior deixava uma outra
\emph{economia}, essa muito ligada ao conceito antigo e etimológico da
organização da casa, para um segundo plano, como desimportante. Mas era
essa economia arcaica e subsidiária que fundamentava a vida colonial
interna, e era ela que determinava em muitos sentidos a própria
inexistência de um impulso comercial dinâmico interno do qual pudesse
surgir uma forma de subjetividade mercantil voltada para o futuro. Era
essa economia secundária que, para Celso Furtado, dificultava, pela sua
dispersão, a própria constituição de uma mão"-de"-obra no Brasil que
pudesse ser usada em atividades mais produtivas economicamente (2007, p.
177).

Ao visitarmos um pouco pelos historiadores a vida doméstica das famílias
coloniais, vemos o peso que tinha essa economia secundária, que tem esse
nome apenas porque o que importava em termos capitalistas era a
primeira, a da grande lavoura -- sem esquecer a mineração. Sem exagero,
porém, pode"-se dizer que era essa economia subsidiária a mais importante
internamente na colônia, uma economia arcaica e desmonetarizada que se
forjava em ``quintais, jardins, pomares e hortas,''(\versal{ALGRANTI}, 1997, p.
91) áreas destinadas ao ``convívio, ao cuidado dos animais e à indústria
doméstica'' de onde vinham os produtos para a subsistência. Se nos
fiarmos em Leila Mezan Algranti, eram pomares e hortas -- muito
presentes ``nos registros dos séculos \versal{XVI} ao \versal{XIX}'' (\emph{Idem,} p. 95)
-- que forneciam muitos alimentos básicos aos colonos. Se era assim nas
grandes propriedades, imaginemos quantas piruetas para a subsistência
não precisavam dar aqueles que não se encaixaram nas duas principais
classes: a de senhores e a de escravos, que estavam entre as duas. Ainda
segundo a autora, era imprescindível a autossuficiência em sítios e
fazendas, ``tendo sofrido muito as zonas monocultoras quando esta não
foi atingida'' (\emph{ibidem,} p. 143). E para Gilberto Freyre, a
própria fartura dentro do país descrita por alguns cronistas precisa ser
relativizada, porque o Brasil dos três primeiros séculos de colonização
não teve nada de um ``país de cocanha'', sendo terra de alimentação
incerta (2003, p. 100). Dito de outro modo, a economia de subsistência
-- relegada a subsidiária daquela que seguia o compasso do mundo
exterior -- era uma espécie de revés obscuro da dita verdadeira
\emph{Economia}, que não determinava as relações sociais em nível
interno senão indiretamente, como já dissemos.

E é o próprio Caio Prado quem afirma ser a economia interna um mero
apêndice, e a grande lavoura o nervo da economia colonial. Ele, porém,
interpreta esse ``apêndice'' como atraso, não como impedimento ao
germinar de relações sociais mercantis. E era esse apêndice que produzia
os ``gêneros destinados à manutenção da população do país, ao consumo
interno.'' (\versal{JÚNIOR}, 2011, p. 149), mantendo uma economia
interna em larga escala desmonetarizada, algo constantemente lembrado e
lamentado por Celso Furtado em sua \emph{Formação}. Diz ele que ainda no
século \versal{XIX}, com o processo de independência, o governo precisava
imprimir papel moeda para fechar o caixa, o que causava inflação e
aumentava o preço dos importados. Mas quem sofria as consequências do
fenômeno era a população urbana, já que no campo os senhores tinham tudo
produzido pelos escravos, cuja força de trabalho extorquida amortecia os
gastos monetários e fazia com que a autossuficiência imunizasse aquele
microcosmo das emissões de papel"-moeda. (\versal{FURTADO}, 2007, p. 148).

Com a produção de mandioca, milho, feijão sendo mero apêndice, não por
acaso gozaram de preço baixo até serem transformados em verdadeiras
mercadorias como todos os gêneros alimentícios. Os produtos que faziam
sobreviver a colônia eram de tal modo desprezados por não terem um valor
no mercado que era um desperdício plantá-los. É Caio Prado quem nos
relata a indignação de um produtor com a lei que proibia a exportação de
bens alimentares entre capitanias. Ele, querendo ocupar sua terra com
mercadorias de fato, não queria desperdiçá-la com produtos que apenas
serviam para comer: ``Não planto um só pé de mandioca para não cair no
absurdo de renunciar à melhor cultura do país pela pior que nele há.''
(\versal{JÚNIOR}, 2011, p. 172).

Como nota Caio Prado, é exatamente a mandioca o maior legado dos índios
e o principal produto de subsistência no Norte"-Nordeste. A mandioca
ocupou durante muito tempo no Nordeste o lugar que o trigo ocupou e
ainda ocupa em muitas áreas. A mandioca talvez seja, pelo menos no
Nordeste, o grande elo com uma economia subsidiária de outros tempos que
se manteve até bem pouco tempo. Eu mesmo tive ocasiões até o início dos
anos 1990 de participar da vida em torno do tripé cana"-de"-açúcar -- cujo
produto principal no Nordeste, fora dos grandes centros, ficou sendo a
rapadura --, mandioca e café -- sendo os dois primeiros não só os menos
monetarizados, mas os mais carregados de aspectos simbólicos e sociais.
A poesia de Patativa do Assaré (\versal{ASSARÉ}, 2008) é cheia desse simbólico
que permaneceu ainda no século \versal{XX}. Ele, por exemplo, chama o motor que
invadia as casas de farinha e engenhos de ``estragêro enxerido'' (p.
29), que acaba as festas animadas das farinhadas, que tira a poesia
daquelas duras lidas, que tira o gosto da rapadura. Também tive a
oportunidade de ver as casas de farinha sob ruínas desde então e as
moendas dos engenhos sendo vendidos como conjunto de ferragens informes.
E não deixa de chamar a atenção que, mesmo sendo uma economia
subsidiária, a mandioca era o gênero essencial que dominava a
subsistência da colônia, já que o milho chega só mais tardiamente ao
Nordeste e ao Norte. Até mesmo na Bahia era raro o milho. Já dali rumo
ao Sul até Santa Catarina, há bastante milho. Depois da mandioca e do
milho, vem o feijão e o arroz. O arroz, embora exportado pelo Maranhão,
é cultivado em toda a colônia para consumo interno.

Em princípio, a população rural ocupada nas grandes lavouras e nas
fazendas de gado, ``e que constitui a maior parte do total dela, provê
suficientemente à sua subsistência com culturas alimentares a que se
dedica subsidiariamente, sem necessidade de recorrer para fora.''
(\versal{JÚNIOR}, 2011, p. 166). Já a situação nos centros urbanos é diferente,
onde há a importação de muitos bens, embora haja muitas propriedades que
cultivam sobretudo nos entornos dessas grandes cidades que se formavam
por causa da grande lavoura, que não deixava espaços cultiváveis por
perto para a agricultura de subsistência, chamada de ``mesquinha
plantação''. Dessa forma, as culturas alimentares estão normalmente
longe das cidades que são, entretanto, seu único ``mercado''. Somente as
áreas que não serviam para a grande lavoura é que eram deixadas para as
``mesquinhas plantações'' da cultura alimentar. E é em torno desses
cultivos que se formam lugarejos estreitamente ligados às necessidades
alimentares dos grandes centros da colônia. Povoamentos, portanto, não
guiados pela produção em larga escala para exportação, mas produtos de
subsistência baratos. Mas também a pecuária, incluindo a criação de
animais de carga, significava também povoação guiada pelo ritmo interno.
Em suma, os principais centros de povoação eram devidos à grande
lavoura, mas outras formas de povoamento eram possíveis, sem esquecermos
os aldeamentos que sobraram das reduções jesuíticas. Diz Caio Prado:
``As boiadas também contribuem, embora em menor proporção, para o
povoamento e uma certa atividade agrícola em zonas que de outra forma
teriam sido desprezadas.'' (\emph{Idem}, p. 171).

O que no fundo interessa ressaltar é que a organização da subsistência
-- embora também haja grandes faixas de terra especializadas nos víveres
internos -- vai desde essa grande propriedade -- a exceção -- até as
chácaras, sítios, roças onde não há assalariados ou escravos e o
proprietário ou agregado é o trabalhador. O agregado é aquele que recebe
do proprietário, ``em geral a título gratuito e em troca apenas de uma
espécie de vassalagem e prestação de pequenos serviços, o direito de se
estabelecer e explorar uma parte inaproveitada do domínio''.
(\emph{Ibidem}, 2011, p. 167). Certamente não se pode ver aí quaisquer
restos feudais, porque não era esse aspecto o que determinava as
relações da colônia. Trata"-se de uma relação particular que se misturava
a uma série de outras relações de matiz não moderno -- incluindo a
escravidão -- que se constituíam como um lado obscuro do sistema
principal, algo sem o qual ele não poderia ir adiante, mas que não era o
seu fundamento como sistema geral.

Assim mesmo, há evidências da existência de áreas de subsistência que
não necessariamente dependiam diretamente do setor principal e que
levavam uma vida isolada na subsistência, embora, evidentemente, a
subsistência tenha se dado na maioria dos casos atrelada à atividade
principal ou ligada a ela no próprio local da grande lavoura, e não em
núcleos de povoados isolados na subsistência. Mas isso não impede que se
possa ver nesse tipo de relação algo bastante corrente pelo interior,
sobretudo em lugares onde a miséria e o declínio de um produto de
exportação jogava essa faixa de terra na subsistência. Essa relação
devia ser muito presente e deixou marcas na sociedade sobretudo
nordestina, dos interiores do Brasil, e que incrivelmente até a difusão
dos meios de comunicação de massa em larga escala a partir dos anos 1980
pareciam estar em outro tempo histórico em termos de relações sociais e
de subjetividade. Não à toa, a mentalidade urbana parecia um universo
quase que completamente distinto da mentalidade daqueles que viviam
longe dos grandes centros da dinâmica mercantil.

E não se pode menosprezar o peso da chamada indústria caseira de
subsistência na formação social brasileira e na inibição de um
desenvolvimento de formas sociais ditas modernas e de uma subjetividade
burguesa. Além da grande lavoura que se relacionava com o mundo exterior
à Colônia, havia internamente uma série de atividades cotidianas ligadas
à ``alimentação, ao vestir, à construção e fabricação de utensílios de
uso diário'' (\versal{ALGRANTI}, p. 143). Sendo assim, não é sem importância para
as relações internas à Colônia a preocupação primordial que recebia a
alimentação por parte de qualquer família, rica ou pobre, que ``exigia
cuidado com os animais e com as roças de subsistência, uma série de
atividades, como a preparação dos alimentos que seriam usados nas
refeições diárias, fosse a preparação da farinha, do milho ou da carne,
ou a própria fiação doméstica, artesanal. Todas atividades que davam
vida à Colônia internamente, sem as quais a grande empresa de exploração
de produtos coloniais para o mercado exterior não teria sido possível,
mas que não tinham características de uma \emph{Economia.}

Podemos também incluir nesse complexo de relações internas o exemplo
também levado em consideração por Sérgio Buarque (2014, p. 70) de uma
relação herdada dos indígenas, o mutirão, que também entra nesse caldo
de relações ligadas a uma economia sem importância. Uma prática fundada
no auxílio mútuo, na partilha das ceias e danças, algo que existiu no
Brasil até pouco tempo e que, no caso particular, não seria impossível
ainda encontrar.

Nesse mesmo sentido de uma economia interna sem dinâmica, há nas regiões
de grande produção do açúcar esse verdadeiro microcosmo social ou, como
diz Caio Prado, ``um verdadeiro mundo em miniatura, em que se encontra e
resume a vida toda de uma pequena parcela de humanidade.'' (\versal{JÚNIOR},
2011, p. 153), o engenho. Homens livres só em pequenos números:
caixeiros -- que faziam as caixas para o açúcar -- mestres, feitores,
purgadores. A cachaça é um produto secundário, mas importante para o
mundo exterior, para a troca nas costas da África por escravos. A
rapadura é feita no sertão do nordeste e substitui o açúcar. E esse
mundo em miniatura, no qual impera a figura do senhor patriarcal dono de
terras, em torno do qual se criam vínculos sociais cultivados pela
``mística de fidelidade ao chefe como técnica de preservação do grupo
social'' (\versal{FURTADO}, 2007, p. 179), não é um microcosmo feudal, mas também
não é um microcosmo de relações capitalistas, longe disso.

O engenho era um organismo completo e autossuficiente. A alimentação
procedia das plantações, das criações, da caça, da pesca. Até móveis
vinham de serrarias ali instaladas. Como poderia surgir dinâmica interna
em locais de tamanha autossuficiência? E isso durou, segundo Sérgio
Buarque, até mesmo no decorrer do século \versal{XIX}: ``Sabemos que, durante a
grande época do café na província do Rio de Janeiro, não faltou lavrador
que se vangloriasse de só ter de comprar ferro, sal, pólvora e chumbo,
pois o mais davam de sobra em suas terras.''(\versal{HOLANDA}, 2014, p. 95).
Claro que esse quadro se prolongou por várias regiões dependendo do
ritmo que o desenvolvimento da economia moderna tomou em cada caso. Mas
o germinar de relações capitalistas nunca foi um mero desabrochar de
germes já contidos nesse tipo de economia primária -- ou uma mera
superação do atraso colonial.

Os economistas que veem o Brasil colônia pelas trocas comerciais, que
olham pelos números em libras das trocas comerciais do Brasil, deixam"-se
muitas vezes obnubilar pelas vultosas somas em detrimento do que se
passa internamente, no cotidiano, nas entranhas sociais. Porque somente
com muita imaginação se pode imaginar um capitalismo no Brasil desde o
começo, a não ser que se pense num capitalismo \emph{sui}
\emph{generis}, cujas repercussões sempre dinâmicas aqui não se faziam
sentir. A própria passagem em repetição por ``pequenas e curtas empresas
de maior ou menor sucesso'' de exploração do território, seja com a
cana"-de"-açúcar, o extrativismo, o ouro, o algodão até chegar ao café,
não consegue ser o aspecto propulsor e ativador de uma sistema econômico
mercantil dentro do Brasil. O que no mínimo leva a consideração
teleológica de Caio Prado de que as raízes do Brasil contemporâneo estão
na estrutura econômica da Colônia a uma avaliação crítica mais apurada.
Porque não parece plausível considerar que ``essa evolução cíclica, por
arrancos, em que se assiste sucessivamente ao progresso e ao
aniquilamento de cada uma e de todas as áreas povoadas exploradas do
país,'' (\versal{JÚNIOR}, 2011, p. 133) possa constituir, fazer vicejar uma
forma"-social e uma forma de subjetividade modernas, porque as bases
sobre as quais se assenta mesmo a dita \emph{Economia} são
``precaríssimas'':

\begin{quote}
Não constitui a infraestrutura própria de uma população que nela se
apoia, e destinada a mantê-la; o \emph{sistema organizado da produção e
distribuição de recursos para a subsistência material dela}; mas um
``mecanismo'', de que aquela população não é senão o elemento propulsor,
destinado a manter seu funcionamento em benefício de objetivos
completamente estranhos. Subordina"-se portanto inteiramente a tais
objetivos e não conta com forças próprias e existência autônoma.
(\emph{Idem}, p. 133)
\end{quote}

Sendo assim, o sucesso econômico de algum produto no mercado
internacional impulsiona o funcionamento da \emph{Economia} dando
ilusão, no dizer de Caio Prado, de riqueza e prosperidade, quando na
verdade o funcionamento interno não se assenta em quaisquer bases das
quais pudesse germinar relações mercantis no sentido moderno. A forma
tomada pela colônia vai afastar qualquer desenvolvimento interno, vai
manter a vida interna sem impulso mercantil, longe de um desenvolvimento
mercantil -- evidentemente Caio Prado pensa que a bases estão
germinadas, apenas sofrem de atraso pela estrutura da colônia.

Um sinal de que nossa tese da falta de organicidade das relações
mercantis dentro do Brasil até o século \versal{XIX} tem algum fundamento pode
estar até mesmo neste trecho de Caio Prado, para quem o caráter ``mais
estável, permanente, orgânico, de uma sociedade própria e definida, só
se revelará aos poucos'', porque permanecerá ``dominado e abafado''
pelos fundamentos dos séculos anteriores, que continuarão ``mantendo a
primazia e ditando os traços essenciais da nossa evolução colonial''.
(\emph{Ibidem}, p. 29). Ora, estamos em contato aqui com um
entrecruzamento entre formas sociais, em que a forma colonial ecoa sobre
as formas modernas que vão se delineando, grandemente imbuídas do
colonial que, em vez de empecilho, é a forma mesma da sociedade evoluir
dentro do enquadramento capitalista. Mesmo essa época de caráter mais
estável -- que deve ser entendida como o Brasil contemporâneo com o qual
o autor se ocupa -- está arraigada no velho colonial. Mas o fato de as
formas coloniais ecoarem na forma social contemporânea, que tem a nosso
ver um fundamento completamente distinto, não confirma a ideia de Caio
Prado de um fim determinado desde a Colônia. O que era o fundamento da
Colônia agora aparece como um resquício arcaico que em nada atrapalha ou
inibe a dinâmica das relações mercantis -- como era o caso na Colônia. É
aí que há um salto de qualidade que é preciso levar em conta. Mas não
parece haver corte para Caio Prado, porque para ele, a partir do começo
do século \versal{XIX}, começa o que chama de Brasil contemporâneo, quando se
alcançam os elementos constitutivos de nossa nacionalidade acumulados
desde o início da colonização. O Brasil contemporâneo seria para ele
erigido sobre aquela base. É como se para o autor se tratasse de mero
avançar ou desabrochar de formas já presentes desde o início, mas sem
organicidade.

Caberia lamentar essa atrofia das relações mercantis desde os primórdios
do Brasil, caberia lamentar a ``atonia econômica'' como fazem Caio Prado
e Celso Furtado em proveito do chamado desenvolvimento econômico em
bases sólidas e modernas? Para nós, não cabe escolher entre os dois
caminhos, o nosso caminho é o da crítica das formas"-sociais mercantis,
sejam as baseadas em formas arcaicas, ou em formas avançadas e
dinâmicas.

\section{O simbólico"-religioso brasileiro: uma~tradição~colonial}

A colonização, como empresa mercantil de exploração dos trópicos dentro
de um quadro de objetivos amplo de modernização da Europa, ela mesma
sacudida pelos questionamentos modernos do fundamento
simbólico"-religioso, não deixou de plantar uma tradição no Brasil. Se
não tínhamos um fundamento feudal, nem tínhamos qualquer germe de
relações sociais modernas, qual seria tal tradição? Bem específica. É
uma tradição em que traços arcaicos cimentam a vida social internamente
misturados com o aspecto do poder ligado à exploração da terra. O que
não significa que nessa tradição bem nossa obrasse a condenação tenaz da
exploração das novas terras. Em verdade, trata"-se de uma tradição
misturada ao interesse econômico, sem ter nada de ética protestante.
Porque se trata de uma vida social em que os sujeitos concretos que
estabeleciam relações com o exterior, ou até mesmo a Metrópole, não
tinham a menor consciência de que estavam participando da marcha moderna
nas suas relações principalmente com a Inglaterra. Essa tradição,
portanto, vai poder se arraigar, porque dentro da Colônia não há uma
dinâmica moderna que possa se lhe opor. Ao contrário, a exploração
mercantil da Colônia não vai se chocar com nosso esteio
simbólico"-religioso. Essa tradição vai deitar raízes no espaço daquela
cisão entre os dois mundos da colonização: o que é voltado para fora,
que submete a vida social à produção para o comércio exterior, e o que
segue o ritmo interno, despregado em termos das consequências econômicas
da dinâmica daquele mundo voltado para fora.

E trata"-se de uma tradição especificamente brasileira que vai fincar
raízes profundas, mas ao mesmo tempo superficiais, maleáveis a
interesses. Trata"-se de uma tradição simbólico"-religiosa entranhada numa
sociedade baseada no poder econômico daqueles que produzem os produtos
ditos econômicos envolvidos em transações mercantis. Uma tradição que
vai fazer com que não seja somente esse poder econômico o que vai
constituir o prestígio, o poder e o que vai determinar o padrão das
relações. No clã patriarcal, há um importante emaranhado de relações
pessoais, ligadas à figura do senhor de terras, que uma leitura
meramente econômica das relações não teria como apreender. As relações
extrapolam o caráter de exploração econômica. Criam"-se relações de
submissão pessoal e simbólica que serão instituídos pela religião: ``o
testemunho nas cerimônias religiosas do batismo e do casamento, que
criarão títulos oficiais para elas: padrinho, afilhado, compadres.''
(\versal{JÚNIOR}, 2011, p. 306). Outra faceta desse processo de cimentamento de
uma tradição na Colônia -- evidentemente herança trazida com os colonos
e desenvolvida pela Igreja -- é o fato de a dita aristocracia colonial
tomar ares de todas as aristocracias, dando importância considerável a
honra, orgulho, tradição, sangue, família. E não somente nos centros em
que aflorou incrivelmente o poder patriarcal em torno dos engenhos se
faz sentir tal tradição. O patriarcado é uma relação generalizada,
embora em Minas, como observa Caio Prado, os proprietários de terra
apareçam mais humildemente, com ares menos aristocráticos e mais rudes
(\emph{Idem}, p. 306), inclusive trabalhando a terra com os filhos --
impensável para os senhores instalados no Nordeste, para os quais a
atividade física era uma desonra.

Já explicitamos em outro lugar que a forma"-social capitalista é uma
forma de síntese social que tem um fundo religioso, não no sentido de
cultos, adorações, mas no sentido de que ele tende a ser o fundamento
simbólico, o elo de ligação e de sentido social total que era o
simbólico"-mítico"-religioso em outras sociedades, o próprio fundamento da
vida na Terra. No Brasil, um outro aspecto que entrava o desabrochar de
um capitalismo interno à Colônia desde os primórdios é o peso que tem na
vida social o fator religioso, um peso que vai ao encontro da ideia de
um descompasso ou de cisão entre as relações que a Colônia e sua
Economia estabeleciam com o mundo exterior e as relações sociais
internas.

A vida na colônia não é só mergulhada na produção de produtos
valorizados no mercado externo, é mergulhada também na religiosidade e
na atmosfera clerical. Há a onipresença de um conjunto de crenças e
práticas que o indivíduo já encontra ao nascer e que o acompanharão até
o fim da vida,

\begin{quote}
mantendo"-o dentro do raio de uma ação constante e poderosa. Ele
participará dos atos da religião, das cerimônias de culto, com a mesma
naturalidade e convicção que de quaisquer outros acontecimentos banais e
diuturnos da sua existência terrena; e contra eles não pensaria um
momento em reagir. Será batizado, confessará e comungará nas épocas
próprias, casar"-se"-á perante um sacerdote, praticará os demais
sacramentos e frequentará gestas e cerimônias religiosas com o mesmo
espírito com que intervém nos fatos que chamaríamos hoje, em oposição,
da sua vida civil. Uma coisa necessária e fatal, como vestir"-se, comer a
certas horas, seguir um regime de vida geral para todo o mundo.
(\emph{Idem}, p. 349)
\end{quote}

A incredulidade ficará em grupos pequenos e fechados. Os incréus, mais
que criminosos, ``\emph{apareceriam como loucos
temíveis.''}(\emph{Ibidem}, p. 349). Pela pesquisa de Mott:

\begin{quote}
a missa obrigatória aos domingos e dias santos de guarda -- um total de
98 feriados! -- a obrigação da desobriga pascal (atestado assinado pelo
vigário que o freguês confessou"-se e comungou ao menos uma vez por
ocasião da Páscoa da Ressurreição), a indispensabilidade da frequência
aos sacramentos, são algumas práticas religiosas amalgamadoras do corpo
místico no Brasil de antanho, um contrapeso socializador significativo
para compensar a dispersão espacial e isolamento social dos colonos na
imensidão da América portuguesa. (1997, p. 159)
\end{quote}

Não só juridicamente a igreja tinha poder, mas também na vida cotidiana
mais íntima. Lembremos ainda que uma das facetas da tradição brasileira
é a sua especificidade de ser ao mesmo tempo um fundamento
simbólico"-religioso e um esteio à exploração mercantil da Colônia. O seu
papel como fundamento simbólico"-religioso tinha um peso considerável.
Basta citarmos o seu papel na constituição mesma da subjetividade do
``cidadão'' da Colônia, um indivíduo que não vivia a religião em termos
modernos -- como questão de fé privada --, mas como um fundamento da sua
constituição como sujeito. Não há um plano na subjetividade das
``relações externas e jurídicas'' e um plano na subjetividade
relacionado ao foro íntimo. A religião se apresenta como um complexo de
práticas e normas em que se emaranham ``o código moral e sacramental''.
(\versal{JÚNIOR}, 2011, p. 316). A própria instituição da excomunhão tinha
consequências funestas para a vida social da pessoa, isolando"-a como
leprosa do meio social (\emph{Idem}, p. 350-51). Na sua \emph{Formação
do Brasil Contemporâneo}, Caio Prado nos escreve interessantes páginas
sobre alguns âmbitos nos quais se imiscuía o poder eclesiástico, da vida
doméstica dos casais, passando pelo ensino, a catequização dos índios, a
assistência social, até o terreno das chamadas diversões públicas pois
``a maior parte das festividades e divertimentos populares se realizava
sob seus auspícios ou direção''.(\emph{Ibidem}, p. 351)

A Igreja forma uma esfera de importância. Emparelha"-se tanto à
administração civil que é certas vezes difícil ou impossível distinguir
na prática uma da outra. Havia uma comunhão de propósitos. O que não
significa uma relação social de tipo feudal, mas também parece
significar outra coisa que não relações mercantis ou capitalistas. É
como se fincassem raízes aqui uma tradição diferente. Uma mescla de
poder senhorial, poder clerical, dominação pessoal com a sede de
riquezas -- que pelo menos em aparência seria digna de reprimenda
religiosa.

É como se a Igreja aqui tivesse criado por séculos um ambiente separado
dos questionamentos que vinha sofrendo na Europa, como se aqui pudesse
gozar de terreno livre para se expandir, para retomar seu poder, ou para
espalhá-lo, fora de um ambiente hostil como estava sendo o europeu. A
terra virgem propiciava o nascimento dessa tradição que perdurou com
força nas entranhas sociais e subjetivas por no mínimo quatro séculos.
Uma tradição \emph{sui} \emph{generis} que gozava da situação colonial
na qual as manifestações de espírito capitalista, por não constituírem
uma organicidade objetiva e subjetiva, podiam conviver com o aspecto
simbólico"-religioso no plano mesmo das relações sociais, enquanto em
países onde o capitalismo se expandia tal tradição já vinha sendo
superada ou jogada para um plano secundário de fé privada. Não é por
acaso que temos um país com raízes conservadoras em todos os âmbitos, o
que só começa a mudar quando um mínimo de corpo interno começa a se
formar tendo como modelo inalcançável as nações ditas civilizadas. Não
que depois tenha mudado, longe disso. Assim como no fundamento econômico
o desenvolvimento de um sistema mercantil internamente incorporou traços
coloniais, do ponto de vista do fundamento subjetivo, o desenvolvimento
da forma"-sujeito burguesa no Brasil tampouco acertou contas com as
raízes dessa tradição.

O mais fundamental é ver que a religião é um fundamento da vida
colonial, sem qualquer relativização como acontecia na Europa. Se
dissemos que não havia um sistema, não havia uma organicidade,
evidentemente a referência é a forma de vida moderna, porque a própria
religião e a subsistência eram uma forma de síntese social num plano
distinto. Não é que o simbólico"-religioso era um entrave à exploração
mercantil aqui realizada, mas um entrave a que relações do tipo moderna
germinassem internamente, como se a economia voltada para o exterior não
dissesse nada sobre a dinâmica interna que se passava sobre outras
bases. Havia uma espécie de caldo cujos ingredientes eram o
primitivismo, a exploração mercantil em vista do comércio e a cultura
deslocada do espírito do tempo, como era a portuguesa.

A própria expulsão pombalina dos jesuítas em 1759, que poderia soar como
uma revolução anticlerical de cariz moderno, faz parte de um contexto
mais amplo de interesses econômicos, não no sentido de desenvolver um
capitalismo internamente, entendido como relações modernas e dinâmicas,
mas de ampliar a submissão da Colônia à sustentação da Metrópole. Se a
Igreja gozava de independência considerável antes disso, depois, fica
entregue ao poder soberano da coroa, sendo ``um simples braço do poder
secular'' (\versal{HOLANDA}, 2014, p. 141), tornando"-se ``simples departamento da
administração portuguesa, e o clero secular e regular, seu
funcionalismo.'' (\versal{JÚNIOR}, 2011, p. 353).

Dito de outro modo, o caráter piedoso da Colônia contrasta de certo modo
com a marcha do mundo que punha o fundamento simbólico"-religioso em
questão. Há que se ressaltar também que Portugal não jogou um papel de
vanguarda nesses questionamentos, ou seja, à parte uma mentalidade
notadamente mercantil, Portugal tinha uma forma de subjetividade em
geral fincada profundamente na tradição religiosa, e sua aprovação sem
restrições das estipulações do Concílio de Trento, que se mantiveram em
vigor no Brasil até a República (\emph{Idem}, p. 350), são indício
importante dessa tradição que se amalgama aqui com a ``tradição'' da
exploração mercantil da terra -- que inclui a dos escravos.

É verdade também que tal caráter ou fundamento religioso da Colônia
tinha um matiz bastante epidérmico e pouco profundo, no dizer de Sérgio
Buarque (2014, p. 181). As próprias cerimônias religiosas públicas
chamavam a atenção pela descompostura da parte dos próprios sacerdotes:
``clérigos e leigos ávidos por aproveitar aqueles preciosos momentos de
convívio intersexual a fim de fulminarem olhares indiscretos, trocarem
bilhetes furtivos e, os mais ousados, tocarem maliciosamente o corpo das
nem sempre circunspectas donzelas e matronas'' (\versal{MOTT}, 1997, p.
161\emph{-}162), o que também foi sublinhado por Sérgio Buarque.
Do mesmo modo, não era raro o assédio sexual da parte dos sacerdotes no
próprio confessionário, ``fazendo do tribunal da penitência alcova para
pecaminosas imoralidades''. Segundo o autor, muitos moradores passavam
anos sem ver um sacerdote, sem participar de ritos religiosos, e isso
levou a certa indiferença em relação a práticas religiosas comunitárias,
e ao ``incremento da vida religiosa privada'', que também não era
desprovida de ``desvios de heterodoxias'' (\emph{Idem}, p.163).

Evidentemente, não devemos imaginar que tal indiferença em relação a
práticas religiosas comunitárias tenha sido a tônica geral, pois, como
já notamos acima, sempre tiveram importância considerável as festas
religiosas em nossa terra, desde que se fixaram com mais estruturas os
aglomerados rurais ou urbanos -- de modo que até hoje, em certas
localidades rurais no Brasil, ainda há uma população, geralmente de mais
idade, que se mobiliza para tais festividades que parecem deslocadas no
tempo para o observador acostumado com a dinâmica das cidades. Ao que
parece, esse caráter epidérmico da tradição religiosa não diz respeito à
falta de religiosidade, mas muito mais uma permissividade aumentada, à
criação de uma subjetividade mais indulgente ainda em relação aos
desvios. O que significa muitas vezes um ardor interno antagônico à
falta de ardor nos atos -- aspecto típico das religiões que aqui
ganharia maior amplitude pelo caráter da fixação da vida social.

Além desse aspecto, ao que parece, dos católicos praticantes autênticos
aos pseudocatólicos, -- ``boa parte dos cristãos"-novos, animistas,
libertinos e ateus'', que só para evitar a repressão inquisitorial
frequentavam rituais -- o que se cristalizava, ``do papa"-hóstias ao mais
irreverente libertino"-agnóstico'', eram ``diferentes tipos de vivência e
práticas privadas'' cujo centro era a religião (\emph{Idem}, p. 175).
Não necessariamente católica, já que são correntes as denúncias contra
aqueles que recorriam a feiticeiros e feiticeiras quando ``os exorcismos
da Igreja e os remédios da botica não surtiam efeito na cura de
variegada gama de doenças''(\emph{Ibidem,} p. 192-193).

É, pois, nesse contexto social, onde ``o medo dos castigos divinos era
uma obsessão generalizada'' (\emph{Ibidem,} p. 176) e a vida beata e
piedosa o antídoto, onde a compra do céu e um polpudo testamento para a
celebração de missas poderiam reduzir ``a permanência no cálido
purgatório'' (\emph{Ibidem,} p. 177), onde as bênçãos proibidas eram
castigadas, e patuás e bolsas de mandingas disputados (\emph{Ibidem,} p.
196), que se desenvolve a vida colonial, e a estrutura econômica que
alguns veem como capitalista. Sem esquecer os desviantes, que tomavam
não pequeno cuidado para não despertar a ``repressão da justiça civil,
episcopal ou inquisitorial.''(\emph{Ibidem,} p. 201).

Podemos lançar mão de outro ângulo de visão para pensarmos o peso do
simbólico"-religioso na Colônia. Sabemos que muitas transgressões na
Bahia do fim do século \versal{XVIII} -- que ``contrastavam poderosamente com o
formalismo dos letrados mineiros quanto aos ritos da vida pública''
(\versal{JANCSÓ}, 1997, p. 420) -- atingiam as normas de comportamento da Igreja
para ganhar sentido político. Talvez a maior transgressão fossem
refeições de carne às sextas"-feiras da Paixão, sobretudo em jantares de
cunho político. Além disso, havia quem divulgasse blasfêmias, para o
espírito pio, segundo as quais não haveria Juízo Final, Inferno ou
Paraíso, que a alma era mortal, que as mulheres casadas não precisavam
ser fieis, entre outras invectivas. Apesar disso, segundo Jancsó, seria
``equivocado ver nas transgressões de feição anticlerical ausência de
religiosidade.'' (\emph{Idem}, p. 418). Para defender tal ponto de
vista, apoia"-se o autor em descrições de bens sequestrados em devassas e
manuscritos de autores das blasfêmias, que comprovam que esses hereges
guardavam privadamente, em si ou em seus baús, fortes sentimentos
religiosos.

Além disso, as horas canônicas eram aqui pelo menos em parte seguidas
pelos moradores. Nos conventos, seguia"-se esse ascético ritmo dos oito
momentos canônicos. Na vida cotidiana, ``quando menos as três principais
horas litúrgicas'' obrigavam os moradores mais pios a lembrar de rezar
conforme as badalas especiais dos sinos das igrejas: às seis da manhã,
``hora do \emph{ângelus --} ao meio"-dia -- \emph{a} \emph{hora que o
diabo está solto --} e às seis da tarde, hora das \emph{ave"-marias.}
(\versal{MOTT}, 1997, p. 164). As cerimônias de benzimento de fundações de
moradas pelos mais abastados ou a própria presença do quarto dos santos
(\emph{Idem,} p. 166) também parecem ir ao encontro desse peso
simbólico"-religioso na vida social da colônia.

As igrejas da Bahia e de Ouro Preto não são exemplos menos expressivos
de devoção. Como se sabe, ambas as cidades, em que se podiam ver duas
das mais ``capitalistas'' da Colônia, cada uma em sua época, contam com
monumentos religiosos de grande monta. O que não invalida o fato de tais
regiões terem um peso econômico considerável, ao contrário, as
construções demonstram exatamente o peso econômico, mas ao mesmo tempo o
quanto esse desenvolvimento econômico não era necessariamente
capitalista no sentido conceitual que aqui estamos dando: de
forma"-social ou sistema social capitalista. E nem precisamos citar que a
mão"-de"-obra de tais construções era em geral escrava.

O espaço reservado para tal devoção, ou para o simples desejo de
grandeza dos poderosos ao ajudarem na ereção de uma igreja ou em obras
nesse sentido, ou ainda a importância dada ao local onde a elite se
postava na igreja, dissociada da arraia"-miúda, são elementos que não
contribuem ao desenvolvimento de relações sociais modernas ou
capitalistas no Brasil, mesmo nos centros mais avançados. O que não
significa que o desenvolvimento de relações mercantis tenha acabado com
esses arcaísmos, que se mantiveram como traço cultural importante. E
essas relações arcaicas devem ser levadas em conta, embora não vão de
encontro à estrutura econômica da Colônia, embora a ela se amalgamem,
numa mistura de tradição simbólico"-religiosa e manifestações modernas,
dentro da exploração capitalista da grande lavoura. O fato de nas
propriedades rurais mais abastadas haver uma igreja demonstrava um
espaço -- impossível numa sociedade realmente capitalista -- para
``status e cumprimento de obrigação religiosa'' (\emph{Ibidem}, p. 168).
Numa sociedade em que o capitalismo se apresenta como sistema, como
forma"-social -- e não como momentos e manifestações devidos mais ao
contexto externo, e que não engrenam uma dinâmica interna -- não há
espaço para esse tipo de veleidade senão em casos particulares. Somente
numa sociedade em que há um entrecruzamento na forma social e um
entrecruzamento na forma de subjetividade é possível haver tal espaço.
Numa sociedade em que a subjetividade burguesa fincou seus pés superando
formas de subjetividades agarradas em tradições, nem tempo nem dinheiro
podem ser despendidos em empreitadas custosas e sem retorno financeiro,
como era a ereção de uma capela. Nem mesmo se daria tamanha importância
simbólica a que uma bula do Papa liberasse a celebração de missa em sua
capela. (\emph{Ibidem}, p. 169). O próprio desenvolvimento do Barroco,
sobretudo em Minas Gerais e em Salvador, não é também menos
significativo. São índice de uma sociedade mergulhada em profundas
mediações simbólicas. E a plasticidade barroca, cada filigrana da
arquitetura, a suntuosidade, a imponência, as sinuosidades dos
elementos, o culto dos santos e anjos, o desejo de provar que os
sentidos enganam, em suma, a resposta barroca aos ventos modernos mais
que demonstram o peso das mediações simbólico"-religiosas por essas
terras -- onde o Barroco não aparece como resposta à Reforma, como na
Europa, mas muito mais como continuação do caráter piedoso que aqui
reinava.

Sérgio Buarque também destaca o papel da religião nas relações sociais
do Brasil Colônia. E uma sociedade com uma dinâmica capitalista
não uniria ``homens notáveis e o povo da vila'' (\versal{HOLANDA}, 2014, p. 69),
como no caso da construção da matriz de Iguape, em fins do século \versal{XVIII},
em que se carregavam pedras da praia até o local da construção. Ou no
caso da matriz de Itu, ``erigida em 1679 com auxílio dos moradores, que
de longa distância levavam à cabeça, em romaria, a terra de pedregulhos
com que foram pilados os muros''? (\emph{Idem}, p. 70).

Não se pretende dizer que a religião seja impedimento a que as relações
capitalistas se desenvolvam, mesmo porque vê-se hoje que as relações
mercantis convivem com a religião sem maiores problemas. Mas, para isso,
é preciso relegá-la a uma questão de fé privada, já que o fundamento da
vida social e subjetiva passa a ser o simbólico"-mercantil. O que se
pretende desde o começo é refletir sobre o quanto o aspecto
simbólico"-mítico"-religioso cria entraves ao desenvolvimento de relações
modernas. Não entraves refletidos, trata"-se mais de entraves devidos ao
fincamento em raízes, que são sempre um empecilho à dinamização das
relações fluidas, desimpedidas, numa palavra, liberalizantes. E essa era
uma caraterística do Brasil, a tal de ponto de nos perguntarmos como
seria possível falar em relações sociais capitalistas se o peculiar da
vida brasileira nessa época, segundo Sérgio Buarque, era uma atrofia, em
proveito do irracional e do passional, das ``qualidades ordenadoras,
disciplinadoras, racionalizadoras. Quer dizer, exatamente o contrário do
que parece convir a uma população em vias de organizar"-se
politicamente.'' (\emph{Idem}, p. 71).

\section{À luz, o tímido germe moderno}

As descrições normalmente feitas acerca do nível cultural da Colônia não
fazem imaginar em geral um quadro dos mais dignos de admiração. Segundo
Caio Prado, ``O nível cultural da colônia era da mais baixa e crassa
ignorância. Os poucos expoentes que se destacavam pairam num outro
mundo, ignorados por um país que não os podia compreender.'' (2011, p.
146). Tal contexto vai ao encontro da tese de Antônio Cândido da falta
de um \emph{sistema literário} dentro do Brasil até o que ele chama de
\emph{momentos decisivos} para a \emph{Formação} de uma literatura
brasileira, de um sistema literário. Tal sistema existiria, para o
autor, quando há a formação de uma continuidade literária à qual se
integra a atividade dos escritores de um dado período. Uma continuidade
que é uma ``espécie de transmissão da tocha entre corredores, que
assegura no tempo o movimento do conjunto, definindo o delineamento de
um todo'' (\versal{CÂNDIDO}, 2000, p. 24).

Assim como afirma Cândido que não havia até o Arcadismo, metade do
século \versal{XVIII}, uma tradição, uma ``transmissão de algo entre os homens'',
uma ``continuidade ininterrupta de obras e autores, cientes quase sempre
de integrarem um processo de formação literária'' (\emph{Idem}, p. 24)
numa dinâmica que forma padrões e vai se impondo ao pensamento, à
subjetividade , também não havia tal \emph{continuidade ininterrupta} em
termos de uma dinâmica mercantil até o século \versal{XIX} -- é o que temos
defendido até aqui. Portanto, assim como faltavam os elos para criar o
sistema literário -- leitor"-obra"-autor --, do ponto de vista do
capitalismo no Brasil faltavam elementos que engendrassem um sistema,
com uma dinâmica própria. E, nesse caso, o principal elemento ausente
era a mão"-de"-obra livre. Sem esse elemento, não havia como ativar o que
Sérgio Buarque chama de ``revolução lenta'', que se processa sem o
alarde de ``algumas convulsões de superfície'' do período republicano e
mesmo anterior, que ele chama de ``revoluções palacianas'' (\versal{HOLANDA},
2014, p. 203). Já essa revolução lenta que entra realmente em marcha em
1888 agiria nas entranhas mesmas do social. Embora não se possa fixar
uma data para um acontecimento que se dá por tensões contínuas, foi
talvez esse momento, segundo Sérgio Buarque, o mais decisivo, pois

\begin{quote}
a partir dessa data tinham cessado de funcionar alguns dos freios
tradicionais contra o advento de um novo estado de coisas, que só então
se faz inevitável. Apenas nesse sentido é que a abolição representa, em
realidade, o marco mais visível entre duas épocas. E efetivamente daí
por diante estava melhor preparado o terreno para um novo sistema, com
seu centro de gravidade não já nos domínios rurais, mas nos centros
urbanos. \emph{(Idem,} p. 204)
\end{quote}

Pelo trecho, Sérgio Buarque parece também ambíguo -- embora sua visão
seja parecida com a dos intérpretes já citados de que a base do Brasil
dito contemporâneo está na colônia. O que não o impede de falar de
``duas épocas'', um ``novo sistema'', como se fosse uma continuação, mas
também um corte. Talvez a falta de uma crítica das categorias fundantes
da vida social moderna em seu todo, talvez uma crítica muito fincada no
atraso em relação ao desenvolvido -- e não na falta em relação ao dito
desenvolvido --, tenha impedido aprofundar que se tratava de um corte em
profundidade com a colônia, embora os traços coloniais se mantivessem,
agora emaranhados numa dinâmica tendente a mercantil internamente.
Diferentemente de antes, quando não ecoava aqui, essa dinâmica germina,
mas avança sempre entrecruzada com o passado colonial que, nem por isso,
impedia que o país se colocasse na marcha mundial moderna a seu modo
meio ``fora do lugar'', mas também dentro do seu lugar.

Antes desse marco importante, assim como a falta de sistema literário
não impede que obras de valor surjam -- numa avaliação retrospectiva do
olhar contemporâneo --, como é o caso de um Gregório de Matos ou de
Antônio Vieira, a falta de um sistema em termos capitalistas também não
impede que aqui tenha havido o que podemos chamar de
\emph{manifestações} \emph{capitalistas} materializadas na chamada
exploração mercantil nos trópicos. Mas o decisivo é que tais
manifestações não chegavam a fazer germinar relações sociais
capitalistas no conjunto da sociedade -- o que é diferente de um ponto
de vista do atraso em relação aos centros capitalistas, um ponto de
vista que já pressupõe a partilha de uma base social comum, o que não
era o caso internamente. O amálgama é o que domina e continua ainda a
cisão entre os dois ritmos de que temos falado.

Esses \emph{momentos decisivos} na formação da literatura brasileira
coincidem em certo sentido com o germinar de uma espécie de espírito
nacional, no seio sobretudo de uma classe mercantil que ganha a
consciência de poder fundar sua própria \emph{empresa mercantil
exploradora dos trópicos}, livre do jugo exploratório da Metrópole.
Trata"-se de um germe de forma"-sujeito burguesa, ainda bastante misturado
a formas de subjetividade voltadas para o passado, mas que já vislumbram
e já têm contato com o que é concebido como moderno. Tal consciência,
não por acaso, em termos literários se iniciou com os árcades mineiros
no meio do século \versal{XVIII} e ganhou nitidez no começo do \versal{XIX}.

Seguindo István Jancsó, em seu ensaio intitulado \emph{A sedução da
liberdade,} por essa época o matiz das contestações à coroa sentem uma
modificação. Se antes as ``irrupções coletivas de rebeldia'' se esvaiam
``no específico de sua motivação imediata'', fossem questões de excessos
fiscais ou questões de legislação, e a Coroa emergia sem maiores
questionamentos após superados tais problemas, o que se viu na passagem
do século \versal{XVIII} para o \versal{XIX} foi o questionamento da própria organização
do poder e desejo de sua substituição. ``Os ensaios sediciosos do final
do século \versal{XVIII}'', para o autor, ``anunciam a erosão de um modo de
vida'', a emergência de outra forma de vida social, o ``despontar de
novas formas de sociabilidade, permitindo penetrar no pulsar do
cotidiano, tanto das elites quanto dos outros diversos segmentos da
sociedade colonial'' (\versal{JANCSÓ}, 1997, p. 389), que entravam em contato com
ideias liberais europeias que, com toda evidência, somente com jogo de
cintura poderiam aqui ganhar corpo. Porque, como vimos acima, a
forma"-sujeito burguesa não foi fruto de um amadurecimento interno
paulatino, porque as condições da vida social reinantes no Brasil até o
século \versal{XIX} não eram eivadas das tensões que pudessem fazê-la avançar. A
ideia de burguesia no Brasil Colônia e até o século \versal{XIX} ainda está
atrelada simplesmente ao ganho mercantil, sem que se crie uma dinâmica
social e subjetiva rumo a uma liberdade sem peias que é a tendência
conceitual da lógica mercantil. Assim, a qualidade dos questionamentos
surgidos a partir da segunda metade do século \versal{XVIII}, não ``anunciam a
hegemonia burguesa'', como pensa Jancsó, embora sejam transformações que
penetram, lenta mas persistentemente, ``no fluir da vida social
organizada.'' (\emph{Idem}, p. 393). Cabe ressaltar o aspecto paulatino
e persistente da transformação, mas falar em hegemonia burguesa aparece
como exagero perante a argumentação defendida neste estudo. O simples
exemplo de um autor como Tomás Antônio Gonzaga, que desempenhava um
papel na agitação política, já é expressivo. Suas Cartas Chilenas
criticam o governo de Vila Rica, mas mantêm no geral o teor tradicional
na forma e no conteúdo, e não atingem a Corte.

O mais dialético e condizente com a história da formação da
subjetividade burguesa é dizer que tínhamos o engatinhar de
tensionamentos que iam florescer durante o século \versal{XIX} para amadurecer no
século \versal{XX}. Até porque, mesmo com as modificações políticas ocorridas, a
estrutura econômica continua colonial, até sofrer a modificação advinda
da abolição. As mudanças que ocorrem no terreno econômico e político se
relacionam muito mais a circunstâncias e tensionamentos exteriores do
que ao amadurecimento de tensões internas. O que significa que as
reivindicações de teor novo não diziam respeito a uma classe burguesa
que se aproxima de seu conceito, mas a uma elite agrária que se vestia
de burguesa na epiderme, e somente porque a marcha do mundo o exigia, e
não porque as relações sociais dentro do país amadureciam em termos
mercantis pelas suas próprias tensões internas\footnote{Como temos
  defendido, a forma"-sujeito burguesa não é para nós a forma de
  subjetividade moderna a cada momento de seu desenvolvimento, sendo
  sempre a cada momento já forma"-sujeito burguesa sem mais. Existe um
  ideal de forma"-sujeito e de forma"-social burguesas, embora não deixe
  de participar da vida social capitalista a sociedade e a subjetividade
  que não alcancem tal ideal de modo perfeito. O passe de entrada é
  partilhar as categorias fundantes da vida social moderna. Aí também o
  peso do concreto, das vicissitudes de cada país jogam um papel na
  forma como esse ideal se realiza. No entanto, como encaramos a vida
  social e subjetiva capitalistas como um caso único na história de
  formas sociais e subjetivas com uma dinâmica interna e, portanto, com
  uma teleologia negativa, quanto mais caminha essa dinâmica interna,
  mais essas formas social e subjetiva se aproximam de seu fim último.
  Por isso, a vida social capitalista não foi a todo momento um vazio
  visível da subjetividade e do social. Mas o seguir adiante da vida
  social e subjetiva capitalista, seu amadurecer é uma tendência ao
  vazio, no sentido da realização de um mundo em que a mercadoria ocupa
  todos os recantos da vida. Quanto mais a vida social moderna avança,
  mais a lógica da mercadoria domina tudo que é vivido, e assim, mais
  nos aproximamos da vida social burguesa, que é quando a lógica da
  mercadoria se torna segunda natureza sem mais.}.

Não por acaso, na ideia da constituição do espírito nacional no
Arcadismo não está a preocupação em termos sociais ou estéticos com um
mergulho na busca do que poderia ser o nacional. O que anima os
escritores neoclássicos, como afirma Antônio Cândido, ``é o desejo de
construir uma literatura como prova de que os brasileiros eram tão
capazes quanto os europeus.'' (\versal{CÂNDIDO}, 2000, p. 26), originando um
espírito nacional que criava uma literatura empenhada que tolhia até o
voo literário. Mas mesmo essa imaturidade ``deu à literatura sentido
histórico e excepcional poder comunicativo, tornando"-a língua geral duma
sociedade à busca de autoconhecimento.'' (\emph{Idem}, p. 27). Ora, essa
busca de autoconhecimento era o próprio sintoma na literatura da
necessidade da formação de uma nacionalidade em termos mais amplos.

Embora um movimento diferente já se pudesse notar, com movimentos
reunidos em Sociedades Literárias sobretudo em Minas, Bahia e Rio de
Janeiro, já no fim do século \versal{XVIII} (\versal{JANCSÓ}, 1997, p. 410), movimentos
literários que tiveram influências em movimentos de revolta como a
Inconfidência Mineira e a Conjuração Baiana -- neste caso, com a
especificidade de que a sociabilidade literária incluía ``homens de
baixa extração e, pelos critérios da época, de poucas luzes.''
(\emph{Idem}, p. 412) --, esse amadurecimento do espírito nacional
sofreu grande impulso com o deslocamento da Corte para esse lado do
Atlântico. Como diz Cândido, a vinda da corte para o Brasil trouxe
consequências importantes para o ``desenvolvimento da cultura
intelectual e artística, da literatura em particular.'' (\versal{CÂNDIDO}, 2000,
p. 215) e, evidentemente, para uma nova dinâmica da vida social como um
todo -- que continuava sobre as mesmas bases inibidoras de um sistema
mercantil interno. De todo modo, foi nesse século triunfal, quando pela
primeira vez configurou"-se no Brasil ``uma vida intelectual no sentido
próprio'' (\emph{Idem}, p. 222), que apareceu o primeiro público regular
de arte e literatura.

É portanto essa aquisição de uma mentalidade, de uma reflexão acerca do
nacional, que veio a criar os tensionamentos para a superação do estado
colonial, para o germinar de formas sociais capitalistas e de uma forma
de subjetividade burguesa -- embora tais tensionamentos não tivessem
sido suficientes sem um tensionamento vindo pelo espelho da própria
marcha dos países avançados em termos capitalistas. A importância é
tamanha da vinda da Corte para o Brasil que a historiadora Laura de
Mello arrisca dizer que foi quando o ``Brasil começou a se fazer
Brasil'' (\versal{MELLO}, 1997, p. 440), quando tivemos hábitos e costumes
modernos integrando um processo social complexo que entrava em tensão
com a forma antimoderna em que vivia ainda a Colônia. Antônio Cândido
lembra que no Rio de janeiro havia, entre 1807 e 1817, 4 livrarias mal
abastecidas. Em 1821 eram 8, mas sempre misturadas com a função de
papelaria. Em Recife, havia uma de livros religiosos em 1815. A maior
parte do saber e dos livros estava concentrado na Igreja (\versal{CÂNDIDO}, 2000,
p. 219).

O chamado período joanino contribui, assim, para esse período de
germinação de um espírito nacional, quando a sociedade vai adquirindo o
que Antônio Cândido chama de ``consciência de si própria'' (\emph{Idem},
p. 221), que no fundo é uma consciência que vai amadurecer nos
tensionamentos objetivos e subjetivos no decorrer do século \versal{XIX}, no seio
de uma elite econômica e intelectual, logo, no seio de uma minoria
esmagadora, que começava a revestir"-se de uma consciência da modernidade
a partir do contato com as ideias liberais europeias -- embora já
tenhamos dito que essa consciência da modernidade precisava de muito
jogo cintura numa estrutura social que negava no geral os aspectos
modernos, e que ``impunha à consciência burguesa uma série de
acrobacias'' (\versal{SCHWARZ}, 2000, p. 42).

A raridade da instrução dava um relevo maior aos intelectuais e só
aumentava a distância entre uma dita massa e essa elite que se
autovalorizava ainda mais e participava do desenvolvimento do espírito
nacional, da vida pensante da sociedade. Como diz Cândido, com a
Independência, o intelectual passa de artista a ``pensador e mentor da
sociedade, voltado para a aplicação prática das ideias.'' (2000, p.
225). O intelectual nesse momento novo do século \versal{XIX} assume um
papel na modelação da sociedade de uma forma diferente. Agora em
consonância com as mudanças na marcha do mundo. Se na tradição colonial
a intelectualidade era formada por padres e bacharéis de formação
clássica, ``detentores de cargos e prebendas, identificados aos
interesses da Coroa, sua patrona.''(\emph{Idem}, p. 225) , o quadro muda
no momento da Independência, pois o número de brasileiros que vão
estudar na Europa, não só em Coimbra, mas também na França, aumenta. Não
só a quantidade aumenta, mas a qualidade dos estudos se diversifica,
passando a figurar entre os estudos as ciências, notadamente a
filosofia. Segundo Cândido, isso representava uma libertação da
mentalidade jesuítica e legista das gerações anteriores, dando à luz
``uma mentalidade progressista'', uma ``concepção mais ousada do papel
da inteligência na vida social e das relações entre Metrópole e
Colônia'' (\emph{Ibidem}, p. 225). Evidentemente, essa visão mais ousada
não se dá no sentido dos questionamentos acerca da organização social da
colônia, mas no sentido da libertação, da independência em relação à
Metrópole sugadora do sangue colonial. O humanismo e o pensamento
moderno agiam no sentido de conformar uma elite à nova marcha no mundo.
E o progressismo se refere à libertação das amarras da economia
nacional, portanto, à libertação da elite nacional das amarras da
metrópole. Dito de outras palavras, era o desejo de libertação do
atraso, com base no idealizado centro civilizatório, notadamente
europeu. Partia"-se do pressuposto, portanto, de que o fundamento da
civilização moderna e de sua democracia liberal, mediada pela
forma"-mercadoria, era o positivo a ser partilhado para que o negativo
atrofiamento de relações modernas pudesse aflorar.

Mas, no geral, a sociedade não mudava substancialmente. Aquela cisão a
que nos referimos acima, entre uma \emph{Economia}, tida por verdadeira
e essencial, de produtos coloniais voltados para fora, e uma
\emph{economia} interna, subsidiária, precária, revés obscuro daquela,
na qual não repercutiam consideravelmente as trocas mercantis com o
exterior, vai trazer consequências para a gestação desse espírito
nacional. Porque tal espírito nacional está ligado à estrutura
econômica, à verdadeira Economia, aquela que representava o Brasil, a
economia de exportação -- aquela ligada aos circuitos mundiais e na qual
havia um fluxo monetário que o espírito nacional queria desviar de fora
para dentro. Não à toa, desse espírito nacional participará aquela elite
nacional que vivenciava a Economia virada para fora, enquanto grande
parte de outras camadas não participará de decisões ou pressões para a
Independência\footnote{O próprio Canudos, tido por antirrepublicano, e
  às vezes por conservador, não o era por questões conceituais, mas pelo
  fato de grande parte das populações não se sentirem concernidas senão
  negativamente pelas mudanças ditas de modernização do país.}. É assim
que Antônio Cândido descreve a situação de modo certeiro: ``Em poucos
momentos, quanto naquele, a inteligência se identificou tão
estreitamente aos interesses materiais das camadas dominantes, (que de
certa forma eram os interesses reais do Brasil), dando"-lhes roupagem
ideológica e cooperação na luta.'' (\emph{Ibidem}, p. 225).

A independência do Brasil dizia, assim, muito mais respeito aos
``interesses reais do Brasil'' voltado para fora, do que os ``interesses
reais do Brasil'' voltado para dentro, que era muito mais amplo, rico e
diverso do que aquele que a Economia não moldara senão indiretamente --
ao fazê-lo parecer mesquinho frente às somas monetárias vultosas que a
grande economia deitava nas mãos de uma dita elite.

Um aspecto que vai ao encontro da ideia de que foi no transcurso do
século \versal{XIX}, a vinda da Corte e a própria Independência fazendo parte
desse processo, que esse espírito nacional se consolida, é que o
Arcadismo da segunda metade do século \versal{XVIII}, que plantara o germe desse
espírito nacional, acabou assimilado à Colônia, enquanto o Romantismo
ficou assimilado à Independência. Embora não se possa ver o rico período
de modo tão esquemático, já que o próprio espírito nacional amadurecido
com o Romantismo tomava a Europa como modelo de vida social e subjetiva,
em suma, de civilização. De todo modo, como diz Cândido, ``o Romantismo
no Brasil foi o episódio do grande processo de tomada de consciência
nacional, constituindo um aspecto do movimento da Independência.''
(\emph{Idem}, p. 281). Trata"-se mesmo assim de um tempo de
amadurecimento. Pois essa consciência nacional que se inspirava no que
havia de mais moderno em termos ideológicos ainda vai conviver muito
tempo com aquela que era tida com a mais arcaica das instituições: o
trabalho não formalmente livre dos negros.

O dilema da formação de uma literatura nacional pode ser relacionado com
o dilema da formação da vida social capitalista no Brasil. A literatura
nacional só vingaria se desenvolvesse os aspectos nacionais. Do mesmo
modo, para se desenvolver um sistema capitalista no Brasil era preciso
ativar uma dinâmica interna ela mesma inibida pela Economia voltada para
fora. Se a literatura durante muito tempo foi um ramo da portuguesa ou
europeia, se a literatura pouco dizia respeito ao ambiente interno, não
se pode negar relação com a própria inexistência de uma dinâmica social
interna que pudesse, como ocorreu a partir do Arcadismo, mas
principalmente com o Romantismo, fazer brotar a necessidade do voltar"-se
para si mesma -- embora tenhamos visto que era mais fácil levantar as
cores do moderno do ponto de vista ideológico do que do ponto de vista
da estrutura social. Tanto o capitalismo brasileiro como a literatura
brasileira na colônia careciam de tensionamentos, de um núcleo dinâmico,
algo que lhes desse movimento, que lhes fizesse existir como sistema.

Como falamos acima, a subjetividade e a sociedade não vão amadurecer por
tensionamentos advindos de suas entranhas, mas de uma mistura entre
tensionamentos internos e o mundo exterior que sempre aparecia como
modelo inexorável. É assim que, não estranho a esse descompasso, vai
surgir a própria ideologia romântica do \emph{Indianismo.} Nada mais
próprio do que o índio para servir de símbolo nacional. Mas também nada
mais impróprio pela artificialidade do símbolo que não tinha que ver com
a redescoberta desse povo esmagado pela colonização, mas com a invenção
de um ser pintado com as colorações míticas europeias -- diferentemente
de Macunaíma de Mário de Andrade que é uma paródia da mitologia
europeia. Voltaremos a esse tema.

O índio, Antônio Cândido cita o movimento em torno da revista Niterói,
precursora do Romantismo brasileiro, era visto ``como o elemento básico
da sensibilidade patriótica'' (2000b, p. 19). Diferentemente do
indianisno identificado com 1922, que se insurgia contra o ``O índio
vestido de senador do Império.[\ldots{}] Ou figurando nas óperas de
Alencar cheio de bons sentimentos portugueses.'' (\versal{ANDRADE}, 1978, p. 16),
o indianismo romântico pintou um índio com traços do conquistador,
tentando equipará-lo a este ``no cavalheirismo, na generosidade, na
poesia.'' (\versal{CÂNDIDO}, 2000b, p.19). Como numa espécie de sonho romântico,
que só poderia acontecer nessas terras, de passar diretamente do índio,
do natural, do primitivo que nos caracterizaria, à civilização europeia,
à segunda natureza, por mero processo mimético. A velocidade da marcha
moderna não permitindo um amadurecimento lento, pelo fato de os modelos
já estarem muito à frente, a sociedade e a subjetividade tentam
acompanhar sempre do jeito que dá. Como na paródia expressa por
Macunaíma que entra num rio e sai tingido de branco e de olhos azuis.
Não precisaria aqui citar os vários trechos de Iracema em que temos o
índio com ares de fidalguia.

No afã de criar uma grandeza heróica tipicamente brasileira, o
\emph{indianismo} romântico criou uma tradição sobressalente que acabou
servindo ``não apenas como passado mítico e lendário [\ldots{}] mas como
passado histórico, à maneira da Idade Média.'' (\emph{Idem}, p. 20) Como
diz Cândido, tanto na poesia de Gonçalves Dias como no romance de
Alencar, lenda e história se fundiram no ``no esforço de suscitar um
mundo poético digno do europeu.'' (\emph{Ibidem}, p. 20).

Nesse sentido, o corte vislumbrado por Antônio Cândido entre um
Arcadismo ainda apegado ``aos dois séculos anteriores pelo culto da
tradição greco"-romana'', e um Romantismo que ``revoca tudo a novo
juízo'', que pretende ``liquidar a convenção universalista dos herdeiros
de Grécia e Roma, em benefício de um sentimento novo, embebido de
inspirações locais'' (\emph{Ibidem}, p. 22), precisa ser nuançado no
caso do \emph{indianismo.} Porque a inspiração local nesse caso estava
muito mais relacionada com um sujeito imaginado como tipicamente
nacional -- o índio, que não tinha nada a ver com isso -- que pudesse
ser o depositário de uma tradição heroica e lendária também de
empréstimo, ``a que os românticos desejavam, numa utopia retrospectiva,
dar tanto quanto possível traços autóctones.'' (\emph{Ibidem}, p. 101).

No caso de um espírito nacional às voltas com o específico de uma terra
que se fizera cindida entre um ritmo para fora e outro para dentro,
ritmos opostos e dependentes, nada melhor do que um amálgama entre o que
havia de disponível, tipicamente da terra, o índio, e o que era invejado
como civilização: o europeu ainda pintado como um sujeito burguês
entrecruzado e idealizador da tradição nobre. Esse sujeito de
empréstimo, por escritores que nada tinham daquele sujeito que queriam
erigir como símbolo da nacionalidade, pode ser visto como um sintoma
desse vácuo em termos de organização social e em termos de subjetividade
moderna numa sociedade que ainda estava em construção depois de séculos
passados sem precisar se preocupar com questões desse gênero, já que o
sentido que fazia a terra ter importância perante a marcha moderna se
dava via a Metrópole, e a vida interna não precisava construir"-se por
si, porque seu ritmo interno inibidor de uma dinâmica mercantil só
contava enquanto sustentáculo, não como linha mestra. Constituir um
espírito nacional significava criar uma forma de aproximar"-se
minimamente da marcha moderna, e esse sentimento ou desejo de
aproximar"-se da civilização moderna, dita capitalista, só pôde
amadurecer quando germinou uma dinâmica social interna, uma dinâmica que
só pôde existir ou se consolidar com a Independência e sobretudo com o
fim da escravidão. De todo modo, difícil é cravar uma data ou evento a
partir do qual tudo muda, uma vez que, germinada no seio social essa
dinâmica interna, até pelo passivo objetivo e subjetivo herdado do tempo
colonial, as mudanças vão marinando vagarosamente, e criando cada vez
mais um descompasso entre o que acontecia de fato na vida social, o que
a dita elite pensante alardeava e a própria marcha do mundo à qual a
vida social brasileira se conformava à sua maneira. Por essas terras
brasileiras, a vida social capitalista germina sem o adubo que há tempos
fertilizava o solo objetivo e subjetivo nas terras onde a marcha do
capitalismo se apresentava adiantada.

\section{Fundamentos para o nascimento de~um~espírito~burguês}

\begin{quote}
O fato de maior relevância ocorrido na economia brasileira do último
quartel do século \versal{XIX} foi, sem lugar a dúvida, o aumento da importância
relativa do setor assalariado. A expansão anterior se fizera seja
através do crescimento do setor escravista, seja pela multiplicação dos
núcleos de subsistência. Em um e outro casos o fluxo de renda, real ou
virtual, circunscrevia"-se a unidades relativamente pequenas, cujos
contatos externos assumiam caráter internacional no primeiro caso e eram
de limitadíssimo alcance no segundo.

\emph{Celso Furtado}
\end{quote}

Se durante quase quatro séculos o Brasil, como temos visto, se manteve
em contato com os circuitos comerciais internacionais, não foi senão na
segunda metade do século \versal{XIX} que o germe de relações mercantis enquanto
um sistema interno pôde sentir um impulso. Mas qual a mudança
fundamental que fez surgir o que poderia ter surgido no mínimo um século
antes? Porque, se há mudança, é preciso primeiramente dizer que a
leitura segundo a qual a chave do contemporâneo está na época colonial
ganha um tom mais que ambíguo. É possível, pela via argumentativa aqui
tomada, entender tanto que 1) o subdesenvolvimento do capitalismo
brasileiro tem sua causa no subdesenvolvimento do capitalismo brasileiro
da época colonial, quanto que 2) o subdesenvolvimento do capitalismo
brasileiro se deve à dificuldade havida na colônia de se constituir uma
forma social mercantil enquanto sistema internamente. São duas
interpretações distintas. A primeira verá o desenvolvimento das relações
capitalistas no Brasil como desabrochar do que já existia de modo
atrasadíssimo desde a colônia sem, no entanto, desenvolver"-se devido à
exploração metropolitana. A segunda verá esse desenvolvimento mercantil
como corte em relação ao fundamento dos séculos anteriores. Podemos nos
apoiar no próprio Celso Furtado, que é mais adepto da primeira visão e
não dá importância fundamental a qualquer corte, para avançar nessa
direção:

\begin{quote}
Observada em conjunto, a nova economia cafeeira baseada no trabalho
assalariado apresenta certas similaridades com a antiga economia
escravista: está constituída por uma multiplicidade de unidades
produtoras que se ligam intimamente às correntes do comércio exterior.
Todavia, se nos fixamos mais de perto no mecanismo dessas unidades,
vemos que são profundas as diferenças. [\ldots{}] Os assalariados
transformam a totalidade ou quase totalidade de sua renda em gastos de
consumo.[\ldots{}] Os gastos de consumo -- compra de alimentos, roupas,
serviço etc. -- vêm a constituir a renda dos pequenos produtores,
comerciantes, etc. Estes últimos também transformam grande parte de usa
própria renda em gastos de consumo. (\versal{FURTADO}, 2007, p. 219)
\end{quote}

A descrição do movimento não é nada anódino, porque é por ele que
aparece o fundamental: uma ampliação sem precedentes da mediação das
relações pelo veio monetário. Essa monetarização é o impulso mínimo,
inexistente na colônia, para que o desabrochar de um sistema mercantil
interno ganhe organicidade e dinâmica. Trata"-se de uma mudança
significativa que vai finalmente se configurar como o elemento ativador
de uma dinâmica mercantil no sentido da constituição de uma vida social
fundada na mercadoria. E não é somente o fim da escravidão, mas também a
absorção de grande parte da mão"-de"-obra do setor de subsistência, que
vai significar uma elevação do ``salário real médio, e ainda mais o
salário monetário médio, pois nesse setor o fluxo monetário era
relativamente muito menor.'' (\emph{Idem}, p. 221). Ainda se pode
ressaltar o aspecto da mudança subjetiva ligada à mudança de níveis de
exigência daqueles que saem da subsistência rumo ao assalariamento. Os
assalariados se viam diante de somas de dinheiro maiores do que os
indivíduos habituados à economia de subsistência, onde o dinheiro só
entra ocasionalmente ou em pequenas quantidades. Desse modo, crescia,
pelo mesmo movimento, a diferença entre os empregados do setor de
exportação -- monetizado -- e os do setor de subsistência -- pouquíssimo
monetizado.

Apesar da grande absorção dos negros no assalariamento, não se pode
imaginar uma conversão imediata. Nesse sentido, são interessantes as
observações de Celso Furtado sobre como de início a abolição da
escravatura não significou a liberação de uma grande quantidade de
mão"-de"-obra e, portanto, uma inclusão imediata dessa parte da população
no fluxo monetário dinamizador de relações mercantis. No âmbito do café,
ele deplora que a libertação dos escravos não necessariamente trouxe um
aumento de renda, porque os escravos tinham um ``reduzido
desenvolvimento mental'' -- entenda"-se por essa frase de Celso Furtado,
reduzido desenvolvimento de uma forma de subjetividade burguesa -- que
os segregará do desenvolvimento do país após a abolição. Não possuindo
hábitos familiares, a ideia de acumulação de riqueza lhe é estranha.
Assim, já que para o escravo o trabalho é ``uma maldição e o ócio um bem
inalcançável, a elevação de seu salário acima de suas necessidades --
que estão definidas pelo nível de subsistência de um escravo --
determina de imediato uma forte preferência pelo ócio.'' (\emph{Ibidem},
p. 204). Ora, podendo trabalhar somente 2 ou 3 dias para sua
subsistência, o ex"-escravo ia para o ócio quando já tinha o bastante
para viver. Furtado vê essa característica como algo extremamente
negativo para o desenvolvimento da vida social burguesa no país. A nosso
ver, eis aí uma visão das mais interessantes, mas que vai se perder,
como veremos adiante, juntamente com a mística do primitivismo dentro da
lógica mercantil do gozo, tão logo o capitalismo se desenvolva como
comunidade mercantil.

Nesse mesmo processo de desabrochar de relações mercantis de forma mais
orgânica na segunda metade do século \versal{XIX}, outros grupos pela primeira
vez surgem para competir ou tensionar de forma organizada com o grupo
agroexportador, único a não sofrer com a depreciação cambial em larga
medida devida à expansão brusca da renda monetária. Fazem parte desses
novos grupos de pressão, muito mais em sintonia com uma forma de
subjetividade burguesa, ``a classe média urbana -- empregados do
governo, civis e militares, e do comércio --, os assalariados urbanos e
rurais, produtores agrícolas ligados ao mercado interno, as empresas
estrangeiras que exploram serviços públicos'', além dos nascentes grupos
industriais preocupados com o preço dos equipamentos importados.
(\emph{Ibidem}, p. 248).

É aqui que vemos o quanto essa diversificação é tardia no Brasil, devido
ao próprio fluxo monetário interno que só vai se intensificar com a
constituição do assalariado, que coincide com a ascensão do café como
produto"-chave e com a descentralização na república que permite a
emissão de dinheiro pelos bancos regionais e a oferta de crédito. A
crise daí advinda comprova, para dizê-lo com Furtado, o quanto ``O
sistema monetário de que dispunha o país demonstrava ser totalmente
inadequado para uma economia baseada no trabalho assalariado.''
(\emph{Ibidem}, p. 245). Na época do regime escravo, o fluxo de renda
monetária era reduzido, e é justamente esse fluxo, que monetariza as
relações sociais, o fundamental para que o desenvolvimento de uma
mentalidade mercantil realmente germine e se desenvolva. Porque somente
quando uma dinâmica interna ganha fundamentos sistêmicos é que as
categorias capitalistas podem se mostrar como tal. Enquanto a vida
social interna se funda largamente em estruturas que no fundo barram a
dinamização desse sistema -- embora ao mesmo tempo a colônia esteja
inserida no capitalismo da época --, somente projetando
retrospectivamente na história passada as categorias econômico"-sociais e
subjetivas contemporâneas se poderá ver nas \emph{manifestações
capitalistas voltadas para o exterior} um sistema capitalista.

Também não significa que o Brasil entrou no século \versal{XX} já a todo vapor
com a máscara de caráter do sujeito burguês. A própria forma como a
política de valorização do café foi implementada provava que o país no
fundo continuava dependente de um produto colonial para exportação -- e
esta parece ser um componente da ``linha mestra'' que Caio Prado vê
desde a Colônia se manter no contemporâneo, embora ele não faça a
diferença qualitativa da passagem, como já dissemos. Os sujeitos
mercantis que emergiram do café e da sociedade baseada no assalariamento
ainda carregam em sua subjetividade individual as marcas das formas
anteriores, evidentemente, mas essas formas anteriores, agora sim, se
veem perante tensionamentos, formas novas, que as impulsionam. O que já
não faz mais delas simples ``formas anteriores''. A forma"-sujeito
burguesa ainda não se manifestava como um espírito abstrato que toma
conta das subjetividades concretas. Esse processo estava em curso. Mas o
que mais há são subjetividades individuais, no nível de cada produtor,
que somente querem maximizar seus lucros, seja com escravo, seja com
assalariados. Do ponto de vista geral, não havia uma forma de
subjetividade que vislumbrasse o limite da monocultura e começasse a
fomentar alternativas à aplicação ``dos lucros obtidos no setor cafeeiro
com uma rentabilidade comparável a este último.'' (\emph{Ibidem}, p.
257-258). Foi preciso uma série de crises para que o capital daí advindo
fosse frutificar nas indústrias principalmente de São Paulo. Apesar
disso, a monetarização da relações sociais -- sem a qual uma
forma"-sujeito burguesa não vinga -- continua se ampliando sobre bases
mais sólidas, mesmo com as crises econômicas oriundas do café até a
crise de 29: ``o fato dinâmico principal, nos anos que se seguem à
crise, passa a ser, sem nenhuma dúvida, o mercado interno.''
(\emph{Ibidem}, p. 278). E é isso que nos toca de perto aqui, a formação
de uma vida social capitalista e de uma forma"-sujeito burguesa, fundadas
nos laços simbólico"-mercantis. É disso que trataremos ao adentrarmos
pelo século \versal{XX}, cujo início será marcado pelo amálgama de ideologia de
modernização do país e luta pela florescimento do espírito nacional de
cariz mais delineadamente burguês -- embora com o contrapeso nacional do
primitivo. Mas, como veremos, não terá nada que ver com o
\emph{indianismo} romântico. Ao contrário, será uma tentativa de
ridicularizá-lo e superá-lo.

\section*{Espírito nacional e antropofagia: uma~síntese~\emph{indígeno"-burguesa}}
\addcontentsline{toc}{section}{Espírito nacional e antropofagia:\\ uma síntese \emph{indígeno"-burguesa}}

Da teoria ambígua da \emph{cordialidade} do povo brasileiro de Buarque
de Holanda, passando pela \emph{malandragem} como conceito positivo
crítico"-afirmativo do ser brasileiro (\versal{CÂNDIDO}, 1970), pela alegoria da
constituição do brasileiro ou da subjetividade brasileira em
\emph{Macunaíma}, até as teses antropófagas, pelas quais se encontraria
o espírito nacional revolucionário, temos exemplos de tentativas de
identificação de especificidades da constituição da sociedade
brasileira. E essas ditas especificidades, e as citadas não são as
únicas, podem apontar para entendimentos distintos: 1) o de que o
entrecruzamento objetivo e subjetivo ou a não"-simultaneidade histórica
da forma"-social e subjetiva burguesa gestou essas especificidades. Ou
seja, as especificidades nasceram do descompasso tenso entre dois
mundos, o da forma de vida herdeira da ``atonia'' interna não"-mercantil
-- sem contar a diversidade cultural -- e o da forma de vida adiantada
em termos capitalistas; 2) e o de que essas especificidades servem de
lastro para fundar uma nacionalidade com caracteres pretensamente
distintos daqueles dos centros capitalistas, embora fique sem abordagem
o fato de que a própria nacionalidade desejada deverá partilhar a seu
modo os fundamentos burgueses. Dito de outro modo, a nacionalidade
pressupõe a modernização, pressupõe a entrada pela vida social burguesa,
ao mesmo que ideologicamente se pretendia negá-la.

Sendo assim, todas essas reflexões encontram sua justeza e partem de uma
realidade. Mas o que caberia trazer como pergunta é até que ponto essas
especificidades são aparências -- transformadas ideologicamente em
essência -- que não se opõem necessariamente à essência do que se
poderia chamar de marcha moderna. A trilha reflexiva nesse momento de
nosso estudo se embrenha pela questão de pensar o quanto a busca pelo
instinto de nacionalidade no começo do século \versal{XX}, materializado no
Modernismo paulista e mais especificamente ainda nos \emph{Manifestos
Pau"-Brasil} (1924) e \emph{Antropófago} (1928) e em \emph{Macunaíma}
(1928), são uma tentativa de síntese problemática da nacionalidade. E o
aspecto problemático estaria em que tal síntese, também ligada a uma
forma de subjetividade brasileira típica -- num período efervescente que
coincide com o ciclo do café e também com certa dinamização industrial
--, pretende se constituir como forte questionamento do modelo europeu,
num nível que nenhum movimento literário ou político ousara fazer, mas
ao mesmo tempo parte da ideia de \emph{deglutição} desses mesmos
pressupostos europeus no plano estético, social e subjetivo como um
conceito crítico da forma de vida dos países avançados em termos
capitalistas: ``Um misto de `dorme nenê que o bicho vem pegá' e de
equações'' (\versal{ANDRADE}, 1978, p. 9).

Evidentemente, no bojo dessa ideologia está a questão da modernização
das estruturas arcaicas do país, intimamente ligada com a estética
modernista da cidade de São Paulo. Mas o problema é que não havia
modelos de modernização alternativos, o capitalismo sempre impôs padrões
de desenvolvimento aos quais os países têm que alcançar para existirem
no mapa. Sendo assim, o que o movimento modernista paulista no geral,
enquanto ideologia de modernização, tentava passar como alternativa eram
na verdade características tidas como bem próprias para dar mais
colorido e malemolência ao sujeito burguês brasileiro e sua modernização
-- que nem por isso deixava de ser cinzenta e dura.

Mas um momento histórico, social, estético nunca vem a lume de chofre,
sem fases de amadurecimento, de sussurros antes que o grito que
normalmente se constitui como marco da história venha a sobrepor"-se ao
caminho por vezes longo percorrido pelos tensionamentos nas entranhas da
sociedade. No caso da formação do espírito nacional representada pelo
modernismo paulista, esse amadurecimento passou pelos chamados
\emph{antigos modernistas} (\versal{HARDMAN}, 2009), aqueles que não entraram no
cânone do modernismo paulista, mas que estavam tensionados com questões
modernas bem antes da Semana Arte Moderna de São Paulo.

Hardman destaca a existência na passagem para o século \versal{XX} de um processo
de \emph{continuum} mental ``feito de múltiplas e contraditórias
combinações'', no qual se poderiam identificar afinidades entre ``o
discurso modernizador do Estado'' (a engenharia de obras públicas
ocupando, aí, lugar de `vanguarda')'', o ``discurso
evolucionista"-progressista da imprensa operária emergente'' e o
``discurso estético"-literário moderno de literatos, ensaístas e críticos
de estilos aparentemente díspares.'' (p. 170).

Mesmo com esse \emph{continuum} identificado pelo autor, nos chamados
\emph{antigos modernistas,} as questões de nacionalidade apareciam num
turbilhão tal -- numa dialética, se preferirmos -- que impossibilitava
uma resolução ou uma síntese como a de certo modo pretendida pelo
modernismo paulista. A história brasileira impedia essa síntese. A
leitura já citada de Robert Schwarz das \emph{ideias} \emph{fora}
\emph{do} \emph{lugar} acerca de algumas obras de Machado de Assis e
mesmo de \emph{Senhora} de José de Alencar demonstram essa
impossibilidade de síntese. Uma impossibilidade que aparecia, por
exemplo, de modo intencional em Machado de Assis, que desvela no nível
formal a impropriedade da subjetividade e da forma social burguesas no
Brasil e assim ``fica em dia com a complexidade objetiva de sua
matéria''. Ou aparecia de modo involuntário, indesejado, ``pelas
frestas, como defeito'', o que Schwarz (2012, p. 37-38) analisa com base
no romance \emph{Senhora} de Alencar. Assim, o desterramento da forma
social e subjetiva burguesa em solo brasileiro cria um contexto para a
intelectualidade do país bastante peculiar, e até artificial. Como
explicita Arnoni (2010), a força da organização social imposta pelo
domínio das chamadas elites, sempre dispostas a bloquear projetos de
``reformas que pusessem em risco sua hegemonia'', é tal que o escritor
vivencia uma divergência, ``que não chega ao antagonismo'', mas que o
``induz a conceber a atividade intelectual como um instrumento eficaz
para aliar"-se à modernização do país, sem romper com o sistema arcaico
que paralisava as ideias.'' (p. 21).

Arnoni exemplifica com a agitação política no seio de algumas revistas
cuja vinculação com as ditas elites criava uma espécie de mundo de
\emph{ideias fora do lugar} ou de incongruência ideológica: ``o ato de
propor a ruptura, desvinculando"-a da realidade local, traz em si um
compromisso ideológico evidente. O modelo do inconformismo é estranho a
seu porta"-voz, mas dá-lhe em contrapartida, a condição de equivalência
com a fonte de origem.'', o modelo europeu (\emph{Idem}, p. 26).

Mas se nessa incongruência notada por Arnoni, ou pela chave de leitura
de Schwarz, o que vem à tona é tensão e inadequação, é a
incompatibilidade entre um mundo das ideias modernas idealizadas e o
mundo real tudo menos moderno, no Modernismo paulista expressado em
manifestos como o da \emph{Poesia Pau"-Brasil} e o \emph{Antropófago},
essas tensões aparecem como menos tensas e mais bem resolvidas.

Dito de outro modo, o tema da nacionalidade, ou da brasilidade,
aparecido desde sobretudo o século \versal{XIX}, parece ganhar contornos mais
amadurecidos em termos de uma dialética nacional de uma subjetividade
burguesa com o debate modernista, que foi pautado nesses termos
principalmente pela cidade de São Paulo. Não por acaso. Era a cidade,
pode"-se dizer, onde essa forma de subjetividade encontrava solo mais
fértil para florescer. A questão de saber qual seria o lugar dessa jovem
nação na marcha do mundo moderno movimentou a chamada intelectualidade
da época. Caberia ao país ser uma mera transplantação dos modelos já
avançados, caberia criar luz própria, e sobre que bases? Ou deveria o
país ser um hibridismo, uma mistura? Colher o fruto dos séculos de
miscigenação e proximidade com a natureza e temperar com as conquistas
das civilizações mais avançadas?

Nesse sentido, vale lembrar a aproximação que Schwarz faz entre a
literatura do Brasil e a da Rússia na virada do século \versal{XX}, pelo menos em
termos de Machado, que ele compara com Gógol, Dostoiévsky, Tchekhov, já
que parece ser muito distinta a tensão que há na literatura russa e a
que há na brasileira, sobretudo naquela que desemboca no modernismo
paulista. Na Rússia, a modernização se perdia num imenso território e na
inércia social, ``entrava em choque com a instituição servil'', que era
experimentada como vergonha nacional. Mas essa vergonha não impedia a
existência de um critério para também ter uma medida do ``desvario do
progressismo e do individualismo que o Ocidente impunha ao resto do
mundo''. O resumo de tal dilema, não encontrado no Brasil que forjava
seu espírito nacional pelo lado da civilização europeia ou deglutindo"-a,
é uma escolha entre Caribda e Cila, ou uma aporia -- só resolvida pela
força da marcha moderna: ``o progresso é uma desgraça e o atraso é uma
vergonha.'' (2012, p. 28). Evidentemente, para os que tomam o lado da
civilização mercantil sem pestanejar, esse dilema não passa de um falso
dilema.

Tão falso que a estética antropofágica encontra uma resolução original
pela recusa da importação dos modelos europeus, dos colonizadores, em
proveito da deglutição desses mesmos modelos, tingidos de cor local --
entenda"-se mestiça, popular, espontânea, primitiva, instintual -- para
que a vida social burguesa pudesse ter no Brasil um espírito brasileiro
e não um caráter meramente transplantado da Europa. Nasceria assim em
solo nacional uma alternativa à rigidez neurótica europeia, num caldo
cultural que junta ``O necessário de química, de mecânica, de economia e
de balística. Tudo digerido.'' (\versal{ANDRADE}, 1978, p. 10). Como diz Antônio
Cândido e Aderaldo Castelo (1975), trata"-se de uma ``verdadeira
filosofia embrionária da cultura'': ``Oswald propugnava uma atitude
brasileira de devoração ritual dos valores europeus, a fim de superar a
civilização patriarcal e capitalista, com suas normas rígidas no plano
social e os seus recalques impostos, no plano psicológico.'' (p. 16).
Gesta"-se a crença ideológica de que o espírito nacional advindo dessa
deglutição -- dita ``Absorção do inimigo sacro'' (\versal{ANDRADE}, 1978, p. 18)
-- seria em si quase anticapitalista, embora mantenha como fundo tácito
o desenvolvimento das categorias capitalistas, sem as quais por outro
lado não há modernização.

No seu ensaio \emph{A crise da filosofia messiânica,} Oswald faz uma
crítica de fato do patriarcado, identificando"-o à civilização europeia,
ao que ele chama de cultura messiânica, à qual ele opõe o matriarcado,
que coincidiria com a cultura antropofágica -- que se distingue do
canibalismo, que não é ritualístico. Mais do que esse aspecto, que não
deixa de ser interessante, salta aos olhos no texto a fundamentação
filosófica da antropofagia dentro de uma dialética. Assim, a \emph{tese}
seria o homem natural, a \emph{antítese}, o homem civilizado, e a
\emph{síntese}, o homem natural tecnizado (\emph{Idem}, p. 79). Para
Oswald, sua época vivia o 2º termo, o estágio da negação. O seu objetivo
é a chegada ao 3º termo, à síntese que seria a deglutição dos aspectos
civilizacionais, técnicos, pelo homem natural, o brasileiro
evidentemente.

\begin{quote}
No mundo supertecnizado que se anuncia, quando caírem as barreiras
finais do Patriarcado, o homem poderá cevar a sua preguiça inata, mãe da
fantasia, da invenção e do amor. E restituir a si mesmo, no fim do seu
longo estado de negatividade [o segundo termo da dialética], na
síntese, enfim, da técnica que é civilização e da vida natural que é
cultura, o seu instinto lúdico. Sobre o Faber, o Viator e o Sapiens,
prevelecerá então o Homo ludens. À espera serena da devoração do planeta
pelo imperativo do seu destino cósmico''. (1978, p. 83)
\end{quote}

Portanto, as revoluções industriais estariam anunciando a volta do
matriarcado e, mais que isso, a realização da síntese do homem natural
tecnizado, um sujeito evoluído para cuja gestação o Brasil estaria em
posição de vanguarda. Nem Marx, em seu \emph{Fragmento sobre as
máquinas} contido nos \emph{Grundrisse}, conseguiu expressar tamanho
otimismo acerca da evolução automática da humanidade. Nada nos permite
negar que seja um pensamento bonito, no entanto, é uma forma de
fundamentar uma estética que só demonstra o quanto falta ao poeta e ao
ensaísta uma abordagem de fundo do processo moderno de produção de
riquezas chamado capitalismo. Além disso, dizer que ``o matriarcado se
anuncia com suas formas de expressão e realidade social que são: o filho
de direito materno, a propriedade comum do solo e o Estado sem classes,
ou a ausência de Estado.''(1978, p. 128) pode parecer bastante
revolucionário, mas choca com os meios como ele pretende que a história
humana esteja caminhando para isso: como uma espécie de consequência
óbvia do progresso técnico e intelectual materializados pela Razão
instrumental.

Evidentemente, nem todos os ícones do Modernismo paulista defendem
abertamente a modernização do país como ``acesso à racionalidade, ao
pragmatismo, enfim, à ética capitalista'', como o faz um Menotti del
Picchia que, conforme Mônica Velloso (1993),

\begin{quote}
defende esses valores e pleiteia a morte necessária do romantismo.: Em
``O último romântico'', o autor lamenta o caráter anacrônico de um
suicídio amoroso, argumentando que os novos tempos exigem que o amor
passe para o domínio de uma simples operação financeira, devendo essa
mesma dinâmica ocorrer no nível da vida pessoal, social e política. (p.
7)
\end{quote}

Há também as reações, como a do Manifesto Regionalista de 1926. Nele,
Gilberto Freyre faz críticas à homogeneização das cidades, ao estilo
citadino de vida, à cultura urbana ocidentalizada, sem todavia
fundamentar uma verdadeira crítica da modernidade senão em tom
conservador, e mesmo assim reconhecendo os arroubos modernos, contanto
que não arrasassem tudo e respeitassem certa tradição.

\begin{quote}
Reconheçamos a necessidade das ruas largas numa cidade moderna, seja
qual for sua situação geográfica ou o sol que a ilumine; mas não nos
esqueçamos de que a uma cidade do trópico, por mais comercial ou
industrial que se torne, convém certo número de ruas acolhedoramente
estreitas [\ldots{}](\versal{FREYRE}, 1996, s. p.)
\end{quote}

Quem não gosta de ruas acolhedoramente estreitas, que nos protegem do
sol e onde podem estar guardadas agradáveis surpresas? Mas defender sua
existência ao lado das ruas largas que rasgam as cidades modernas não se
constitui em crítica à marcha arrasadora do progresso, e se aproxima
mais a um pedido para que essa marcha não abocanhe todos os cantos da
vida.

À parte essas manifestações estéticas e políticas em outras partes do
Brasil, vão se diferenciar dois tipos de modernismo no centro agitativo
principal: um mais cosmopolita, identificado com a Semana de 22 e um
mais regionalista, localista, identificado com a doutrina verde"-amarela
(\versal{VELLOSO}, 1993, p. 9). Esse dito regionalismo tem como umbigo a cidade
de São Paulo e não deve ser confundido com o regionalismo nordestino --
que expressava em geral muito mais um outro Brasil em relação ao surto
modernizador -- de figuras como Rachel de Queiroz, Graciliano Ramos ou
José Lins do Rego. Trata"-se de um regionalismo paulista ligado à
modernização de São Paulo, em que o Verde"-amarelismo aparece como
corrente ideológica mais nacionalista dentro do modernismo, defendendo
um projeto de cultura nacional calcado num ``retorno idílico às
tradições do país.'' (\emph{Idem}, 1993, p. 10). Nesse localismo
Verde"-amarelo, a busca da nacionalidade passa pela afirmação do ser
nacional ligado a leituras bastante discutíveis da época colonial como
mundo apaziguado: ``Para os verde"-amarelos, foi São Paulo que deu início
ao processo nacionalizador. Através da epopéia das Bandeiras, em pleno
século \versal{XVI}, o Estado partiu para a conquista do território. Cabe a São
Paulo, portanto, coordenar todas as vozes regionais, assegurando a
comunhão brasileira.'' (\emph{Ibidem}, 1993, p. 16). Mas concentremo"-nos
no que aparecia como vanguarda de fato, cuja nacionalidade defendida na
melhor das boas intenções burguesas não tem nada que ver com o ufanismo
com conotações de fascismo tupiniquim do grupo Anta verde"-amarelo.

Haroldo de Campos, na sua introdução às \emph{Poesias Reunidas} de
Oswald de Andrade, destaca um depoimento do poeta que é dos mais
significativos para uma compreensão da ligação entre a vanguarda
modernista e o impulso de uma subjetividade burguesa no Brasil:

\begin{quote}
``Se procurarmos a explicação do por que o fenômeno modernista se
processou em São Paulo e não em qualquer outra parte do Brasil, veremos
que ele foi uma conseqüência da nossa mentalidade industrial. São Paulo
era de há muito batido por todos os ventos da cultura. Não só a economia
cafeeira promovia os recursos, mas a indústria com a sua ansiedade do
novo, a sua estimulação do progresso, fazia com que a competição
invadisse todos os campos de atividade''. É o retrospecto de Oswald, em
1954. (1974, p. 11)
\end{quote}

Como diz Alfredo Bosi, na sua interpretação de \emph{Macunaíma,} foi em
São Paulo, onde estava a burguesia mais aberta às influências
internacionais, ``o segmento cultural mais avançado do país'', nessa
cidade reeuropeizada pela moda do \emph{art nouveau} e pela imigração,
``e já entrada na era da máquina e das relações capitalistas, onde pôde
gerar"-se uma revisão drástica do tom e do teor neoparnasiano ainda
vigente nas províncias e no Rio de Janeiro da Academia Brasileira de
Letras.'' (1988, p. 174).

Pretende ainda Haroldo de Campos, sem corar, que a poesia de Oswald de
Andrade é ``Uma poesia de tipo industrial, diríamos, por oposição ao
velho artesanato discursivo, institucionalizado em modelos retóricos
pelo parnasianismo'' (1974, p. 15). Ou seja, há um empenho modernizador
que se reflete diretamente no terreno estético, inclusive na sequência
de gravuras feitas por Tarsila para as \emph{Poesias Reunidas.} Cada
imagem, assim como os poemas, trazem o avançar da sociedade Brasileira.
Do momento em que só havia os navios e pequenos barcos chegando às
costas, até os \emph{postes da Light,} que simbolizam o progresso, em
que já se vê na gravura prédios e imagens de postes e indústrias e
bondes. É a parte da obra em que a modernização de São Paulo aparece de
forma expressiva. No poema \emph{Atelier}, por exemplo, há um tom
apologético da mistura de carros, klaxons que dão a cor da verdura, que
já não existe. Uma verdura que agora é buzina, arranha"-céus e o café,
que está ficando para trás. Mas é uma modernidade, como lembra Schwarz
ao analisar um dos poemas da seção, em que o `` progresso é inegável,
mas a sua limitação, que faz englobá-lo ironicamente com o atraso em
relação ao qual ele é progresso, também.'' (1987, p. 15). De todo modo,
o desenvolvimento ou a afirmação da nacionalidade tinha que passar pela
modernização nos moldes inconfessadamente europeus -- mas evidentemente
com tonalidade malemolente, pretensamente primitivista, espontânea, com
todas as cores nacionais que negariam a ideia de importação de
civilização.

O primeiro poema da seção Loide Brasileiro é o \emph{Canto de regresso à
pátria.} O que o poeta chama de pátria não é mais o Brasil, como no caso
de Gonçalves Dias, mas é São Paulo. Isso não teria nada a ser sublinhado
se o poema não fosse ao encontro de um processo de entronamento da
cidade como símbolo brasileiro, como verdadeira alegoria nacional, como
a locomotiva. E a última estrofe do poema com seu ``Não permita Deus que
eu morra/ Sem que volte pra São Paulo/ Sem que veja a Rua 15/ E o
progresso de São Paulo'', (\versal{ANDRADE}, 1974, p. 144) é bastante expressiva
dessa ideologia que não pode ser vista apenas como tirada poética.

Nesse mesmo sentido, os manifestos do período modernista pretendiam
colocar os artistas em pé de altura, para não dizer em nível mais
elevado, em relação às vanguardas europeias. Diferentemente da retórica
de uma busca da alma tupi e de uma formação nacional pacífica presente
no \emph{Manifesto} \emph{Nhegaçu Verde Amarelo}, os \emph{Manifestos
Antropófago} e da \emph{Poesia Pau"-Brasil} têm como lastro a apologia da
engenharia, da modernidade, da cidade, do concreto, do progresso, embora
com a tonalidade local, irreverente e questionadora do passado colonial.
A disputa é entre um Brasil que importa e copia, e um Brasil que
inventa, ``Poesia Pau"-Brasil, de exportação.''(\versal{ANDRADE}, 1978, p. 7). Um
Brasil que ganha um lugar de protagonista e não de copiador. Ou seja, o
modernismo aparece nesses manifestos como uma forma de dar cor
primitiva, indigenista, negra, mestiça, à vida social burguesa no Brasil
-- dito de outro modo, às categorias capitalistas no Brasil. E é aí que
se encontra o aspecto ideológico do modernismo paulista, uma ideologia
aparentemente aporética do progresso, que tenta sintetizar dois termos
inconciliáveis, já que é impossível conciliar progresso e primitivismo,
a menos que este seja incorporado como estética de superfície -- como é
o caso do Modernismo de São Paulo.

À diferença das vanguardas européias, cuja empreitada pretendia pôr em
questão a sociedade burguesa -- embora tal empreitada tenha significado
um passo no caminho de um movimento culminado nas contestações
contraculturais dos anos 1960 que, sem ter esta pretensão, acabaram por
liberar o sujeito burguês das amarras tradicionais -- o tipo de
vanguarda expressa nos \emph{Manifestos da Poesia Pau"-Brasil} e
\emph{Antropófago} está mais para um acerto de contas com os modelos, um
acerto de conta com o pai"-Metrópole, do que para um questionamento da
ordem burguesa. Assim, a afirmação de Haroldo de Campos de que a poesia
de Oswald é uma poesia de \emph{ready} \emph{made} (1974, p. 28), uma
poesia em consonância com as vanguardas europeias aparece como
exagerada, senão na forma, no conteúdo. Se estas viam no mundo burguês a
decadência do mundo, a prisão dos sentidos, a \emph{Poesia Reunida} de
Oswald vê a decadência no fato de ainda não termos um mundo burguês e
uma subjetividade burguesa -- embora pretenda também criticar o mundo
burguês. Assim, a fórmula ``ver com olhos livres'' (\versal{ANDRADE}, 1978, p. 9)
tem na Europa um sentido, e no Brasil outro, embora a frase seja
vanguardista lá e cá. Mas, num caso, o desejo é se livrar do mundo
burguês, no outro, o desejo é a vinda desse mundo, preferível àquele dos
doutos arcaicos identificados com o parnasianismo, o gabinetismo, o
jurisconsulto, o domínio dos senhores rurais. Lá, a oposição era feita
aos aos últimos laços arcaicos que a vida burguesa trazia -- o
regramento extremo da vida, a rigidez, o cálculo, a racionalidade, a
negação do sonho em proveito da consciência da realidade --, ao que ela
anda tinha de protestantismo e ascetismo. Aspectos que, enquanto
entrecruzamento de uma subjetividade virada para o passado em tensão
contínua com outra virada para o futuro -- que compõe a dialética da
subjetividade burguesa --, apenas se ramificavam na vida social
brasileira.

Assim, a própria estética da antropofagia, como vanguarda, parece uma
impropriedade. Inconfessadamente, o ato ritual antropofágico era feito
com a forma"-sujeito burguesa para tirar dela suas virtudes e criar algo
tipicamente nacional. Em vez de um questionamento de vanguarda ao modo
de vida burguês, o que nasce é o germe de um sujeito burguês
\emph{parodiado}, do qual Macunaíma seria um protótipo bastante
contemporâneo. Uma paródia que, como veremos, não inibe o
desenvolvimento da vida social burguesa.

E é por isso que a revolução Caraíba, maior que a Revolução francesa
(1978, p. 14), pensada por Oswald de Andrade, não só aparece como
utópica, mas como uma contradição em termos -- o final melancólico de
\emph{Macunaíma}, de Mário de Andrade, que se torna a constelação Ursa
Maior, parece apontar para essa impossibilidade da existência mesma dos
povos originários, depois de os objetivos mercantis aportarem numa
terra. Como o caraíba poderia fazer uma revolução para modernizar o
Brasil? Como o caraíba iria fazer uma revolução com os ideais que
fundamentaram a sociedade que os extinguiu no fim das contas? Um caraíba
operário de fábrica, pintor de quadros ou fazedor de poemas não parece
verossímil. A não ser como \emph{carnavalização} do sujeito burguês, uma
dessacralização para torná-lo malemolente, flexível aos ditames
modernos.

\section{\emph{Macunaíma} e a brasilidade burguesa~de~limbo}

\begin{quote}
O pecurrucho tinha cabeça chata e Macunaíma inda a achatava mais batendo
nela todos os dias e falando pro guri:
--- Meu filho, cresce depressa pra você ir pra São Paulo ganhar muito
  dinheiro.

\emph{Mario de Andrade}, \emph{Macunaíma}
\end{quote}

Para Alfredo Bosi (1988), em \emph{Macunaíma}, imbricado ao aspecto
estético, há ``o desejo não menos imperioso de pensar o povo brasileiro,
\emph{nossa} \emph{gente}, percorrendo as trilhas cruzadas ou
superpostas da sua existência selvagem, colonial e moderna, à procura de
uma identidade que, de tão plural que é, beira a surpresa e
indeterminação; daí ser o herói sem nenhum caráter.'' (p. 171). Ou seja,
para Bosi, é preciso entender \emph{Macunaíma} pelas duas motivações: a
de narrar, lúdica e estética; e a de interpretar, histórica e
ideológica.

Nossa tentativa de leitura de \emph{Macunaíma} tenta enveredar mais pelo
caminho da interpretação do Brasil -- histórica e ideológica -- que
aparece na obra, sem necessariamente ser obra de vontade do autor.
Assim, não nos interessa se \emph{Macunaíma} tem inspiração da obra que
o alemão Koch"-Grümberg escreveu a partir de sua viagem ao norte do
Brasil até a Venezuela. Não nos interessa provar que Macunaíma é um
personagem"-símbolo da terra chamada Brasil, ou genuinamente dessa terra,
mas muito mais refletir sobre tal personagem como uma \emph{alegoria} ou
metáfora da subjetividade brasileira em confronto com a vida social
burguesa amadurecida. Sendo assim, trata"-se de uma reflexão para cuja
validade somente a época contemporânea pode contribuir. Porque as
questões que tentaremos levantar não se apresentavam de modo algum para
o autor. De certo modo, trata"-se de confrontar a obra com com o
desdobrar"-se da forma"-social burguesa no Brasil. É ao que nos
dedicaremos agora.

Como diz Bosi, no momento do modernismo, a Civilização mais requintada
se encanta com o pensamento selvagem, chamado de pensamento primitivo:

\begin{quote}
Algo de comum ou, mais precisamente, de analógico, vai se articulando
entre esse universo, colonizado e oprimido havia séculos, e as novas
estéticas cujo horizonte de sentido era a denegação da mente
racionalizadora imposta ao planeta inteiro desde que se consolidara o
modo de viver e pensar capitalista. Nessa rede de afinidades, entende"-se
o veio neo"-indianista e neo"-folclórico do modernismo brasileiro.'' (p.
174)
\end{quote}

É aí que começa a se complicar o emparelhamento do modernismo paulista
com as vanguardas europeias, pois aqui o que se pretendia denegar era
muito mais a mentalidade patrimonialista, rural, personalista, atrasada
do que a racionalidade capitalista que por aqui apenas andara dando
algumas voltas até então. É nesse sentido que se pode dizer que o
modernismo brasileiro é também um programa de modernização, um projeto
de gestar a racionalidade capitalista no Brasil, embora com o nome menos
assustador de desenvolvimento e saída do atraso, e com toques
malemolentes, preguiçosos, malandros, primitivos, em vez de simplesmente
copiá-la em sua forma já existente nos países mais à frente no
desenvolvimento da forma de viver capitalista: ``O modernista passa a
aspirar ao brilho agudo do aço da civilização industrial em expansão,
que os manifestos futuristas proclamam, ou então a uma forma selvagem de
contracultura que vinha elegendo os seus emblemas entre os símbolos do
inconsciente.'' (\versal{BOSI}, 1988, p. 174.). Notemos que no trecho anterior
Bosi falava em ``denegação da mente racionalizadora'', enquanto agora
fala em ``brilho do aço da civilização industrial'' como algo aspirado
pelo modernista, como se civilização industrial e a racionalização do
mundo não andassem de mãos dadas.

Se em Alencar os índios ganham caráter de nobreza europeia, em Mário de
Andrade o índio Macunaíma deglute o caráter da burguesia europeia, e
assim fica praticamente no mesmo pé a questão da identidade que o
\emph{indianismo} do modernismo paulista pretendia resolver. A diferença
é que o \emph{indianismo} de Mário de Andrade está em consonância com os
tempos modernos, é revolucionário, virado para o futuro, é uma síntese
mal"-acabada indígeno"-burguesa, enquanto a síntese mal"-acabada e não
menos problemática desenhada por Alencar, o indígeno"-nobre, é virada
para o passado, é reacionária.

Assim, o que se delineia em \emph{Macunaíma} é uma tentativa a um só
tempo bem"-sucedida e malfadada de síntese da identidade brasileira.
Bem"-sucedida porque logrou a construção de uma ideologia de cultura
própria fundada em raízes próprias; mas malfadada porque o que é tido
por próprio não desemboca na alternativa à subjetividade e à sociedade
burguesas prometida, mas é incorporado pela forma"-sujeito burguesa. O
que mostra a dificuldade da empreitada de construção de uma identidade
nacional num mundo que tende a anular o que é próprio ou delegá-lo ao
plano do enfeite secundário.

Mesmo assim, é forçoso concordar em parte com Bosi quando diz ``que não
há em Macunaíma a contemplação serena de uma síntese''. Em parte, porque
no decorrer da obra, embora o autor insista ``no modo de ser incoerente
e desencontrado desse `caráter' que, de tão plural, resulta em ser
`nenhum'" (\versal{BOSI}, 1988, p. 178), é esse caráter que parece não tensionar
por onde passa. Macunaíma parece mover"-se com desenvoltura tanto no mato
quanto na cidade. Vemos sempre Macunaíma \emph{se virando} de todo
jeito, sempre se adaptando às situações com jogo de cintura, com ginga.
Nesse sentido, sua alta capacidade de adaptar"-se aos ambientes e se dar
bem parece ser, com todas as incoerências e desencontros, sua forma
brasileira de estar no mundo burguês. Nesse sentido, a síntese pode até
não ser contemplada serenamente, já que Macunaíma sente certo enjoo de
viver no mundo, mas não deixa de pretender ser uma síntese.

O fio condutor da obra é o que Gilda Mello chama de busca do Graal, ``a
busca do objeto miraculoso'' (2003, p. 60), a busca de Macunaíma pela
pedra"-amuleto \emph{muiraquitã} que ganhara da rainha das Icamiabas, no
momento de sua morte. A forma com que domina a rainha da tribo é
apresentada como uma espécie de paródia de um processo de colonização,
misturando violência e ``brincadeira'' -- sempre sinônimo de aventura
sexual. Perdendo o talismã numa luta e sabendo que uma tartaruga pescada
por um mariscador o havia encontrado e vendido a um rico fazendeiro
residente em São Paulo, decide ir com os irmãos para lá.

Macunaína chega a São Paulo comicamente, numa postura paródica do
conquistador, como se fizesse o caminho inverso dos bandeirantes para
conquistar São Paulo: ``Na frente [da embarcação] Macunaíma vinha de
pé, carrancudo, procurando no longe a cidade.'' (\versal{ANDRADE}, 1988, p. 36).
Tendo só em aparência deixado sua identidade na foz do Rio Negro, ao
entrar na cidade, toma um banho bastante metafórico:

\begin{quote}
[\ldots{}]a água era encantada porque aquele buraco na lapa era marca do
pezão do Sumé, do tempo em que andava pregando o evangelho de Jesus pra
indiada brasileira. Quando o herói saiu do banho estava branco louro e
de olhos azuizinhos, água lavara o pretume dele. E ninguém não seria
capaz mais de indicar nele um filho da tribo retinta dos Tapanhumas.
(\versal{ANDRADE}, 1988, p. 37)
\end{quote}

É um trecho dos mais significativos, em que salta aos olhos -- mais do
que uma ironia em relação aos imitadores que querem se pintar das cores
do mundo exterior -- uma alegoria da forma"-sujeito burguesa enquanto uma
fôrma macho"-branco"-ocidental, como podemos ler num interessante estudo
de Roswitha Scholz (1997). É desse protótipo que vai se revestir
Macunaíma na água encantada, para poder entrar na cidade, no limiar
entre sua subjetividade anterior e a posterior. O colorido de chofre,
dá-lhe um tom meio de imperfeição à sua adaptação para entrar na cidade.
Mas essa imperfeição não o impede de entrar e mover"-se muito bem nela,
numa bela demonstração de jogo de cintura. Para o autor, isso aparece
como uma paródia, como uma crítica dessa vontade de se revestir do
europeu, algo criticado em várias metáforas no decorrer da obra. E seus
irmãos, achando bonito, querem seguir Macunaíma:

\begin{quote}
Nem bem Jiguê percebeu o milagre, se atirou na marca do pezão do Sumé.
Porém a água já estava muito suja da negrura do herói e por mais que
Jiguê esfregasse feito maluco atirando água pra todos os lados só
conseguiu ficar da cor do bronze novo. Macunaíma teve dó e consolou:

---~Olhe, mano Jiguê, branco você ficou não, porém pretume foi"-se e
antes fanhoso que sem nariz. Maanape então é que foi se lavar, mas Jiguê
esborrifara toda a água encantada pra fora da cova. (\versal{ANDRADE}, 1988, p.
37)
\end{quote}

O desejo, assim, vai além do herói e é alimentado pelos irmãos, que
ficam decepcionados por não poderem se lavar das belas cores e ganhar
pelo encanto/desencanto as cores necessárias para se dar bem na
cidade"-símbolo da vida burguesa. Não à toa, diferentemente de Macunaíma,
os irmãos penarão bem mais para mover"-se em solo citadino, e serão o
antípoda do herói sem caráter. A água suja deixada por Macunaíma aos
irmãos após seu banho pode ainda, sem forçar a leitura, ser uma metáfora
da impossibilidade de todos se darem bem numa sociedade mundialmente
guiada pela concorrência desenfreada. Nem todos cabem no centro da
marcha moderna, há os que estão no pelotão da frente, e os que
imperfeitamente enquadrados objetiva e subjetivamente ficam em pelotões
intermediários. Mas mesmo os atrasados sempre procuram se identificar
com os que estão na frente, embora normalmente seja de maneira paródica
ou no mínimo desengonçada, típica de quem chega atrasado a um baile de
máscaras -- no caso moderno, o baile exige máscaras de caráter do
sujeito burguês --, com a máscara meio torta, mas não menos aferrada ao
rosto no essencial para partilhar o baile.

Esse desejo de identificação é parodiado criticamente por Mário de
Andrade em Macunaíma, mas de modo algum como crítica aos fundamentos da
vida social moderna importados, de modo algum como crítica da
subjetividade burguesa, mas como crítica da importação acrítica desses
fundamentos, que deveriam a seu ver submeter"-se a uma coloração local,
às especificidades locais.

Apesar de tudo, o alvejamento de Macunaíma não lhe garante uma
subjetividade europeia como por mágica. Não. Ele se mantém como amálgama
e é isso que o mantém atual numa interpretação da subjetividade burguesa
brasileira. Tendo uma preguiça estrutural, Macunaíma quer voltar para
casa ao saber que vai precisar trabalhar, já que sua moeda não voga na
cidade. Mas o ato não é nada de revolucionário, pois não dispensa os
prazeres que o dinheiro pode comprar. Acostumado à sua moeda natural,
bagues de cacau, que brotam da natureza, não se habitua à mediação do
trabalho e quer viver na cidade essa mesma relação imediata com a
natureza. Assim, o que era inadequação da subjetividade à sociedade
burguesa num estágio de sujeito neuróticos e ascéticos, vira adequação
extrema à sociedade burguesa de sujeitos narcísicos de desejos
imperiosos.

Também nesse sentido, há uma metáfora do instinto nacional brasileiro
bastante expressiva no segundo capítulo. Ao ser castigado pela mãe por
uma malvadeza que fizera com os irmãos, Macunaíma se embrenha no mato e
encontra seu ``avô'' Currupira, que também sofre com suas maldades.
Chamado de Menino"-home por Currupira, por fazer maldades de gente
grande, Macunaíma chega a sua ``avó'' Cotia, que o castiga ao saber da
ruindade que fizera com Currupira. Joga nele caldo de mandioca
envenenado, do qual só consegue livrar a cabeça. O veneno faz seu com
que seu corpo cresça e sua cabeça fique de criança, com cara enjoativa.
Há que se ver aí uma dialética entre Corpo"-Espírito, entre
Brasil"-Europa. O corpo desenvolvido expressa a dimensão, a força, o
gigantismo, o crescimento, o amadurecimento do país enquanto forma,
enquanto modernização das estruturas. A cabeça de criança expressa muito
mais o lado primitivo, liberto, direto, livre das imposições morais, do
represamento do eu típico das sociedades avançadas capitalisticamente
que já passaram por essa modernização estruturante. Mas ao mesmo tempo,
a cabeça de criança, livre desses imperativos sociais, pode se sentir
livre para agir livre de qualquer senso de comunidade, como um ser
social, mas que traz interiorizados em sua subjetividade os automatismos
da natureza, que age de acordo com os instintos, sem atentar para a vida
de fato em sociedade, que é vista como o melhor meio para impor o seu
eu, para fazer florescer suas potencialidades narcísicas.

Nesse sentido, cabe uma reflexão sobre essa aproximação em Macunaíma
entre primitivo e moderno. Se atentarmos, por exemplo, para o
vocabulário usado por Macunaíma para falar das coisas das cidades,
veremos que não é de modo algum anódino. Expressa uma fusão desejada
entre natureza -- o nacional -- e sociedade -- a civilização europeia:

\begin{quote}
A inteligência do herói estava muito perturbada. Acordou com os berros
da bicharia lá em baixo nas ruas, disparando entre as malocas temíveis.
[\ldots{}] De"-manhãzinha ensinaram que todos aqueles piados berros
cuquiadas sopros roncos esturros não eram nada disso não, eram mas
cláxons campainhas apitos buzinas e tudo era máquina. As onças pardas
não eram onças pardas, se chamavam fordes hupmobiles chevrolés dodges
mármons e eram máquinas. Os tamanduás os boitatás as inajás de curuatás
de fumo, em vez eram caminhões bondes autobondes anúncios"-luminosos
relógios faróis rádios motocicletas telefones gorjetas postes
chaminés\ldots{}Eram máquinas e tudo na cidade era só máquina! (\emph{Idem},
p. 40)
\end{quote}

Trata"-se aqui de uma forma de ver o mundo que no fundo lhe impregna de
tons naturais, numa mistura entre primeira e segunda natureza. Macunaíma
faz uma transmutação distendida entre os bichos da natureza e os
``bichos'' completamente distintos da cidade -- um gênero de metáfora
que também aparece numa poesia de Oswald de Andrade, em que ele chama os
bondes de ``Grandes cágados elétricos'' no poema \emph{O combate} (1974,
p. 125). Chega"-se ao cúmulo quando Macunaíma, após uma viagem fantástica
que subverte completamente as regiões do país a ponto de perder"-se, pede
ao primo tuiuiú que o leve para casa: ``Logo o tuiuiú se transformou na
máquina aeroplano [\ldots{}]'' (\versal{ANDRADE}, 1988, p. 109).

Assim, Macunaíma passa do primitivo ao civilizado com grande
desenvoltura, operando mesmo uma síntese entre os dois mundos, algo que
se mostra mais expressivo num momento em que a vida social capitalista
amadurece e se desdobra como segunda natureza. Ele não tensiona com seu
entorno, ao contrário, quer ser imperador dos filhos da mandioca. Seu
raro momento de reflexão, em que chega à conclusão de que ``os homens é
que eram máquinas e as máquinas é que eram gente'' (p. 41), serve apenas
para fazê-lo cair na gargalhada com a descoberta, depois da qual ele
``ligou pros cabarés encomendando lagosta e francesas.'' (p. 41).

Mesmo nessa viagem fantástica pelas regiões podendo representar um
contraponto em relação ao modernismo que coloca São Paulo como centro, é
São Paulo quem dá o tom moderno, quem fornece o espelho no qual, de
certo modo, as outras regiões deveriam desejar se ver refletidas. É isso
que Mônica Velloso parece perder de vista ao dizer que Macunaíma não
``não segue a lógica dos roteiros possíveis, mas inventa uma espécie de
`utopia geográfica' que vem corrigir o grande isolamento em que vivem os
brasileiros.'' (1993, p. 11). E é nesse sentido que a observação de
Gilda Mello parece mais certeira, embora não necessariamente esteja na
direção de nosso entendimento. Ela vê no sobrevoo de Macunaíma pelo
Brasil no Tuiuiú-aeroplano, em que o personagem descortina o mapa do
país, como ``a projeção de um desejo profundo do autor: desejo de
estabelecer a identidade entre o habitante rico do Sul e o pobre
seringueiro do Norte'' (2003, p. 33), entre as cidades prósperas e
superpovoadas e o vasto interior, onde reinaria ainda a pobreza, a
incultura e o deserto.

Nesse mesmo sentido de fusão natureza e cultura é que o herói narra a
origem da máquina automóvel, que teria surgido da natureza. Ao fugir de
um tigre, uma onça teria engolido um motor, posto uns vagalumes nos
dentes, engolido gasolina e óleo de mamona para não feder. A onça é
quase a metáfora da sociedade moderna, já que anda sempre fugindo e não
pode parar para o tigre não pegá-la: ``Anda sempre com roda nos pés,
motor na barriga, purgante de óleo na garganta.'' (\versal{ANDRADE}, 1988, p.
131).

Essa mistura pouco tensa entre natureza e sociedade, ou entre natureza e
cultura, não é um aspecto secundário na obra. Após ser morto pelo
gigante que detém seu amuleto e ressuscitar pela mistura de sangue
sugado pela formiga e o resto de seus ossos, o herói vai armar"-se. Vai
ao encontro dos ingleses que vendiam garruchas que davam em pés de
árvores, assim como as balas e o uísque. Esse mito da árvore do paraíso,
trazida para o mundo das mercadorias, serve bem para naturalizar o mundo
da cultura mercantil.

Estando sempre disposto a vender, a trocar, a adaptação à cidade é algo
que impressiona em Macunaíma. Ele se aventura pelo mundo sem
ancoradouro, e sua natureza, de onde ele vem, não parece chocar"-se com a
natureza da cidade. Ao contrário, ele parece até dar mais fulgor a essa
natureza.

Mas é preciso também entender qual o movimento crítico do autor através
da obra. Como já dissemos, Mário de Andrade não faz qualquer apologia
aberta da modernidade ou da forma"-sujeito burguesa, porque talvez
pretenda fazer uma crítica, mas uma crítica que se perde na crítica da
importação acrítica dos modelos. A carta que o rei Macunaíma manda para
seus súditos, em vocabulário de empréstimo, em tom empolado que não cola
com o personagem, num tom imitativo da linguagem do colonizador, é uma
forma de paródia do vocabulário lusitano e dos intelectuais brasileiros
que prezavam essa forma de expressão -- identificada com o
parnasianismo. Ao final da carta, pede dinheiro às súditas para poder
brincar:

\begin{quote}
Bem podereis conceber, pois, quanto hemos já gasto; e que já estamos
carecido do vil metal, para brincar com tais difíceis donas. Bem
quiséramos impormos à nossa ardida chama uma abstinência, penosa embora,
para vos pouparmos despesas; porém que ânimo forte não cedera ante os
encantos e galanteios de tão agradáveis pastoras! (\emph{Idem}, p. 75)
\end{quote}

Mesmo sua crítica à cidade de São Paulo, onde ``insetos devoram as
mesquinhas vidas da ralé e impedem o acúmulo de desocupados e
operários'' (p. 80), não impede Macunaíma de pretender fundar junto às
Icamiabas uma civilização parecida. Por mais que seja uma cidade que
ainda tem duas línguas, uma falada e uma escrita -- que apaga as
asperezas e espontaneidades da falada --, por mais que ainda peque pela
falta de originalidade, está à frente das outras regiões, é uma
cidade"-luz.

Do mesmo modo, ao saber que o gigante de quem quer tomar de volta o
amuleto fora para a Europa, decide ir em busca dele. O narrador pinta um
quadro irônico de como Macunaíma deveria se apresentar na Europa.
Fingir"-se de pianista? Não, de pintor: ``No outro dia pra esperar a
nomeação matou o tempo fazendo pinturas'' (\emph{Ibidem}, p. 112). A
passagem abre pelos menos dois caminhos de interpretação: no primeiro,
mais óbvio, o autor faria aqui uma mordaz crítica aos artistas
brasileiros da época sempre a imitar os gostos europeus. No segundo,
Macunaíma seria um retrato de um personagem sagaz que se vira de todo
modo em toda e qualquer circunstância: se é preciso ser pianista,
pintor, dá-se um jeito. Não tendo apoio do governo para ir à Europa, em
busca do Piaimã que está com seu talismã, Macunaíma desiste dizendo:
``Não vou à Europa não. Sou americano e meu lugar é na América. A
civilização europeia decerto esculhamba a inteireza de nosso caráter.''
(\emph{Ibidem}, p. 114-115). A frase parece soar imprópria da boca de
quem é chamado de \emph{herói sem nenhum} caráter. Mas no fundo, é o
contrário: a Europa é vista pela ideologia modernista como aquela que
vai endurecer o caráter, que é o verdadeiro significado de ``esculhambar
o caráter'', enquadrá-lo em categorias racionalizadoras. Assim, seria
preferível manter a falta de caráter, no sentido de rigidez, manter a
fluidez, a malemolência, a ginga para se opor a essa esculhambação
europeia. A crítica, portanto, representa certo movimento pendular,
notado por Gilda Mello, que faz parte da oscilação presente na obra
``entre o modelo europeu e a diferença brasileira'' (\versal{MELLO}, 2003, p.
62), entre a atração pela Europa e a fidelidade ao Brasil -- como se a
chamada fidelidade ao Brasil estivesse necessariamente em choque
profundo com a Europa. Assim, Macunaíma aparece às vezes como
personagem"-crítica, ou como alvo da crítica pela sua entrega à
civilização europeia. O tom paródico e crítico toca mais as posturas
copiadoras brasileiras voltadas sempre para o exterior e que esquece sua
pretensa essência, sem que seja uma crítica de profundidade, pelo menos
para nossa leitura hoje, à luz de uma crítica da forma social moderna.

É o caso mais uma vez da visão tida por Macunaíma na fonte da praça
Carlos Gomes. Na visão, Macunaíma vê um transatlântico cheio de
``marujos foçudos, argentinos finíssimos, donas lindíssimas pra gente
brincar até enjoar'' (\versal{ANDRADE}, 1988, p. 121). Macunaíma se sente atraído
pelo navio e quer partir para a Europa ``que é milhor''. No entanto,
antes de subir, todos os tripulantes soltam uma grande vaia para o herói
e seguem sem levá-lo. Macunaíma, assim, é pintado como o personagem que
não tem um caráter nacional bastante formado para resistir aos encantos
europeus, para assumir sua brasilidade sem titubear. O problema é que,
se em termos teóricos ou estéticos a questão pode se resolver em termos
de escolhas e da busca pelo instinto de nacionalidade, em termos da vida
social, o pobre Macunaíma entra num mundo com fundamentos estruturados,
aos quais precisa enfrentar com suas artimanhas, que só contam como
lubrificantes para que a forma"-sujeito burguesa funcione melhor, não
para opor"-se a ela.

Se esteticamente podia haver escolhas de posturas voltadas para o
chamado instinto nacional, em vez de simplesmente copiar o estrangeiro,
em termos do fundamento da vida social, desde que se instala a marcha
moderna, não há grande margem de escolha. Temos visto que a forma social
burguesa e sua forma"-sujeito vão paulatinamente consumindo o que é
vivido e aproveitando essas pretensas especificidades como enfeite
superficial. Nesse sentido, a proposta modernista é uma alternativa
sobre os pressupostos já dados. É o dilema de toda modernização,
notadamente no caso brasileiro. É preciso primeiro passar pelos
fundamentos capitalistas para questioná-los. E ainda sim, não nos seus
fundamentos mesmos, não no conjunto de aspectos destrutivos trazidos
pela vida social moderna, mas na sua tendência ao imperialismo ou outras
coisas mais. A crítica brasileira em geral não se dirige -- e não
somente na época do modernismo, ainda hoje parece ser assim
majoritariamente -- ao fato de a modernidade tender a apagar ou consumir
ou mercantilizar e esvaziar o que há de específico culturalmente -- de
onde até poderia germinar uma crítica mais radical. A crítica é feita
geralmente em proveito de uma estética e de uma economia nacional, não
menos destrutiva, mas com cores nacionais. Assim, a vida social tornada
comunidade mercantil não passa a ser problema senão quando regulada por
interesses internacionais ou da elite, e a tentativa de criar esse
instinto nacional acaba desembocando num ``capitalismo cordial'', num
capitalismo malandro.

Finalmente, Macunaíma consegue matar o gigante Piaimã, metáfora na obra
do capitalista, comedor de gente, que tira o sangue da gente para
sobreviver, para comer. Recuperando a muiraquitã, vai voltar para sua
mata, não sem antes fazer um feitiço para transformar um prédio em bicho
preguiça de pedra. Mas antes de ir embora,

\begin{quote}
Depois de muito refletir, Macunaíma gastara o arame derradeiro comprando
o que mais o entusiasmara na civilização paulista. Estavam ali com ele o
revólver Smith"-Wesson o relógio Pathek e o casal de galinha Legorne. Do
revólver e do relógio Macunaíma fizera os brincos das orelhas e trazia
na mão uma gaiola com o galo e a galinha. Não possuía mais nem um tostão
do que ganhara no bicho porém lhe balangando no beiço furado pendia a
muiraquitã. (\emph{Idem}, p. 136)
\end{quote}

Do revólver e do relógio, Macunaíma fizera brincos. Já a muiraquitã,
levava nos lábios como lembrança amorosa, não como talismã. Ou seja,
aquilo que foi o objetivo da viagem do herói não aparece senão como mera
paródia do Graal, daquilo que o tornava um verdadeiro marupiara. Ao
contrário, sua identificação com o progresso, embora de forma paródica,
fez com que ele, adestro na proa, fazendo o caminho de volta, fosse
tomando ``nota das pontes que carecia construir ou consertar pra
facilitar a vida do povo goiano.'' (\emph{Ibidem}, p. 136). O fato de
trazer o revólver como brinco não necessariamente significa um desdém do
herói em relação a essa invenção moderna. Ao contrário, mostra mais uma
vez a naturalização de tais invenções nada ingênuas. Quando vai ao
rancho de Oibê na mata, sente medo, ``uma friagem por dentro'', não
esquece de lembrar ``do Smith"-Wesson'' (\emph{Ibidem}, p. 140) e cria
coragem para pedir pousada. De onde viria a coragem senão da
possibilidade do uso real da arma?

O pêndulo crítico de que falamos acima torna a obra por vezes ambígua.
Certa vez parece estar zombando do contexto estético e social da época,
explícito na \emph{Carta pras Icamiabas}. Outra vez, parece estar
descrevendo apologeticamente o caráter sem caráter de um instinto
nacional, marcado pela espontaneidade, o primitivismo, os instintos mais
naturais -- marcante na sede insaciável do herói pela brincadeira. Seria
talvez possível fazer a objeção de que isso tudo só acontece porque
Macunaíma deixara sua consciência na foz do Rio Negro antes de se tingir
de branco em São Paulo. Todavia, mesmo antes, ele já dava provas da
incapacidade de qualquer represamento dos desejos e da insatisfação com
qualquer desapontamento. Nesse sentido, num contexto de capitalismo
desenvolvido, de uma segunda natureza desenvolvida, a proximidade com a
natureza torna as relações capitalistas ainda mais fortes e automáticas,
como que brotando de fato da própria natureza. Diferentemente daquelas
subjetividades que passaram por séculos de traumas de disciplinamento
para fundar o ascetismo fundamental para forjar o sujeito burguês de um
determinado momento do capitalismo -- traumas que desembocam nas tensões
entre indivíduo e sociedade --, aquelas formas tidas por espontâneas e
próximas da natureza contemporaneamente parecem mais próximas de uma
segunda natureza representada pelo capitalismo desenvolvido. Isso se
torna ainda mais patente no fato de Macunaíma não se importar em não
encontrar a consciência deixada na foz do rio. Sem lamentar, ele ``pegou
na consciência dum hispanoamericano, botou na cabeça e se deu bem da
mesma forma.'' (\emph{Ibidem}, p. 148). Há nessa passagem quase um ápice
da falta de caráter de Macunaíma que, sem atributos para chamar de seus,
pode lançar mão do caráter que lhe estiver mais ao alcance da
subjetividade líquida, flutuante feito o igareté no rio Negro, para
postar"-se na marcha do mundo burguês do jeito que der.

Mas não salta só aos olhos o fato de o herói ser sem caráter próprio,
quase um \emph{Homem sem qualidades} brasileiro, mas também o fato de
ele não ter caráter mesmo, no sentido do senso de comunidade, de
caracteres comuns partilhados por uma coletividade. Trai os irmãos
repetidas vezes, é mentiroso, é dilapidador da natureza, não gosta de
trabalhar -- o que poderia ser libertário --, mas não deixa de gostar de
dinheiro e de seus prazeres. É medroso, desleal, mentiroso, injusto,
ganancioso. É o contrário do herói de cavalaria que ele parodiza,
segundo Gilda Mello, não tem nobreza, coragem, lealdade, verdade,
justiça e desprendimento. Mesmo sua saga, sua busca mítica pela
muiraquitã se dá não pela comunidade das Icamiabas, mas por uma questão
individual sua, por causa da ``saudade da danada''.

E não é senão no fim da obra que a síntese da subjetividade brasileira
antropófaga se depara com a dificuldade de viver no mundo sem ser nem
índio, nem branco, mas como mistura que beira à indefinição, pois
Macunaíma não é mais um ser originário, é já fruto de algumas
assimilações, de algumas \emph{deglutições} do espírito europeu. Ele se
encontra num limbo em termos subjetivos, mas isso não o impede de
mover"-se aparentemente bem na cidade, de vestir a seu modo a máscara de
caráter do sujeito burguês, não o de sua época, mas o que veio a se
desdobrar quase 100 anos depois de a lógica mercantil espalhar se manto
sobre o mundo.

Para Bosi, fica irresoluta a tensão fundadora da obra entre o ponto de
vista \emph{civilizado}, \emph{moderno} e \emph{racional} de um Mário
que compõe uma figura que vale como sátira das idealizações românticas,
mas também do ponto de vista \emph{arcaico}, \emph{primitivo},
\emph{mitopoético} de outro Mário que critica o progresso da cidade. No
entanto, parece que, diferentemente do que pensa Bosi, essa tensão chega
a uma sutil síntese, embora capenga: um sujeito moderno"-burguês,
antropófago e pau"-brasil, cheio de ginga, que vai se revelar, numa
leitura contemporânea, mais adaptado perante a vida social capitalista
do que o sujeito burguês tradicional, que não tem essa malemolência. Ele
se aproxima mais da forma de vida mercantil moderna pela sua proximidade
com os automatismos da natureza -- que também são automatismos da vida
de mercado, onde reina uma segunda natureza. \emph{Macunaíma} vive na
cidade com os mesmos automatismos das leis naturais com que vivia no
mato e consegue lidar bem com as parafernálias modernas enquanto esteve
por lá.

Perdido, por ter perdido seu talismã, seu Graal, como diria Gilda Mello
(2003), e ao mesmo tempo se dando bem em solo citadino, no fundo parece
``enjoado de viver'', num limbo -- nem mais índio, nem branco:
entrecruzado. Nem seu expressivo ritual de braqueamento, quando de sua
chegada a São Paulo, pode lhe dar um caráter, senão um de penduricalho.
Segundo Bosi (1988),

\begin{quote}
Para Macunaíma, nem a cidade representa uma saída para a selva, nem a
selva representa uma saída para a cidade. \emph{O sentido é de impasse;
e dor pelo impasse}. Nem é feliz o mundo amazônico (metonímia de um
vasto Brasil que vive fora da civilização moderna), porque nele os
embates contra as intempéries, as pragas e as pestes, os monstros e os
espíritos vingadores, podem levar à míngua e à morte; nem a vida urbana,
mal europeizada e já semi"-americanizada, consegue dar à nossa gente um
habitat acolhedor\ldots{} E por todo lado o que se vê é a máquina hostil, a
política rasteira e o dinheiro.'' (p. 180)
\end{quote}

Embora capte bem o impasse e a dor do impasse, o autor se mantém na
mesma linhagem de crítica do modernismo, mesmo tendo algumas décadas a
seu favor. Porque mesmo notando bem que Macunaíma parece infeliz na mata
e na metrópole -- o que aponta para um sujeito mal"-ajustado, para uma
tensão, embora não chegue a ser decisiva, já que o mais notável é um
Macunaíma que anda com desenvoltura por onde anda -- os motivos
apontados parecem frágeis: a dureza da natureza de um lado, e a vida
urbana ``mal europeizada e já semi"-americanizada'', do outro.

O decisivo parece ser que, mesmo capenga, por lhe terem comido uma perna
as piranhas, Macunaíma segue adiante na sociedade que foi obrigado a
abrigar, embora ``enjoado de tanto viver'', por não se achar. A metáfora
da Ursa Maior em que se transforma no final é a expressão do enjoo de
viver nesse limbo, um enjoo que já se manifestara no capítulo \versal{VIII} da
narrativa.

O epílogo ao final dá o tom conclusivo da aventura: derrota da saga de
Macunaíma, derrota da própria ideologia antropófaga, embora saia daí uma
subjetividade macunaímica para levar adiante a modernização capitalista
nacional, em vez de seu questionamento. Mas do herói e sua história,
resta a memória, a do intelectual, que recolhe a história de um
papagaio, único sobrevivente. Como pode daí sobrar brasilidade que se
possa opor à Europa? Desse desgarramento, desse mundo abismal, que é o
mundo de Macunaíma?

À primeira vista, no personagem de Macunaíma, pode"-se ver uma aporia de
subjetividade, por ser penetrado por aspectos tão díspares e impossíveis
de chegarem a uma síntese, e até mesmo por seu fim melancólico. Mas,
atentando bem, Macunaíma, mesmo em seu limbo, não sendo branco, nem mais
índio, representa essa síntese estranha de quem não é fruto de um
desenvolvimento em séculos da modernidade, mas que é obrigado a
degluti"-la de supetão para poder virar"-se bem na vida moderna, até o
enfado invadir"-lhe a existência.

Não tendo passado pelos séculos de amadurecimento das categorias
fundantes do capitalismo, o país sempre foi visto como atrasado do ponto
de vista dos centros capitalistas. Em pleno século \versal{XXI}, não estamos mais
atrasados. Não me refiro aqui à posição econômica, mas ao
desenvolvimento no Brasil de uma subjetividade vanguardista em termos de
segunda natureza, a mercantil. A forma"-sujeito burguesa entranhada agora
na subjetividade de todas as classes, com a difusão contínua da
\emph{sociedade do espetáculo} (\versal{DEBORD}, 1997) nas mãos de cada pessoa em
suas televisões, tábletes e celulares, pode se desenvolver até mesmo
onde a realidade concreta está mais para arcaica do que para burguesa.
Assim, ironicamente, as \emph{ideias entram no lugar}, o atraso da
estrutura sócio"-econômica não impede mais o florescimento de uma
subjetividade burguesa, porque o \emph{espetáculo} só precisa do
\emph{locus} abstrato da subjetividade para fazer morada. O
primitivismo, a espontaneidade, a não adaptação aos enquadramentos
sociais rígidos, a malandragem, a não repressão dos instintos, a
flexibilidade do eu, agora já não se opõem à subjetividade da sociedade
capitalista desenvolvida, antes são aspectos fundamentais de sua
componente narcísica. Sendo uma sociedade também primitiva, a sociedade
capitalista, ao se aproximar de seu conceito, engendra uma subjetividade
com essas características. Uma sociedade anômica, de sujeitos valendo
por si mesmos, que constituem uma comunidade mercantil com tendência a
ser regida apenas pelos desejos dos narcisos que se relacionam
socialmente por meio de seus eus inchados. Não se trata de narcisos cujo
delírio de onipotência foi tolhido bem ou mal pelo social. São narcisos
cujos eus pretendem ao máximo fazer um retorno impossível ao narcisismo
primário, quando não havia diferença entre seu \emph{Eu} e o
\emph{Mundo} (\versal{FREUD}, 2010b). O capitalismo desenvolvido é uma
comunidade em que os sujeitos não pretendem sair dessa menoridade. O
desejo é realizar"-se sobre os demais como acontece na vida natural, onde
não há diferença entre o animal e seu mundo, ambos naturais. Não
passando pelos séculos de amadurecimento as categorias mercantis,
passamos em certa medida de uma natureza a outra, mais elevada que a
primeira, mas não menos feroz, pois seus mecanismos destrutivos e
automáticos são incontroláveis e racionais. Se Macunaíma era no
Modernismo paulista e por tempos depois o símbolo da mentalidade
antiburguesa, o desenvolvimento da sociedade burguesa levou exatamente à
fusão natureza"-cultura que são o mundo de Macunaíma, apesar do desejo do
autor.

Na obra, a espontaneidade, o primitivismo, as relações pouco mediadas
por fatores sociais, o elogio dos instintos naturais, a certa
dificuldade de as relações passarem pela autoridade de uma instituição
supraindividual poderia até ser positivo, se não fosse ao encontro de um
estágio ainda mais primitivo de natureza, o momento da realização da
forma social mercantil e da forma"-sujeito burguesa. Por mais elevada que
pretenda ser a sociedade moderna, com o desabrochar da inventividade
jamais visto, ao se desenvolver e amadurecer, mostra em suas entranhas
mesmas o quanto ela somente é a mais desenvolvida das sociedades
primitivas que a pré-história humana já conheceu.

\section*{Entre Ulisses e Leonardo ou a dialética da~forma"-sujeito burguesa malandra}
\addcontentsline{toc}{section}{Entre Ulisses e Leonardo ou a dialética da forma"-sujeito burguesa malandra
\medskip}

O capitalismo venceu a malandragem. Não na forma da repressão dos anos
1930, quando as músicas eram censuradas pela apologia do \emph{dolce far
niente}, mas na da assimilação -- como é típico de uma forma de
organização social que tende a se entranhar em todos os recônditos da
vida social e subjetiva. Se agora o capitalismo se impõe muito mais pela
sedução e pelo gozo nas mercadorias, e menos pela repressão pura e
simples, tudo aquilo que outrora parecia ser um ataque direto à forma de
vida burguesa acaba sendo incorporado como estilo de vida entre os
muitos disponíveis na democracia moderna que representa o poder do povo
numa comunidade mercantil.

A busca pelo instinto nacional, que se manifesta também numa certa
ideologia da especificidade brasileira, ainda se mostra patente meio
século depois do modernismo paulista na \emph{Dialética da malandragem}
(1970) de Antônio Cândido -- no mais um texto extremamente rico. O
ensaio de Antônio Cândido é uma leitura do romance \emph{Memórias de um
sargento de milícias} (1854) de Manuel Antônio de Almeida que ele chama
de \emph{romance malandro}, especificamente nacional. O espaço histórico
é o período joanino no Rio de Janeiro. Cândido lê o romance como
verdadeiro panorama das relações sociais e do mundo subjetivo do Brasil
na época, um mundo em que reinam relações personalizadas, em que os
expedientes estão na cabeça de todos os homens livres que precisam se
virar para sobreviver, que precisam do favor -- menos na dos negros que
não aparecem no romance. Como já vimos, estavam na quase totalidade
incrustados na estrutura fixa de submissão aos senhores.

Cândido lê as \emph{Memórias} entrevendo ali personagens sem caráter
fixo, pendulares, sempre oscilando entre a ordem e a desordem,
personagens para os quais os limites entre o bem e o mal, o lícito e o
ilícito são tênues: ``O remorso não existe, pois a avaliação das ações é
feita segundo sua eficácia'' e a repressão moral ``só pode existir fora
das consciências'', como questão de polícia, como repressão externa, e
``se concentra inteiramente no major Vidigal, cujo deslizamento cômico
para as esferas da transgressão acaba, no fim do romance, por baralhar
definitivamente a relação dos planos.'' (\versal{CÂNDIDO}, 1970, p. 85). No
romance, é descrita uma sociedade em que se consegue as coisas mais
facilmente pelo jeito, pela manha, do que pela lei.

Ainda pela leitura do autor, as \emph{Memórias} ``criam um universo que
parece liberto do peso do erro e do pecado.'' (\emph{Idem}, p. 84), um
mundo em que todos têm defeitos e ninguém merece qualquer censura.
Porque é uma sociedade fluida e flexível que foge à escolha dos pares
antitéticos impostos pelas sociedades rígidas entre o ``lícito ou
ilícito, verdadeiro ou falso, moral ou imoral, justo ou injusto,
esquerda ou direita política e assim por diante'' (\emph{Ibidem,} p.
84). Pela leitura de Cândido, nas \emph{Memórias,} bem e mal nunca
aparecem em sua inteireza, há sempre um balanceio em que um compensa o
outro. Nesse sentido, as \emph{Memórias} contrastariam com a literatura
de seu tempo, de personagens que vivenciam a repressão mutiladora da
personalidade:

\begin{quote}
Uma sociedade jovem, que procura disciplinar a irregularidade da sua
seiva para se equiparar às velhas sociedades que lhe servem de modelo,
desenvolve normalmente certos mecanismos ideais de contenção, que
aparecem em todos os setores. No campo jurídico, normas rígidas e
impecavelmente formuladas, criando a aparência e a ilusão de uma ordem
regular que não existe e que por isso mesmo constitui o alvo ideal. Em
literatura, gosto acentuado pelos símbolos repressivos, que parecem
domar a eclosão dos impulsos. É o que vemos, por exemplo, no sentimento
de conspurcação do amor, tão freqüente nos ultra"-românticos.
(\emph{Ibidem}, p. 85-86).
\end{quote}

Cândido opõe nesse sentido a constituição dos Estados Unidos e do
Brasil. Ali, desde o início, há uma presença coercitiva da lei. É uma
sociedade de eleitos, que pertencem à mesma lei, um endurecimento do
grupo que reforçaria a identidade e a resistência, mas desumanizaria a
relação com os outros tidos como fora do grupo. Já no Brasil, os
indivíduos nunca teriam estado em contato com tais formas, ``nunca
tiveram a obsessão da ordem senão como princípio abstrato, nem da
liberdade senão como capricho.'' A sociabilidade aqui sempre ganhou
formas espontâneas que atuaram como amortecimento ``dos choques entre a
norma e a conduta, tornando menos dramáticos os conflitos de
consciência.'' (\emph{Ibidem}, p. 86).

Assim, as \emph{Memórias} não se enquadrariam em ``nenhuma das
racionalizações ideológicas reinantes na literatura brasileira de então:
indianismo, nacionalismo, grandeza do sofrimento, redenção pela dor,
pompa do estilo etc.'' O romance seria uma vasta acomodação que desbasta
os extremos, ``tira o significado da lei e da ordem'', mostra uma
interpenetração de grupos, idéias e atitudes mais díspares, que criam
``uma espécie de terra"-de"-ninguém moral, onde a transgressão é apenas um
matiz na gama que vem da norma e vai ao crime.'' Numa frase, segundo
Cândido, o romance é um caso raro da literatura do século \versal{XIX} que não
``exprime uma visão de classe dominante''. (\emph{Ibidem}, p. 87).

O que sobressairia das \emph{Memórias} é uma irreverência popularesca
que se articularia com uma ``tolerância corrosiva'' tipicamente
brasileira e que se manifestaria na literatura ``sob a forma da piada
devastadora.'' Essa comicidade popular fugiria das esferas sancionadas
pelas normas burguesas e iria ao encontro de certa irreverência e
amoralidade presente em certas expressões populares e amainaria as
quinas, dando ``lugar a toda a sorte de acomodações (ou negações), que
por vezes nos fazem parecer inferiores ante uma visão estupidamente
nutrida de valores puritanos, como a das sociedades capitalistas; mas
que facilitará a nossa inserção num mundo eventualmente aberto. ``
(\emph{Ibidem}, p. 88).

É exatamente nesse ponto que a leitura de Cândido parece perder sua liga
-- pelo menos aos olhos críticos que a leem mais de quarenta anos depois
de sua escrita. Se à sua época poderia até parecer uma leitura avançada,
até por ainda se viver uma ditadura no país, hoje, essa leitura parece
bastante perecida. Toda a leitura de Cândido das \emph{Memórias} é
bastante original e rica em argumentos, mas perde todo seu vigor quando
sorrateiramente, saindo do universo do romance, do período joanino,
pretende fazer uma generalização antropológica\footnote{O aspecto da
  generalização antropológica, já notara Robert Schwarz em artigo de
  1979 intitulado \emph{Pressupostos, salvo engano, da '‘Dialética da
  malandragem’',} embora sem refletir em sua época, que é nosso esforço
  aqui, o quanto as formas de subjetividade tidas por alternativas à
  máscara de caráter da forma"-sujeito burguesa são, em verdade, a
  própria forma que toma a forma"-sujeito burguesa em seu desdobramento
  narcísico contemporâneo.} que coloca a figura do personagem Leonardo
como verdadeira alegoria positiva da subjetividade brasileira e a
malandragem como componente de uma pretensa natureza humana brasileira.
Assim como o modernismo paulista fundamentou uma pretensa alternativa à
civilização europeia pela antropofagia, pela valorização do que seria
bem próprio do Brasil, o primitivismo, a espontaneidade, a imunidade a
qualquer repressão do eu, a força dos instintos, por essa leitura de
Antônio Cândido, a malandragem era alçada ao aspecto que nos
diferenciaria e nos colocaria na frente perante um mundo aberto, perante
uma possível revolução.

Sendo assim, a análise de Antônio Cândido acerca das relações entre o
malandro -- tal qual formulou a partir das \emph{Memórias de um sargento
de milícias --} e o ambiente da formação da subjetividade brasileira nos
tempos coloniais não parece equivocada. Não pareceria tampouco
equivocado trazer certas dessas características para o contemporâneo. O
que parece equívoco é a generalização com teor apologético de um traço
que pretensamente resumiria o ser brasileiro e o colocaria em situação
privilegiada num possível nível superior de sociabilidade, graças à
fluidez das relações brasileiras que contrastaria com o rijo e
cristalizado dos países avançados.

Nesse sentido, podemos quase imaginar que a \emph{Dialética da
malandragem} é uma espécie de contraponto malemolente e pretensamente
superior, posto que nacional, à \emph{Dialética do Esclarecimento} de
Adorno e Horkheimer. Se nessa obra os autores refletem, a partir do
personagem \emph{Ulisses} da Odisseia de Homero, o domínio da natureza
interior como aspecto fundante da subjetividade burguesa, Antônio
Cândido em sua \emph{Dialética} avança uma alternativa a esse
enrijecimento do eu a partir do personagem Leonardo: a de que a
constituição subjetiva do brasileiro foge a esse domínio da natureza
interior e que por isso estaria mais adiante na superação da vida
burguesa. Mas, no fundo, o que se viu foi o desenvolvimento da
forma"-social burguesa gestar uma estranha fusão entre Ulisses e
Leonardo, em que Leonardo, em vez de contraponto nacional, vai vencendo
aos poucos Ulisses e sua dominação da natureza interna para se colocar
como o protótipo mais próximo de uma forma"-sujeito burguesa
desenvolvida, livre de qualquer amarra para se realizar como sujeito
mercantil e narcísico.

Se por um lado a visão de Antônio Cândido é generalizante, a crítica de
Akira Goto, que pretende ser demolidora em sua \emph{Malandragem
revisitada} (1988), peca pela restrição e o teor tradicional. Se Cândido
traça uma linha positivadora do caráter fluido, malemolente, tipicamente
brasileiro que teria alcançado no ``Modernismo as suas expressões
máximas, com Macunaíma'' (1970, p. 88), para Goto, tal generalização é
criticável, mas somente porque o verdadeiro malandro não são todos os
brasileiros, porque a ``a malandragem lucrativa tem sido a malandragem
da classe dominante'' (\versal{GOTO}, 1988, p. 97). Já aquele malandro
``associado ao brasileiro pobre que vive de expedientes para assegurar
sua precária liberdade diante da conscrição do mercado de trabalho'',
teria dado lugar ao ``malandro oficial'', ``com gravata e capital que
nunca se dá mal.'' (\emph{Idem}, p. 97).

Seguindo esse raciocínio, perdemos de vista a forma"-sujeito burguesa num
capitalismo desenvolvido contemporâneo e regredimos a uma visão que
projeta todos os males do mundo numa classe e idealiza a outra perdendo
de vista que o desenvolvimento da vida social burguesa impõe uma máscara
de caráter abstrata que todas as classes precisam vestir,
independentemente de sua posição perante a riqueza mercantil. Seguir uma
crítica da forma"-sujeito burguesa não deve significar fechar os olhos
para o fato de que existe uma camada social cuja malandragem é mais
rentável, que está mais bem posicionada perante a vida social mercantil,
e cujas manobras prejudicam uma camada mais mal posicionada. Deve
significar sobretudo uma crítica do fato de que as diferenças objetivas
e subjetivas que se poderiam notar entre o mundo popular -- como mundo
de uma integração imperfeita à ordem, de uma internalização imperfeita
das exigências burguesas -- e o mundo dito burguês foram diminuindo, ao
menos em termos da subjetividade. Eis as verdadeiras quinas que se foram
amainando historicamente e dando lugar ``a toda a sorte de acomodações
(ou negações)'', que agora, em vez de nos fazerem ``parecer inferiores
ante uma visão estupidamente nutrida de valores puritanos, como a das
sociedades capitalistas'', colocam"-nos em posição até superior,
facilitando nossa inserção, ``num mundo aberto'', que é o do capitalismo
contemporâneo de um extremo desabrochar do Eu regressivo, narcísico.

De modo que o desejo do malandro de baixo é normalmente ser um malandro
de cima, e não necessariamente criar um espírito coletivo que se oponha
a uma vida cujas entranhas são cada vez mais invadidas pelas coerções
mercantis. As coerções mercantis são vistas em geral como criticáveis
quando impedem o acesso ao gozo mercantil, não porque são uma forma
racionalmente irracional de produção tautológica de riquezas.

E neste ponto há que se concordar com Akira Goto: é necessário fazer uma
diferença histórica entre o malandro ou pícaro do século \versal{XVI} ao \versal{XIX}, e o
malandro do século \versal{XX}, sobretudo no século \versal{XXI}. Mas somente se for para
apreender, o que Goto não enfrenta, o quanto o elogio da plasticidade do
caráter brasileiro parece casar muito mais com a sociedade capitalista
posterior aos anos 1960, da flexibilidade e da liberdade ilimitada e
imediata -- dentro dos limites e da mediação da mercadoria. Assim, o
malandro no seu sentido mais puro já não se choca com a vida social
contemporânea em que o gozo mercantil tende a substituir o esforço
mercantil. Agora que o capitalismo se aproxima cada vez mais da
realização de sua utopia de sociedade -- uma sociedade livre de toda e
qualquer amarra no plano subjetivo, já que os laços em uma comunidade
mercantil são frouxos e aparentes, em que cada sujeito precisa muito
mais do outro para escorar o seu eu -- o malandro aparece como o sujeito
perfeito para uma inserção cheia de ginga nesse mundo ``aberto'', mas
que vive ao mesmo tempo uma crise profunda cuja aceitação só pode ser
feita com jogo de cintura para virar"-se cotidianamente.

O que temos visto é que a forma"-sujeito burguesa que se desdobra com a
forma social burguesa, capitalista, quanto mais se aproxima de seu
conceito, vai se despregando de todos aqueles enrijecimentos que tinham
sido necessários para sua fundação, num momento em que a subjetividade
burguesa ainda significava um entrecruzamento entre as formas de
subjetividade passadas e o tensionamento rumo ao progresso, ao
desvencilhamento de tudo que pudesse servir de amarras ao pleno
desabrochar da vida social burguesa e da forma"-sujeito burguesa. Se a
irreverência, o popular, a fluidez, a flexibilidade localizados por
Cândido no malandro podiam ser vistos como um ato de oposição ao mundo
burguês quando a subjetividade burguesa arrastava o peso do
enrijecimento secular necessário para fundar o sujeito que vale por si
mesmo na comunidade mercantil, com o desdobrar"-se da forma social
mercantil, em que reina o imperativo do \emph{Eu}, do gozar a qualquer
preço, da realização do \emph{Eu} nas mercadorias também elas portadoras
de subjetividade, essa forma de oposição já não o é. Porque faz parte da
realização do conceito de capitalismo a realização contínua da sociedade
mais livre que já existiu, com a condição de que a liberdade total fique
aprisionada dentro dos imperativos das leis mercantis.

\chapter*{O homem sem qualidades à~espera~de~Godot}
\addcontentsline{toc}{chapter}{\large\versal{O HOMEM SEM QUALIDADES\\ À ESPERA DE GODOT}}
\hedramarkboth{O homem sem qualidades à espera de Godot}{}

\begin{flushright}
\scriptsize{\emph{Ad maiorem gloriam mercibus.}}
\end{flushright}

Talvez pudéssemos dizer que Beckett apreendeu, sem necessariamente ter
sido seu desejo estético, ainda no século \versal{XX}, num contexto muito diverso
do de Musil, o que vivemos contemporaneamente, e que, na época em que
escreveu, ainda aparecia como entrecruzamento de subjetividades. Nosso
propósito nesse capítulo é discorrer sobre o século \versal{XX} pós o \emph{homem
sem qualidades} Ulrich. Ou seja, pós a barbárie cientificamente levada a
cabo na Segunda Guerra. De que modo a mercadoria passou a dominar no
século \versal{XX} para que possamos dizer que no século \versal{XXI} estamos mais
próximos da realização da forma"-sujeito moderna, a forma do vazio da
subjetividade? Podemos, portanto, dizer que tentaremos refletir sobre a
afirmação de Lipovestky de que a realização plena do indivíduo coincide
com sua dessubstancialização. Embora saibamos que esse autor tem
tonalidades mais apologéticas que críticas, parece"-nos uma sentença que
aponta para dois sentidos que estamos buscando aqui. Primeiro, a via do
desdobramento de uma forma"-sujeito na história moderna, a forma do
Indivíduo moderno que vai ganhando sempre mais terreno sobre as
individualidades concretas. Segundo, a via de que esse desdobramento
paulatino vai coincidindo com seu esvaziamento, que é sua realização, ao
mesmo tempo que sua crise. Em suma, a realização plena da forma"-sujeito
burguesa coincide com sua dessubstancialização, com sua crise. Essa
ideia é concatenada com aquilo que estamos desenvolvendo como
devir"-sujeito"-dessubstanciado desde o começo como de fato um devir,
nunca um dado absoluto. Nem mesmo agora esse vazio está realizado.
Estamos refletindo sobre o desenvolvimento moderno como um processo
paulatino de aproximação desse vazio. Portanto, aqui nesse momento de
nossa reflexão, trata"-se primeiramente da passagem da moral ascética
para a moral pretensamente hedonista de um mundo liberal -- que, por
vezes, confunde"-se com libertário. Dito de outro modo, trata"-se da
passagem da sociedade da repressão aberta para a repressão pela sedução,
pela aceitação concordante, embora a sociedade não abra mão da violência
aberta.

Para chegar a esse esvaziamento que expressam as peças de Beckett sobre
a qual refletiremos, retomaremos o fetichismo da mercadoria sob bases
também contemporâneas, ou seja, entendendo a mercadoria como portadora
também de subjetividade, o que significa uma espécie de
\emph{mais"-fetichismo,} ou \emph{duplo fetichismo}, como conceituou
Severiano (2007). Teremos também, visto que falamos de
desdobramento de uma forma"-sujeito, que falar do desdobramento da lógica
mercantil moderna. Nesse sentido, não é só a forma"-sujeito burguesa que
se realiza coincidindo com sua crise. Também a forma"-social burguesa se
realiza coincidindo com sua crise, não como crise cíclica, mas como uma
crise intrínseca à lógica mercantil. Ou seja, quanto mais o capitalismo
se realiza, mais encontra sua lógica de exaurimento de si mesmo, da vida
social que ele instituiu e da própria subjetividade. Esse rodar em
falso, esse vazio, essa crise, esse empanturramento que limita os
movimentos de todo gênero, em suma, esse impasse é o que está em
Beckett. Não é mais a ruína da guerra real, é a ruína da vida social.
Mas são ruínas que se mostram como uma barbárie em fogo baixo, não
necessariamente com grandes catástrofes, mas com um aviltamento
cotidiano dos corações. Bem como com uma violência de gestão de crise
por parte dos gestores de plantão. ``É isso que eu acho maravilhoso.
Como o homem se adapta. Às condições que mudam'' (\versal{BECKETT}, 2010b, p.
46), diz a personagem Winnie em \emph{Dias Felizes.} Nosso intento é,
portanto, localizar a tautologia do vazio que preenche a \emph{espera de
Godot}, os \emph{dias felizes} e o \emph{fim de partida} como algo de
nossa época e não como um problema existencial transhistórico. Enfim,
por que o homem contemporâneo espera Godot, nos destroços do \emph{Fim
de Partida}, embora pense viver \emph{Dias Felizes}? Por que ele é um
homem sem qualidades?

\section*{O fetichismo da mercadoria em Marx~e~seus~desenvolvimentos}
\addcontentsline{toc}{section}{O fetichismo da mercadoria em Marx e seus\\desenvolvimentos}

Podemos dizer que o desenvolvimento de nosso raciocínio teve como fio
condutor o nosso entendimento do conceito de fetichismo da mercadoria em
termos de Marx. Tentamos, assim, deixar claro que o fetichismo da
mercadoria em Marx não é o fetichismo da coisa, de um objeto qualquer. O
fato de entender as mercadorias com as quais nos relacionamos, a todo
momento, na vida social capitalista não como meras coisas é fundamental
para que se possa mostrar fundamentado nosso intento de sustentar uma
ideia de desdobramento da forma de subjetivação moderna -- ao modo de
uma máscara de caráter, uma forma"-sujeito moderna -- juntamente com o
desdobramento da própria objetividade capitalista. Significa que o
capitalismo é por nós pintado como a única sociedade a conter de fato
uma verdadeira teleologia da história, sem que seja positiva. Do mesmo
passo, o capitalismo é visto como uma sociedade que tem uma dinâmica
interna, como um organismo \emph{vivo --} embora seu núcleo seja sem
vida --, que não segue adiante somente porque seus sujeitos o conduzem,
mas porque ele impõe ser levado adiante pelos sujeitos, sob pena de toda
a vida social desabar -- de tão fincada se mostra no movimento do
dinheiro.

Embora já tenhamos explicitado qual nossa compreensão do conceito de
fetichismo da mercadoria de Marx, cremos ser importante retomar seus
aspectos principais para relacioná-lo com o desdobramento na
contemporaneidade. Pelo que expusemos, notamos a diferença em relação à
compreensão tradicional desse conceito, ao qual nunca foi dada muita
atenção por ser julgado ``muito filosófico'' e sem consequência para o
proletariado que deveria compreender o mais importante, a mais valia, a
parte de seu trabalho não pago. Embora ambíguo, Marx fala do fetichismo
como dominação sobre ``os homens'' (1985, p. 76), e não sobre uma
determinada classe.

Já deixamos claro, portanto, que, diferentemente dos fetiches primitivos
ou anteriores à modernidade, que eram estáticos, o fetiche moderno é
dinâmico, nunca pode parar no mesmo ponto, sempre precisa encontrar um
novo ponto de expansão, numa lógica sem limites, que sempre tende a
romper barreiras, tanto do ponto de vista da objetividade, quanto do
ponto de vista da própria subjetividade. E tal qual o Esclarecimento
analisado por Adorno e Horkheimer (1986), ele é a vitória da forma sobre
o conteúdo, do abstrato sobre o concreto sensível. Seguindo Menegat, o
concreto na sociedade moderna ``é o resultado da imposição da forma
valor em sua necessidade de materialização, e não a produção de um valor
de uso sequestrado pelo capital ou desviado da realização de uma
essência humana.'' (2011, p. 5).

Desse modo, existe uma forte relação entre o movimento do fetichismo da
mercadoria e o que Debord conceituou como espetáculo, momento em que a
mercadoria se desenvolve e passa a dominar tudo o que é vivido:

\begin{quote}
Por esse movimento essencial do espetáculo que consiste em retomar nele
tudo o que existia na atividade humana em estado fluido para possuí-lo
em estado coagulado, como coisas que se tornaram o valor exclusivo em
virtude da formulação pelo avesso do valor vivido, é que reconhecemos
nossa velha inimiga, a qual sabe tão bem, à primeira vista, mostrar"-se
como algo trivial e fácil de compreender, mesmo sendo tão complexa e
cheia de sutilezas metafísicas, a mercadoria. (\versal{DEBORD}, 1997, p. 27)
\end{quote}

Assim, poderíamos afirmar que a sociedade do espetáculo é o momento em
que o fetichismo da mercadoria chega a um alto grau de desenvolvimento,
em que a mercadoria adquire um ponto tal de autodeterminação que
consegue construir um mundo à parte por meio de imagens idealizadas.
Citando Kehl, ``na sociedade do espetáculo, as imagens, em sua forma
mercadoria, é que organizam prioritariamente as condições do laço
social''. (2009, p. 93).

Ainda uma vez é preciso sublinhar que o problema no conceito de
fetichismo da mercadoria em Marx não consiste no fato de a mercadoria
encantar e ser venerada em termos das vitrines, no fato de ela
contemporaneamente portar aspectos subjetivos; esse ``encanto'', esse
\emph{sex}-\emph{appeal do inorgânico,} como veremos, é um outro estágio
de fetichismo mais ligado à aparência das mercadorias no mercado. O que
``fascina'' na mercadoria, no nível de Marx, é aquilo que está escondido
nela, que é tido por evidência axiomática, embora seja uma projeção
social nossa: quantidade de trabalho abstrato transformável em dinheiro.
É essa substância abstrata, bem moderna, de onde o mundo moderno haure
sua base de existência primordial.

E nós já sabemos o que a modernização causou aos vínculos sociais
baseados em tradições locais (\versal{MARX}, 1985; \versal{CROSBY}, 1999, \versal{DUFOUR}, 2005;
\versal{JAPPE}, 2006; \versal{KURZ}, 2010), no universo simbólico religioso, que foram
corroídos em proveito de vínculos mediados pela mão invisível típica do
universo simbólico mercantil. Uma mudança de universo que não aconteceu
sem criar um transtorno psíquico nos indivíduos, sobretudo naqueles que
tinham todo o seu mundo transtornado pela marcha mercantil. O transtorno
traumático não significava apenas a expropriação direta e indireta das
formas de sobreviver, mas também a entrada na marcha moderna, sem poder
mais se reportar a qualquer transcendência como fundadora, pois o céu,
de certo modo, precisava ser esvaziado, para que o sujeito moderno
nascesse e desse a impressão aos indivíduos concretos que vestir sua
máscara equivalia a ser ``livre'' --- de amarras pessoais e religiosas,
principalmente --- para encarar os outros como mercadoria concorrente
(\versal{JAPPE}, 2006; \versal{KURZ}, 2010). E a única coisa que pretende mediar essa
concorrência é a mão invisível do mercado.

Se a lógica descrita acima teve um início avassalador, tão totalitária
quanto a Razão Instrumental (\versal{ADORNO} \& \versal{HORKHEIMER}, 1986), submetendo as
matérias diversas à calculabilidade, transformando, como explicitamos,
tudo que já existia em mercadoria, ela não tardou a mostrar que não se
satisfazia facilmente: o abstrato tende sempre ao infinito e a esgarçar
o concreto à sua mercê.

Depois de avançar sobre a vida social e a subjetividade anteriores feito
fogo que avança numa floresta, a marcha moderna adentra pelo século \versal{XX}
encarando"-o como um arco do triunfo. Se a primeira grande guerra
demonstrava o ar de irracionalismo que pairava por trás da Razão
moderna, a crise de 29 representou de certo modo a primeira vez em que o
sistema produtor de mercadorias esbarrou contraditoriamente em sua
própria força: uma enorme e crescente produtividade linear não ia ao
encontro de pessoas ``aptas'' para consumir também crescentemente: ``A
produção maciça de mercadorias em abundância sempre crescente requeria,
agora, um mercado também maciço para absorvê-las, sob o risco de um
colapso geral[!] do sistema''. (\versal{SEVERIANO}, 2007, p. 69).

Ora, a subjetividade daquela geração que viveu a primeira metade do
século \versal{XX}, em geral ainda mantinha relação forte com o uso dos objetos,
resquício não corroído da própria tradição anterior à vida social
mercantil, ou seja, era uma subjetividade entrecruzada de aspectos
``atrasados'' e aspectos ``progressistas'', o próprio sujeito
protestante, embora em sintonia com o desenvolvimento das relações
capitalistas, não era a forma"-sujeito em estado puro, ainda continha
muitas das amarras da tradição que ele tentava conjugar com o
desdobrar"-se de uma vida mercantil sem peias. Além do mais, mesmo os
objetos já sendo mercadorias e o contexto sendo capitalista, as pessoas
geralmente consumiam aquilo de que precisavam. Até essa época, como
explicita Robert Kurz (2002), havia uma não"-simultaneidade interna e
externa do capitalismo. O capitalismo ainda era largamente misturado
objetiva e subjetivamente a resquícios pré-modernos. Quer dizer, tanto o
sistema não estava desenvolvido por igual dentro dos próprios países,
quanto a discrepância em relação ao desenvolvimento entre os países era
considerável. O que significa ainda muito espaço por onde a marcha
moderna poderia avançar.

Desse modo, para a lógica mercantil moderna seguir adiante, era preciso
agir tanto objetivamente quanto subjetivamente\footnote{Essa afirmação
  em verdade não passa de uma lapalissada, uma vez que não encarar o
  capitalismo como sistema econômico, mas como vida social já significa
  entendê-lo como sendo uma forma de organização social cuja dinâmica
  pressupõe a formação de uma subjetividade. Dito de outro modo, o
  capitalismo entendido como vida social mercantil é uma forma social
  total.}. Ou seja, objetivamente era necessário fazer com que a
mercadoria chegasse a toda parte; subjetivamente, era imperativo não um
novo sujeito, mas desenvolver a forma"-sujeito burguesa de fato, fazer
com que ela se despregasse das amarras de passado atrasadas e
``reacionárias'', e de seus restos de pensamento, ou de ``melindre'' de
tradição. Não se trata necessariamente de uma ação orquestrada pelos
capitalistas do mundo, mas muito mais das consequências de se colocar a
mercadoria com sua dinâmica no centro da vida social. Desde que nos
vestimos da máscara de caráter do sujeito burguês, passamos a querer
subitamente resolver os problemas que a mercadoria se coloca com sua
lógica sem limites. De modo que o amadurecimento da forma"-social e da
forma"-sujeito burguesas foi levado a cabo pelos indivíduos reais, na sua
dialética cotidiana sob exigência impessoal de um imperativo de
desenvolvimento da forma mercantil, dentro da qual parece estar contido
o desenvolvimento em termos humanos.

Se a forma"-social e a forma"-sujeito puderam amadurecer ou desdobrar"-se é
porque ainda havia largo terreno na vida social e subjetiva a ser
conquistado pela mercadoria, que ainda não tinha mostrado tudo que pode
fazer com uma cidade ou com a subjetividade\footnote{Diz Guy Debord, em
  relação à Paris de antes das mudanças urbanas da segunda metade do
  século \versal{XX}, em seu filme \emph{In girum imus nocte et consumimur igni}:
  ``A mercadoria moderna ainda não tinha vindo nos mostrar tudo o que se
  pode fazer de uma rua''. E ainda: Paris era ``uma cidade que era então
  tão linda que muita gente preferiu nela morar sendo pobre a viver rico
  alhures.''}. Assim, não foi uma escolha dos conspiradores
capitalistas, mas a própria lógica mercantil que criou uma nova
estrutura de consumo, agora ``massivo''. Nesse novo momento do
desenvolvimento da vida social mercantil, a forma"-sujeito burguesa
começa a não exigir apenas que os indivíduos se unidimensionalizem no
trabalho, mas que eles desenvolvam e descubram um mundo de novas
potencialidades por meio das mercadorias. Esse amadurecimento da marcha
moderna abre margem para uma nova etapa da lógica mercantil em que são
integrados amplos setores da população antes marginalizados do consumo
de mercadorias. ``Essa sociedade desponta nas décadas de 20 e 30 nos
Estados Unidos, mas somente alcança difusão generalizada nos países
desenvolvidos, a partir da segunda metade do século \versal{XX}, após a Segunda
Guerra mundial'' (\versal{SEVERIANO}, 2007, p. 71). É o início do chamado Estado
do bem"-estar social, com pleno desenvolvimento da economia e grandes
conquistas sociais e salariais por parte dos trabalhadores que se
integravam cada vez mais na sociedade a que eram pretensamente
predestinados a abolir -- pela teleologia marxista. Dito de outro modo,
esse é um momento em que a forma"-sujeito burguesa começa a mostrar"-se em
seu conteúdo, ou seja, como uma forma de subjetividade abstrata perante
a qual pode até haver diferenças materiais entre os indivíduos, mas não
diferença em termos subjetivos. Ou que significa dizer um embaçamento
daquilo que fazia a diferença entre as classes sociais, que pareciam
viver esferas distintas, universos distintos e culturas distintas. Era
como se a classe dos despossuídos guardassem na memória o trauma de
várias gerações que foram forçadas a entrar na marcha moderna, onde não
possuíam nada mais para sobreviver senão uma quantidade abstrata de
cérebros, nervos, músculos e sentidos que poderiam se materializar em
qualquer mercadoria nas indústrias. A era burguesa do consumo é também a
era do esquecimento desse trauma primordial, em que todas as classes se
igualam perante a riqueza mercantil. O trauma passa a ser não participar
da divisão da riqueza mercantil e não mais o fato de que sua produção
causa um transtorno na vida social. Mas foram necessárias várias
gerações para que esse movimento de esquecimento do trauma se
processasse.

A crescente entronização da mercadoria (não de simples objetos) na vida
social, que ganhou o belo nome de \emph{Estado do bem"-estar social --}
denominação que mostra sem rodeios o atrelamento entre
``desenvolvimento'' e lógica mercantil -- trouxe consequências
subjetivas de importância digna de nota: o sujeito analisado por Weber
(1964-2010), norteado pela ética protestante, ascético, dedicado ao
trabalho árduo e sistemático, um sujeito de um grande domínio de si, um
sujeito que recusa a desmedida, o luxo e o gozo dos bens --- que devem
ser acumulados --- começava a ser destronado. Ou seja, aquele estágio de
desdobramento da forma"-sujeito burguesa ainda se mostrava em larga
medida misturado com o peso do mundo passado, com a ideia da avareza
cuja filosofia parecer:

\begin{quote}
ser o ideal do homem de honra que tem não só seu crédito reconhecido,
mas também a ideia de que o dever de todos é aumentar seu capital, isso
sendo concebido como um fim em si. Com efeito, não é simplesmente uma
forma de traçar seu caminho no mundo que se prega aqui, mas uma ética
particular. Violar as regras não somente é insensato, mas deve ser
tratado como uma espécie de esquecimento do dever. (p. 47)
\end{quote}

Como vimos, esse \emph{ethos,} essa forma de estar no mundo dominou
grande parte do desenvolvimento do capitalismo, quando o acúmulo de
capital não era possível sem sacrifício e violência. Mas a crise de 29
tinha mostrado que essa moral poderia ir de encontro ao desenvolvimento
do próprio capitalismo ao qual pretendia servir em princípio. Dito de
outro modo, o destronamento do sujeito ascético de Weber aponta para
esse devir da subjetividade na modernidade. Não se pode falar de um
corte na estrutura básica da forma"-sujeito, pois essa mudança deve ser
entendida muito mais como momento lógico de desenvolvimento da
forma"-sujeito, um momento de sua dinâmica. Nesse sentido, não existe
morte do sujeito, mas realização paulatina do sujeito. A forma"-sujeito é
a forma universal de subjetividade para agir no invólucro da identidade
com a forma"-social burguesa. Como vimos, em princípio, os indivíduos
concretos, particulares, não são idênticos a essa forma. Eles sempre têm
uma margem de manobra maior ou menor, o que é índice também dessa
individualidade. Mas essa parte não idêntica à forma"-sujeito se torna
cada vez mais um espécie da lado obscuro que ajuda a forma"-sujeito em
seu reinado, em vez de ser uma resistência a essa forma subjetiva
impositiva.

É nesse sentido que se observa, sob a capa de uma pretensa ruptura com a
moral burguesa, a exigência do gozo tensionar com as exigências
ascéticas, de sacrifício, de adiamento das satisfações. Não se pode
evidentemente falar que o sistema capitalista simplesmente adaptou a
subjetividade à base de um automatismo, que felizmente não existe. Mais
uma vez, o destronamento do aspecto ascéstico da forma"-sujeito se deu
por tensionamentos, num processo paulatino, mas constante. Naquele
momento pós"-29, essa economia psíquica fundada no sacrifício começava a
ver surgir (\versal{LEBRUN}, 2004; \versal{MELMAN}, 2009), e a ela se opor, uma outra,
mais adequada à falta de limites da lógica da mercadoria, mas que apenas
ganharia muita força nas últimas décadas do século \versal{XX}.

Se antes a moral recomendava ao sujeito o comedimento, o acúmulo do
excedente, uma nova impõe o consumo do agora e até do futuro, com a
difusão do crédito, que nada mais é do que o consumo do futuro no
presente. Essa mudança subjetiva já tinha, há algum tempo, marginalmente
aparecido nos primórdios da modernidade sem que conseguisse ainda
terreno fértil, porque o tensionamento da época apontava mais para um
arraigamento na tradição do que uma desprendimento rumo ao futuro. Quero
dizer que já no século \versal{XVIII} começou a aparecer essa ideia de que os
apetites insaciáveis do ser humano, antes condenados como pecado,
infortúnio pessoal e fonte de instabilidade social, eram o combustível
da máquina econômica. Significa dizer que:

\begin{quote}
As atitudes, os caracteres e os comportamentos considerados
repreensíveis em nível individual (tais como a cobiça, o gosto pelo
luxo, um ritmo de vida dispendioso, a libertinagem\ldots{}) estão para a
coletividade na origem da prosperidade geral e favorecem o
desenvolvimento das artes e das ciências. (\versal{DUFOUR}, 2008, p. 261)
\end{quote}

O inglês Bernard de Mandeville parece um dos menos sutis em apresentar
ainda no século \versal{XVIII} bem mais do que uma ideologia liberal: ele
apresenta uma verdadeira moral do desejo que somente 2 séculos depois
teria terreno subjetivo para se realizar. Evidentemente, ele defendia a
liberação do vícios humanos porque sabia que isso não ia de encontro à
vida mercantil. O que aparece em \emph{Uma} \emph{investigação sobre a
origem da virtude moral} é a ideia de que se o homem é ``um conjunto de
paixões, as quais o governam'', paixões estas que, se provocadas, ``vêm
à tona'' (2013, p. 87), que se explore esse terreno para o bem da
economia. Essa visão está expressa de forma explícita na Fábula das
abelhas de Bernard de Mandeville, cuja moral já está expressa no título:
os vícios privados fazem a felicidade pública.

Mas as moral de Bernard de Mandeville precisou esperar dois séculos para
encontrar terreno fértil, para casar com um momento em que era preciso
desenvolver --- para dar conta da dinâmica da lógica da mercadoria ---
uma subjetividade que não julgasse os excessos --- ligados
principalmente ao ato de mercado --- algo condenável, mas antes como
conduta normal, signo de virtude, e naturalmente humana. Christopher
Lasch (2006) explica que, para os moralistas do século \versal{XVIII} ---
guardadas as devidas diferenças de desfaçatez mercantil entre Bernard
Mandeville David Hume e Adam Smith ---, era exatamente o caráter
auto"-reprodutivo, portanto, ilimitado das novas expectativas, das novas
necessidades e gostos que:

\begin{quote}
Favorecia a emergência de uma sociedade capaz de manter uma expansão
infinita [\ldots{}] A inveja, o orgulho e a ambição empurravam os seres
humanos a querer mais do que aquilo de que tinham necessidade, no
entanto, esses ``vícios pessoais'', ao estimularem a indústria e a
inventividade se tornavam ``virtudes públicas''. A ideia de poupar e o
esquecimento de si, contrariamente, significavam a estagnação econômica.
(p. 62-63).
\end{quote}

É Lasch que também ressalta a lucidez de Hume quanto ao perigo dessa
filosofia da superabundância de mercadorias --- quando se estava longe
do que vivenciamos hoje. Ele já refletia que poderia significar um golpe
na tendência geral que era a de um adiamento da satisfação (p. 69). Não
poderíamos dizer que essas características prefiguradas por esses
\emph{amoralistas} econômicos se coaduna com a máscara de caráter da
forma"-sujeito moderna? Não com o burguês ``clássico'', ainda não
desenvolvido, mas com o burguês que, mais do que uma classe, é um modo
de estar no mundo, uma forma de subjetivação que tensiona com qualquer
individuação concreta que não esteja tencionada a aderir ao \emph{homo
homini lupus est.}

Com essa nova moral, a forma"-sujeito dava um salto importante em seu
desdobrar"-se, pois acrescentava, além da máscara de trabalhador
disciplinado, uma nova exigência aos indivíduos aos quais ela pretendia
aferrar sua máscara de caráter: a exigência do consumo, do gozo nas
mercadorias, a liberação das paixões, uma segunda forma de entronização
da mercadoria que também se torna ingrediente da constituição subjetiva.

Sempre somos obrigados a lembrar nossa visão de que isso não se deu de
modo automático, como por um anúncio de chamada para \emph{recall de
subjetividades.} Com toda evidência, o sistema não chamou sua
forma"-sujeito para um \emph{recall.} Abstratamente, o sistema gostaria
de proceder assim, seria mais simples. Mas como as relações sociais e a
própria subjetividade tem um quê de concreto, foi preciso um processo
paulatino, mas sempre contínuo de corrosão dos rastros pesados do
passado. E não foi pela escolha consciente dos capitalistas. Eles
puseram em prática cotidianamente uma lógica que simplesmente é tida por
óbvia. Do mesmo modo, os ditos ``não"-capitalistas'' nem sempre foram um
polo negativo e de fricção, e também agiram na direção de entronizar as
exigência capitalistas. O desenvolvimento de uma grande área da
criatividade de publicidade para a chamada ``captura da subjetividade''
não deve ser vista como uma conspiração capitalista. Trata"-se da
forma"-sujeito burguesa resolvendo os problemas da mercadoria no terreno
da concorrência, característica fundamental do capitalismo que se aguça
com sua crise. Além disso, mais uma vez, quando a vida social mediada
pela mercadoria é tida por óbvia pelos indivíduos, nada mais normal que
trabalhem consciente e inconscientemente por ela como a coisa mais
prosaica do mundo.

Nesse processo, jogou papel fundamental toda a indústria cultural, todo
o aparelho de informação que foi também se desenvolvendo. A crítica dos
frankfurtianos à Indústria Cultural trouxe à tona, com efeito, muito do
que ainda estava em germe, ou muito do que ainda não se mostrava
enquanto fenômeno totalizante no mundo. Para essa crítica, na sociedade
capitalista contemporânea, todas as produções do espírito nas artes, na
literatura, no teatro, no cinema, etc., tornaram"-se integralmente
mercadorias voltadas inteiramente para produzir lucro (\versal{SOARES}, 2007).
Adorno e Horkheimer, ao escreverem a \emph{Dialética do Esclarecimento},
evitaram usar o termo cultura de massa em proveito do termo Indústria
Cultural por eles cunhado. A opção por esse oxímoro --- cultura e
indústria eram termos que não casavam na época --- deve"-se ao fato de o
termo cultura de massa pressupor que a cultura provém das massas,
exatamente o contrário do que querem sublinhar: a cultura passara a
fazer parte de todo um sistema cujo fim último é ``o poder absoluto do
capital'' (\versal{ADORNO} \& \versal{HORKHEIMER}, 1986, p. 113), restando pouco ou nenhum
espaço para a ``produção independente''. Como diz Hannah Arendt no que
ela chama de \emph{crise da cultura:}

\begin{quote}
Talvez a principal diferença entre a sociedade e a sociedade de massas
esteja em que a sociedade sentia necessidade de cultura, valorizava e
desvalorizava objetos culturais ao transformá-los em mercadorias e usava
e abusava deles em proveito de seus fins mesquinhos, porém não os
consumia. [\ldots{}] A sociedade de massas ao contrário, não precisa
de cultura, mas de diversão, e os produtos oferecidos pela indústria de
diversões são, com efeito, consumidos pela sociedade exatamente como
quaisquer outros bens de consumo. (2009, p. 257.)
\end{quote}

Fica claro que Arendt chama de sociedade o período anterior ao século
\versal{XX}. Para ela, a crise da cultura tem direta relação com uma sociedade
cujos objetos produzidos ``são tratados como meras funções para o
processo vital da sociedade, como se aí estivessem somente para
satisfazer a alguma necessidade. [\ldots{}] (p. 261). Como alertam
Adorno e Horkheimer (1986), com a indústria cultural, nada mais tem
direito de existir sem aparecer na publicidade: ``a publicidade é, hoje
em dia, um princípio negativo, um dispositivo de bloqueio, tudo aquilo
que não traga seu sinete é economicamente suspeito.'' (p. 152). Segundo
Debord (1997, p. 17), na sociedade do espetáculo, ``o que é bom aparece
e o que aparece é bom''. E \emph{aparecer,} para \emph{parecer,}
jogando a questão do \emph{ser} para as calendas, tornou"-se tão
fundamental que as programações da indústria cultural como um todo
apenas servem para deixar o cérebro do espectador disponível: ``Isto é,
diverti"-lo, relaxá-lo, para prepará-lo entre dois anúncios. O que
vendemos à Coca"-cola é tempo de cérebro humano disponível. Nada é mais
difícil de obter do que essa disponibilidade.'' (\versal{FONNET}, 2003, apud
\versal{DUFOUR}, 2008, p. 33).

E essa disponibilização dos cérebros não é só uma façanha da televisão.
As tecnologias de comunicação, notadamente aquelas possibilitadas pela
internet, cumprem cada vez mais esse papel de disponibilização, ao mesmo
tempo que dão a impressão de que qualquer indivíduo pode ser
protagonista desse mundo -- que é ao mesmo tempo uma tentativa de
negação de qualquer real protagonismo. A internet somente criou um
universo novo para a indústria cultural, um universo onde os átomos
flutuantes podem ter seu terreno de pretensa individuação. Onde todos
podem ser heróis, artistas, críticos do mundo, enfim, onde finalmente a
sociedade de massas dá lugar a uma sociedade de indivíduos pretensamente
ricos de possibilidades que eles só precisam explorar. A pretensa
interatividade e protagonismo contemporâneos demonstram o quanto a
sociedade mercatnil crê em seu próprio enraizamento social. Assim como
em outros terrenos, o espectador não pode ter a necessidade, nem a ele
pode ser deixado o tempo, de um pensamento próprio, pois a mercadoria já
prescreve toda a reação possível: ``Toda ligação lógica que pressuponha
um esforço intelectual é escrupulosamente evitada'' (\versal{ADORNO} \&
\versal{HORKHEIMER}, 1986, p. 128).

É importante sublinhar que o sentido psíquico de que é investido o
objeto"-fetiche"-mercadoria pelo sujeito é mediado pelo social. Há uma
relação forte entre esse sentido psíquico e aquilo em relação a que o
sujeito moderno -- \emph{subjectum} -- se constitui. Não se pode dizer
que a indústria cultural impõe simplesmente sua manipulação aos
espíritos em tudo contrários a ela, já que ela vai ao encontro de um
terreno de fato existente na estrutura psíquica que é o da tendência
sempre à espreita de um regresso ao \emph{eu} onipotente típico do
narcisismo primário. No entanto, tampouco se pode simplesmente dizer que
ela apenas capta o que já é próprio da estrutura mental; essa
argumentação também é por demais reduzida. Se assim o fosse, não haveria
tanta publicidade dirigida às crianças, principalmente nos intervalos de
desenhos animados e nos canais de televisão ditos ``infantis''. Essas
publicidades aparentemente inocentes vão moldando o sentido psíquico que
esses objetos ganham na vida dos sujeitos desde cedo. A formação do
sujeito burguês começa desde a mais tenra infância, e em todos os lares
onde desde cedo se abre a porta para a mercadoria fazer sua pregação.

Não é preciso se surpreender se, conforme Russell Jacoby (2007), as
brincadeiras vêm se tornando cada vez mais privadas, se o principal
local de lazer vem mudando dos espaços públicos para o quarto. Sem
contar o fato de haver uma enorme quantidade de jogos que permitem às
crianças brincarem sozinhas, ou pelo menos em um quadro já determinado,
o que deixa, segundo Jacoby, pouco espaço para a imaginação:

\begin{quote}
Será possível que o tempo de brincadeiras desestruturadas que dão espaço
à imaginação tenha diminuído? O tédio não parece ter desaparecido,
entretanto, o ``tédio'', entendido como uma fantasiosa tarde de domingo
com nada a fazer, pode ser abreviado por uma troca de canais de
televisão ou por jogos de computadores. `O tédio é o pássaro dos sonhos
que faz eclodir o ovo da experiência', escreveu Walter Benjamin em 1936.
[\ldots{}] A expressão `estou entendiado', quando dita por uma criança,
não é mais um fato, mas uma acusação que significa: `entretenha"-me'''
.(\versal{JACOBY}, 2007, p. 58-59).
\end{quote}

Para este autor, o ``estar sem fazer nada'' passou a tornar"-se
inaceitável principalmente quando a diminuição das famílias e a ida ao
mercado das mães deixaram um vazio que foi preenchido pelos produtos
manufaturados como gibis, filmes, televisão, jogos eletrônicos,
computador\footnote{``Os gastos com publicidade na internet brasileira
  deverão crescer 91\% entre 2013 e 2017, segundo projeções da
  consultoria PwC, representando \versal{US}\$ 3,1 bilhões, uma taxa média anual
  de crescimento de 18,6\%.'' Segundo reportagem do jornal Folha de São
  Paulo de 05 de junho de 2013.}, que são responsáveis para deixar a
criança em excitação o máximo de tempo, fazendo algo o tempo todo, sem
que haja um espaço de ``tédio'' tenso que torne possível imaginar o
diferente daquilo que já lhe foi dado pronto, como um mundo à parte ---
sem de fato sê-lo --- que ela precisa viver.

Poderíamos dizer que esses produtos manufaturados compõem também as
agências extrafamiliares de que fala Marcuse (1955-2009, p. 96-97), com
as quais a família --- e mesmo a escola --- não teriam mais condições de
competir.

\begin{quote}
Agora as crianças `têm pronta a maior parte de seus sonhos'. Ou como
coloca Gary Cross em sua História dos Brinquedos [\ldots{}] Novos objetos
de brincar incorporaram o sonho de crescer rapidamente em um fantástico
mundo do consumo ou em uma esfera heróica do poder e do controle'.
(\versal{JACOBY}, 2007, p. 61).
\end{quote}

E a quantidade de publicidade dirigida ao público infantil, adolescente
ou adulto não é pequena. E há uma unidade. Elas são responsáveis por
criar um mundo em que não há limites para se alcançar tudo que se
queira, mesmo em detrimento do que quer que seja. Segundo Anselm Jappe
(2013), o capitalismo pós"-moderno representa a única sociedade na
história que foi capaz de promover uma \emph{infantilização} massiva de
seus membros e uma \emph{dessimbolização} em larga escala. Segundo ele,
agora tudo contribui para manter o ser humano em uma condição infantil:
da revista em quadrinho à televisão, da publicidade aos
\emph{vídeo}-\emph{games}\footnote{É preciso notar que muitos desses
  jogos também contribuem para uma adultização dessimbolizada, em todas
  as faixas etárias, no que diz respeito à banalização de imagens de
  violência.}, dos programas escolares ao esporte de massa:

\begin{quote}
Tudo contribui para a criação de um consumidor dócil e narcísico que vê
no mundo inteiro uma extensão de si mesmo, que vê o mundo como algo
governável com um clique de seu \emph{mouse}. A pressão contínua dos
\emph{mass média} e a eliminação contemporânea tanto da realidade quanto
da imaginação em proveito de uma chata reprodução do existente, a
``flexibilidade'' imposta em permanência aos indivíduos e o
desaparecimento das perspectivas tradicionais de sentido, a
desvalorização simultânea do que constituía outrora a maturidade das
pessoas e daquilo que fazia o encanto da infância, substituídas por uma
adolescência eterna e degradada: tudo isso produziu uma verdadeira
regressão humana em larga escala e que poderia muito bem ser chamada uma
barbárie quotidiana. (p. 222)
\end{quote}

Além disso, ele argumenta que a indústria da diversão --- da televisão
ao rock, do turismo à imprensa de celebridades --- joga um papel
importante de pacificação social e de criação de consenso. Ele nos dá a
saber um conceito bastante contemporâneo, mas nem tão discutido: o
``tittytainment''. Mas o que é isso?

Em 1995, segue ele, reuniu"-se em São Francisco o primeiro ``State of the
World Forum'' do qual participaram por volta de 500 pessoas dentre as
mais poderosas do mundo (entre outros participaram Gorbatchev, Bush
júnior, Thatcher, Bill Gates\ldots{}) para discutir sobre a seguinte questão:
o que fazer no futuro com os 80\% da população mundial que não serão
mais necessárias à produção de mercadorias? Um ex"-conselheiro do
presidente americano Jimmy Carter teria então proposto como solução o
que ele cunhou como ``tittytainment'':

\begin{quote}
Às populações ``supérfluas'', e potencialmente perigosas por causa de
sua frustração, será destinada uma mistura de comida suficiente e
diversão, uma mistura de \emph{entertainment} embrutecedor, para obter
um estado de letargia feliz semelhante à que sente um recém"-nascido que
mama no seio materno (\emph{tits} no jargão americano) (p. 218)
\end{quote}

Dito de outro modo, a estratégia, sem que se possa falar simplesmente em
complô, é instigar a regressão ao período de fusão do bebê com a mãe na
primeira infância, período caracterizado exatamente pela negação de
qualquer falta: ``O bebê lactante ainda não separa seu Eu de um mundo
exterior, como fonte de sensações que lhe sobrevêm'' (\versal{FREUD}, 1930-2010,
p. 18). Se aos poucos essa diferenciação se dá com a intervenção da
cultura, paradoxalmente, é essa mesma cultura que incita, em nosso
tempo, à sensação --- já que a volta é de fato impossível --- de
novamente poder ter o seio materno e, portanto, poder gozar daquela
situação de completude. Comprar e divertir"-se já se confundem com
entreter"-se, ou manter"-se em estado de letargia -- o que não significa
inação, apatia física, mas apenas mental. O que Maria Rita Kehl disse
das drogas poderia muito bem ser dito do papel das mercadorias de todo
gênero na contemporaneidade:

\begin{quote}
A lógica da droga é a lógica das sensações alucinantes. Dos prazeres
estonteantes. A ética da droga é a ética do \emph{no limits.} Do quanto
mais, melhor. Tudo ao mesmo tempo agora. A gente adora ficar muito
louco. De cerveja, de refrigerante, de música acima dos decibéis
suportáveis, de sol, de tesão, de adrenalina. (2011, p. 112).
\end{quote}

A indústria cultural, portanto, jogou um papel fundamental na
entronização da mercadoria na vida social e subjetiva, sempre
ultrapassando limites. Ela está sempre encontrando novas formas de dizer
que a mercadoria é o \emph{nec plus ultra} da vida. Foi assim que ela
construiu uma indústria paralela que Marcondes Filho denominou
``Indústria do desejo abstrato'', ou seja:

\begin{quote}
Uma unidade da produção preocupada em mexer com elementos guardados no
psiquismo dos indivíduos, acomodados desde a remota infância cheia de
recordações afetivamente carregadas para, a partir disso, desencadear
nas pessoas, ``desejos inconscientes'' e imperiosos por rádios,
televisores, automóveis, roupas, bebidas, principalmente os bens
supérfluos que estavam sendo relegados a um segundo plano nas opções de
compra dos consumidores desempregados, endividados e empobrecidos.
(\versal{FILHO}, 1988, p.144).
\end{quote}

Evidentemente, menos importa de que mercadorias específicas se trata. O
mais importante é o processo regressivo em marcha. Porque explorar as
possibilidades do terreno do desejo, ao explorar o terreno das paixões
ou das pulsões, que são inerentes ao homem em sociedade, segundo
Mandeville, acontece um processo de regressão ao estágio associal do
homem. Ou seja, a vida social já não age como mediador entre a
satisfação e o adiamento da satisfação, mas como terreno onde o
indivíduo com a máscara de caráter do sujeito burguês pode pretensamente
se realizar. Catucar nos espíritos o que ele tem de mais íntimo, para
regredir ao estágio de um pretenso Eu"-grandioso, onipotente, um Eu que
precisa ter seus desejos sempre saciados, pois ele encara seus desejos
como imperiosos, é em verdade regredir ao estágio infantil do homem. Ou,
para dizer de outro modo, é uma forma de realização de uma segunda
natureza, o que significa uma tendência ao encerramento das
possibilidades de transcendência ao universo estabelecido da palavra e
da ação (\versal{MARCUSE}, 1978). A pretensa realização de todas as faltas, de
todos os espaços não preenchidos da existência por meio das mercadorias,
tende a uma distensão entre indivíduo e sociedade. Entre o indivíduo com
a máscara de caráter do sujeito burguês em seu desdobramento e a
sociedade tende a já não haver mais tensão -- que só germina porque os
indivíduos veem na sociedade uma castradora do desabrochar de suas
possibilidades. Quanto mais o indivíduo sente tal máscara de caráter
ajustada ao seu rosto, sua tensão vai dando lugar ao sofrimento psíquico
de um filho mimado que se revolta porque sua mãe lhe prometeu um objeto
de satisfação e não cumpriu. A tensão dá lugar ao sofrimento psíquico,
ao átomo flutuante, porque o filho mimado não se constituiu num processo
de atrito com a vida social, mas num processo de exigência que ela o
incluísse em suas benesses. A imaginação de outro mundo dá lugar à
realização de melhoras no interior deste mundo já corroído.

Além disso, a saciedade contínua que essa sociedade promete, nem que
seja pelas imagens, é realizada num objeto que não tem qualquer
substância a oferecer a esse Eu que pretende nele buscar algum sentido.
Substância, filosoficamente, é o que permanece. Mas, na mercadoria,
substância é o abstrato que se materializa em dinheiro. As mercadorias
têm a característica da fungibilidade, do momentâneo, pois sua dinâmica
interna não permite arraigamento.

\section{Forma-sujeito burguesa e~segunda~natureza}

Se a sociedade moderna pretende se apresentar como um afastamento cada
vez maior em relação à natureza a partir de sua dominação, há uma
maneira de pensar para a qual a possibilidade de superação do quadro
negativo da vida social moderna, notadamente a destruição (ou consumo)
da natureza, passa por uma reaproximação do homem em relação à natureza.
Para essa forma de pensar, o grande problema do capitalismo entendido
como organização social moderna seria seu completo afastamento da
natureza no afã de dominá-la, posto que saber é dominação, como vimos na
\emph{Dialética} \emph{do} \emph{Esclarecimento}, uma dominação que
passa pelo domínio da natureza interna para poder alcançar o da natureza
externa. Então, para se opor a tamanho e prejudicial afastamento da
natureza, seria preciso criar uma espécie de sociedade \emph{avatar} na
qual o homem viveria em harmonia com a natureza, a tal ponto de poder
com ela se conectar, numa espécie de retorno da \emph{mimesis} com a
natureza.

Uma tal visão das coisas parece não permitir uma reflexão de maior
impulso que encare a forma social moderna como ela também uma forma de
natureza, embora social. Portanto, como uma organização social das mais
desenvolvidas, mas cujo desenvolvimento e progresso se dão num processo
automático e inconsciente, como se brotassem dos próprios instintos -- e
o darwinismo social seria apenas uma das facetas dessa forma
naturalizante de organizar a sociedade.

É nesse sentido que cabe dizer que o sujeito moderno, a forma"-sujeito
burguesa, tem também uma constituição natural. Como diz Robert Kurz, o
pensar enquanto concepção, planejamento, construção intelectual deveria
preceder ao agir, e seria isso que distinguiria justamente o ser humano
das abelhas, segundo Marx. O problema é que, segundo Kurz, na forma
social moderna, devido ao seu movimento automático e autoevidente de
multiplicação do dinheiro, dá-se exatamente o contrário: somente em
aparência as pessoas se constituiriam como mestres"-de"-obras no seu
metabolismo com a natureza. Em termos profundos, elas se aproximariam,
em sua organização social, mais das relações que estabelecem entre si as
abelhas. Esse caráter apiário da organização social significa uma
estrutura social

\begin{quote}
na qual já não há unidade entre ``concepção'' e ``execução'' na acção
(nem mesmo ``experimental''), pois esta última é pressuposta \emph{a
priori} de acordo com sua forma, tal como no caso das abelhas. Sob essas
condições, a reflexão (teórica) surge forçosamente como esfera
\emph{subordinada} à ``práxis prática'' e consequentemente dela
\emph{separada}. Por esse motivo, também se registra que as pessoas,
embora ainda capazes de reflectir, desesperam"-se com as consequências
ecologicamente destruidoras das suas próprias acções compulsivas e
apenas \emph{a posteriori} susceptíveis de ser reflectidas e
``trabalhadas''. (\versal{KURZ}, 2008, s.p)
\end{quote}

Significa dizer que os padrões de ação já são estabelecidos \emph{a
priori} sem que um trabalho intelectual reflexivo anterior e consciente
possa julgar esses padrões de ação que ``são quase \emph{ontologicamente
pressupostos} à reflexão.'' (\emph{Idem}, s.p.). O que se argumenta aqui
não tem nada a ver com o elemento pulsional do homem, como terreno de
seu psiquismo que ele não dominaria. Trata"-se em verdade do elemento
pulsional, do terreno inconsciente da própria sociedade, cuja
organização não passa pelo filtro da reflexão coletiva sobre seus meios
e fins. Tudo é autojustificado e pressuposto à reflexão.

O que não tem nada a ver com uma discussão sobre se o ser humano carrega
consigo instintos parecidos com o dos animais ou se são ``instintos''
socialmente mediados. A discussão aqui se refere ao fato de o sujeito
moderno, por mais evoluído que seja, por mais distante que pareça em
relação à natureza que tão fortemente em aparência ele domina e modela,
ainda é um ser sócionatural. Ele é natural no sentido de que age -- na
sociedade que ele crê e pretende dominar nos mínimos detalhes -- como o
ser mais forte, que vence a natureza externa e o outro, como o vencedor
na ``evolução'' darwinista aplicada feito lei natural à vida social.
Assim, ao que parece, numa análise mais atenta, o sujeito moderno só
domina a natureza no sentido de vencê-la no mesmo terreno dela, como ser
também natural, numa disputa de quem consegue ser o ser mais evoluído e
mais poderoso -- mas no terreno da evolução natural. É por vencermos a
natureza por um pretenso instinto de sobrevivência, que está
completamente emaranhado com a forma"-mercadoria, que acabamos
constituindo uma sociedade natural, ou uma natureza social.

Por esse pensamento que adiantamos aqui, o ser humano com a máscara de
caráter do sujeito burguês -- mas também em outras formas sociais na
história -- ainda é um animal -- não no sentido corrente de racional --
mesmo com sua racionalidade dita culminante. Nesse sentido, não é
possível qualquer oposição à sociedade moderna racional sem uma ruptura
com esse aspecto natural, com essa natureza social, ou com essa
sociedade natural. Em vez de uma \emph{mimesis} com a natureza, em vez
de uma tentativa de igualar"-se à natureza para salvá-la, talvez seja
preciso problematizar uma real separação entre \emph{homem} e
\emph{natureza,} ou seja, uma nova transcendência em relação à natureza,
o que poderia ser chamado de nova humanização como superação da segunda
natureza, social e inconsciente até agora reinante.

O ser humano que desenvolve um aspecto regressivo de ser natural -- que
vive em sociedade como se fosse na natureza -- deve tornar"-se perverso e
viver no esteio de seu aspecto natural instintivo. A primeira
consequência é o não reconhecimento da realidade exterior como diferente
de si, é o embaçamento entre o \emph{Eu} e o mundo. É nesse sentido que
a realização da segunda natureza enquanto desdobramento da forma"-social
mercantil e da forma"-sujeito mercantil encontra um terreno inconsciente,
no sentido social. Nesse mesmo movimento, vai ironicamente ao encontro
da realização do Espírito hegeliano na história, já que ``é no teatro da
história universal que o Espírito atinge sua realidade mais concreta.''
(\versal{HEGEL}, 2007, p. 74). Mas que relação se poderia fazer entre esse
Espírito autodeterminado que é para si mesmo seu objeto, que não conhece
limites e o inconsciente social?

Ora, esse Espírito de Hegel não se materializa num indivíduo humano
singular -- obviamente Hegel fala que essa materialização na história se
dá num povo. Mas não nos parece extrapolação de pensamento ver esse
Espírito como a forma"-social e a forma"-sujeito burguesa, únicas formas
na história com pretensões universais a pretender fazer da história e da
subjetividade o teatro de sua realização. Esse Espírito só tem por
objetivo ``manifestar"-se no mundo sensível tal como ele é realmente para
si'', ou seja, ele busca ``produzir um mundo espiritual que seja
adequado a seu conceito, ele busca cumprir e realizar sua verdade''
(\emph{Idem,} p. 83). Claro que Hegel fala de uma materialização do
Espírito numa religião, num Estado. Mas sua especulação abstrata de que
o ``o mundo físico não tem qualquer verdade'' perante o mundo do
espírito é em verdade a descrição do desenvolvimento moderno -- embora o
próprio Hegel pretenda abordar toda a história. Essa visão de mundo
segundo a qual a ``história é apenas o processo pelo qual o Espírito
realiza o seu Si, o seu conceito (\emph{Idem,} p. 94), só pode
significar uma visão totalitária do mundo segundo a qual a realidade
concreta deve se moldar imediatamente aos conceitos. O que equivale a
apagar a diferença entre mundo do pensamento e mundo vivido. Podemos
lançar mão também de Freud e seu conceito de inconsciente, embora ele se
centre no aspecto individual. O mundo do insconsciente tal como descrito
por Freud é esse universo onde também não há diferença entre os mundos
reais e possíveis, entre abstrato e concreto, é um mundo onde ``a
realidade do pensamento é equiparada à realidade externa, o desejo à sua
realização, ao acontecimento, tal como sucede sob o domínio do velho
princípio do prazer.'' (\versal{FREUD}, 2010a, p. 119).

Essa argumentação, que pode parecer abstrata, vai na direção de entender
essa tendência à realização da segunda natureza como coincidindo com a
própria crise da forma"-social e da forma"-sujeito burguesa. Pode"-se dizer
que toda forma de sociedade se constitui como corte na pretendida
onipotência do indivíduo. Nesse sentido, o que se descreve como sujeito
neurótico, que quer ganhar dinheiro e poupar, acumular, é apenas uma das
sínteses num determinado momento de desenvolvimento da forma"-sujeito
burguesa. Do mesmo modo, um capitalismo que ainda deixa vários recantos
da vida sem colonizar é apenas um momento de imaturidade que seu
desenvolvimento lógico trata de resolver. Em verdade, a forma"-sujeito
burguesa mais próxima de sua realização conceitual é caracterizada pela
falta de limites, pelo narcisismo, pela quebra de qualquer mediação
social que emperre a realização do seu \emph{Eu.} Nas sínteses mais
contemporâneas da forma"-sujeito burguesa, não há mais idealização da
sociedade do trabalho, que significava um sacrifício típico de segunda
natureza. A tendência agora é a uma ancoragem no gozo, na facilidade, na
imediatez, numa passagem direta à satisfação dos impulsos mais imediatos
-- o trabalho não passando de um mal necessário, para aqueles que ainda
se apegam ao conceito moral de honestidade. Se o \emph{ideal} impossível
da sociedade mercantil, que corresponde à segunda natureza realizada, é
passar de uma quantia de dinheiro a uma maior sem passar pela mediação
do trabalho, é esse o sentido do capital fictício, essa quebra da
mediação concreta é também característica da forma"-sujeito. Mas, nos
dois casos, esse \emph{ideal} faz parte de um processo de realização
dentro de um quadro geral também de crise. Como formas do Espírito,
tanto a forma"-sujeito burguesa quer realizar seu \emph{Eu} nas
individualidades concretas, como a forma"-social burguesa quer realizar
seu \emph{Eu} na vida social concreta. Isso significa um automovimento
do vazio na esfera da reprodução social e na esfera subjetiva. Dito de
outro modo, uma vida social aos moldes dos automatismos da natureza.

Nesse sentido, o caminho talvez seja o de uma verdadeira diferenciação
com a natureza, em que o ser humano realmente veja a natureza como
\emph{outro que não ele}, para não subjugá-la enquanto mero ente natural
que briga pela sobrevivência numa selva, que nesse caso é realmente de
pedra.

Refletir sobre essa segunda natureza, a social, especialmente a moderna,
significa trazer à mesa a capacidade reflexiva do ser humano como o
aspecto distintivo. Evidentemente, o próprio fato de chamar de segunda
natureza a forma de organização humana historicamente constituída até
aqui já significa que \emph{social} e \emph{natural} são dois níveis
muitos distintos. O social se faz como transcendência em relação à
natureza, como diferenciação do reino animal onde reinam os automatismos
animais, os instintos mais simplórios ou primários. Mas isso não
significa que essa distinção esteja resolvida tão logo tenhamos passado
por essa transcendência. A regressão aos automatismos é sempre algo que
espreita o social, pois o inconsciente social e individual herdam
aqueles caracteres instintuais \emph{mais simplórios e primários} da
primeira natureza, embora transmutados pelo social em segunda natureza,
como inconsciente, não mais como instinto puro e simples. Em outras
palavras, a justificação de acontecimentos sociais por meio de
argumentos fundados no automatismo instintivo da natureza sempre ronda a
organização social. Do mesmo modo, não seria exagero afirmar que o
estado de letargia do animal em sua vida natural, movida por esforços
primários, é algo que também espreita a subjetividade humana. Assim,
essa distinção do homem em relação à natureza não está dada de uma vez
por todas. Embora a regressão a automatismos naturais não signifique uma
volta à primeira natureza, ela significa uma segunda natureza cada vez
mais próxima da primeira natureza, não no que ela tem de pouco elaborado
materialmente, mas no que ela tem de autodeterminado naturalmente num
nível de elaboração e produção material nunca visto. Ou seja, não sendo
dada de uma vez por todas, a diferenciação homem"-natureza, no sentido de
seguir rumo à transcendência da segunda natureza, faz"-se como
amadurecimento de uma capacidade reflexiva e autocrítica que passa por
estágios lentos e paulatinos.

Quero dizer que o homem nunca é um ser reflexivo \emph{em si}, ele
precisa tornar"-se pela mediação da sociedade em que vive. Não deve
transparecer nessa reflexão qualquer defesa da ideologia da mente como
tábua rasa ou da ideia do \emph{amorfismo humano} de Robert Musil. Pelo
contrário, trata"-se de problematizar uma tendência que sempre puxa o ser
humano -- e contra a qual ele deve lutar constantemente -- a uma
diferenciação mínima, porque os automatismos da natureza são menos
exigentes que os estágios de reflexão formadores de uma cultura -- nesse
caso uma cultura no sentido largo, crítica, que assume nas mãos a
organização da sociedade de forma coletiva. O estágio do
narcisismo primário descrito por Freud talvez seja, num quadro social, o
estágio mais próximo da natureza a que o ser humano chega, no sentido de
seu movimento de refestelar"-se em qualquer coisa que não lhe tire o
sossego letárgico. Como se sabe, desse narcisismo ele só sairá rumo ao
narcisismo mediado socialmente por meio de choques contínuos com a
realidade que desbastam seu delírio de onipotência. Significa dizer que
parece haver uma tendência no ser humano a ficar mais perto dos reflexos
naturais, uma tendência contra a qual é preciso resistir. Se tal
tendência não for combatida -- e historicamente as formas sociais com
suas instituições de enquadramento significavam um freio a essa
tendência -- a subjetividade que aflora desses sujeitos naturais só pode
engendrar uma sociedade funesta. A tendência da sociedade mercantil
contemporânea é exatamente manter os sujeitos num patamar ao menos
parecido com esse estágio primário, em que cada desejo mercantil é
transformado em gozo, em refestelar"-se na letargia.

Diante dessa argumentação, uma pergunta poderia ser posta. Seria o
capitalismo a primeira sociedade a realizar a natureza humana? Por vezes
há muita pressa em responder a essa pergunta para afastá-la feito uma
pestilenta que incomoda. Em verdade, essa é uma ideia liberal para
justificar a sociedade mercantil, dizem alguns. Mas se pensarmos com um
pouco mais de vagar, e sem ver nisso a positividade liberal, com efeito
a forma"-social burguesa vai ao encontro da natureza humana, mas apenas
no sentido de que as esferas natural e humana se misturam nessa forma de
vida social. Dito de modo distinto, apenas se entendermos o capitalismo
desenvolvido como uma forma de sociedade que promove uma regressão
social e subjetiva a uma vida psíquica rebaixada, mais próxima dos
reflexos naturais, dos ``vícios instintuais'' em detrimento do que até
hoje se constituiu como sendo considerado virtude. Ou seja, o
capitalismo desenvolvido tende a corroer as formas de mediação social
que são uma vitória sobre os impulsos mais próximos do reino da
natureza\footnote{Permito"-me lembrar aqui uma expressão bem cearense:
  ``rebolar no mato''. A expressão não tem nada a ver com uma dança no
  mato, significa simplesmente jogar no lixo. Mas se cavarmos um pouco
  mais fundo, ela em verdade tem relação com a passagem de uma relação
  social mediada pelo campo, em contato maior com a natureza, para uma
  relação social mediada diretamente e de chofre pela mercadoria. Se
  antes falava"-se ``rebolar no mato'' para uma casca de banana,
  embalagens de papel, ou outro lixo orgânico com que as pessoas tinham
  contato, hoje utiliza"-se essa mesma expressão para dizer jogar no lixo
  os mais diversos tipos de dejetos, sobretudo plásticos, como se fossem
  ser devorados pelo ``mato'' da mesma forma que o lixo de outrora. É
  como se se passasse de uma natureza a outra sem que qualquer conflito
  daí brotasse.}. A forma"-sujeito burguesa não tem como se aproximar de
sua realização, de seu ideal de subjetividade, sem um pretenso regresso
do indivíduo a um estágio similar ao natural, não no sentido de se
tornar animal, mas no sentido de viver uma vida animalmente social. É
somente nesse sentido que se pode dizer que a realização da
forma"-sujeito é a realização da natureza humana, cujo habitat é a
forma"-social burguesa, a vida econômica naturalizada\footnote{Não seria
  sintoma dessa regressão uma tendência na música dita de massa
  contemporânea a ritmos bastante primários, com a predominância de
  batidas, tambores sobre a melodia? Sem contar a apologia que se vê em
  tais músicas de relações sexuais cada vez mais dessimbolizadas e
  imediatas, descritas com minúcias como pura descarga de energia
  instintual sobre o outro usado como material de depósito?}.

Para romper a segunda natureza social moderna, mais do que nunca, é
preciso um movimento contínuo de superação da tendência ao sócionatural.
Nesse sentido, não se pode negar que a reflexão filosófica moderna -- no
que tem de questionamento da autoridade da Igreja, do poder despótico,
do fetiche da dominação pessoal mediada pela terra e pelo fetiche do
sangue -- levou o ser humano a um patamar de pensamento mais elevado;
por mais que tais questionamentos em proveito do mais alto nível de
liberdade humana tenham tido como lastro o tensionamento criado pela
forma de vida mercantil -- uma forma de vida que tende à liberdade
irrestrita. Mesmo assim, o que interessa dizer aqui é que são no fundo
esses choques de pensamento, de mentalidade, de subjetividades, que
questionam radicalmente as bases de uma sociedade, que elevam a
capacidade reflexiva e criam a possibilidade de superação da segunda
natureza. O esforço de enfrentamento feito pelos indivíduos contra as
formas de dominação da sociedade fetichista anterior ao capitalismo pode
ser considerado, enquanto movimento de choque com o estabelecido, algo
fundamental a uma transcendência.

Entretanto, esses questionamentos apenas criam a possibilidade. Como
sabemos, os questionamentos modernos foram ao encontro da organização
social moderna com sua tendência intrínseca à liberdade mediada apenas
pela mão invisível da lógica mercantil. Ou seja, o complexo desse
conjunto de questionamentos é que o salto que chegou a impulsionar
desembocou ainda numa forma de menoridade, relativa a aspectos mais
sutis, mas não menos imperativos. Assim, a transcendência não ocorreu,
ou ocorreu apenas de um nível de fetichismo para outro, mais elaborado,
mas não para uma forma de sociedade organizada diretamente pelos
indivíduos.

Para uma transcendência da forma social moderna naturalizada, seriam
necessários saltos nessa capacidade reflexiva, para se vislumbrarem
possibilidades para além da realidade dada, como tensionamento
permanente com a vida social naturalizada. Mas há quanto tempo não se
veem esses saltos? Não se assistem a grandes rupturas impulsionadas e
impulsionadoras desses saltos desde que a modernidade pôs a economia
como centro fundante da vida, como forma natural de organização social.
Enquanto as discussões girarem em torno da gestão do existente, com
nuanças que conservam a essência do imperativo categórico moderno, será
rude aventura causar rachaduras nessa selva de pedra, nessa natureza
colonizada pelo social, ou melhor, nesse social dominado pelas leis da
natureza.

O \emph{fim da história} humana portanto é muito diferente do \emph{fim
da pré-história} humana. A ideologia do fim da história é a da aceitação
cínica do aprisionamento na segunda natureza social plasmada pelas
categorias capitalistas. É a aceitação do jugo da subjetividade e da
vida social ao movimento naturalizado, automático e autojustificado da
multiplicação do dinheiro. Em suma, é o fim de qualquer verdadeira
transcendência ao universo estabelecido, ou seja, é o fim dos
verdadeiros tensionamentos na vida social. Já o fim da pré-história como
vislumbrada por Marx diz respeito exatamente à superação de uma história
humana em que são as coisas que governam e os seres humanos que agem
dentro de um invólucro coercitivo que lhes escapa. O fim da pré-história
seria o momento em que a coletividade humana assume o controle da
própria organização social, um momento no qual a história se materializa
numa forma de práxis coletiva, não mais subjugada aos determinismos da
natureza, da escassez e da mediação do movimento de multiplicação do
dinheiro que faz os indivíduos concorrerem como se estivessem brigando
pela sobrevivência na selva. Seria assim o romper"-se da matriz
apriorística do fetichismo da mercadoria, portanto, o fim desse dito
caráter apiário da reprodução social.

\section{O valor-signo como usura do simbólico}

Já dissemos que uma mercadoria não é uma coisa qualquer, qualquer
objeto. Mas sim um objeto que precisa realizar em dinheiro o tempo de
trabalho abstrato que contém. Por isso, quanto mais ela é entronizada na
sociedade, passa a ficar cada vez menos importante seu lado útil
concreto -- que não é seu objetivo. Além disso, ela não tem qualquer
pretensão simbólica que as coisas como coisas poderiam ter.

Ora, o aspecto simbólico diz respeito a um construto social que tem
certa solidez, está fundado em certa tradição, não prescindindo das
trocas sociais que a sustentem, pois o objeto simbólico tem seu sentido
construído apenas no social, daí seu caráter perene. Um objeto simbólico
pode passar de geração em geração. Pode significar uma ligação forte com
determinados objetos que não têm nada a ver com fetichismo mercantil,
mas sim com o peso simbólico que adquirem determinados objetos por
criarem laços de sentido entre vidas e experiências. Podemos pensar em
roupas presenteadas, roupas feitas por um parente, um móvel herdado, um
livro que nos tocou, até o simples respeito aos mais velhos pelo aspecto
simbólico da sua experiência.

O avançar da forma"-social burguesa sobre a vida social concreta também
implica a corrosão paulatina desse processo simbólico em proveito de
relações sígnicas e dessimbolizadas, imediatas. Isso não só na relação
com as mercadorias, mas na relação de alteridade. Michéa, autor pouco
conhecido no Brasil, contribui com uma reflexão importante em seu livro
\emph{O ensino da ignorância}:

\begin{quote}
A magia do dinheiro -- como equivalente geral -- reside na possibilidade
que ele oferece ao sujeito de se livrar de qualquer dívida em relação a
um doador a partir do momento em que o serviço prestado foi pago na
hora. [\ldots{}] Ficando quite no ato mesmo daquilo que eu devo, o
que compro igualmente \emph{é o tempo implicado pela obrigação
tradicional de esperar para retribuir} e, portanto, pelo mesmo ato,
compro \emph{o direito de não ter história} com aqueles que me prestaram
algum serviço. (2006, p. 88)
\end{quote}

O autor aqui desenvolve a temática da \emph{dívida simbólica} que é uma
mediação social, um tempo transcorrido dentro do qual sentidos são
construídos, laços são amarrados. Evidentemente, ele fala aqui de uma
trama social chamada de \emph{dom, dádiva,} no sentido de Marcel Mauss,
cujo enlaçamento simbólico ou \emph{dívida simbólica} se dá pelo
\emph{dar, receber, retribuir.} Para Michéa, a modernidade criou toda
uma série de \emph{obrigações sem reciprocidade} (\emph{Idem}, p. 91)
sob o nome de civismo, disposição ética, que são uma dessimbolização que
contribui para a máquina andar mais macia.

Ou seja, trata"-se de relações mais adaptadas à dinâmica cada vez mais
frenética que exige que não nos apeguemos ``a tralhas'', que nos
desapeguemos em prol de algo cheirando a novo, sempre novo. Essas
relações sígnicas, sobretudo com o objeto, nada têm a ver com um
processo de construção, porque pretendem prescindir -- o que não é
possível -- da relação com o outro com quem estabeleço o laço simbólico.
O simbólico pressupõe que o indivíduo respeite o laço, portanto, o
simbólico é um freio no narcisismo que é afagado nas relações com
objetos sígnicos, pois são objetos que se pretendem personificados --
não com um valor simbólico, mas um valor sígnico (\versal{BAUDRILLARD}, 2008). O
valor"-signo é, diferentemente do símbolo, algo que não permanece, mas
algo que se adapta à rapidez da dinâmica mercantil. Sua fugacidade, sua
imediaticidade, deve"-se fundamentalmente à sua arbitrariedade. O signo
está aberto a todas as significações imagináveis, é flexível --- como os
tempos contemporâneos. Já o símbolo tem a característica de se fixar nas
relações sociais e dificulta as transações sempre móveis. O simbólico se
transforma num peso, porque é uma mediação social.

Poderíamos adentrar por reflexões sobre o quanto a mediação simbólica,
por mais fetichista que tenha sido, construiu coisas belas na história.
Arendt reflete sobre isso defendendo que uma sociedade não pode
deixar"-se prender no mundo da utilidade, que é passageira, não
permanece:

\begin{quote}
As catedrais foram construídas \emph{ad maiorem gloriam Dei;} embora,
como construções, sirvam decerto às necessidades da comunidade, sua
elaborada beleza jamais pode ser explicada por tais necessidades que
poderiam ter sido servidas igualmente por um outro edifício qualquer.
Sua beleza transcendia todas as necessidades e as fez durar através dos
séculos. (2009, p. 261).
\end{quote}

Não sendo a necessidade, o que estaria por trás de belas construções
arquitetônicas do passado? Não seriam o simbólico, as mediações sociais
-- por mais que fossem justificadoras de dominação pessoal mediada pelo
fetichismo religioso? Em cada detalhe elaborado numa obra, em cada curva
burilada no mínimo detalhe, em cada filigrana não estaria uma negação da
imediatidade, do reto, do linear, tipicamente modernos? Como já
dissemos, as obras até o Renascimento, e bem depois, para não falar do
Barroco brasileiro, tendiam a agradar um ser transcendente, um deus, um
ser mitológico. Em verdade, ao que parece, a história anterior ao
capitalismo se deu no teatro do simbólico"-mítico"-religioso. A sociedade
projetava seu poder social para o terreno transcendente. No entanto, a
ação social era feita pelos próprios homens, e o simples fato de terem
que se reportar respeitosamente a essa esfera superior criava no seio
social uma série de mediações criadoras de cultura. Desejava"-se oferecer
o melhor àquele ser que tinha um peso simbólico forte nas relações
humanas.

A modernidade deve ser, por outro lado, entendida como também fundada
num universo simbólico, mas mercantil. Como a transcendência moderna
vive diretamente na terra, mediando as relações diretamente na terra, e
os indivíduos se creem livres, o deus a quem se pretende agradar para
acalmar sua possível fúria é o deus mercantil, para cuja glória não é
preciso qualquer mediação. Além disso, o deus moderno não exige nenhuma
postura moral de seus fieis. Ele é o próprio vazio em movimento. As
catedrais que ele exige em sua honra são mais do estilo de
\emph{shoppings} \emph{centers}, e a postura que ele exige dos fieis é a
do desprendimento em relação a qualquer mediação social inútil. Como
esse deus moderno simboliza um vazio, o que já significa um oxímoro
chamar seu universo de simbólico, as obras grandiosas em seu nome são a
expressão desse vazio, reto, funcional, imediato. Mesmo porque, as obras
contemporâneas pretendem vir de sujeitos que se acham livres de toda
ligação simbólica. As obras parecem vir de sujeitos plenos de
significado em si mesmos. A dívida simbólica agora, se é que se pode
chamar simbólica, é do sujeito consigo mesmo, com o narcísico, ou então
dele com o mercado, onde estão suas dívidas ``simbólicas''.

E não se trata de nostalgia ou de apologia do passado. No universo
simbólico mercantil, em sua fase mais madura, a mercadoria desenvolve
também um valor"-signo. Ou seja, significados são atribuídos às
mercadorias como se deles emanassem, como se de fato tivessem passado
por uma mediação social -- constituindo"-se, assim, como
pseudosimbólicos. Esse sentido não é um construto social no sentido de
que é estruturado pelas relações sociais, mas é imposto às relações e,
portanto, ao sujeito que apenas aceita porque já não tem em si tão forte
espaço que se contraponha a essa arbitrariedade. ``A grande
peculiaridade do objeto"-signo é que seu sentido não está mais referido a
nenhuma relação humana, mas sim, `en la relación diferencial respecto a
otros signos''' (\versal{SEVERIANO}, 2007, p. 51). Assim, o objeto"-signo pretende
diferenciar e dar estatuto num processo ilimitado, posto que arbitrário:
o objeto"-signo se torna depósito para as projeções idealizadas de
onipotência humana. (\emph{Idem,} p. 51).

É fundamental ressaltar que, mesmo percorrendo esse caminho, a
mercadoria não perdeu aquilo que faz dela mercadoria: sua
essência"-valor. Antes do mais, é essa essência que, para se desenvolver,
precisa adaptar a realidade objetiva e subjetiva a seus desígnios.
Contrariamente ao que deixa pensar Baudrillard, o valor"-signo não se
sobrepõe ao valor mercantil como numa evolução.

A fase do valor"-signo --- que se liga com um fetichismo no nível do
desejo (\emph{Idem}, 2007) -- é a fase do espetáculo, este sendo
compreendido como o ``capital em tal grau de acumulação que se torna
imagem''. (\versal{DEBORD}, 1997, p. 25).

\section{O tempo tornado abstrato cobra~sua~conta?}

Já explicitamos a revolução no sentido do tempo na aurora da
modernidade. E vimos também que somente com um processo de
tensionamentos rumo à modernidade o tempo abstrato passou a dominar. E,
de fato, o capitalismo não teria podido se instalar sem essa revolução,
pois o capitalismo é uma sociedade baseada no trabalho medido pelo tempo
-- sem levar em conta no que esse tempo é gasto --, que é a essência
abstrata do valor, que, por sua vez, é a essência da mercadoria.
Portanto, o dinheiro é a materialização do tempo abstrato contido em
toda mercadoria.

A questão do tempo adquire uma importância fundamental também, porque se
trata de mais um elemento que perdeu substância, que perdeu suas
qualidades na modernidade. Assim como as coisas perderam sua dignidade
ao serem transformadas em mercadoria, assim como as atividades
produtivas perderam sua dignidade ao serem transformadas em trabalho,
assim como os indivíduos perdem, pouco a pouco, sua dignidade ao
vestirem a máscara de caráter da forma"-sujeito, o tempo perdeu sua
dignidade ao deixar de ser uma variante dependente, concreta, para se
tornar uma variante abstrata independente e coercitiva das atividades.
Ora, o transtorno da experiência com o tempo é marca da modernidade.
Quando o tempo se torna a mônada abstrata que se realiza em dinheiro,
desgarrando"-se das relações concretas das quais dependia, não pode mais
deixar de ser transformado em dinheiro, em vazio, em vazio tautológico,
como se fosse uma verdadeira maldição da qual não se pode escapar. Mas,
ao mesmo tempo, é o tempo que consome a própria sociedade que o quis
aprisionar. Ou seja, podemos dizer que é o próprio tempo abstrato que
está corroendo a vida capitalista e, ao mesmo tempo, inundando a vida de
mercadorias que exigem ser compradas\footnote{Arendt achava que por não
  ``haver suficientes bens de consumo para satisfazer aos apetites
  crescentes de um processo cuja força vital [\ldots{}] precisa ser gasta
  no consumo'', seria como se a vida se esgotasse, ``valendo"-se de
  coisas que jamais foram a ela destinadas.'' (2009, p. 264). Ela achava
  que pelo tempo menor dispendido no trabalho, não haveria como dar
  conta do aumento do consumo. Para nós, pretensiosamente, o esgotamento
  da vida se dá exatamente pelo fato de a sociedade ter feito e fazer
  tudo para que os ditos bens de consumo -- a riqueza -- chegassem e
  cheguem a todos os recantos.}. E o tempo abstrato é o tempo do
presente encerrado numa temporalidade unidimensional -- essa é a
temporalidade nas peças de Beckett. Vejamos o caminho percorrido por
essa abstração no social.

No seio de cada mercadoria mora sua substância que é o trabalho. Mas não
qualquer trabalho, um trabalho medido pelo tempo. Cada mercadoria tem
seu valor retirado do tempo socialmente necessário para produzi"-la
(\versal{MARX}, 1985). Mas o que acontece quando a produtividade aumenta em
função da maquinaria e o tempo de trabalho contido em cada mercadoria
--- que é a verdadeira riqueza no sentido capitalista --- diminui? Para
exemplificar de forma simples: se a média social de produção de 30
camisas é de uma hora, para o capitalismo, em média global, cada camisa
vai conter 2 minutos de valor. Portanto, 2 minutos de \emph{riqueza
social}. Com a introdução de uma nova tecnologia que possibilite a
produção de 60 camisas em uma hora, o valor de cada camisa cairá para 1
minuto\footnote{O exemplo pretende apenas ser didático, porque nunca vai
  ser possível determinar o valor de uma mercadoria individual, uma vez
  que o processo capitalista se dá em termos globais.}. Essa mudança de
produtividade não é anódina. Ela vai criar dois problemas para o
capitalismo. O primeiro: a \emph{riqueza social} estagnará, pois ela é
medida não pela quantidade de coisas -- que tende a aumentar --, mas
pela quantidade de tempo de trabalho contido nas mercadorias, pelo valor
que elas contêm -- um valor que tende a diminuir independentemente do
amontoado concreto. Para fugir dessa estagnação trágica para o
capitalismo, é preciso uma compensação dessa diminuição inundando ainda
mais a vida social de mercadorias. Cada vez mais mercadorias serão
produzidas num intervalo menor de tempo, ou seja, o tempo será cada vez
mais recortado em intervalos abstratos menores e iguais no qual se
materializará uma mercadoria.

Que o valor de cada mercadoria tende a diminuir a um mínimo (\versal{MARX}, 2011,
p. 588), essa foi uma das grandes reflexões de Marx. Ou seja, Marx
problematizou uma contradição interna ao capitalismo -- embora ao lado
dessa reflexão radical ele alimentasse também a ideia de que o
proletariado daria cabo ao sistema antes que essa contradição se
desenvolvesse. Significa dizer que, embora a \emph{riqueza material}
medida pela quantidade de objetos aumente, a \emph{riqueza social}
diminui se não houver uma ampliação constante e imperativa. Essa não é
uma contradição das menores no interior do sistema, que historicamente
sempre precisou lutar para compensar essa perda de riqueza.

O outro problema, decorrente do primeiro: para manter"-se de pé, o
capitalismo tem que fazer comprar não mais 30, mas no mínimo 60 camisas
para manter o mesmo nível de riqueza de antes da introdução da nova
invenção tecnológica (\versal{POSTONE}, 2009) -- no caso de nosso exemplo
particular. Além de inundar a vida social de mercadorias que imploram
para serem compradas, há uma crescente exigência feita aos sujeitos no
cotidiano, como se algo nunca fosse o bastante, como se cada pedaço de
tempo precisasse ser materializado em mais coisas, mais atividades, como
se as próprias atividades que desempenhamos precisassem compensar parte
da riqueza que perdeu. Dito de outro modo, é como se nossas atividades
ficassem cada vez mais pobres.

Esse é um processo que se dá continuamente desde o início do
capitalismo, mas é um processo que nunca volta ao mesmo ponto, pois
sempre são estabelecidos novos padrões de produtividade, ou seja, sempre
novos limites precisam ser quebrados. Ora, se pelo exemplo o sistema
terá que fazer comprar o dobro apenas para manter a mesma riqueza
social, para poder se expandir, vai ter que ir além da duplicação da
construção de um quadro favorável ao consumo ilimitado. Dito em termos
simples, a vida social capitalista só se mantém pela inundação cada vez
maior da vida cotidiana com mercadorias. Sem contar que, para compensar
essa perda, cada vez mais recursos naturais são necessários para manter
o progresso capitalista.

Podemos, assim, sustentar a ideia de que o capitalismo precisa, a todo
custo, fazer comprar não apenas para aumentar seus lucros -- que é a
percepção da consciência cotidiana --, mas para fugir antecipadamente de
uma crise interna que lhe é intrínseca. Um sistema que é dinâmico por
essência não pode encontrar um ponto de equilíbrio. Consequentemente,
estamos a falar de uma verdadeira crise lógica prevista por Marx, uma
crise que o capitalismo precisa cotidianamente combater em seu próprio
âmago e que a contemporaneidade vivencia sempre mais fortemente. A
aceleração do tempo, não somente nas fábricas, mas na vida social, tem
uma relação estreita com a sua abstratificação, com o fato de que não
são os acontecimentos da vida que indicam o tempo ou que são
simplesmente marcados por ele, e que fazem dele algo qualitativo, mas o
tempo é que determina os acontecimentos, o tempo é que determina as
atividades -- as que \emph{valem} são chamadas trabalho.

Como o capitalismo não consegue ir adiante sem inundar a vida social de
mercadorias, é preciso que essas mercadorias sejam consumidas, ou
melhor, é preciso que os sujeitos se relacionem de modo cada vez mais
íntimo com as mercadorias. Para isso, ela precisa ser transformada num
objeto animado, com alma, não só na produção, mas na circulação. Ela
passa a defender seus direitos de cadáver frente aos vivos que consomem
as subjetividades atreladas às mercadorias. E esse processo acaba
concorrendo para a confusão entre desejo e objeto de consumo (\versal{SEVERIANO},
2007), portanto, acaba contribuindo para o esvaziamento do pensamento
que só surge nesse espaço de tensão. Não havendo espaço -- ou melhor,
não havendo tempo de espera -- entre o desejo e o objeto do desejo, não
há espaço nem tempo para que os indivíduos se erijam como seres
reflexivos. Pensar pode virar um fardo, já que é preciso gozar a vida. E
o mundo inundado de mercadorias pretende poupar"-nos da reflexão,
pretende poupar"-nos do choque de realidade que nos constrói como seres
minimamente tensos com a realidade, e minimamente fortes para
enfrentá-la. O apelo ao inconsciente, ao mundo onírico onde tudo é
possível aponta para essa suspensão do pensamento em proveito do mundo
vazio da mercadoria e do encerramento no universo da segunda natureza,
como tentamos refletir. É como se a forma"-social burguesa, para se
realizar, precisasse desenvolver um \emph{princípio de prazer mercantil}
sem o qual ficaria difícil fazer comprar tanta mercadoria.

Esse apelo ao onírico, que poderia chocar à forma de subjetivação
ascética, muito presa no senso de realidade, de responsabilidades
neuróticas, foi muito bem incorporado à venda de mercadorias, no sentido
de desenvolver um fetichismo pós"-produção. Esse lado onírico, prezado
como inovação estética pelo Surrealismo e normalmente presente nas
vanguardas, parece ir ao encontro ironicamente da fase atual do
desenvolvimento capitalista. Como explicita Tiedemann:

\begin{quote}
Os primeiros surrealistas tinham, pelo sonho, despojado a realidade
empírica em geral de seu poder; tratavam sua organização teleológica
como uma simples matéria onírica cuja linguagem só pode ser decifrada
indiretamente. Dirigindo a ótica do sonho para o mundo despertado,
dever"-se"-ia trazer à lume as ideias latentes, escondidas, que dormiriam
em seu seio. Benjamin queria aplicar um procedimento semelhante à
apresentação da história: tratar o mundo das coisas do século \versal{XIX} como
se se tratasse de um mundo de coisas sonhadas. Quando é regida pelas
relações de produção capitalistas, a história é, em todo caso,
comparável à ação inconsciente do indivíduo que sonha: ela é certamente
feita pelos homens, mas sem plano nem consciência, como num sonho.
(\versal{TIEDEMANN}, 1997, p. 15)
\end{quote}

Mas quando o sonho é invadido pelo desejo mercantil, o terreno
inconsciente já não pode ser um terreno de aposta para se opor ao
próprio universo mercantil. Se a exploração do sonho para os
surrealistas era a descoberta de novos terrenos criativos, que se
opunham a uma sociedade moralmente coercitiva, quando restam poucos
terrenos não colonizados, é preciso buscar noutro campo a oposição ao
universo simbólico mercantil\footnote{Nossa argumentação, portanto,
  distingue"-se daquelas que pretendem ainda nos tempos atuais que o
  inconsciente é o terreno libertário, o não"-idêntico não suportado pela
  forma"-social mercantil.}. Mas esses novos terrenos criativos, a
sociedade capitalista tomou para si ironicamente, do mesmo modo que os
desejos de liberdade dos movimentos dos anos 60. Hoje podemos ter a
medida de que a repressão de uma vida regrada contra a qual lutavam os
movimentos daquela época eram apenas resquícios de um entrecruzamento
objetivo e subjetivo de fases anteriores do capitalismo. No
desdobramento da vida social moderna, cada geração foi ficando mais
parecida com seu tempo do que com seus pais, como disse Debord (1997) e
foi, uma após outra, num processo tenso, mas sempre tendendo para
frente, convivendo menos estranhamente com a ligação embaçada entre
sonho, desejo, liberdade e mundo mercantil. Aproximamo"-nos hoje, cada
vez mais, de uma geração de desejos imperiosos, que pretende equiparar
seus desejos (no invólucro mercantil) à realidade sem qualquer mediação
com o mundo exterior. Todas as mediações necessárias ao estabelecimento
da reflexão são, nesse processo, achatadas pelo imediatismo que impõe
esse princípio. No princípio de prazer não há pensamento, apenas prazer
imediato, ensina Freud. Apenas repetição do mesmo tempo de realização
dos desejos encerrada numa temporalidade única.

É ao que Adorno (1986b, p. 123) se refere em seu artigo ``Sobre música
popular'' ao afirmar que tudo aquilo que se reveste de individualidade,
no fim de tudo foi previamente escolhido e imposto por outrem. Por isso
justifica"-se o que ele chama de pseudoindividuação. Se ele se referia à
\emph{standardização musical}, pode"-se estender essa reflexão à
individuação em tempos atuais, que não se constitui sem a mediação da
forma"-sujeito burguesa.

Desse modo, aqueles que se creem no chamado desabrochar da
individualidade no mundo contemporâneo ignoram que são individualidades
de prótese, pois, em verdade, cada vez mais são as mercadorias que são
individuais e que pretendem ter uma individualidade e transmiti"-las a
quem comprá-las.

Por essa argumentação, seria possível defender que o desdobramento da
forma"-sujeito burguesa é uma tendência à vacuidade. Uma vacuidade que
não se dá necessariamente por um vazio absoluto, ausência de reflexão
total, mas pelo enraizamento da forma"-social burguesa como segunda
natureza. Essa vacuidade tem estreita relação com o processo de
superação de um estágio de desenvolvimento da forma"-sujeito burguesa no
qual o ideal era a negação da desmedida, do gozo, do luxo, vistos até
como diabólicos, enquanto o acúmulo e a ascendência social eram vistos
como signos de eleição divina. Este estágio descrito por Weber
(1905-2013), como já refletimos, começou a desaparecer nas sociedades de
massas e, desde há algum tempo, começou a tensionar com ele e a ganhar
terreno, inaugurando um novo \emph{ethos}, uma forma de subjetivação que
coloca exatamente o gozo, a desmedida, o luxo e o rompimento de todos os
limites como ideais. Essa ideologia da falta de limites, ao invés de ser
uma subversão do capitalismo, serviu"-lhe de terreno fértil para
subverter seus próprios limites -- o próprio capitalismo é uma
forma"-social que não pretende reconhecer qualquer limite.
Portanto, o que o capitalismo exigia do sujeito quando de sua
implantação, não se mostra mais adequado quando um nível de
desenvolvimento precisa sempre ser suplantado por outro mais adiantado.
Nesse contexto, essas características protestantes, ``reacionárias'',
apenas obstam a realização do espírito do capitalismo --- espírito
``revolucionário''.

\begin{quote}
Um novo estádio do individualismo vem a assentar"-se: o narcisismo
designa o surgimento de um perfil inédito do indivíduo em suas relações
consigo mesmo e com seu corpo, com outrem, com o mundo, com o tempo, no
momento em que o ``capitalismo autoritário'' cede seu lugar a um
capitalismo hedonista e permissivo. A idade de ouro do individualismo,
concorrencial no nível econômico, sentimental no nível doméstico,
revolucionário no nível político e artístico, chega ao fim. Um
individualismo puro se desenvolve, desvencilhado dos últimos valores
sociais e morais que coexistiam ainda com o reino glorioso do \emph{homo
oeconomicus}, da família, da revolução e da arte; (\ldots{}) Se a modernidade
se identifica com o espírito empreendedor, com a esperança futurista,
está claro que o narcisismo inaugura, por sua indiferença histórica, a
pós"-modernidade, a última fase do \emph{Homo Aequalis} [tradução
nossa] (\versal{LIPOVETSKY}, 2008, p. 71-72).
\end{quote}

Para Lipovetsky, essa mudança subjetiva inaugura a pós"-modernidade. Para
nós, trata"-se de um desdobramento da modernidade, da forma"-sujeito
burguesa, cuja lógica é um desvencilhamento contínuo dos valores sociais
e morais que coexistiam na subjetividade concreta atravessada pelo
passado e pelas exigências do futuro. Hoje, a tendência é que até esses
resquícios de ética se esmaeçam -- sem que se possa chamar de um avanço
humano. É de Maria Rita Khel (2005) que vem uma explicação que muito nos
interessa aqui. Ora, se o imperativo do gozo nas mercadorias vem
substituindo o adiamento -- típico do sujeito ascético --, ao mesmo
tempo ele pretende poupar as pessoas do esforço de individuação. E a
mercadoria aqui precisa ser vista como comprada materialmente ou
comprada por via das imagens:

\begin{quote}
Ocorre que o tipo de produção de sentido que é próprio das imagens induz
o sujeito a um modo de funcionamento psíquico que prescinde do
pensamento. Brevemente eu diria que isso ocorre porque o imaginário
funciona segundo a lógica da realização dos desejos. Cada imagem
apresentada proporciona ao espectador um microfragmento de gozo --- e a
cada fragmento de gozo o pensamento cessa. Por que o pensamento não
acompanha o gozo? O pensamento é um longo rodeio que o ser humano é
obrigado a fazer em busca de um objeto de satisfação que se perdeu;
(2004, p. 89-90).
\end{quote}

Mas a vida não se torna um amontoado, um vazio, uma superfluidade sem
esse rodeio fundante? Não é esse rodeio uma mediação que também
caracteriza o simbólico? Mas é exatamente essa espécie de suspensão do
pensamento cada vez mais constante na vida que é um dos aspectos
característicos da forma"-sujeito moderna na contemporaneidade.

A tensão típica desse rodeio só pode se manter quando de fato o Eu
mantém seu caminho de afastamento do seu Eu infantil. Portanto, a tensão
leva à sublimação, ao pensamento, à reflexão: ``cultura é sublimação''
(\versal{MARCUSE}, 1980, p. 15). O estado a"-conflitivo sempre prometido pela
mercadoria atualmente leva à regressão psíquica e social visto exigir o
reino da imediatez, o reino que poupa o ser da individuação. No reino da
imediatez não há reflexão. O reino da imediatez não permite nada faltar,
não permite o rodeio que caracteriza o pensamento que deseja refletir
sobre essa falta, que deseja o desejo --- que é em si também uma
mediação com o social. E o pensamento apenas pode surgir se houver
espaço de tensão, espaço de não"-imediatez, e é exatamente isso que vem
caracterizando a existência humana historicamente. Nessa existência, é
como se algo tivesse ficado vago e precisasse ser preenchido, um espaço
vazio que surge a cada momento de encontro com a realidade externa e o
outro que não possibilitam que mundo psíquico e mundo real coincidam.

Mas a relação social de suposta saciedade sempre contínua e possível ---
nem que seja por meio de imagens ---, embora o dinheiro continue a
condição \emph{sine qua non,} desemboca no que Marcuse chama de
\emph{dessublimação repressiva}. Marcuse argumenta que a experiência da
renúncia e da sublimação eram experiências muito pessoais, um estágio
importante na individualização, algo que deixa ``cicatrizes dolorosas''.
O que Marcuse ressalta é que ``a formação do superego maduro parece
agora saltar por cima do estágio da individualização: o átomo genérico
torna"-se imediatamente um átomo social.'' (\versal{MARCUSE}, 1955-2009, p. 97).
Quer dizer, os sujeitos não mais guardariam a \emph{consciência infeliz}
por terem sido obrigados a desviar a sua satisfação pulsional em
proveito da socialização --- ou da civilização, como defende Freud. Uma
\emph{consciência infeliz} decorrente do \emph{Mal"-estar} (\versal{FREUD},
1930-2010), ele mesmo decorrente do fato de o \emph{Eu} ter sido forçado
pela realidade externa e pelo outro a desbastar sua onipotência. Antes,
haveria o desenvolvimento de uma \emph{consciência feliz} em razão das
liberdades satisfatórias, inclusive de ordem sexual erógena, concedidas
pela sociedade que utiliza a própria liberdade como dominação.

\section{Beckett e a estética da vida mercantil~desenvolvida}

Para Fredric Jameson (1997), o chamado pós"-modernis-mo, pode ``ser %PALAVRA ESCAPANDO DA MANCHA
pensado como o momento no qual a antiga subjetividade (agora plenamente
coletivizada) desaparece inteiramente''. O pós"-modernismo representaria
o momento em que se esvai a tensão ``que constituía o minimalismo de
Beckett tanto quanto o momento expressionista de Schoenberg -- o grito
silente de dor.'' Assim, ficaria a ``produtividade e a tecnologia
coletiva avançada livres para `expressar' apenas a si mesmas: um
processo cujo produto final não é mais obra de arte, mas mercadoria.''.
(p. 258-9). Talvez se pudesse objetar ao crítico que a chamada
pós"-modernidade -- que não passa de um capitalismo amadurecido -- realça
em verdade com maior força não só o minimalismo de Beckett, mas
sobretudo sua forma teatral da tautologia. Ou seja, o teatro de Beckett
parece ganhar um novo significado exatamente porque o mundo da
``produtividade e da tecnologia coletiva'' já expressam apenas a si
mesmas. Nesse sentido, cabe diferenciar uma obra de arte que é
mercadoria e uma obra de arte que expressa como uma espécie de sintoma o
mundo da mercadoria. O teatro de Beckett estaria nesse segundo rol.
Porque toda arte na modernidade tende a ser uma mercadoria, mas somente
na contemporaneidade elas tendem a expressar com maior clareza em seu
conteúdo de verdade o movimento da mercadoria: pura tautologia do vazio.
É nesse sentido que o teatro de Beckett ganha novo ar de vanguarda em
pleno século \versal{XXI}.

Segundo Jameson, nas últimas peças de Beckett, o choque está

\begin{quote}
em descobrir, no coração desses espetáculos eternamente recorrentes, uma
situação empírica -- um casamento infeliz, intoleráveis lembranças da
juventude, uma estrutura familiar banal, com nomes e personagens
impossíveis de deduzir, o burguês chocando a ninhada em determinado
período, os acontecimentos biográficos pontuais que se enfrentam,
inabalavelmente, o fracasso de uma obscura e lamentável vida -- que
poderia ter fornecido a matéria para um romance soturno e realista e, em
lugar disso, persiste como fatos brutos e indigestos, aos quais a forma
reverte, várias vezes seguidas, na sua vã tentativa para dissolvê-los''.
(2005, p. 242)
\end{quote}

Para nós, o choque está no fato mesmo de Beckett colocar em cena a
tautologia da existência na própria forma de expressão -- uma tautologia
especificamente moderna. Não é o conteúdo, é a forma mesma que expressa
todo o conteúdo do absurdo de uma vida encerrada numa dimensão única,
que roda em falso. Nesse sentido, as três peças de Beckett aqui
referidas são encenadas num teatro em ruínas -- esse teatro é o mundo
contemporâneo.

Por essa via interpretativa, caberia talvez perguntar em que medida
realmente o romance é a forma estética da vida burguesa. Luckács (2009)
coloca o romance como a forma estética burguesa, que expressa o
afastamento com o fundamento simbólico"-religioso que atravessava as
formas estéticas anteriores: ``pois a forma do romance, como nenhuma
outra, é uma expressão do desabrigo transcendental''. (p. 38). Porque o
fundamento simbólico"-religioso que a modernidade foi corroendo era o
abrigo transcendental da vida social em seu todo, que se manifestava
também nas expressões estéticas. Por isso somente na modernidade se diz
que a obra de arte alcança uma autonomia, somente na modernidade ela se
livraria do fardo de expressar diretamente o fundamento transcendente da
sociedade. Mas, é possível sustentar, essa autonomia é apenas aparente e
não é total, pois a arte vai se libertar de prestar contas com um
universo simbólico transcendente, mas não de prestar contas
inconscientemente com o universo transcendental da vida social burguesa.
A arte continua entranhada na sociedade, mas numa sociedade que se
pretende livre e da máxima expressão do indivíduo.

Se a epopeia é o ápice da congruência entre interior e exterior, entre
eu e mundo, entre alma e ação, enquanto a tragédia e a filosofia já são
sintomas da cisão entre esses pares, a Igreja teria dado origem a uma
espécie de ``nova \emph{polis''}, no sentido de pretender ser esse elo
de unificação, de fundamento social total. De modo que em:

\begin{quote}
Giotto e Dante, em Wolfram de Eschenbach e Pisano, em São Tomás e São
Francisco o mundo voltou a ser uma circunferência perfeita, abarcável
com a vista, uma totalidade: o abismo perdeu o perigo das profundezas
efetivas, mas todas as suas trevas, sem nada a perder da luz sombria,
tornaram"-se pura superfície e assim se inseriram à vontade numa unidade
integrada de cores (2009, p. 35).
\end{quote}

Embora a unidade inicial do mundo helênico não pudesse mais ser
encontrada, o mundo simbólico"-mítico"-religioso que se constituiu
posteriormente até a modernidade não deixam de ser formas fundamentais e
totalizantes a dar sentido ao mundo.

Para Luckács, o romance é exatamente essa ``epopeia de uma era para a
qual a totalidade extensiva da vida não é mais dada de modo evidente,
para a qual a imanência do sentido à vida tornou"-se problemática, mas
que ainda assim tem por intenção a totalidade.'' (\emph{Idem}, p. 55). O
indivíduo épico -- o que já é um contrassenso, já que ``o herói da
epopeia nunca é, a rigor, um indivíduo'' (\emph{Ibidem}, p. 67) -- o
herói do romance, nasceria de um alheamento em relação ao mundo. Num
mundo intrinsecamente homogêneo, como o da Antiguidade Clássica, os
homens não diferem qualitativamente entre si: ``claro que há heróis e
vilões, justos e criminosos, mas o maior dos heróis ergue"-se somente um
palmo acima da multidão de seus pares, e as palavras solenes dos mais
sábios são ouvidas até mesmo pelos mais tolos.'' (\emph{Ibidem}, p.
66-67). A vida própria da interioridade, o desabrochar da importância
social de uma individualidade só é possível e necessária, ``quando a
disparidade entre os homens tornou"-se um abismo intransponível''.
(\emph{Ibidem,} p. 67). Ou seja, é aí que se funda a chamada
Individualidade, o desgaramento do sentido comunitário contido no
simbólico"-mítico"-religioso. O traço essencial da epopeia, ressalta
Luckács, é o fato de seu objeto não ser o destino pessoal, mas o de uma
comunidade.

\begin{quote}
E com razão, pois a perfeição e completude do sistema de valores que
determina o cosmos épico cria um todo demasiado orgânico para que uma de
suas partes possa tornar"-se tão isolada em si mesma, tão fortemente
voltada a si mesma, a ponto de descobrir"-se como interioridade, a ponto
de tornar"-se individualidade''. (\emph{Idem}, p. 67).
\end{quote}

Lukács viu em Dante ``uma transição histórico"-filosófica da pura epopeia
para o romance (\emph{Ibidem}, p. 69), porque nele ``seus personagens já
são indivíduos que resistem conscientemente e energicamente a uma
realidade que a eles se fecha'', tornando"-se ``verdadeiras
personalidades''. (\emph{Ibidem}, p. 69). O romance, como diz Lukács,
``é a forma da aventura do valor próprio da interioridade; seu conteúdo
é a história da alma que sai a campo para conhecer a si mesma, que busca
aventuras para por elas ser provada e, pondo"-se à prova, encontrar a sua
própria essência.'' (\emph{Ibidem}, p. 91). Trata"-se assim de uma busca
de valores autênticos num mundo degradado.

Lucien Goldmann, no esteio de Luckacs e René Girard, levanta a hipótese
de uma sociologia do romance segundo a qual a forma romanesca parece ser
``a transposição para o plano literário da vida cotidiana na sociedade
individualista nascida da produção para o mercado''. Ou seja, existe uma
relação intrínseca entre a forma literária do romance e a ``relação
cotidiana dos homens com os bens em geral e, por extensão, dos homens
com os outros homens, numa sociedade produtora para o mercado.'' (1986,
p. 36).

A análise materialista de Goldmann -- que tem o mérito de relacionar sua
sociologia do romance com o conceito de fetichismo da mercadoria de Marx
e de notar a tendência existente na forma"-social capitalista a ``uma
diminuição progressiva da ação da consciência sobre a vida econômica.''
(\emph{Idem}, p. 45) -- relaciona, portanto, forma literária e forma
social, que ele entende como de mercado, em que os bens passam pelo
mercado. Nesse sentido, a forma"-social a qual ele pretende relacionar a
forma estética nos aparece hoje como a sociedade mercantil numa fase de
seu desenvolvimento, em que a degradação das relações qualitativas se
dão pela preponderância dos valores de troca. Para Goldmann,

\begin{quote}
no plano consciente e manifesto, a \emph{vida econômica} se compõe de
pessoas orientadas exclusivamente para os valores de troca, valores
degradados, aos quais se juntam na produção alguns indivíduos -- os
criadores em todos os campos -- que permanecem orientados essencialmente
para os valores de uso e que por isso mesmo se situam à margem da
sociedade e se tornam \emph{indivíduos problemáticos.} (\emph{Ibidem},
p. 38).
\end{quote}

Quem são os indivíduos problemáticos? Para Goldmann, são aqueles cujo
``pensamento e comportamento permanecem dominados pelos valores
qualitativos.'' (\emph{Ibidem}, p. 47). O indivíduo problemático também
tem como característica a vivência da contradição entre o
``individualismo como valor universal engendrado pela sociedade burguesa
e as limitações importantes e penosas que essa sociedade impunha na
realidade às possibilidades de desenvolvimento dos indivíduos.''
(\emph{Ibidem}, p. 49).

A questão que se pode colocar é se esse \emph{indivíduo problemático},
típico do romance, não é fruto da tensão entre o indivíduo e a própria
sociedade no caso específico dos transtornos da modernidade num
determinado nível de seu desenvolvimento. Um indivíduo que ainda vive,
como diz Goldmann, um mundo ``de conformismo e convenção'' e empreende
uma ``busca por valores autênticos'' (\emph{Ibidem}, p. 24), ao mesmo
tempo que pretende romper as amarras de uma sociedade que represa seu
pleno desenvolvimento. Ora, o mundo convencional dito burguês é cada dia
dinamitado, e a busca pelo qualitativo na contemporaneidade já não é uma
tensão presente no social para se expressar na arte. Dito de outro modo,
o \emph{indivíduo problemático} do romance está resolvendo seus
problemas com a sociedade burguesa.

Nesse sentido, a tese de que se deve analisar a forma literária ou
estética em estreita relação com a forma"-social permanece válida como
ideia geral, mas o problema é o entendimento da forma"-social que deve
ser enriquecido, porque a forma"-social desenvolveu, amadureceu novos
elementos na sua tendência totalizante que na época de Luckács e
Goldmann ainda não tinham se mostrado com a força da contemporaneidade.

É assim que arriscamos afirmar: se o romance ``é a epopeia do mundo
abandonado por deus'' (\versal{LUCKÁCS}, 2009, p. 89), o teatro de Beckett se
revela a epopeia do mundo em que a mercadoria se torna o simbólico que
medeia as relações sociais e o fundamento da subjetividade entregue à
segunda natureza. O teatro de Beckett é a epopeia de uma comunidade
atrelada ao movimento cotidiano do vazio, uma comunidade empanturrada
com seus progressos destrutivos, uma comunidade em que os indivíduos são
consumidos pela forma"-sujeito e já não se mostram como seres
\emph{problemáticos.}

Se no universo do romance o mundo abandonado por deus ainda gestava a
subjetividade em busca do sentido perdido, em busca da falta de
substância, no teatro de Beckett o sentido não se apresenta mais como
sentido a ser buscado, porque, como diz Adorno, é o próprio sentido que
é posto em questão. A busca de sentido do romance só encontrava solo
fértil porque a própria vida social burguesa ainda era entrecruzada com
camadas das formas anteriores que ela precisava corroer para seguir em
frente. A busca de sentido que se vê em cena no romance tem muito dessa
tensão de uma forma"-social e de uma forma"-sujeito em movimento,
corroendo a realidade exterior não"-idêntica. A busca do sentido do
romance se dá no espaço de não"-identidade entre conceito e realidade, um
espaço que o avançar da vida social moderna foi tornando exíguo. Com uma
proximidade maior com o verdadeiro \emph{sentido} burguês na
contemporaneidade, com a realização tendencial da forma"-sujeito
burguesa, com a diminuição do espaço entre conceito e realidade, o
\emph{sentido} moderno, a Individualidade e a liberdade modernas podem
desnudar"-se como de fato se apresentam em seu conceito: como vazio
tautológico que se entranha da vida social cotidiana que gira em falso.

Nesse sentido, a consideração de Luckács de que o romance é a forma de
expressão do mundo burguês, desse sujeito alheado do mundo, parece se
relacionar com a época de desenvolvimento da forma social e subjetiva
burguesa, enquanto ainda são formas entrecruzadas com formas anteriores
-- talvez fosse nesse entrecruzamento que aflorava o chamado indivíduo
problemático. A impressão de que todos são livres de qualquer
transcendente, de qualquer conteúdo supraindividual, de que todos estão
livres de quaisquer laços para concorrer no mercado é apenas um lado
aparente da questão da defesa moderna da Individualidade. Como vimos, a
Individualidade é essa forma moderna de subjetividade que veste a
máscara de caráter do sujeito burguês. No fundo essa liberdade moderna é
mediada subrepticiamente pelo \emph{a priori} tácito da lógica mercantil
que vai se desdobrando no seio social. A liberdade moderna é mediada
portanto por esse fantasma que medeia as relações mas que não se
apresenta como constrangimento de uma vida de comunidade. No fundo, o
capitalismo não rompeu com todas as formas de comunidade, ele fundou a
sua forma de comunidade, ele criou uma tendência à perfeição e à
completude de seu sistema social. Só que a comunidade é mediada por
deuses e mitos que não são tidos como tais. Todos os sujeitos agem por
si mesmos como se agissem pela sua autodeterminação. O \emph{a priori}
tácito da forma"-mercadoria não é visto como aprisionamento explícito. Do
mesmo modo, a sociedade burguesa não parece ter criado ``a primeira
forma de consciência radicalmente anestética.'' (\versal{GOLDMANN}, 1986, p. 55),
porque a forma"-social burguesa tem ela própria a sua estética, a
estética da mercadoria, que não deve ser confundida com um estética de
obras vendáveis no mercado. Uma estética da mercadoria significa muito
mais a materialização do espírito da forma"-sujeito nas próprias obras,
mesmo naquelas que se pretendem questionadoras, mesmo naquelas que não
se pretendem vendáveis.

No teatro clássico, Sófocles, por exemplo, põe conscientemente em cena a
ação dos deuses e dos homens elevados. O inconsciente em termos
artísticos joga um papel quase nulo, porque o autor tem consciência dos
constrangimentos sociais transcendentes aos quais está submetido e aos
quais seu teatro deve prestar contas. Na arte moderna, por outro lado, o
artista é livre, não tem necessariamente consciência dos
constrangimentos sociais que vê muitas vezes somente nas relações de
opressão de classe. A autonomia da arte moderna significa que ela não
tem de prestar contas com qualquer transcendente supraindividual -- como
era o fundamento simbólico"-mítico"-religioso -- ao contrário, quanto mais
a sociedade burguesa amadurece, mais o artista pode dar asas a seu
interior pretensamente rico, à sua subjetividade pretensamente livre de
prestar contas com o social. Nos termos dessa liberdade, o inconsciente
joga um papel mais destacado -- por isso alguns significados artísticos
escapam dos próprios artistas. Por essa liberdade -- que é apenas
relativa e superficial -- o artista não tem obrigação de expressar
diretamente os constrangimentos de uma coletividade social. Ele não tem
consciência de que os constrangimentos sociais que eram transcendentes
agora são transcendentais, ubíquos, terrenos, que dormem entre os homens
sussurrando"-lhes desavenças.

Diante disso, poderíamos dizer que não é mais o romance a forma estética
da sociedade burguesa, ele o foi num período de seu desenvolvimento --
que somente hoje podemos ver como ainda imaturo. A forma estética da
forma"-social burguesa desenvolvida, quanto mais se aproxima de seu
conceito, é um novo tipo de epopeia, que não tem nada a ver com a
clássica. É aquela que coloca novamente em cena, não o destino pessoal,
mas o de uma comunidade. Mas uma comunidade de átomos flutuantes envolta
numa dinâmica social fundada na tautologia do vazio da mercadoria. A
forma do teatro de Beckett expressa a forma"-sujeito burguesa não mais em
desenvolvimento, não mais entrecruzada e carregando o peso de formas de
subjetividade anteriores. Ele é uma alegoria da realização plena da vida
social burguesa e seu sujeito no auge, um auge que culmina com sua crise
de dessubstancialização. Portanto, o teatro de Beckett não é apenas um
teatro da crise do sujeito moderno, mas um teatro em que crise e
realização se confundem. Porque no teatro de Beckett a crise não é
devida a um defeito, a uma impropriedade, a um desajuste superável, mas
ao fato mesmo de uma irracionalidade ter alcançado seu mais alto nível
perfeição.

Assim, a verdadeira forma estética burguesa é essa forma que coloca o
sentido em questão, que coloca em cena a própria tautologia abocanhando
a vida social em suas entranhas. É isso que o teatro de Beckett encena.
O fato de as peças se relacionarem com um mundo pós"-catástrofe da guerra
não impede em nada que se possam lê-las -- ao contrário -- com
referência ao mundo catastrófico da realização da lógica mercantil. A
forma da repetição, do nada de novo a dizer, a prisão na segunda
natureza, na pura imanência, a perda de qualquer outra dimensão que não
seja a da realidade imediata, do silêncio que atordoa e não gesta
reflexão, que não abre brecha para o entendimento do absurdo da vida, o
silencio que atordoa pela sua vacuidade, simplesmente por ser uma
continuidade do silêncio da própria existência, em suma: o ruir da vida
como narrativa exprimível. O silêncio atordoa pelo vazio que transmite,
não pela expressividade contida. É o grau zero do silêncio, quase o
inorgânico.

Comparativamente ao teatro de Beckett, a forma teatral de Brecht faz
parte ainda de um tempo em que a forma"-social e a forma"-sujeito
burguesas se apresentavam imaturas. Porque agora, já não cabe somente
uma tomada de partido pelos oprimidos que só dispõem de seu trabalho e
que são praticamente coagidos a vendê-la -- e em geral, em Brecht, esse
é o nível da apreensão do constrangimento social. A forma teatral que
põe em cena os mais frágeis economicamente fazia parte de um momento em
que as classes se distinguiam não só materialmente. O teatro de Brecht
pressupõe como ação interna a luta de classes, portanto, uma vida social
capitalista com horizontes. À medida que o capitalismo se desenvolve nos
termos de seu conceito, de sua lógica interna, à medida que realiza seu
ser como \emph{a priori} da vida, como espírito que tende ao universal,
como fundamento sagrado da vida, as diferenças entre as classes passam a
ser cada vez mais a posição perante a riqueza mercantil que não é vista
mais como estranhada. As diferenças subjetivas se amainam, e é a
mercadoria que se torna medida da existência\footnote{É digno de nota
  que Goldmann tenha evidenciado a impossibilidade de nascimento de uma
  cultura autêntica da parte do proletariado, já que ele já estava
  integrado na sociedade mercantil. (\versal{GOLDMANN}, 1986, p. 44).}. O teatro
de Brecht perde um pouco de seu encanto nesse momento. Por outro lado, o
teatro de Beckett ganha em esplendor na sua forma quanto mais se
desenvolve a forma social mercantil que deixa o rastro de vazio e
tautologia na vida social. Não se trata simplesmente de jogar o teatro
de Brecht na lata da história, mas de vislumbrar o teatro de Beckett
como portador de uma expressividade que ganha realce de estética da vida
social mercantil amadurecida, sem que o próprio autor o tivesse
desejado.

É importante esclarecer que não há como defender o teatro de Beckett
como estilo positivo, porque seu teatro é o desabar do estilo, é a forma
do desabar da estética encenada como estética da tautologia. É a
libertação da forma perante o conteúdo. Ou a vitória da forma sobre o
conteúdo que se expressa diretamente na forma da tautologia. Talvez em
Beckett tenhamos um exemplo grandioso da relação mais intrínseca entre
forma"-social e forma estética, quando se pode perceber, cavando em meio
às ruínas da forma estética, as entranhas da vida social. Decerto não se
pode afirmar que Beckett tenha desejado deliberadamente fazer uma
alegoria da vida social burguesa desenvolvida. Mas talvez tais sinais se
manifestaram nas brechas do inconsciente artístico, pois a construção da
destruição na forma teatral é por demais chocante. Ele não põe em cena
um cenário destruído pela guerra com pessoas buscando sua dignidade, sua
sobrevivência sobre escombros, uma nova reconstrução, uma busca de
sentido em meio à catástrofe. Não é a dignidade humana perante a guerra
ou a catástrofe. Mas do que o pós"-guerra, seu teatro pode ganhar maior
riqueza se lido como alegoria da forma social burguesa desenvolvida,
amadurecida. É o sujeito que está em ruínas e que leva a vida sobre
ruínas. Seguindo a alegoria no sentido de Benjamin, o teatro de Beckett
faz com que o espectador ou o leitor tenham ``diante de si a
\emph{facies hippocratica} da história como paisagem primordial
petrificada.'' Assim, no teatro de Beckett, ``a história, com tudo
aquilo que desde o início tem em si de extemporâneo, de sofrimento e de
malogro, ganha expressão na imagem de um rosto -- melhor, de uma
caveira.'' (\versal{BENJAMIN}, 2013, p. 176).

Como Beckett conseguiu tal proeza? Onde ele vivenciou a experiência da
tautologia, que é a marca das suas peças, de suas epopeias, de seus
dramas? Onde ele foi buscar a ideia de um sujeito em trapos? Na guerra?
Ou na sociedade que lhe apontava elementos que ele apenas sentia?

\pagebreak

\section*{No \emph{fim de partida}, esperamos \emph{Godot}~em~\emph{dias~felizes}?}
\addcontentsline{toc}{section}{No \emph{fim de partida}, esperamos \emph{Godot} em \emph{dias felizes}?
\bigskip}

\begin{flushright}
\scriptsize{\emph{Ad maiorem gloriam vacui.}}
\end{flushright}

Aqui, tentaremos encetar uma reflexão em que as três peças de Beckett,
\emph{Esperando Godot}, \emph{Fim de Partida} e \emph{Dias Felizes},
aparecem num novo contexto: o da devastação do mundo contemporâneo
colonizado pelo movimento tautológico da mercadoria. Não seria exagero
afirmar que as três peças se movem por um vazio tautológico, em que a
falta de conteúdo gira em torno de si mesma e tenta significar, embora
sem conseguir substância. É a encenação do esvaziamento do tempo, ou o
encerramento do tempo no presente contínuo que já corroeu as referências
que apontam para além do universo estabelecido da palavra e da ação. Em
vez de ver nisso um destino humano, parece"-nos mais rico estabelecer uma
relação crítica entre esse esvaziamento do mundo vivido e a forma"-social
contemporânea entendida como \emph{the waste land}\footnote{``Abril é o
  mais cruel dos meses, germina/ Lilases da terra morta, mistura/
  Memória e desejo, aviva/ Agônicas raízes com a chuva da primavera.
  [\ldots{}]'' (\versal{ELIOT}, 1922-1981, p. 89).}. Evidentemente,
devastação é aqui apreendido não somente no sentido de natureza, mas no
sentido de infertilidade das relações sociais, de potencial humano para
a construção de relações que tendam a transcender tal devastação, para o
enriquecimento dessas relações, em suma, devastação também do terreno
subjetivo.

Essas peças de Beckett são costumeira e corretamente relacionadas ao
contexto do pós"-guerra. Mas o que nos toca aqui é relacioná-las com o
contexto do século \versal{XXI}, como se elas amadurecessem juntamente com a vida
social moderna. Afinal, nada nos obriga a lê-las apenas relacionadas com
o momento histórico de meados do século \versal{XX}, pois elas não são uma obra
histórica, embora possam remeter para a história. São obras que remetem
para momentos extremos e, guardadas as devidas proporções, encaramos o
capitalismo desenvolvido tal como se apresenta contemporaneamente como
um momento extremo. Ou seja, queremos relacionar essas peças de Beckett
com a terra arrasada que está sendo deixada pela entronização da
mercadoria na vida social, uma mercadoria que guerreia contra qualquer
forma de qualidade concreta, seja qualidade de coisas concretas, seja do
tempo concreto, seja de subjetividades concretas.

Já Karl Kraus, ao esclarecer sua peça \emph{Os últimos dias da
humanidade} (1927-2003, p. 15), cuja temática é a destruição deixada
pela guerra, deixa claro que o que está em jogo é mais o estado de
declínio geral da humanidade quando ela aceita a guerra do que a guerra
em si. A barbárie é menos o ato violento materializado, e mais a
disposição subjetiva para que tal ato tenha lugar e encontre
justificativa: ``Por que não descreveria o destino trágico da humanidade
conduzida para a morte por sua falta de imaginação? -- esse problema
está no centro de meu drama.'' (p. 16). Não queremos defender a tese de
que Beckett quis pôr em cena o estado de ruínas a que tende uma vida
social e subjetiva cada vez mais em contato íntimo com a mercadoria e
seus nulos atributos, embora seu contexto do pós"-guerra seja de
crescimento econômico grande. Não queremos propor uma interpretação
correta em detrimento de outras, mas abrir as possibilidades de
entendimento desses textos. Não nos interessa tanto o que o autor quis,
mas o que ele conseguiu, muitas vezes \emph{malgré lui}. O que não
significa simplesmente projetar nossas interpretações no texto, mas
refletir se a terra arrasada representada nas peças de Beckett, se a
tautologia que resumem essas peças podem ser relacionadas com a
tautologia do vazio produzido pela mercadoria na vida social. Não se
trata de buscar nas peças uma teoria ou uma tese que lhes sejam
exteriores, mas de buscar, cavando nas entranhas dessas peças, o que se
apresenta apenas de modo latente, de modo sintomático, que permite
brechas interpretativas relacionadas com o todo social moderno.

Portanto, lemos Beckett encarando o tempo presente como tempo de
extremos, tempos de crise, não uma crise de escassez, mas de excesso de
mercadoria, de excitação e de vazio ao mesmo tempo: \emph{blasément.} É
de outra destruição que estamos falando, portanto. Essa tentativa
encontra a ideia de Sartre expressa na apresentação da Revista
\emph{Temps} \emph{Modernes} segundo a qual ``Tudo o que é escrito
possui um sentido, mesmo se esse sentido é bastante diverso daquele que
o autor sonhara'' (\versal{SARTRE}, 1999, p. 118). Seguindo Todorov, o escritor
não impõe uma tese ao dar forma a um objeto ``mas incita o leitor a
formulá-la: em vez de impor, ele propõe, deixando, portanto, seu leitor
livre ao mesmo tempo em que o incita a se tornar mais ativo.'' (\versal{TODOROV},
2009). E Beckett aqui nos incitou a essa ``ação''. O exercício é tentar
vislumbrar, no que talvez esse autor sonhara de forma distinta, por ser
outro contexto, aquilo que se realiza hoje, mas somente porque os
contextos mantêm ligação. O que liga a época de Beckett e a nossa é a
mercadoria como fundamento social. Mas em sua época, embora a crítica
social dos situacionistas em torno de Guy Debord já denunciasse a usura
do pensamento e o quanto a miséria da vida está no fato de vivermos de
uma maneira que nos escapa, ainda não se vivia a \emph{era do vazio}. A
realidade ainda tinha bastante força perante o conceito, e muitos
espaços de não"-identidade ainda existiam na própria sociedade mercantil.
Ou essa era do vazio não estava de modo tão desdobrada quanto hoje.
Mesmo hoje estando mais desdobrada, não significa dizer realizada --
seria o fim de tudo. Mas podemos certamente afirmar que, de lá para cá,
fomos perdendo substância, conteúdos, experiências, saberes, cuidados,
dignidades, éticas. Dito de outro modo, a forma"-sujeito avançou com sua
forma de subjetividade unidimensional sobre os indivíduos concretos.

Poderíamos fazer uma pergunta com base no que Simmel analisou como
decadência da cultura em meio a tantos objetos: por que tamanha riqueza
de objetos sem um desabrochar das almas no processo de contato com esses
objetos? Como explicar esse fenômeno, ``se toda cultura das coisas, como
vimos, é uma cultura dos homens, de tal modo que formando as coisas nós
mesmos nos formamos'' (\versal{SIMMEL}, 2009, p. 574)? Poderíamos
arriscar uma resposta dizendo que não há mais quase coisas no mundo que
possam, sendo coisas, ter uma dignidade,\footnote{``No reino dos fins
  tudo tem um preço ou uma dignidade. Quando uma coisa tem um preço,
  pode"-se por em vez dela qualquer outra como equivalente; mas quando
  uma coisa está acima de todo o preço, e, portanto, não permite
  equivalente, então ela tem dignidade.''
(\versal{KANT}, 1785-2005, p. 65).)}
mas sobretudo mercadorias -- o que já vimos ser fundamental distinguir.
Significa dizer que esses objetos modernos não poderiam simplesmente ser
comparados com aquelas coisas produzidas pela atividade humana na
\emph{vita activa}, coisas que condicionam a existência historicamente.
Como afirma Arendt, ``por ser uma existência condicionada, a existência
humana seria impossível sem as coisas, e estas seriam um amontoado de
artigos incoerentes, um não"-mundo, se esses artigos não fossem
condicionantes da existência humana.'' (\versal{ARENDT}, 1993, p. 17). As coisas
modernas conseguem exatamente transformar o mundo em seu mundo, ou seja,
em não"-mundo, em mero amontoado abstrato de sentido vago e fugaz, pois a
relação que se constrói com elas não se dá pela sua especificidade, mas
pelo seu caráter geral de ser portador de uma abstração --- valor.
Alguém poderá nos objetar que não existe só o valor e o tempo de
trabalho abstrato na mercadoria e sua materialidade em dinheiro, há
também sua utilidade, e que pessoas, de fato, fazem um uso concreto
delas. É possível, e é realmente assim que a consciência cotidiana
apreende, mas esse lado concreto é considerado pelo social como mero
portador daquelas abstrações, o que realmente importa.

Ao mesmo tempo, não significa que a condição humana não se dá mais nessa
relação com as coisas. Antes o contrário, o problema é exatamente que a
condição humana está condicionada por essas coisas, muitíssimas coisas
que, na modernidade, deixaram de ser simples coisas para serem pura
forma sem conteúdo.

E a compreensão de que há uma abstração na raiz da sociedade é
fundamental para compreender que a falta de substância do sujeito é um
vazio formado a partir do fato de que, se tudo se equivale, se tudo tem
uma substância igual, então tudo se nega ao mesmo tempo. Não é que
sejamos vazios pura e simplesmente, mas sim que a experiência cotidiana
fundamentada no movimento da mercadoria vai nos fazendo perder a
capacidade de fazer distinções significativas. Isso é algo típico do
caráter \emph{blasé,} como vimos com nosso \emph{homem sem qualidades.}

Poderíamos, então, dizer que as mercadorias ganharam uma guerra
histórica contra as coisas. Uma mercadoria não pode ter uma dignidade,
porque em essência ela é uma forma sem conteúdo, uma mercadoria é um
puro dispêndio de nervo, músculo e cérebro (\versal{MARX}, 1985). Uma mercadoria
é um objeto totalmente indiferente ao uso. Não importa se esse dispêndio
é feito em bomba, relógio, pão, remédios, estádios fantasmas, o
importante é que o fim último que é a valorização do dinheiro se
realize. Como pensar em coisas que tenham dignidade se elas não têm
conteúdo? Se a única relação que estabelecem entre si e com o mundo é a
da troca, ou a de pretender transmitir aos sujeitos pretensas
subjetividades? Se a cultura das coisas é uma cultura dos homens, se ao
fazer e se relacionar com as coisas formamo"-nos nós mesmos, o que somos
nós se a coisa é essa forma sem conteúdo?\footnote{Alguém poderia
  evidentemente objetar que há um exagero nessa afirmação. Poderia ser
  no caso específico. Mas o sistema é um todo e só leva em conta o todo.
  O capitalismo é a primeira forma de sociedade para a qual o caráter
  concreto do objeto, seu uso, só conta como mal necessário, porque é
  portador do que de fato interessa, o valor que se realiza em dinheiro
  -- esse é o \emph{ser em si} da coisa moderna.}

É essa a questão que podemos nos colocar perante as peças de Beckett.
Para nós, a tautologia presente em suas peças é própria do girar em
falso da sociedade baseada ela mesma numa tautologia: o processo
contínuo de valorização do dinheiro que acaba contaminando todas as
esferas sociais -- embora haja quem defenda que o consumo desmedido é
mera saciedade das necessidades intrínsecas ao ser humano que estavam
represadas.

Essa tautologia acelerou sua dinâmica no século \versal{XX}, que representou um
sempre maior ganho de importância da mercadoria, uma entronização da
mercadoria na vida social e não das coisas -- as passagens descritas por
Benjamin como ruas lascivas, onde se adorava o luxo no século \versal{XIX}, agora
parecem mero conto lúdico de um país distante.

E é desse contexto que pretendemos falar ao analisar as peças de
Beckett. Nesse sentido, não achamos que a contemporaneidade seja uma
corrupção do espírito reflexivo que o sujeito burguês ainda dava provas
nos séculos \versal{XIX} e início do \versal{XX}. Portanto, é para nós importante nuançar
a afirmação de Fábio de Andrade segundo a qual a ficção de Beckett
expressaria na linguagem ``a falência do sujeito burguês, a dissolução
dos indivíduos como sedes de reflexão, perdidos num mundo coisificado''
(2001, p. 30). Para nós, as peças de Beckett estudadas aqui não
expressam a falência do sujeito burguês, porque seríamos levados a
encará-lo como uma positividade que faliu. E não podemos encarar como
positividade uma forma de subjetividade cuja característica principal é
a razão instrumental"-mercantil, uma forma de subjetividade abstrata que
encara o mundo como massa modelável à multiplicação. Para nós, antes do
mais, as peças de Beckett expressam em negativo a vitória da
forma"-sujeito burguesa quando desdobrada, uma vitória que coincide com
sua falência, é verdade. Aquela reflexão, de que podia dar provas o
sujeito burguês, não lhe pertencia, provinha do entrecruzamento
subjetivo que temos tentado tratar. O sujeito burguês, de fato, ainda
não existiu em estado puro porque se trata de uma abstração
generalizante de subjetividade. Aquela reflexão era um amálgama de uma
subjetivação passada, construída dentro de outras mediações, e a forma
de subjetivação que tensiona para frente, que tenciona encontrar"-se com
o ideal forma"-sujeito burguesa. Porque a forma"-sujeito burguesa, se
permitirmos que ela exista em estado puro, sem tensão com uma
individualidade formada numa trama mais variegada -- o que cada dia é
mais difícil -- só pode ser uma forma de subjetividade coadunada com a
forma mercantil. E o que seria uma subjetividade puramente coadunada com
a forma mercantil senão a forma de subjetivação do próprio vazio?

Seguindo a ideia de entrecruzamento da subjetividade que temos tentado
desenvolver, podemos dizer que nesses tempos subjetivos que estamos
expondo, esteve presente em germe, rondando o indivíduo concreto, essa
forma"-sujeito. Foi essa forma"-sujeito que veio se desenvolvendo
juntamente com a sociedade calcada em outras relações que -- por mais
crítica que devamos fazer, pela dominação que justificava --, bem ou
mal, davam à pessoa um quadro de mediações que possibilitavam o
desenvolvimento de certas qualidades (não necessariamente boas, mas pelo
menos havia mais formas de julgar se eram ou não boas).

Na sociedade de consumo do capitalismo desenvolvido, a cada vez maior
efusão de mercadorias, que se transformam e nos transformam a todo
tempo, apenas dão a impressão de que se está mudando ou ``progredindo''.
Mas, no fim das contas, a sensação generalizada, mesmo que inconfessa, é
de que se patina, de que se está empanturrado de objetos (e de seus
modelos de subjetividade), de que o tempo e a vida estão mais velozes
que nós, de que a vida social tem um sentido que escorre por nossas
mãos.

É o que sentimos na leitura das peças de Beckett, especificamente
\emph{Fim de Partida} (2010a), \emph{Esperando Godot} (2005) e
\emph{Dias Felizes} (2010b). O que está em jogo nessas peças é a falta
de sentido num momento específico da história humana -- e não uma falta
de sentido atemporal --, é a falta de distinção, a falta de substância
para além da igualdade abstrata -- é assim que entendemos o sentido de
\emph{qualidade.} Os personagens de Beckett patinam no vazio tanto
quanto Ulrich. Mas Ulrich ainda tinha otimismo com o futuro que tinha
pela frente, ainda queria fazer suas experimentações e esperava viver o
desabrochar de sua individualidade na modernidade. Já os personagens
dessas três peças já experimentaram a modernidade sem que sua
individualidade se tornasse mais rica de possibilidades humanas, e agora
a vivem como ruína sem qualquer otimismo ou fé no futuro. Os personagens
dessas peças são aquilo no que se tornou Ulrich: sem qualidades
literalmente.

Tanto Ulrich quanto os personagens de Beckett, mesmo que de forma
diferente, por viverem épocas diferentes, vivem à deriva, uma deriva
contínua sem se saber bem porquê. Ulrich continua agarrado ao seu
otimismo com a razão científica, já os personagens dessas peças de
Beckett simplesmente não têm ao que se agarrar -- a não ser em seus
restolhos de linguagem, de criação repetitiva que lhes ajuda apenas a
passar o tempo, que não passa, por ser sempre igual. Os próprios
cenários em que seus restos de ação se desenvolvem são terra arrasada.
Em \emph{Fim de Partida} (2010a), a humanidade está tão destruída
que nem mais há marés, nem mesmo ratos. Diz Adorno (2003, p. 274): ``A
fase de completa reificação do mundo, na qual já não resta mais nada que
não tenha sido feito por homens, é indistinguível de um sucesso
catastrófico suplementariamente provocado exclusivamente por homens, no
qual a natureza foi anulada e depois do qual nada mais cresce''. Nem
mesmo as sementes jogadas no chão brotam -- terra infértil, onde nem
mais o abril traz o cheiro da mudança, como no poema de Eliot. O
contexto das peças de Beckett é, portanto, de um mundo em que a natureza
já não é mais natureza, é terra arrasada.\footnote{A natureza é cada vez
  mais transtornada em seus ciclos para ser tornada cada vez mais mero
  suporte para produção de mercadorias agrícolas. É \emph{wast land}
  literalmente. O desdobramento mercantil já produz para si uma
  pós"-natureza (\versal{MENEGAT}), artificial, a partir do uso da engenharia
  genética na agricultura -- que já se degenerou em agronegócio.}

Para nós, ressaltar o fato de existir uma abstração no seio da
sociedade, mas não uma abstração que fica no campo das ideias, mas uma
abstração que se realiza no concreto social, é fundamental. Porque a
abstração primordial que possibilita a troca, no dizer de Adorno (2008),
é o que abre caminho para a indiferenciação entre pessoas e coisas,
portanto a um mundo indiferenciado, já que, nessa troca, necessariamente
se abstrai essa configuração específica dos objetos para que possam,
apenas assim, ser trocados entre si. Quer dizer, a dinamização do tempo,
essência do dinheiro multiplicado, de tão frenética, acaba por não se
saciar apenas com o presente mas precisa adiantar o futuro, de modo que
tudo fica concentrado num pretenso presente contínuo. Mas, como escreve
Fábio Andrade (2010a, p. 25) sobre \emph{Fim de Partida}: ``O simulacro
de atividade frenética encobre a convicção profunda de que o fim da ação
é vazio, desprovido de finalidade, frustrante e conhecido de antemão
[\ldots{}]''. A falta de ação nessas três peças de Beckett em verdade é
uma alegoria de uma vida social em que as ações humanas contam cada vez
menos quando não vão ao encontro da marcha do todo. É a encenação da
impotência dos indivíduos num mundo em que suas ações passam cada vez
menos por um filtro crítico. Se a ação é o princípio da tragédia
clássica, como sustenta Aristóteles (2007, p. 25), e se os personagens
adquirem seus caracteres por meio das ações, a tragédia de Beckett se
funda exatamente no princípio da \emph{não ação} humana. Não se trata de
inação literal, de imobilidade literal, mas de ações submetidas a um
emaranhado totalizante que deixa pouca margem de manobra para a
liberdade das ações. A inação é também uma subjetividade estancada.

Falar em inação parece um contrassenso, já que o sujeito moderno é
pretensamente o mais ativo da história. Mas a questão fundamental é que
sua ação se dá no invólucro da vida social atrelada à dinâmica
capitalista, uma dinâmica que chega ao paroxismo na contemporaneidade,
quando, de tão rápida, acaba por consumir a si própria e a rodar em
falso, embora passe normalmente a impressão de mudança de ciclo que é,
em verdade, o ciclo da repetição. Esse ciclo da repetição está bem
resumido na frase ``O fim está no começo e no entanto continua"-se,''
dita por Hamm em \emph{Fim de Partida} (2010a, p. 113). O
expresso nessa frase pode significar simplesmente: \emph{o fim de tudo
está começando, mas seguimos em frente.} Mas pode também significar: o
ciclo repetitivo que significa o fim, o objetivo, está sempre
recomeçando do começo e nos consumindo. Essa frase tautológica é quase
uma paródia de um palíndromo, mas um palíndromo social, que não é só
textual ou estilístico. É como se a vida social estivesse literalmente
presa num palíndromo e só pudesse ir do começo ao fim e do fim para o
começo, como o título em forma de palíndromo do filme de Guy Debord:
\emph{In girum imus nocte et consumimur igni} (Rodamos na noite e somos
consumidos pelo fogo). Um aprisionamento que se resume na fala do
personagem central de uma novela de Beckett intitulada \emph{O Fim}
(2006, p. 82): ``[\ldots{}] sem coragem de terminar nem força para
continuar.''

E a vida se torna uma tautologia inexorável, não apenas no trabalho, que
é repetitivo, mas nas relações cotidianas mais diversas. Uma
inexorabilidade que também se expressa na troca de falas entre Clov e
Hamm, em \emph{Fim de Partida} (2010a): ``Vamos parar com esse jogo'',
diz Clov. Hamm não diz nada além de um ``Nunca!''. (p. 121) Para Adorno,
somente quando o processo que se inicia com a transformação do trabalho
em mercadoria passa a permear todos os homens -- o que transforma cada
impulso em objeto, em algo comensurável e, deste modo, trocável -- é que
a vida passa a reproduzir"-se de acordo com a produção. Sua organização
integral exige uma união de mortos. A vontade de viver encontra"-se na
dependência da negação da vontade de viver (\versal{ADORNO}, 1992, p. 201).

E essa ``união de mortos'', esse rodar em falso, esse vazio tautológico,
parece estar em cena com Beckett. Seus personagens são como \emph{Os
homens ocos} de Eliot (1981, p. 117):

\begin{quote}
\forceindent{}Nós somos homens ocos

Os homens empalhados

Uns nos outros amparados

O elmo cheio de nada. Ai de nós!

Nossas vozes dessecadas,

Quando juntos sussurramos,

São quietas e inexpressas

Como o vento na relva seca

Ou pés de ratos sobre cacos

Em nossa adega evaporada

Fôrma sem forma, sombra sem cor,

Força paralisada, gesto sem vigor;
\end{quote}

Nesse mesmo sentido, mesmo o ambiente das três peças sendo de pobreza,
de terra arrasada, mesmo o próprio cenário se limitando ao mínimo, a um
mesmo ambiente austero e destruído, mesmo a própria comida sendo mínima
-- em \emph{Esperando Godot,} é a cenoura que se repete, em \emph{Fim de
partida} é um biscoito --, mesmo errantes e mendigos, sofrendo as
violências físicas do mundo da rua, o que os personagens mendigam está
muito além de algo material. Essa penúria material ou essa mutilação
física -- no caso de \emph{Fim de Partida}, até os personagens são
mutilados fisicamente -- carrega"-nos para uma penúria e uma mutilação
muito mais aprofundadas que dizem respeito a uma penúria, uma mutilação,
uma danificação (\versal{ADORNO}, 1992) da própria existência subjetiva. Os
personagens de Beckett erram numa mendicância do próprio \emph{Eu}, eles
mendigam a própria condição humana num meio social hostil e vago:
``Estamos sempre achando alguma coisa, não é, Didi, para dar a impressão
de que existimos'' (\versal{BECKETT}, 2005, p. 138), diz Estragon em
\emph{Esperando Godot} sem receber resposta. A marca corrente nas peças
é não haver uma discussão sobre a própria condição em que os personagens
se encontram. Os personagens parecem de fato átomos flutuantes na
miséria material e subjetiva. Em \emph{Fim de Partida,} diz Hamm: ``Não
estamos começando\ldots{} a\ldots{} significar alguma coisa?'', ao que Clov
responde sem qualquer ilusão: ``Significar? Nós, significar! (\emph{Riso
breve}) Ah, essa é boa!'' (\versal{BECKETT}, 2010a, p. 73-74).

Quando em \emph{Fim de Partida} Hamm fala: ``Mas nós respiramos,
mudamos! Perdemos os cabelos, os dentes! A juventude! Os ideais!'' não é
para encetar uma discussão sobre essas ideias. Não há diálogo sobre
isso. Isso é apenas um restolho de reflexão que se acaba num fragmento.
A reflexão, quanto existe, é apenas um fragmento. Cada sujeito parece
querer ver no outro uma bengala para sua existência. Os pares de
personagens Estragon e Vladimir, Pozzo e lucky em \emph{Esperando
Godot;} Hamm e Clov, Nagg e Nell, em \emph{Fim de Partida;} e Winnie e
Willie em \emph{Dias Felizes} parecem ter uma relação de
interdependência que não é por laços ditos sociais ou até de família.
São frágeis bases de apoio subjetivo recíprocos -- o que faz do resto de
poder que tentam dar a ver mero jogo de cena dentro da cena. As falas
acabam se tornando \emph{falas}, como ``deixas'' de teatro literalmente,
mas na vida real. É como se dentro da peça se encenasse o tempo todo
para se conseguir seguir adiante. Em \emph{Fim de Partida} (2010a), Clov
chega a perguntar para que ele próprio serve. Hamm responde: ``Para me
dar deixas''. (p. 101). E Winnie de \emph{Dias Felizes} (2010b) diz
claramente que o que lhe dá forças é ter o marido a quem falar, não com
quem falar, embora nem sempre seja ouvida. A simples presença dele já é
seu ``paraíso na terra'' (p. 44): ``Enquanto se você morresse --
(\emph{Sorriso}) -- ah, o grande estilo! -- (\emph{fim do sorriso}) --
ou fosse embora, me abandonasse, o que eu faria então, o que poderia
fazer o dia inteiro [\ldots{}]?'' (\versal{BECKETT}, 2010b, p. 37). Segue
ela: ``[\ldots{}] é tudo que preciso, apenas sentir sua presença ao
alcance da voz e entre os que, presumivelmente, vivem [\ldots{}]'' (p.
41).

Diz Hamm em tom profético:

\begin{quote}
Um dia você ficará cego, como eu. Estará sentado num lugar qualquer,
pequeno ponto perdido no nada, para sempre, no escuro, como eu. (Pausa)
Um dia você dirá, estou cansado, vou me sentar, e sentará. Então você
dirá, tenho fome, vou me levantar e conseguir o que comer. Mas você não
levantará [\ldots{}] Estará rodeado pelo vazio do infinito, nem
todos os mortos de todos os tempos, ainda que ressuscitassem, o
preencheriam, e então, você será como um pedregulho na estepe. (\versal{BECKETT},
2010a, p. 77-78)
\end{quote}

Mas a isso, como para fugir de qualquer reflexão, Clov responde com uma
fala deslocada: ``Não posso me sentar'' (\emph{Idem}, p. 78). Os
personagens fogem o tempo todo de qualquer reflexão, eles passam de fato
o tempo. Ou seja, não importa, de fato, para os personagens pensar no
profundo oco que os consome, um vazio que se apresenta em tal grau que
já até as estratégias de preenchimento sofrem usura. A própria reflexão
mínima que demonstram parece mero \emph{instrumento} para passar o
tempo, meras deixas para rodar o tempo que sempre volta a seu começo, ou
para preencher o tempo vazio dentro do qual levam suas vidas, ou são
arrastados por ela. Clov pergunta: ``O que você tem hoje?'', responde
Hamm: ``Sigo meu curso''. (\emph{Ibidem}, p.84).

Em \emph{Esperando Godot} (2005), uma estrada no campo, uma só árvore
cinzenta. Um entardecer. Por que não uma aurora? Por que dar esse
sentido de ocaso que se complementa com a primeira frase do personagem:
``Nada a fazer'' (p. 17)? Pelo contexto geral da peça, esse ``Nada a
fazer'' não está relacionado simplesmente com a impotência face a sua
bota, com a qual luta, mas a uma impotência perante a vida como um todo,
vida que os aprisiona dentro de uma espera sem fim. E tanto a questão da
busca de sentido, quanto o sentido da espera precisariam ser encarados
como historicamente determinados. Como diz Andrade (2005, p. 9): ``Houve
quem buscasse um Deus oculto em Godot; outros, uma eterna busca da
condição humana; outros ainda procuravam alusões mais diretas a um
contexto histórico determinado.'' Quanto a nós, não buscamos alusões a
um contexto determinado, mas ler o todo das peças como metáfora do
desdobramento de uma época histórica. Do mesmo modo, não seria erro ver
em Godot a condição humana ou deus. Mas não no sentido transhistórico,
não no sentido de um deus tradicional, mas somente se o relacionarmos
com a condição humana moderna, com o deus moderno, que dirige a vida na
terra sem que os homens mais ateus consigam notar\footnote{Essa
  dificuldade em notar o deus moderno, já que a vida moderna é marcada
  pelo esvaziamento do céu, é sintomática no próprio Beckett, que também
  não sabia quem era Godot: ``Se eu soubesse quem era Godot, teria dito
  isto na peça.'' (\versal{MARFUZ}, 2014, p. 9)}.

Não nos interessa aqui uma reflexão sobre o sentido transhistórico da
espera -- como pretende Esslin (1967) em seu famoso livro sobre o teatro
do absurdo -- ou da busca por sentido, não se trata de um problema
existencial em si, mas de um problema existencial determinado
historicamente. Encaramos, portanto, essa busca de sentido e o vazio que
preenche a espera como tipicamente modernos, pois é somente na
modernidade que uma tautologia toma conta do social, é somente na
modernidade que a troca de equivalentes tende a se tornar universal.
Portanto, ``A eterna repetição, sem significação nem direção'' (\versal{PRONKO},
1963, p. 44) deve ser relacionada a nosso ver a uma forma"-social
específica. É somente na modernidade que os laços sociais são
paulatinamente mediados pela forma vazia que possibilita a troca e
elimina as especificidades de pessoas e coisas. Inclusive a
especificidade do tempo é abolida. Nas três peças, os personagens estão
aprisionados no tempo presente e são obrigados a esperar, não podem ir
embora, estão esperando Godot (\versal{BECKETT}, 2005 p. 28-29):

\begin{quote}
\forceindent{}Estragon: E se não vier.

Vladimir: Voltamos amanhã.

Estragon: E depois de amanhã.

Vladimir: Talvez.

Estragon: E assim por diante.
\end{quote}

O aprisionamento no tempo presente, num tempo sem qualidades
específicas, característico somente da sociedade moderna que tornou o
tempo abstrato, desnorteia os personagens (\emph{Idem}, p. 30-31):

\begin{quote}
\forceindent{}Estragon: Tem certeza que era hoje à tarde?

Vladimir: O quê?

Estragon: Que era para esperar.

Vladimir: Ele disse sábado. (\emph{Pausa}). Acho.

Estragon: Depois do batente.

Vladimir: Devo ter anotado. (\emph{Procura nos bolsos repletos de
porcarias de todo tipo})

Estragon: Mas que sábado? E hoje é sábado? Não seria domingo? Ou
Segunda? Ou sexta?
\end{quote}

Os personagens não sabem nem mesmo se no dia anterior estiveram no local
da espera. Esse apagamento da memória é algo marcante nas três peças.
Não há passado nem futuro, apenas presente. Quase uma paródia do
presenteísmo contemporâneo que pretende encerrar toda a vida num
presente eterno, um presente que pretende ser realizador de todos os
desejos do Eu"-imperioso.

Nos raros momentos de reflexão sobre a vida que levam, quando há uma
fala que pode parecer uma reflexão sobre a existência que estão levando,
vê-se logo que é um fragmento desconexo, que, mais do que uma reflexão,
é uma escora para ajudar a passar o tempo da espera. É como se cada um
dos personagens estivesse ali apenas para lembrar ao outro que estão
esperando Godot em quem apostam que resolva todos os problemas. Diz
Vladimir:

\begin{quote}
O certo é que o tempo custa a passar, nestas circunstâncias, e nos força
a preenchê-lo com maquinações que, como dizer, que podem, à primeira
vista, parecer razoáveis, mas à quais estamos habituados. Você dirá:
talvez seja para impedir que nosso entendimento sucumba. Tem toda razão.
Mas já não estaria ele perdido na noite eterna e sombria dos abismos sem
fim? É o que me pergunto às vezes. Está acompanhando meu raciocínio?
(\emph{Ibidem}, 2005, p. 161).
\end{quote}

O angustiante é que o raciocínio não é acompanhado, não é seguido,
desenvolvido -- tensionado. Não há qualquer resposta que possa dar
prosseguimento a essa conversação que poderia se revelar aclaradora,
porque não estamos diante das \emph{individualidades problemáticas} do
romance. É como se houvesse uma impossibilidade de se apreender o estado
a que chegaram, ou foram levados a chegar. As maquinações não impedem
que o entendimento sucumba. Em verdade, as maquinações para passar o
tempo, para preencher o tempo esvaziado, são consequência de o
entendimento já ter sucumbido, ou estar em vias de sê-lo. O restolho é
também de condição reflexiva. Uma vida encerrada temporalmente, sem
referências com as quais se possa estabelecer uma relação para uma
construção subjetiva mais rica e minimamente sólida pode levar também a
um apagamento das qualidades, das distinções. ``E se a gente se
enforcasse?'' (\emph{Ibidem}, p. 34), diz Estragon, dando uma ideia para
passar o tempo da espera, como se fosse apenas outra ideia qualquer. Uma
ideia que logo se mostra mera estratégia para passar o tempo, aliás,
bastante recorrente na peça:

\begin{quote}
\forceindent{}Vladimir: Então, o que fazemos?

Estragon: Nada, é o mais prudente.

Vladimir: Esperar para ver o que ele nos diz.

Estragon: Quem?

Vladimir: Godot.

Estragon: Isso.

Vladimir: Vamos esperar até estarmos completamente seguros.
(\emph{Ibidem}, p. 36-37).
\end{quote}

Ao mesmo tempo, nem sabem o que querem de Godot, embora dele pareça
depender a própria existência em ruínas na terra exaurida. Por isso
Estragon pergunta se não estariam amarrados (\emph{Ibidem}, p. 40-44).
Vladimir acha que não, embora se sinta o peso de uma amarra abstrata, a
amarra do vazio da existência, a amarra de rodar em falso continuamente,
a amarra da impossibilidade de construir um sentido impulsionador de uma
transcendência a essa imanência sufocante. É esse vazio da existência
que torna uma compreensão da relação entre Pozzo e Lucky como metáfora
da relação entre servo e senhor um tanto superficial para o contexto da
peça. Do mesmo modo, a leitura de Jameson (2005) que vê na cena uma
alegoria de uma questão nacional -- com Pozzo representando o Império
Britânico ``em sua relação com as colônias'', e Lucky como alegoria da
Irlanda -- parece pouco para uma leitura que traga forma e conteúdo da
peça para o contemporâneo. O contexto geral transmite muito mais um
profundo desterramento, um mal"-estar de existir que não é devido a
qualquer repressão das pulsões, mas devido à ausência de referências, de
determinados limites, sejam temporais ou espaciais, com as quais os
indivíduos possam friccionar sua própria existência. Isso faz com que a
própria relação de poder pessoal expressa entre Pozzo e Lucky torne"-se
irrisória, uma paródia, mera cena dentro da cena frente ao poder
impessoal que os conduziu àquelas paragens onde só há o nada. Um nada em
relação ao qual se deve tentar construir alguma subjetividade. Nesse
sentido, por mais que o autor não relacione sua leitura com uma época
histórica determinada, não deixa de ser interessante a reflexão de
Pronko de que as relações ``aviltantes entre mestre e escravo os reduz a
ambos à animalidade e à impotência.'' (1963, p. 55). Mais do que uma
metáfora de servo e senhor, a cena parece representar a possibilidade de
uma regressão em larga escala em tempos de crise dos automatismos da
segunda natureza, uma regressão tal que o ser humano é rebaixado à
condição de porco -- é como é tratado Lucky --, com uma corda no
pescoço, já em carne viva de tanto a corda ser puxada. Isso não é
servidão no sentido tradicional, não é relação de dono para animal, vai
além disso. Mesmo com o personagem Lucky o sentido de humanidade se vê
embaçado, uma vez que não parece tensionar com sua situação até mesmo
subanimal, embora seja bastante complexa sua personagem. Ele que pode
dançar, recitar, cantar e pensar -- mas de modo automático, dependendo
da ordem de seu amo. O pensamento que Lucky é ordenado a expressar nada
mais encerra do que um emaranhado desconexo. São palavras que estão no
texto sem se relacionar, sem construírem sentido, quase que para
corroborar a própria desconexão dos sujeitos, ou a ideia de que não há
pensamento coerente na peça porque o contexto espaço"-temporal das
personagens não permite senão o automatismo e ``A eterna repetição, sem
significação nem direção'' (\versal{PRONKO}, 1963, p. 44).

Já a humilhação a que o personagem Hamm de \emph{Fim de Partida} submete
seus pais e Clov, mais do que uma tensão impulsionadora de uma
resistência, apenas contribui para passar o tempo: ``[\ldots{}] a
capacidade de reação aparece aqui neutralizada por um conformismo e uma
desestruturação interior, uma incapacidade de protesto que se traduzem
no aspecto fisicamente mutilado das personagens''. (\versal{ANDRADE}, 2010a, p.
24).

Esse embaçamento dos limites da distinção se delineia também com
Estragon, que leva um chute de Lucky e cospe"-lhe na cara -- uma reação
que mais se aproxima do asco sentido pela personagem que aparece como
figura humana ainda mais degradada dentro da degradação; e com Vladimir,
que questiona Lucky por ele querer deixar um patrão tão bom quanto
Pozzo. Esse embaçamento, essa indistinção está também presente no menino
que traz o recado de Godot. Ele não sabe nem mesmo se é feliz ou não,
embora seja o felizardo que vive com Godot. A incapacidade distintiva
culmina na avaliação da cena subanimal de Lucky. Vladimir diz
simplesmente que ``ajudou a passar o tempo''; já Estragon diz que
``teria passado igual'' (\versal{BECKETT}, 2005, p. 93). E a respeito da Kakânia,
explica o narrador de Musil:

\begin{quote}
A marcha do tempo nos domina. Andamos com ela dia e noite, e fazemos
dentro dela todo o resto; nos barbeamos, comemos, amamos, lemos livros,
exercemos nossa profissão, como se quatro paredes estivessem imóveis; e
o inquietante é saber que as paredes se movem, sem notarmos nada, lançam
seus trilhos à frente como longos fios sinuosos, tateiam, sem que se
saiba para onde. (\versal{MUSIL}, 2006, p. 50).
\end{quote}

Não estaria na vida de Ulrich, de modo menos extremo, sem que ele ainda
pudesse perceber, o mesmo tédio da tautologia que está presente nas
peças de Beckett? Até mesmo Pozzo, ironicamente, reconhece a condição
dos dois errantes à espera de Godot e, depois de reconhecer que Estragon
e Vladimir foram decentes com ele, pergunta o que poderia ``fazer por
estes bravos homens, prestes a morrer de tédio?'', o que poderia fazer
``para que o tempo lhes pareça menos arrastado?'' (\versal{BECKETT}, 2005, p.
77). No entanto, o próprio Pozzo está inserido nessa temporalidade. Por
isso sua fala se perde como mais uma \emph{fala} em meio a tantas outras
que ajudam a passar o tempo, e que em nada ajudam na construção de
sentido.

Nesse contexto de impessoalidade, não se conhece bem Godot (\emph{Idem},
p. 49), mas é por ele que se leva adiante a existência, é ele que
carrega o futuro, o passar dos dias, ou seja, é ele quem determina o
tempo, pois os personagens estão encerrados temporalmente. Em \emph{Fim
de Partida} (\versal{BECKETT}, 2010a, p. 41), Hamm não sabe nem mesmo nomear
aquilo de que está cheio: ``Desse\ldots{} dessa\ldots{} disso.'' Sentimento que é
afirmado pela frase de Clov adiante: ``Alguma coisa segue seu curso''
(\emph{Idem}, p. 73). E é ``Esse\ldots{} essa\ldots{} isso'' que eles acham que já
durou bastante, e que faz de seus dias um eterno retorno do mesmo
cinzento, diante do qual estão impotentes: ``Enquanto durar.
(\emph{Pausa}) A vida toda as mesmas tolices.'' (\emph{Ibidem}, p. 88) A
vida se apresenta tão cinzenta que os corpos vivos já cheiram a cadáver,
de modo que o próprio sentido da morte se encontra achatado, perdeu a
força de sua significação, de mudança de estado, o que corrobora a ideia
do aplainamento das distinções: ``Você já fede. A casa toda já fede a
cadáver.'' (\emph{Ibidem}, p. 89), diz Hamm.

Diante dessa divisa tênue entre morte e vida, vida putrefeita como diz
Clov, até o otimismo só pode se mostrar, quando aparece, como resíduo
amarelecido usado para passar o tempo que não passa: Hamm repete seu
``Estamos progredindo'' como mantra vazio, assim como Winnie repete seu
`` O velho estilo'', suas ``Grandes bênçãos'' seu ``Dia feliz''.

Em \emph{Dias} \emph{Felizes} (2010b), o cenário não é menos arruinado.
A personagem Winnie está enterrada numa montanha que parece um grande
formigueiro, inclusive está sendo comida por formigas e castigada pelo
sol. Como os demais personagens das peças aqui citadas, está perdida
temporalmente, o tempo virou uma geleia incompreensível, indistinguível.
Apesar disso, ela permanece incorrigivelmente otimista, diferentemente
dos personagens das outras peças. Seu otimismo foi relacionado com o
espírito feminino, ou seja, somente uma mulher se manteria otimista
nessa situação. Há também a leitura de que o aprisionamento de Winnie e
seu otimismo são ``uma alegoria unívoca da histeria de uma mulher presa
ao casamento burguês.'' (\versal{ANDRADE}, 2010b, p. 12). Mas o que ``impede o
voo de Winnie vai muito além da possessividade e da castração
machistas'' (\emph{Idem}, p. 12). E se esse otimismo, mesmo na
catástrofe, for signo de uma inconsciência social, que vai fazendo os
sujeitos aceitarem a barbárie por ela ir se dando aos poucos? E se esse
otimismo diante da destruição for decorrência da euforia perpétua, do
gozo nas mercadorias que nos deixa com caráter \emph{blasé --} embora
ativos com nunca? Parece que a estrutura coloidal do ser humano, para
falar com Musil, diz respeito não só às possibilidades positivas de um
desabrochar da individualidade, como também ao caráter negativo da
adaptabilidade do ser humano a situações que deveriam causar repulsa.

De qualquer modo, é um otimismo frágil que, no caso de Winnie, parece
próximo dos fervores religiosos, de que aliás ela dá provas, embora de
modo mais desencantado no segundo ato: ``É, confesso que rezava''
(\versal{BECKETT}; 2010b, p. 56). Um otimismo esmaecido com tonalidade de humor
negro que também está em \emph{Fim de Partida}, numa fala de Clov, ao
mesmo tempo uma negação de qualquer otimismo: ``Vejo\ldots{} uma multidão\ldots{}
delirando de alegria. (\emph{Pausa}) Isso é o que eu chamo de lentes de
aumento.'', após apontar a luneta para o público. (\versal{BECKETT}, 2010a, p.
69).

Os personagens das peças, presos no presente perpétuo, parecem viver
também nos escombros da história. Quando se falou tudo, quando não há o
que falar, repete"-se, é a crise da história, do pensamento, é a ruína do
próprio tempo. Hamm diz: ``Mas de que falam? De que ainda se pode
falar?'' (\versal{BECKETT}, 2010a, p. 62). E Winnie: ``Há tão pouco de que se
possa falar (\emph{Pausa.}) Falamos de tudo.'' (\versal{BECKETT}, 2010b, p.
56)\footnote{Não se trata tanto aqui de entender qual a fonte de
  inspiração de Beckett para \emph{Dias} \emph{Felizes}, se foi a imagem
  do Inferno da \emph{Divina Comédia} pintado por Doré, se foi a
  inspiração no filme de Buñuel e Dali, \emph{O cão andaluz}, em que
  duas mulheres estão enterradas até o meio do corpo na areia.
  (\versal{KNOWLSON}, 2012, p. 23).}.

Numa vida em que existir se torna difícil tarefa, pela fragilidade
subjetiva dos personagens, o revólver está lá, primeiro na bolsa, depois
fora, como para lembrar que pode ajudar a abreviar aquele desterro. Diz
Winnie em \emph{Dias Felizes} (2010b): ``Não deixa de ser um consolo
saber que está por aqui, mas estou farta de você.'' (p. 45). Willie já
tinha sido tentado a ``acabar com m(s)eu martírio.'' (p. 45). Esse mesmo
sentimento de dar cabo à existência ronda os personagens de
\emph{Esperando Godot}, numa espécie de enforcamento por hipótese para
fugir da angústia do encerramento, mas também em \emph{Fim de Partida:}
``É, é isso mesmo, está na hora disso tudo acabar e mesmo assim eu ainda
hesito em ter\ldots{} (boceja)\ldots{} fim.'', Diz Hamm (2010a p. 39). A carga
pesada de que se reveste o existir se mostra claramente numa espécie de
desejo pelo inorgânico. Diz Hamm: ``[\ldots{}] você ainda não estava no
mundo dos vivos.''; Clov: ``Bons tempos'' (\emph{Idem}, p. 87). E logo
adiante, é Hamm quem demonstra o que significa sua vida ao ser
perguntado se acredita em vida depois da morte: ``A minha sempre foi''
(\emph{Ibidem}, p. 93). A vida mediada pelo movimento da mercadoria
apenas tem como se tornar totalizante e totalitária se a relação social
se der como uma espécie de `` união de mortos. (\versal{ADORNO}, 1992, p. 201).

Ao mesmo tempo, não se poderia também relacionar o desejo de onipotência
de Hamm, que submete seus pais e Clov ao seu desmande, com o que
poderiam ser os restos do desejo de onipotência expressos na pretensa
supremacia do Eu na contemporaneidade? Uma supremacia em sintonia com um
mundo de consumo que roda também em falso, numa tautologia de uma
temporalidade vazia? Mesmo assim, por mais que a contemporaneidade
aponte para uma supremacia do Eu, poderíamos dizer que se trata de um eu
frágil, tão frágil quanto os pedintes de subjetividade das peças de
Beckett. São narcisos às avessas, que contemplam a própria imagem, mas
como uma imagem desgastada, por ser imagem cada vez mais mediada pela
indústria cultural e pelo espetáculo.

Não seria exagero ver uma relação entre os sujeitos da \emph{Vida para
consumo} (\versal{BAUMAN}, 2008) e os sujeitos das peças de Beckett, ambos como
personagens reduzidos ao mínimo: mínima linguagem, mínimas
possibilidades, mínima reflexão, mínima ação, enfim, mínimo"-Eu. Nas
peças, há tentativas de construir sentido a partir de uma teia de
retalhos. Mas os retalhos estão longe de ter aquele colorido bonito de
uma colcha de retalhos secando em frente a uma casa de interior. Ao
contrário, não têm cor e, como simples retalhos que ninguém soube
juntar, permanecem incongruentes e cinzentos, sua tessitura é frágil. Ao
mesmo tempo em que os personagens se falam mais para permanecer vivos do
que para transmitir qualquer sentido. Como diz Fábio Andrade, as
palavras se dissolvem com Beckett, seus textos ``dissolvem os projetos
em palavrório, burburinho, rumor, ordenado e simétrico, sim, mas que se
reconhece e se mostra inútil, pondo em cena heróis armados de uma razão
tortuosa e sem finalidade.'' (\versal{ANDRADE}, 2001, p. 105). Ora, por mais
pobres, as pessoas sempre têm uma história. Mas os personagens das peças
de Beckett não parecem ter história de vida, porque essa história parece
ter ruído\footnote{Diz Benjamin em seu texto \emph{O Narrador}
  (1936-2010): ``Ora, é no momento da morte que o saber e a sabedoria do
  homem e sobretudo sua existência vivida -- e é dessa substância que
  são feitas as histórias -- assumem pela primeira vez uma forma
  transmissível. Assim como no interior do agonizante desfilam inúmeras
  imagens -- visões de si mesmo, nas quais ele havia se encontrado sem
  se dar conta disso --, assim o inesquecível aflora de repente em seus
  gestos e olhares, conferindo a tudo que lhe diz respeito aquela
  autoridade que mesmo um pobre"-diabo possui ao morrer.'' (p. 207).}.

Não só a linguagem e a ação, mas também o pensamento está em ruínas. Não
há grandes monólogos -- como no teatro de Sófocles ou Shakespeare -- que
pudessem concatenar uma reflexão pessoal, são quase monossílabos, pois
até mesmo a linguagem fugiu dos personagens. ``Pare de falar agora,
Winnie, fique um pouco quieta, não desperdice todas as palavras do dia,
pare de falar e faça alguma coisa só para variar.'' (\versal{BECKETT}, 2010b, p.
49), diz Winnie a si mesma. A todo momento, ela tenta declamar poemas,
mas as palavras lhe fogem: ``Como eram aqueles versos extraordinários?
(\emph{Pausa}.) Vá, me esqueça pois não sei quê sobre não que mais
deitará sua sombra\ldots{}vá, me esqueça\ldots{} por que a tristeza\ldots{} sorrirás
radiante [\ldots{}]'' (\emph{Idem}, 2010b, p. 60). Não poderíamos
relacionar essa ruína da linguagem com a pobreza da linguagem na
contemporaneidade, quando, sob a capa de uma pretensa informalidade, as
línguas vão se depauperando, tornando"-se mínimas, quase onomatopaicas,
com um vocabulário de uso cada vez mais restrito, e com um vocabulário
inutilizado e esquecido cada vez mais imenso?

As peças de Beckett expressam, como já disse Adorno, uma corrosão da
própria narrativa, pelo fato de que a própria sociedade se encontra
corroída. Essa falta de narrativas diz respeito à falta de momentos
significativos, falta da própria experiência transmissível de que fala
Benjamin. E a falta de momentos significativamente construídos torna as
subjetividades mais abertas ao espetáculo (\versal{DEBORD}, 1997), abertas a
contemplar imagens que se tornam um mundo à parte, mas que se pretende o
real por excelência. Num paralelo com o cotidiano mercantil, as imagens
do espetáculo, ou da indústria cultural, são comparáveis às mesmas
historietas ou falas que os personagens de Beckett nas peças contam uns
para os outros, como tentativa de ludibriar a falta de sentido, ou a
tautologia de sua própria existência.

Mas Winnie, além das historietas, lança mão de outros artifícios para
manter"-se viva, e já falamos de seu otimismo incorrigível. Diz ela:

\begin{quote}
As palavras faltam, há momentos em que até mesmo elas nos faltam. Não é
verdade, Willie, que até mesmo as palavras faltam às vezes? O que fazer
então até que elas voltem? Pentear e escovar o cabelo, se já não se fez
isso antes, ou, caso paire alguma dúvida, lixar as unhas, se estiverem
precisando de lixas, coisas assim nos levam no embalo. [\ldots{}] É
isso que acho maravilhoso, que não se passe um único dia -- ah, o velho
estilo! Quase nenhum sem um mal -- (\emph{Willie desaba na colina, sua
cabeça desaparece} [\ldots{}]) -- que venha para o bem. (\versal{BECKETT},
2010b, p. 39)
\end{quote}

Winnie é descrita como uma mulher na casa dos cinquenta, bem conservada,
um pouco acima do peso, decote amplo, seios fartos, colar de pérolas
(\emph{Idem}, 2010b, p. 29). O que indicaria \emph{aparentemente} uma
pessoa com \emph{dignidade} na sociedade da mercadoria. Exatamente por
isso, ela se relaciona mais com suas coisas do que com seu marido
Willie, que não partilha em nada o seu otimismo, embora não pareça ser
por qualquer consciência reflexiva. A bolsa parece conter o resto de
existência de Winnie, é a única coisa que parece ter sobrado da ruína.
Sua bolsa conta mais que seu marido, que ganha da bolsa por ter um
ouvido. Sobre a bolsa diz ela:

\begin{quote}
Especialmente nas profundezas, quem adivinharia os tesouros.
(\emph{Pausa.}) Os consolos. (\emph{Vira"-se para olhar a bolsa}.) É, há
a bolsa. (\emph{Novamente de frente}.) Mas alguma coisa me diz: Não
exagere com esta bolsa, Winnie, aproveite"-a para continuar\ldots{} tocando,
quando estiver sem saída, lógico, mas pense no futuro, Winnie, no dia em
que as palavras faltarem [\ldots{}] (\emph{Idem}, 2010b, p. 44)
\end{quote}

Parece exagerada essa troca subjetiva com a bolsa, que se transforma em
ser animado, vivo, redentor da existência de Winnie. Como diz Dumont:

\begin{quote}
Na maioria das sociedades, em primeiro lugar nas civilizações superiores
ou, [\ldots{}] nas sociedades tradicionais, as relações entre os homens
são mais importantes, mais altamente valorizadas, do que as relações
entre os homens e as coisas. Esta prioridade é invertida no tipo moderno
de sociedade, onde as relações entre os homens são, ao contrário,
subordinadas às relações entre os homens e as coisas (2000, p. 16).
\end{quote}

Alguém poderá objetar que o autor pretende muito mais falar de uma
burguesa falida, que pretende manter a aparência e nada mais. No
entanto, nossa interpretação nos permite também ver em Winnie uma
paródia da relação dos sujeitos contemporâneos com a mercadoria e sua
tautologia, que lhes fornecem referencial subjetivo. E a mercadoria só
consegue alcançar esse nível de fetichismo na vida social porque as
próprias relações sociais se mostram rachadas, em ruínas. O
desdobramento da forma"-sujeito moderna na contemporaneidade teria esse
ingrediente novo de fetichismo, e os personagens de Beckett
expressariam, podemos arriscar, o que resta da subjetividade numa vida
social cada vez mais mediada pela mercadoria.

A própria falta de mobilidade de Winnie em sua colina poderia ser
relacionada, para além da imobilidade física, com uma imobilidade
intelectual. Afinal, o desdobrar"-se da forma"-sujeito moderna parece
desembocar no contrário daquilo que foi a caixa de toque do
Esclarecimento, a autonomia do sujeito moderno. Do mesmo modo, o fato de
Winnie estar enterrada objetivamente é um belo contraste com seu
\emph{desterramento} subjetivo. Diz Winnie: ``A razão. (\emph{Pausa.})
Não perdi a razão. (\emph{Pausa.}) Não ainda. (\emph{Pausa.}) Não de
todo. (\emph{Pausa.}) Alguma coisa resta. (\emph{Pausa.}) Sons.
(\emph{Pausa.}) Como pequenos\ldots{} deslizamentos, pequenos\ldots{}
desabamentos.'' (\versal{BECKETT}, 2010b, p. 58).

Com essa forma de escrita do desabamento, das sobras, Beckett se insere
numa nova linguagem que subverte a escrita tradicional, não no sentido
simplesmente de fugir à representação, ou ao teatro crítico. Seu teatro
nada tem a ver com Sófocles, Molière, Shakespeare ou Brecht. Seu
conteúdo já está expresso na própria \emph{forma} de construção em que o
\emph{conteúdo} está em ruínas. A reflexão rompe a pura imanência da
forma, como expressa Adorno em suas \emph{Notas sobre literatura}: ``A
nova reflexão é uma tomada de partido contra a mentira da representação,
e na verdade contra o próprio narrador, que busca, como um atento
comentador dos acontecimentos, corrigir sua inevitável perspectiva. A
violação da forma é inerente ao seu próprio sentido'' (\versal{ADORNO}, 2003, p.
60).

Essa violação da forma expressa fortemente por Beckett nas três
referidas peças não é um elogio do \emph{nonsense}. Se as peças são
absurdas, não é pelo fato de nelas estar ausente todo e qualquer sentido
--- nesse sentido elas não teriam importância, não se trata de uma
apologia da abstração ou da falta de sentido ---, ``mas porque põem o
sentido em questão'' (\versal{ADORNO}, 1993, p.176). Parece ser isso que está em
jogo, o sentido unidimensional que está imbricado no progresso e na
Razão instrumental, na unificação dos sentidos e das substâncias
distintas numa só: o princípio da calculabilidade. E isso é uma
característica moderna. Sendo assim, nossa análise considera as ruínas
do teatro beckettiano como consequência do funcionamento moderno, por
isso precisa nuançar opiniões dignas de interesse, mas que tendem a uma
transhistorização desse fenômeno -- o que tentamos evitar para nosso
intento. É o caso de Esslin com seu \emph{Teatro do absurdo} (1967), mas
também de Eugène Webb:

\begin{quote}
O homem, em nossa época, é herdeiro de séculos de análises que deixaram
a análise em fragmentos, e que fizeram do homem um estranho num universo
ininteligível. A literatura do absurdo é uma expressão do estado mental
produzido por essa situação. Somos impelidos por nossa própria natureza
a buscar entendimento, mas a razão, único instrumento que temos com que
buscá-la mostrou"-se um instrumento desajeitado e frágil. (\versal{WEBB}, 2012, p.
28)
\end{quote}

Para nós, esses fragmentos, esse estado de estranhamento do homem num
universo ininteligível é algo tipicamente moderno, não podendo ser
reportado igualmente a séculos anteriores à modernidade como a
Antiguidade ou o medievo. Além disso, temos de nos resguardar com o
termo \emph{literatura do absurdo} para não considerar pura forma
estética o que em verdade é o absurdo incrustado numa forma de vida
social.

Adorno, em sua \emph{Tentativa de entender o Fim de partida,} desenvolve
argumentos em relação à questão do sentido, mas mais uma vez como uma
questão histórica, portanto que atravessa a razão instrumental:

\begin{quote}
A \emph{ratio}, convertida totalmente em instrumental, despojada de
reflexão sobre si e sobre o que é por ela desqualificado deve se
perguntar pelo sentido que ela mesma suprimiu. Mas na situação que
obriga a essa pergunta não fica outra resposta além do nada, que como
forma pura ela já é. (\versal{ADORNO}, 2009, p. 308).
\end{quote}

Nesse sentido, os personagens dessas três peças de Beckett expressam a
própria humanidade separada, alienada de si mesma. São personagens que
portam o despojo de Ulrich --- representante da sociedade burguesa ---
``os restos do otimismo da razão iluminista transformada em bugiganga''
(\versal{ANDRADE}, 2010a, p. 12)

Ulrich conseguia justificar seu mundo, desejava ser importante, mesmo
sem saber o que isso significa direito; ele lançava mão da ordem
científica do mundo, não se propondo a pensar a especificidade das
coisas, belo e feio, bom ou ruim, amar qualquer coisa (\versal{MUSIL}, 2006, p.
1184).

Por outro lado, nem mais esse discurso resta aos personagens de Beckett
que já não conseguem justificar o mundo em que vivem, um mundo onde ``A
plenitude do instante perverte"-se em repetição sem fim, convergindo com
o nada'' como destaca Adorno (1993, p. 43). É o tempo que já se
esvaziou, danificou"-se, bem como o própria subjetividade apressadamente
chamada de burguesa que ainda era sede de reflexão, mesmo que
instrumental: ``[\ldots{}] a fala profética da abertura [de Esperando
Godot] ---`Nada a fazer' --- retorna regularmente, lembrete paradoxal
tanto da necessidade de preencher o vazio, quanto da inocuidade deste
esforço'' (\versal{ANDRADE}, 2005, p. 11). Esse ``nada a fazer'', aliás, é
repetido por Winnie em \emph{Dias Felizes.}

Em função disso, o silêncio, marcante nas peças de Beckett, para nós,
aponta mais para lacunas não possíveis de serem preenchidas, para a
impossibilidade de preenchê-las o tempo todo pelos sujeitos -- que já
precisam de grande malabarismo para preencher a passagem do tempo com
seus restos de linguagem -- do que pausas para a reflexão, do público ou
dos próprios personagens.

Prossegue Adorno sua análise de Beckett:

\begin{quote}
Indiferente ao cliché dominante do progresso, Beckett toma como tarefa
mover"-se num espaço infinitamente pequeno, num ponto sem dimensões. Este
princípio de construção seria, enquanto princípio do \emph{il faut
continuer}, para lá da estática, e enquanto finca"-pé, para lá da
dinâmica [\ldots{}] O telos da dinâmica do sempre"-semelhante é
apenas infelicidade; a poesia de Beckett olha"-a de frente. A consciência
percebe o caráter limitado do progresso que se basta sem fim a si mesmo
como ilusão do sujeito absoluto [\ldots{}]. (\versal{ADORNO}, 1993, p.
252).
\end{quote}

O que aborda Adorno como o princípio do \emph{il faut continuer} em
relação com a dinâmica do sempre"-semelhante é um forte resumo do teor de
Beckett: a angústia de rodar em falso, a angústia do real reificado.
Nada de peripécia, nada de intriga, nada de enredo, nada de evolução,
nada de tensão rumo a um desfecho, mas monotonia, tautologia, forma que
mais expressaria o fato de que nada há a expressar, pois é a própria
vida em sociedade que parece ter perdido substância ou se perdido em
tartamudeios que já não conseguem tecer uma narrativa coerente.

Em \emph{Esperando Godot,} diz Vladimir: ``Então diga o que fizemos
ontem à tarde?''. Depois de algumas palavras trocadas, responde
Estragon: ``Isso, me lembrei, ontem à tarde ficamos falando das botas. A
mesma conversa, há cinquenta anos'' (2005, p. 130). Em \emph{Fim de
Partida}, ao ser perguntado que horas são, Clov responde: ``A mesma de
sempre'' (2010, p. 41). Nas três peças, as pequenas histórias inventadas
para ajudar a passar o tempo, a preencher a passagem do tempo, seriam
apenas: ``Entusiasmo fingido, relações humanas ensaiadas, raivas de
mentira [que] servem às tentativas, desesperadas, de dar sentido a
um mundo desprovido de significado (\versal{ANDRADE}, 2001, p. 80) Mas há também
muitas pequenas falas representativas dos restos de linguagem. Eis a
fala de Winnie em \emph{Dias} \emph{Felizes}: ``Podemos falar em tempo
ainda? Dizer que agora já faz muito tempo, Willie, que não vejo você.
Que não ouço mais você. Podemos? Falamos. O velho estilo! Há tão pouco
de que se possa falar. Falamos de tudo'' (2010b, p. 58). ``É, são dias
felizes'' (p. 60). É a mesma Winnie que, não estabelecendo nem mesmo
diálogos com o marido, prefere passar o tempo com seu objetos: o
espelhinho, a sombrinha, a bolsa.

O que Fábio Andrade escreve sobre \emph{Fim de Partida} também poderia
ser aplicado às duas outras peças, no que tange à temporalidade: ``O
girar em falso do relógio, a negação da novidade e da mudança, sugere um
processo de entropia, uma decadência irreversível e irremediável''
(\versal{ANDRADE}, 2010a, p. 15). Claro que mudanças ocorrem, mas tanto nas peças
quanto na vida social capitalista contemporânea, as mudanças são um
eterno retorno do mesmo sempre mediado.

Embora possam, muitas vezes, pelo absurdo de sua construção, parecer
risíveis, de fato, as peças de Beckett não pretendem transmitir qualquer
moral ou ensinamento, elas nos possibilitam muito mais um contato com o
mal"-estar diante da repetição, da falta de objetivo, da repetição do
nada, dos diálogos que apenas preenchem o tempo à espera vã do tempo
supremo. Elas são uma espécie de \emph{fábula negativa} como argumenta
Gunther Anders em seu livro de título bastante significativo,
\emph{L'obsolencence de l'homme}:

\begin{quote}
Para contar uma fábula acerca dessa forma de existência que não conhece
mais nem forma nem princípio e na qual a vida não mais avança, ele
[Beckett] destrói ao mesmo tempo a forma e o princípio da fábula: a
fábula destruída, quer dizer, aquela que não mais avança, torna"-se a
fábula mais apropriada para dizer que a vida não mais avança. Para
resumir a inversão operada por Beckett, pode"-se dizer que sua parábola
absurda do homem se torna uma parábola do homem absurdo. (\versal{ANDERS}, 2002,
p. 244).
\end{quote}

O homem absurdo de que fala Anders é aquele que está a meio caminho:
``sem coragem de terminar nem força para continuar''. Um homem que deixa
o tempo se valer por si só, um homem que de tanto encarar, com a mesma
indiferença com que encara o mundo, as palavras, elas acabam até mesmo
por abandoná-lo.

Seria risível se não dissesse respeito ao nosso tempo, e o mal"-estar
opõe"-se à possibilidade de rir dos personagens de Beckett. Rir, nesse
caso, parece mais um rir de si mesmo, um rir tonto (\versal{ADORNO}, 2009),
apesar de que Barthes já em 1954 analisa o fato de ``agora, Godot faz
rir de cara aberta'' (\versal{BARTHES}, 2003, p. 88). O riso talvez pudesse
indicar duas possibilidades de entendimento: uma que diz respeito a uma
fuga de encarar o mal"-estar que é dar"-se conta de que vivemos no
cotidiano o que os personagens alegorizam em seus diálogos; outra que
pode dizer respeito ao fato de que a peça não estava amadurecida para
sua época, o vazio tautológico ainda não se apresentava de modo tão
patente, talvez. Será que peças como a de Beckett fariam algum sentido
na época de Shakespeare, na época de Molière, na época de Sófocles, na
época da queda do império romano? Talvez nem mesmo na época do próprio
Beckett ela ainda pudesse ganhar todo o sentido que Adorno conseguiu
expressar na sua interpretação. A especificidade histórica da sociedade
moderna parece apontar para o fato de que essa tautologia expressa na
forma e no conteúdo de suas peças somente poderia adquirir sentido numa
sociedade em que essa tautologia também se expressa no social. Não se
trata evidentemente de mero reflexo da forma"-social na forma estética.
Ao contrário. O próprio fato de Beckett não ter necessariamente desejado
pôr em cena nessas três a tautologia da vida moderna desenvolvida já
demonstra que não há esse reflexo. Mas sua forma teatral da ruína não
pode ser vista somente como sintoma de uma época de choque devido à
guerra que colocava a condição humana em questão. A experiência
cotidiana sentida e inagarrável dessa tautologia pode fazer amadurecer
nos espíritos mais sensíveis a expressão dessa experiência que
permanece, apesar de tudo, inominável para o artista que a expressou.
Assim como continua inominável para os indivíduos que vivem a
experiência do ruir de um tempo histórico com a mesma impotência dos que
esperam Godot.

\chapter{Reflexões finais}

\begin{flushright}
\scriptsize{Não estimo que o homem seja capaz de formar em seu espírito mais vão e
mais quimérico projeto do que pretender, ao se lançar a escrever certa
arte ou ciência que seja, escapar de toda sorte de crítica, e cativar a
aprovação de todos os leitores.

\emph{La Bruyère. Os caracteres de Teofrasto.}}
\end{flushright}

O que nos anima nesse estudo é o desejo de refletir sobre uma crise da
subjetividade aprisionada na forma"-sujeito burguesa, com o objetivo de
contribuir para que esse processo de esvaziamento não siga seu curso,
mas possa ser questionado por uma subjetividade que ponha como
importante a reflexão cotidiana sobre a forma social em que vive e sua
própria subjetividade. Dito de outro modo, seria preciso que os
indivíduos concretos se revoltassem contra a máscara de caráter abstrata
da forma"-sujeito burguesa que tensiona com sua subjetividade feito um
encosto. Certamente assim, não naturalizando a marcha do tempo, não
aderindo a ela de forma irrefletida, não querendo simplesmente
aperfeiçoá-la e corrigir seus excessos, poderemos ter esperança de que o
prosseguir da humanidade não será o aprofundamento da barbárie, cuja
condição principal é o vazio do pensamento típico da forma"-sujeito
burguesa amadurecida. Mas para que a ave da esperança possa não ser
derrubada em pleno voo, é preciso que ela aninhe sempre em seu seio a
crítica negativa.

A tese que perseguimos nesse estudo foi a de que a dinâmica da
modernidade -- a modernização -- é em si um devir"-vazio tautológico.
Desenvolvemos a ideia de que esse vazio tautológico se entranha na
realidade cotidiana quanto mais a mercadoria é entronizada na vida
social. Dito de outro modo, a vida social sobre o fundamento
simbólico"-mercantil tende a desenvolver relações mediadas por esse
\emph{simbólico} que é em verdade a liberação de qualquer laço simbólico
em proveito do vazio. Se esse vazio não se manifestou desde o início do
desenvolvimento da forma"-social moderna, foi porque ainda havia largos
terrenos de vida social fundada no simbólico"-religioso a serem
aplainados. Foi o que chamamos de entrecruzamento objetivo e subjetivo.

Tentamos fundamentar nossa tese desse devir"-vazio tautológico na
argumentação de que a vida social moderna é em verdade uma dupla
abstração que pretende se realizar. Uma abstração de sociedade, a
forma"-social burguesa, e uma abstração de subjetividade, a forma"-sujeito
burguesa. A primeira avança sobre todas as formas de socialização que
não sejam ainda fundadas na mercadoria. A segunda avança sobre as
individualidades concretas para lhes impor uma forma de subjetividade
unitária, em consonância com a vida social moderna sempre dinâmica e
aberta ao novo. A modernidade foi o processo lento de imposição dessas
formas abstratas de vida social e subjetiva.

Como se trata de formas abstratas e dinâmicas, sua realização na
história, feito o Espírito hegeliano, não se faz sem um esgarçamento da
concretude social e subjetiva. Dito com outras palavras, tanto a vida
social, incluindo o meio ambiente, quanto a subjetividade não têm como a
longo prazo acompanhar o ritmo de uma abstração desembestada. O nível de
suportabilidade do concreto tem limites que o movimento do abstrato não
respeita. Do ponto de vista subjetivo, já é notável a gestação de um
mal"-estar inconfesso relacionado não apenas à evolução de relações
sociais embrutecidas em todos os níveis, não somente à falta de um
sentido que não seja o ditado pela dinâmica mercanti, mas também à
exigência feita aos indivíduos concretos para fazerem sacrifícios --
seja trabalhando, seja sempre consumindo -- para que a forma"-social
burguesa possa seguir adiante. É nesse sentido que há um nível de
insuportabilidade que não se expressa só no limite da natureza, mas no
limite da subjetividade esgarçada pelas exigências da forma"-sujeito
burguesa. O que não significa que essa insuportabilidade sentida pelo
indivíduo perante as exigências sacrificiais de uma sociedade
desemboquem em busca de saída emancipatória. Se a marcha da vida social
e subjetiva moderna é objetiva, a resistência e a tentativa de
transcendendê-la não o são. Como o desdobramento da vida social moderna
é uma tendência a uma regressão à pura imanência da segunda natureza, o
ser humano pode seguir a tendência \emph{natural} de se adaptar à
barbárie em vez de se opor a ela. Porque a oposição ao vazio que
possibilita a barbárie é uma vitória sobre esse encerramento do
pensamento crítico na segunda natureza, é uma vitória sobre a tendência
sócionatural. Em verdade, como a trama social escapa aos seus sujeitos,
essa insuportabilidade pode se materializar nos mais diversos tipos de
sofrimento psíquico e autoculpabilização por um lado, e em violência
cotidiana do outro, uma violência que em muitos casos vai ter relação
com os próprios sujeitos burgueses e seus desejos imperiosos que não
suportam qualquer mediação. Os narcisos estão em todas as esferas da
vida social, do executivo ao narcotraficante, do artista vendido ao
artista engajado.

Como pudemos refletir, a crise da vida social mercantil e da
forma"-sujeito mercantil não fazem parte de um desfuncionamento, mas de
sua realização. Uma realização que ainda não está completada, embora os
indivíduos com a máscara de caráter do sujeito burguês pareçam dispostos
a fazer atos sacrificiais para salvá-la. Assim como a forma"-social ainda
não está plenamente realizada -- porque se permitirmos sua realização
ela coincide com o aniquilamento da realidade concreta -- também a
forma"-sujeito ainda não está de todo realizada. Ou seja, não se trata
ainda de um vazio absoluto, mas de uma tendência à dessubstancialização
do sujeito quanto mais ele se aproxima de seu conceito de agente social
dentro do invólucro da relação mercantil -- como segunda natureza em
realização.

Não pretendemos, ao refletir sobre a ideia de uma dessubstancialização
do sujeito, defender a ideia de uma essência humana autêntica, positiva,
transhistórica, uma essência \emph{a priori}, que simplesmente teria
sido destruída pela modernidade, algo que, em verdade, não passaria de
uma ficção. Até porque essa essência humana é humana porque é mediada
pelo social. Mas defendemos que o fato de o sujeito contemporâneo dar
mostras de estar aberto a todas as possibilidades que os ideais
mercantis oferecem apenas aponta para o fato de que ele se torna
adaptável a todas elas, sem a possibilidade de julgar tais
possibilidades que, só em serem várias e estarem à disposição, é como se
já bastassem. Por outro lado, quando esse sujeito se pretende crítico,
ataca muito mais o caráter desigual, injusto da vida social do que suas
raízes irracionais.

Nesse sentido, o caráter \emph{livre} das relações sociais num
capitalismo desenvolvido e das construções de subjetividades sem
ancoragem é um índice dessa \emph{dessubstancialização}. Pois os
``indivíduos não buscam um confronto com sua própria existência'',
situam suas vidas num presente eterno que se multiplica e, assim,
``deixam de elaborar uma consciência marcante deste mesmo presente, como
coisa vivida que se torna aos poucos parte de uma memória pessoal e
social que pode gestar frutos (\versal{SOARES}; \versal{EWALD}, 2004, p. 04). Dessa forma,
a liberdade passa a ser não mediada pelas próprias relações sociais, mas
imposta às relações sociais na forma de uma liberdade imediata como
\emph{pseudonatureza}. É uma liberdade, em última análise, que é
utilizada para melhor dominar (\versal{MARCUSE}, 1973).

Contrariamente ao que os críticos literários formalistas e pós"-modernos
dirão, nosso estudo não foi um mero uso da literatura para provar uma
tese. Não buscamos uma tese nas obras, mas os espaços nas obras em que o
sintoma de uma vida social aparece nas entrelinhas. Entrelinhas, porque
não necessariamente era o objetivo de cada autor aquilo que o tempo fez
amadurecer como interpretação tempos depois. No caso de Molière, é óbvio
que ele não escreveu peças para tratar do entrecruzamento subjetivo
entre o ideal nobre e a nova classe burguesa. Do mesmo modo, Musil não
tinha em sua época vivenciado o desdobramento da vida social moderna
para ver que a postura do \emph{homem sem qualidades}, mais do que uma
postura de oposição ao mundo, se tornaria uma postura prezada pelo mundo
da mercadoria, que é a única forma"-social que tende para uma liberdade
sem freios do indivíduo que se vê livre para desabrochar seu \emph{eu}
dentro dos limites do universo simbólico mercantil.

No caso de \emph{Macunaíma} de Mário de Andrade, que fecha de certo
modo, juntamente com a \emph{Dialética da malandragem} de Antônio
Cândido, nossa tentativa de contribuição para entender a forma"-sujeito
burguesa no Brasil, trata"-se de entender o quanto a relação com o
primitivismo, com a espontaneidade, o não"-racional, não necessariamente
se opõem à sociedade moderna primordialmente europeia. Quase cem anos
depois da escrita da obra, estamos numa posição mais privilegiada para
ver o quanto a ideologia do modernismo paulista era em verdade uma
tentativa de síntese problemática entre cultura e natureza, adiantando
de modo vanguardista o próprio significado da sociedade moderna como
segunda natureza.

Em Beckett, como tentamos refletir, não se tratava de interpretar o que
o autor disse ou pretendeu com suas peças, mas o que ele conseguiu para
além de sua pretensão imediata. O teatro de Beckett nas peças que
analisamos encenam a vida social moderna desdobrada, quando a mercadoria
não deixa mais espaços de não"-identidade, mais espaços de ação para os
homens na sociedade.

É por isso que dizemos que \emph{o homem sem qualidades espera Godot.} O
sujeito contemporâneo se vê na seguinte encruzilhada: tem à disposição
todas as ``qualidades'', qualidades que lhe são exteriores e acidentais,
cambiantes e não essenciais; qualidades que lhe são disponibilizadas de
forma heterônoma, portanto não são construídas nas relações sociais
diretas. Mas é exatamente por ter à disposição todas as qualidades, como
Ulrich, que ele se torna \emph{sem qualidades} constitutivas, definidas,
torna"-se massa modelável, homem flexível tão prezado pelo nosso tempo,
apto a apagar rastros e traçar outros (ou aceitar outros passivamente).
Como diz Jorge Coelho:

\begin{quote}
Somos, assim, não como vítimas manipuladas, mas como cúmplices, à nossa
maneira e com a nossa feição histórica e em larga escala, algo bem
parecido com o ``homem sem qualidades'' de Musil. Como ele, o homem
hipermoderno é um homem sem atributos, sem a clareza moral do que pode
ser, do que deve ser, que lhe permita sobreviver e se adaptar às
exigências da modernidade líquida, que lhe escorre pelos dedos e não
deixa rastros a seguir. (2013, p. 30).
\end{quote}

Esse sujeito coincide com aqueles átomos flutuantes esvaziados pela
circulação de modelos, como afirma o apologético Lipovetsky (2008, p.
154), o que demonstra a fluidez do próprio processo de subjetivação, não
enquanto possibilidade de mudar na dialética com o mundo, mas enquanto
uma constituição subjetiva que vai perdendo as condições de julgar
qualitativamente aquele novo que se lhe apresenta. Ou seja, é o momento
negativo da dialética, é a tensão criada quando os diferentes se
encontram que se fragiliza.

Ora, a sociedade capitalista na sua fase contemporânea pretende
disponibilizar várias ``essências'' consumíveis, cabendo ao consumidor
com sua liberdade a escolha da que lhe convém. Mas a ``essência'' que
pretendem dar as mercadorias são, em verdade, a ``essência'' de uma
forma sem conteúdo, portanto, ``inessência'', negação de qualquer
essência que possa nascer de uma construção de sentido na relação com o
mundo e com o outro. A construção de sentido com o mundo não"-identico a
si é o que a modernização ao desdobrar"-se impossibilita no seu afã de
igualar tudo a grandezas abstratas.

Ao mesmo tempo, esse sujeito contemporâneo espera avidamente algo de
decisivo, sua vida coincide com a \emph{espera de Godot}. E nada
preenche essa espera que deve ser apenas tornada menos penosa --- como
na peça de Beckett. De certa forma, as qualidades que as mercadorias dão
a comprar, como ideais identitários ou máscaras psicossociais, são um
falso Godot. Falso porque cada mercadoria específica, como diz Debord,
``luta por si mesma, não pode reconhecer as outras, pretende impor"-se em
toda parte como se fosse a única'' (\versal{DEBORD}, 1997, p. 44), o que
significa que todas precisam prometer contemporaneamente o momento
supremo, o ``encontro com Godot''. E tal qual na peça de Beckett, na
qual quem entra é confundido com Godot, aquele que pode salvar, acontece
em nossa vida quotidiana. Nós marcamos um encontro imperdível com a
mercadoria no consumo na vã tentativa de obturar o vazio do mundo da
mercadoria e de nossa vida como mercadoria, sem entender que o vazio do
mundo da mercadoria não pode ser preenchido por esse mesmo vazio. É
assim que nosso encontro cotidiano com a mercadoria no consumo é
constantemente marcada pela desilusão de que aquela ainda não era a
última verdade --- ou a substância --- como prometia, ela ainda não era
\emph{Godot} que tanto esperávamos e pelo qual tanto deixamos de viver
na espera.

Ao fim desse estudo, o ir além desse estado de coisas só parece
vislumbrável se para já se começar um processo de tensionamento com a
vida social mercantil. Se a revolução operada pelo capitalismo na vida
social foi um processo lento, numa dialética contínua, sua superação,
entendida como superação da segunda natureza, não pode ser obra de uma
revolução no sentido tradicional. Seria preciso um tensionamento
cotidiano no sentido de construir uma subjetividade e uma forma de
organização social que realmente tensionassem cotidianamente e pudessem
ir corroendo a forma"-social mercantil. Mas para isso seria preciso em
primeiro lugar não tentar arrombar as portas que o próprio capitalismo
já abriu. É preciso saber que liberdade queremos, é preciso que
conteúdos distintivos em meio a tanta equalização possam ter lugar à
mesa.

Por fim, é importante dizer que o universo do discurso objetivo não
conseguiu despir o autor dessas linhas de sua melancolia crítica em
proveito da quietude teórica do observador do mundo. Tendo encontrado as
pessoas que deveria, meus anos em contato com esse universo não fizeram
senão alimentar tal sentido, mas num nível crítico mais acurado e
inquieto. De modo que ao fim dessa jornada não consigo entrever em meio
a essa bruma desalentadora e eufórica que encobre nossos dias na
comunidade mercantil senão a divisa de Pessoa para seguir adiante:
``Tudo vale a pena se a alma não é pequena.'' Mas somente até que um
sujeito muito desejoso de levar adiante esse parêntese histórico queira
lançar mão dessa divisa para vender suas quinquilharias que cada vez
menos vão conseguir dar sentido à tautologia de uma vida tornada
quinquilharia.

\chapter*{Posfácio\\A essência da coisa}
\addcontentsline{toc}{chapter}{Posfácio}
\hedramarkboth{Posfácio}{}
%\section*{A essência da coisa}

\begin{flushright}
\emph{Marildo Menegat}
\end{flushright}

Pode parecer estranho se falar de \emph{homens sem qualidades}. Por pior
que eles sejam, no limite, seus defeitos poderiam fazer a vez do que
lhes falta. Mas não é o caso. Trata"-se da forma neutra que, como
resultado de um todo absurdo, lhes confere com frequência qualidades
negativas que não são suas, mas meros empréstimos de uma estrutura de
dominação impessoal que faz de homens e mulheres seus suportes -- estes
também, frequentemente, absurdos. Este ponto de partida de Robson de
Oliveira é desprovido de qualquer antropologia apoiada numa visão
essencialista de que a história e o ser humano devam ter sentidos
transcendentais. Interessante caso de uma crítica social da literatura
que não tem pretensões além de demonstrar uma incômoda figura do
espírito presa num emaranhado de fios, tal como uma marionete
voluntária. A exposição da constituição do sujeito burguês assim
empreendida, difere de outras chaves conceituais sobre este tema, que se
assentam primariamente no estudo da formação ideológica, enquanto Robson
tem como referência implícita o estudo das máscaras de caráter que os
homens vestem no capitalismo. O burguês, situado deste modo, é uma
existência desassossegada que se explica a partir de sua interioridade
em permanente conflito com as transformações da estrutura social, sendo,
por seu turno, ele próprio um resultado adequado desta. As relações
entre as determinações da forma social e a adaptação das ideias a estes
imperativos, figuram como uma perspectiva de leitura de um objeto que,
ao fim, não cabe a este nenhuma simpatia. Por ser ele o desdobramento de
uma dupla máscara -- a do caráter do capital em determinado tempo da sua
história, e a do personagem literário, que tem o dilema de viver sua
existência situado neste tempo -, este resultado é frequentemente
risível (ou ridículo!), ainda quando o drama é pesado.

A sequência de autores mobilizada para este fim não é aleatória. Todos
os 4 (Molière, Musil, Mário de Andrade e Beckett) escreveram em momentos
decisivos do desenvolvimento deste dramalhão insano que é a história da
modernidade. São escritas de épocas de transição social. Para alguns
deles, esta transição se combinou com tremendas crises do capital.
Porém, nestas mudanças, algo de `substancial' do conteúdo sempre
permaneceu. O pressuposto desta concepção reside no modo como a
realidade se estrutura. Em \emph{História e consciência de classe}
Lukács afirma que ``\emph{só no terreno do capitalismo, da sociedade
burguesa, é possível reconhecer na sociedade a realidade}''\footnote{Cf.
  \versal{LUKÁCS}, G. \emph{\emph{História e consciência de classe -- estudos
  de dialética marxista}.} Lisboa: Editora \versal{ESCORPIÃO}, 1974, p. 35.}. E,
aprofundando esta afirmação, Jamenson arremata: ``\emph{Quando passamos
de tal contexto [de uma sociedade pré-industrial, \versal{MM}] para a
literatura da era industrial, tudo se altera. Os elementos da obra
começam a abandonar seu núcleo humano: uma espécie de dissolução do
humano se manifesta, uma espécie de dispersão centrífuga na qual, a cada
ponto, os caminhos levam ao contingente, à matéria e ao fato bruto, ao
não"-humano''}\footnote{Cf. \versal{JAMENSON}, F. \emph{\emph{Marxismo e forma --
  teorias dialéticas da literatura no século \versal{XX}}.} São Paulo: Editora
  Ática, 1985, p. 132.}. O substancial da realidade no capitalismo não é
o ser humano, por isso, as metamorfoses do sujeito burguês são, antes de
tudo, metamorfoses das formas fundamentais das coisas que governam os
homens\footnote{``[O]s trabalhos privados só atuam, de fato, como
  membros do trabalho social total por meio das relações que a troca
  estabelece entre os produtos do trabalho e, por meio dos mesmos, entre
  os produtores. Por isso, aos últimos aparecem as relações sociais
  entre seus trabalhos privados como o que são, isto é, não como
  relações diretamente sociais entre pessoas \emph{em seus próprios
  trabalhos, senão como relações reificadas entre as pessoas e relações
  sociais entre coisas''. Cf. \versal{MARX}, K. \emph{O capital: crítica da
  economia política, }}\emph{Livro I, vol. 1}. São Paulo: Nova
  cultural, 1985, p. 71}. A mercadoria não existe sem ser produto do
trabalho. Como fruto dessa atividade humana abstrata, ela é um gasto de
energia num tempo objetivado no valor de troca, que apenas pode
preservar sua existência no momento em que se torna dinheiro para seu
proprietário. Mas o dinheiro, que é a essência deste processo, precisa
realizar"-se na forma capital, voltando ao início das metamorfoses,
quando se transforma novamente em produção de mercadorias, e assim por
diante. Esta seria a matéria substancial da realidade. Ocorre que ela,
como elemento dinâmico de uma sociedade sistêmica, se altera. Na busca
incessante da ampliação quantitativa de riqueza, o capital precisa
realizar mudanças qualitativas da produção de valor. Com as renovações
tecnológicas ele atinge parcialmente este fim, eliminando quantidades
crescentes de trabalho do processo de produção. Ora, mas é justamente
ele, cujo tempo dá a medida abstrata da substância, que sustenta o tal
``devir tautológico da modernidade''. Assim, quanto mais se desenvolve o
capitalismo, mais rarefeita se torna a vida nele vivida. Não porque no
passado tivesse sido verdadeira, mas sim, porque no passado ela tinha
algo de uma promessa de verdade que a movia, e hoje dela apenas soa o
som puído de sua falsificação.

É muito original sustentar este devir tautológico a partir de um estudo
de obras - e autores - nas quais as relações humanas passaram a se
apoiar cada vez mais na mediação monetária. Este é um tema essencial
para a compreensão das formas narrativas da modernidade. Nelas se faz
presente, como expressão artística, o processo histórico em que homens e
mulheres em suas relações sociais passaram a deixar de lado suas
qualidades de seres humanos que desenvolvem suas potências, para se
tornarem as qualidades transmitidas pelo dinheiro\footnote{``O que é
  para mim pelo \emph{dinheiro}, o que eu posso pagar, isto é, o que o
  dinheiro pode comprar, \emph{isso sou eu}, o possuidor do próprio
  dinheiro. Tão grande quanto a força do dinheiro é a minha força. As
  qualidades do dinheiro são minhas [\ldots{}] qualidade e forças
  essenciais. O que eu sou e consigo não é determinado de modo algum,
  portanto, pela minha individualidade. Sou \emph{feio}, mas posso
  comprar para mim \emph{a mais bela} mulher. Portanto, não sou
  \emph{feio}, pois o efeito da \emph{fealdade} [\ldots{}] é anulado
  pelo dinheiro''. \versal{MARX}, K. \emph{\emph{Manuscritos
  econômicos"-filosóficos}.} São Paulo, Boitempo, 2004, p. 159.}. Em
Molière, por exemplo, esta é a matéria que organiza os conflitos da
consolidação da passagem de relações de obrigação baseadas na honra,
para relações de obrigação mediadas pelo `vil metal'. Que seja deste
contexto a primeira menção aos \emph{homens sem qualidade}, na pele de
arrivistas entesourados que buscavam legitimidade social com casamentos
de interesse e relações com a aristocracia quebrada economicamente, não
é um mero acaso. Esta menção expressa, ao mesmo tempo, um ressentimento
destes aristocratas com a perda do privilégio do reconhecimento social,
que era monopólio da sua condição de homens superiores e inatingíveis e,
de outra parte, o anuncio de um nivelamento das condições de se arrastar
atrás de si a admiração social centrada nas qualidades das novas formas
de riqueza. Tal problema, implicado na ascensão do capitalismo, em que
os conflitos pessoais precisaram ceder a este \emph{telos} tautológico,
desde então esteve em andamento. As relações monetárias passaram a
operar como uma verdadeira \emph{tábula rasa} da distribuição do
\emph{status quo}. Pensado do ponto de vista de um narrador onisciente,
tudo seria risível, não fosse a experiência dolorosa em curso que
haveria de humilhar inclusive os possuidores de dinheiro, já que este é,
por natureza, desapegado e infiel com as mãos que o carrega.

Neste contexto constitutivo da chamada cultura burguesa, se realizou a
elaboração e internalização de normas comuns, em que, segundo Roswitha
Scholz, as formas dominantes do moderno patriarcado produtor de
mercadorias se consolidaram. A literatura foi um espaço em que o
conflito com o que deve ficar dissociado das formas fundamentais de
dominação se manifesta como dolorosa mutilação das potências,
particularmente as das mulheres, que são desvalorizadas a priori pelas
qualidades sociais de seus fazeres. Como o capitalismo é um modo de vida
originado na Europa e movido pelo poder exercido por homens brancos no
espaço externo à intimidade da casa, o qual inclui a autonomia da esfera
econômica e seu domínio sobre as demais esferas da vida social, deste
poder acabam dissociados também todos os grupos étnicos diferentes dos
envolvidos na origem. O livro de Musil é assombroso por estas tensões,
que o sobrecarregam e por meio dele transbordam o que no seu tempo ainda
viria a acontecer. Partindo de uma Europa do início do século \versal{XX} - que
ainda há pouco se orgulhava de ser o berço da civilização e seu ponto
mais elevado, mas que, não obstante, caminhava a passos céleres para o
aniquilamento -, as mulheres, nesta condição histórica, ou eram
exterminadas, como no cruel assassinato de uma prostituta por
Moosbrugger, ou faziam parte de um jogo sedutor e conflituoso de
potências mobilizadas para celebrar o velho patriarca, numa conjunção
entre o último suspiro da aristocracia e a necessária adequação a moldes
mais atualizados do patriarcado, agora exclusivamente alinhado ao poder
do dinheiro. O \emph{homem sem qualidades} dos tempos de Molière é,
neste romance, a afirmação positivada do declínio do sujeito burguês --
e não mais a forma ambígua e conflituosa do anúncio de uma promessa de
felicidade. O quadro de uma guerra total (1914-18), que era preparada
para realizar o espírito deste mundo, em que a crise de superacumulação
de capital se conjugou violentamente com elementos de `persistência da
tradição' do antigo regime, encontrou nesta ausência de qualidades
justamente as qualidades necessárias para a continuidade do devir
tautológico. A sociedade do valor"-dissociação, para permanecermos nos
termos de Roswitha Scholz, ao perseguir inconscientemente o fim em si da
valorização do valor, precisou transformar massas humanas em modos
passivos e desvalorizados de existência para, em seguida, encaminhá-las
ao autoextermínio. Neste ponto, o romance moderno abandonou sua origem
de epopeia da vida burguesa. O sarcasmo de Musil torna este gênero
absolutamente incômodo ao apaziguamento dos sentimentos com os conflitos
sociais que nesta nova constelação se produzem. O distanciamento
jornalístico com que Musil descreve o assassinato por Moosbrugger, de
certo modo, antecipa o sentido e a frieza com que a Europa se preparava
para a continuidade da era dos assassinatos em massa (1939-45).

As transformações dentro da era industrial, desde meados do século \versal{XIX},
mas especialmente no último quarto daquele século e o início do \versal{XX},
transtornaram a vida social de tal sorte que as patologias coletivas se
tornaram muito frequentes. Hannah Arendt observou que o surgimento da
ralé, como um fenômeno novo da cultura europeia, ocorreu nesta época.
Uma de suas causas teria sido a passagem em ato, com uma força até então
pouco vista, das formas fictícias do capital. Esta modalidade de uma
`acumulação em excesso', desesperada por manter seu vínculo ativo com a
valorização num mercado de possibilidades de investimentos estreitadas,
se deve ao fato de que `a substância do valor' já neste momento foi se
tornando cada vez mais difícil de ser produzida e, com isso, tornou
supérflua a função da forma de existência de milhões de seres humanos. A
ralé se forma quando o desespero dessas massas passa a ser organizado
politicamente para um direcionamento destrutivo de tudo o que pode levar
à reflexão e impedir a continuidade desta vida de marionetes
voluntárias. O \emph{homem sem qualidades} deste momento histórico é uma
explosiva transmutação desta decadência do sujeito burguês, assomada a
um incurável \emph{niilismo passivo} que a \emph{tábula rasa} das
relações de obrigação por meio do dinheiro criou, além da nova e
ampliada experiência do que Lukács chamou de reificação. Não há muito
para onde se fugir. A destruição da guerra total passa a ser um
canhestro desejo de morte, como escreveu Freud, acerca desta
encruzilhada, no seu \emph{Por que a guerra?.} Não faltaram na
literatura outras elaborações embaraçosas desta situação. Na Rússia, por
exemplo, o debate sobre o verdadeiro significado histórico do
\emph{niilismo} e seus `demônios', na versão de Dostoiévski, chegou ao
ponto de profetizar algumas das consequências que a ausência de
qualidades do homem moderno poderia vir a criar\footnote{Este debate se
  iniciou com Ivan Turguêniev, no seu romance \emph{Pais e filhos}. Mais
  tarde, um \versal{DOSTOIÉVSKI} já mordido pelo tema, não perdeu a ocasião de um
  `funesto acontecimento', em que uma célula da organização clandestina
  \emph{Justiça Sumária do Povo} executa, em 1869, um estudante que se
  afastava do grupo por divergências, para escrever \emph{Os demônios}.
  A certa altura deste romance, um de seus personagens \emph{niilistas}
  proclama: ``No mundo só falta uma coisa: obediência. A sede de
  educação já é uma sede aristocrática. [\ldots{}]. Vamos eliminar o
  desejo: vamos espalhar a bebedeira, as bisbilhotices, a delação; vamos
  espalhar uma depravação inaudita; vamos exterminar todo e qualquer
  gênio na primeira infância'' (2004: 407). Sobre as `profecias' de
  Dostoiévski ver \versal{BEZERRA}, P. ``Um romance e profecia''; in:
  \versal{DOSTOIÉVSKI}, F. \emph{\emph{Os demônios}.} São Paulo, Editora 34,
  2004, pp. 689-697.}. Quando mais tarde o stalinismo tornou um
acontecimento pitoresco de província, numa realidade nacional\footnote{No
  mesmo contexto da passagem anterior (nota supra), o referido
  personagem dostoiévskiano diz ainda: ``No esquema dele [trata"-se dos
  escritos de outro personagem] cada membro da sociedade vigia o outro
  e é obrigado a delatar. Cada um pertence a todos, e todos a cada um.
  Todos são escravos iguais na escravidão''. Cf. \versal{DOSTOIÉVSKI} (2004:
  407). O surpreendente é que tudo isso continue atual, mesmo que as
  `realidades nacionais' sejam outras!}, foi possível um entendimento
ampliado daquilo que Lukács queria dizer com ``\emph{só no capitalismo é
possível reconhecer na sociedade a realidade}''. O leitor não deve
esquecer que a Revolução de 1917 foi a solução a Leste desta situação
histórica. Sua novidade já nasceu envelhecida, como pôde ser finalmente
constatado em 1991. Mas ela está longe de ser distinta do que ocorre no
Ocidente e que tem tragado o capitalismo para um dos buracos negros da
história.

O que estes autores estudados por Robson de Oliveira não deixam
esquecer, é que o devir tautológico é um processo progressivo que
resulta na crescente ampliação deste estado explosivo. Não é aleatório
para um leitor de Adorno que um estudo desta natureza termine em
Beckett. É possível se encontrar nas peças deste uma espécie de mensagem
em garrafa lançada ao mar, que guarda a notícia do futuro
próximo"-passado da morte da terra. Em tal cenário, seus personagens
perdidos puderam se furtar, não se sabe como, da hecatombe final, mas
não poderão deixar nenhuma pista além de suas impressões deste fim, que,
aliás, ignoram as causas e ignoram que seja um fim, apesar de
desconfiarem não lhes caber a espera de nada, a não ser a de um
infindável prolongamento do vazio. Svetlana Alieksándrovich no seu
\emph{Vozes de Tchernobil} descreve, entre outros testemunhos, a
história de uma comunidade de 4 pessoas que se formou na região
contaminada pela radioatividade, dentro do deserto da zona de exclusão
do acidente nuclear. Não é ficção. Estas pessoas vivem sem maiores medos
além dos humanos, como o medo da morte e da fragilidade da velhice. As
portas de suas casas podem ficar escancaradas e eles mesmos definem sua
existência como um `verdadeiro comunismo'. São, em diversos aspectos
fundamentais para se analisar a vida moderna, surpreendentemente livres:
nenhum Estado ou mercado os importuna ou coage. Porém, esta comunidade
não sobreviverá e nem poderá transmitir ao futuro o legada das misérias
que são a causa de seu presente venturoso. Alguma consciência de mundo
ainda os habita, mas ela é inócua, não pode ser legada, e se for, não
podemos fazer nada com sua boa nova, pois ela dependeria de um tempo a
ser vivido como o da morte da terra. Esta realidade é um intermundo que
se formou dentro do devir tautológico da modernidade. Há nela vida, se
bem que consciente de se tratar de um sussurro que se despede. No lado
de cá da cortina, a cegueira do sujeito burguês ante este estado de
coisas sequer procura sua cura. Ele desistiu das promessas do passado e
se empenha em levar ao fim a obra que com ele se iniciou. O capitalismo
desde os anos 1980 apenas sustenta sua dinâmica de reprodução por meios
especulativos. As formas ficcionais do capital agora não são mais uma
ocasional aparição, mas a própria estrutura e o motor que empurra a
sociedade ao breu. A acumulação permanente somente pode ser simulada
nestas condições e, mesmo assim, exige um grau incomensurável de
destruição da natureza e dos laços sociais. Tem sido inviável, nestas
condições, delimitar a borda do real neste modo intransponível de
funcionamento do fim da acumulação.

Pode parecer estranho se falar de uma irrealidade do real. Por mais
esquisito que seja, no limite, este irreal poderia ser um tipo de real.
Mas não é o caso. Trata"-se do resultado de um absurdo que contou com a
cumplicidade de homens e mulheres sem qualidades. A substância que
estruturava esta realidade se esvaneceu, porque a sua produção se tornou
impossível no atual nível de desenvolvimento tecnológico - como
consequência do seu devir bem sucedido.

Já não sei se para hoje Godot virá\ldots{}

\chapter{Referências bibliográficas}


\begin{Parskip}
\versal{ADORNO}, Theodor. \emph{Intento de entender Fin de partida.In}: Theodor
Adorno\emph{, Notas sobre literatura.}Tradução de Alfredo Muñoz. Obras
completas vol. 11, 1º reimp. Madri: Akal, 2009.

\_\_\_\_\_\_. I\emph{ntrodução à sociologia}. Tradução de Wolfgang Leo
Maar. São Paulo: Unesp, 2008.

\_\_\_\_\_\_. \emph{Notas de literatura I}. Tradução de Jorge de
Almeida. São Paulo: Duas Cidades; Ed. 34, 2003.

\_\_\_\_\_\_. \emph{Educação e emancipação}. Tradução de Wolfgang Leo
Maar. 2. ed. Rio de Janeiro: Paz e Terra, 1995.

\_\_\_\_\_\_. \emph{Teoria Estética}. Tradução de Arthur Mourão.
Lisboa: Edições 70, 1993.

\_\_\_\_\_\_. \emph{Mínima Moralia}. Tradução de Luiz Eduardo Bicca.
São paulo: Ática, 1992.

\_\_\_\_\_\_. Sobre música popular. In: \versal{COHN}, Gabriel. \emph{Theodor
Adorno: sociologia}. São Paulo: Ática, 1986. P. 68-71.

\versal{ADORNO}, Theodor W. \& \versal{HORKHEIMER}. \emph{Dialética do esclarecimento:
fragmentos filosóficos}. Tradução de Guido Antônio de Almeida. 2. Ed.
Rio de Janeiro: Jorge Zahar Editor, 1986.

\versal{ALGRANTI}, Leila Mezan. \emph{Famílias e vida doméstica. In:}
\emph{História da vida privada no Brasil.} Org. Laura de Mello Sousa.
São Paulo: Companhia das Letras, 1997.

\versal{ANDERS}, Gunter. \emph{L'obsolescende de l'homme.} Paris: Éditions
Ivrea, Paris. 2002.

\versal{ANDRADE}, Mário. Macunaíma: o herói sem nenhum caráter. Edição crítica de
Telê Porto Ancona Lopez. Rio de Janeiro: Livros Técnicos e Científicos,
1988.

\versal{ANDRADE}, Oswald. \emph{Do pau"-brasil à antropofagia e às utopias.} Rio
de Janeiro: Civilização Brasileira, 1978.

\_\_\_\_\_\_. \emph{Poesias Reunidas.} Rio de Janeiro: Civilização
brasileira, 1974.

\versal{ANDRADE}, Fábio Sousa. \emph{Matando o tempo: o impasse e a espera}.
\emph{In:} \versal{BECKETT}, Samuel, \emph{Fim de Partida}.Tradução de Fábio de
Sousa. São Paulo: Cosac Naify, 2010a.

\_\_\_\_\_\_. A felicidade desidratada. \emph{In}: \versal{BECKETT}, Samuel,
\emph{Dias Felizes}. Tradução de Fábio de Sousa. São Paulo~: Cosac
Naify, 2010b.

\emph{\_\_\_\_\_\_. Prefácio}. \emph{In:} \versal{BECKETT}, Samuel,
\emph{Esperando Godot}. Tradução de Fábio de Sousa. São Paulo: Cosac
Naify, 2005.

\_\_\_\_\_\_. \emph{Samuel Beckett: o silêncio do possível.} São
Paulo: Ateliê, 2001.

\versal{AQUINO}, Tomás de. \emph{La Somme Théologique de Saint Thomas.} Paris:
Librairie éclésiastique et classique, 1855.

\versal{ARENDT}, Hannah. \emph{Entre o passado e o futuro}. Tradução de Mauro
Barbosa. 6. ed. São Paulo: Perspectiva, 2009.

\_\_\_\_\_\_. \emph{A condição humana}. Trad. Roberto Raposo. 6. Ed.
Rio de Janeiro: Forense Universitária, 1993.

\versal{ASSARÉ}, Patativa do. \emph{Antologia Poética.} Org. Gilmar de
Carvalho. Fortaleza: Edições Demócrito Rocha, 2008.

\versal{BARTHES}, Roland. \emph{Écrits sur le théâtre}. Paris: Éditions du
Seuil, 2003.

\versal{BECKETT}, Samuel. \emph{Fim de Partida}. Tradução de Fábio de Sousa.
São Paulo: Cosac Naify, 2010a.

\_\_\_\_\_\_. \emph{Dias} \emph{Felizes.} Tradução de Fábio de
Sousa. São Paulo: Cosac Naify, 2010b.

\_\_\_\_\_\_. O fim. \emph{In}: \emph{Novelas.} Tradução de Eloísa
Araújo. São Paulo: Martins Fontes, 2006.

\_\_\_\_\_\_. \emph{Esperando Godot}. Tradução de Fábio de Sousa. São
Paulo: Cosac Naify, 2005.

\versal{BAUDRILLARD}, Jean. \emph{O sistema dos objetos}. Tradução de
Zumira Tavares. 5.ed. São paulo~: Editora Perspectiva, 2004 .

\versal{BENJAMIN}, Walter. \emph{Magia e técnica, arte e política}. In: Obras
Escolhidas. Tradução de Sérgio Paulo Rouanet. 7. ed. 12. reimp. São
Paulo: Brasiliense, 2010.

\_\_\_\_\_\_. \emph{Passagens.} Trad. de Irene Aron. Belo Horizonte:
Editora da \versal{UFMG}, 2006.

\versal{BAUMAN}, Zygmunt. \emph{Vida Para Consumo}. Tradução Carlos Albert
Medeiros. 1. Ed. Rio de Janeiro: Jorge Zahar Editor, 2008.

\versal{BOSI}, Alfredo. \emph{Situação de Macunaíma. In:} \versal{ANDRADE}, Mário.
\emph{Macunaíma: o herói sem nenhum caráter.} Edição crítica de Telê
Porto Ancona Lopez. Rio de Janeiro: Livros Técnicos e Científicos, 1988.

\versal{BOUVERESSE}, Jacques. \emph{La voix de l'âme et les chemins de
l'esprit. Dix études sur Robert Musil.} Paris: Seuil, 2001.

\versal{BRAUDEL}, Fernand. \emph{Civilisation matérielle, économie et
capitalisme -- \versal{XV}-\versal{XVIII} siècle.} (Vol. \versal{I} e \versal{III}). Paris: Armand Colin,
1979.

\versal{BRUCKNER}, Pascal. \emph{A euforia perpétua. Ensaios sobre o dever de
felicidade}. Tradução de Rejane Janowitzer. Rio de Janeiro:
\versal{DIFEL}, 2002.

\versal{BRUYELLE}, Roland. \emph{Les personages de la comédie de Molière.}
Paris: Rene Debresse éditeur, 1946.

\versal{BUCCI}, Eugênio, \versal{KEHL}, Maria Rita. \emph{Videologias}. 1 Ed. 1.
Reimp. São Paulo: Boitempo Editorial, 2005.

\versal{CAMPOS}, Haroldo de. \emph{Uma poética da radicalidade. In:} \versal{ANDRADE},
Oswald. \emph{Poesias Reunidas.} Rio de Janeiro: Civilização
brasileira, 1974.

\versal{CÂNDIDO}, Antônio. \emph{Formação da Literatura brasileira.} Belo
Horizonte~: Itatiaia, 2000.

\emph{\_\_\_\_\_\_. Dialética da Malandragem. Caracterização das
Memórias de um sargento de milícias}. in: Revista do Instituto de
estudos brasileiros, no 8, São Paulo, \versal{USP}, 1970, pp. 67-89.

\versal{CÂNDIDO}, Antônio \& \versal{CASTELLO}, Aderaldo. Presença da literatura
brasileira: modernismo. São Paulo: Difel, 1975.

\versal{CANOVA}, Marie"-Claude. \emph{Présentation et représentation dans Le
Bourgeois gentilhomme, ou le jeu des images et des rôles.} In:
\emph{Molière: trois comédies ``morales'' - Le misanthrope, George
Dandin, Le bourgeois gentilhomme.} Paris: Klincksieck, 1999.

\versal{COMETTI}, Jean Pierre. \emph{Musil Philosophe. L'utopie de
l'essayisme.} Paris: Éditions du Seuil, 2001.

\versal{CROSBY}, Alfred. \emph{A mensuração da realidade. A quantificação e a
sociedade ocidental 1250-1600.} Trad. Vera Ribeiro. São Paulo:
Unesp/Cambridge, 1999.

\versal{CUNHA}, Euclides da. \emph{Os sertões.} São Paulo: Ateliê editorial,
2001.

\versal{DAWLIANIDSE}, David. \emph{Transformation de la réalité poétique. In:}
\emph{Cahier de L'herne.} Paris: Éditions de L'herne, 1981.

\versal{DEBORD}, Guy. \emph{A sociedade do espetáculo}. Tradução Estela
dos Santos Abreu. 1. Ed. Rio de Janeiro: Contraponto, 1997.

\_\_\_\_\_\_\_\_. \emph{Oeuvres cinématographiques complètes}.
1. Ed. Paris: Gallimard, 1994.

\versal{DEFAUX}, Gérard. \emph{Rêve et réalité dans le bourgeois gentilhomme}.
In: \emph{Molière: trois comédies ``morales'' - Le misanthrope, George
Dandin, Le bourgeois gentilhomme.} Paris: Klincksieck, 1999.

\versal{DELEUZE}, Gilles \& \versal{GUATARI} Félix. \emph{O anti"-édipo. Capitalismo e
esquizofrenia 1.} Tradução de Joana Moraes e Manuel Maria. Lisboa:
Assírio \& Alvin, 2004.

\versal{DESCARTES}, René. \emph{Discours de la méthode.} Paris: Flammarion,
1966

\versal{DUARTE}, Rodrigo. \emph{Mímesis e Racionalidade}. São Paulo: Loyola,
1993.

\versal{DUBY}, Georges \& \versal{MANDROU}, Robert. \emph{Histoire de la civilisations
française -- \versal{XVII}-\versal{XX} siècles.} Paris: Armand Colin, 1958.

\versal{DUFOUR}, Dany"-Robert. \emph{O Divino mercado. A revolução cultural
liberal}. Tradução Procópio Abreu. 1. Ed. Rio de Janeiro:
Companhia de Freud, 2008.

\_\_\_\_\_\_\_. \emph{A arte de reduzir as cabeças: sobre a nova
servidão na sociedade ultraliberal}. Tradução de Sandra Regina
Felgueiras. Rio de Janeiro: Companhia de Freud editora, 2005.

\versal{DUMONT}, Louis. \emph{Homo} \emph{aequalis}. Trad. José Leonardo
Nascimento. Bauru: Edusc, 2000.

\versal{ECKERMANN}, Johann Peter. \emph{Conversation de Goethe}. Trad. Émile
Délerot. Paris: Cherpentier, 1863. Tome premier.

\versal{ELIAS}, Norbert. \emph{A sociedade dos indivíduos.} Trad. Vera Ribeiro.
Rio de Janeiro: Zahar, 1994.

\_\_\_\_\_\_\_. \emph{O processo civilizador.} Trad. Ruy Jungmann. Rio
de Janeiro: Zahar, 1993.

\versal{ELIOT}, Thomas S. \emph{Poemas.} São Paulo: Nova fronteira, 1981.

\versal{ESSLIN}, M. \emph{O Teatro do Absurdo}. Tradução de Bárbara
Heliodora. Rio de Janeiro: Jorge Zahar , 1968.

\versal{FILHO}, Ciro Marcondes. \emph{A linguagem da Sedução}. São Paulo:
Perspectiva, 1988.

\versal{FREYRE}, Gilberto. \emph{Casa Grande e Senzala}. São Paulo: Global
editora, 2003.

\_\_\_\_\_\_. \emph{Manifesto regionalista.} 7 ª ed. Recife: \versal{FUNDAJ},
ed. Massangana, 1996. p. 47-75. Acesso em 10 de dezembro de 2015:
\emph{http://www.ufrgs.br/cdrom/freyre/freyre.pdf}

\versal{FREUD}, Sigmund. \emph{O Mal"-Estar na civilização}. Tradução de
José Octávio de Aguiar Abreu. Rio de Janeiro: Companhia das Letras,
1930-2010a

\_\_\_\_\_\_. \emph{Introdução ao narcisismo. In:} \emph{Obras
completas.} Tradução Paulo César de Sousa. São Paulo: Companhia das
Letras, 1914-2010.

\_\_\_\_\_\_\_\_\_\_. \emph{Formulações sobre os dois princípios do
funcionamento mental (1911)}. Tradução Paulo César Lima. São Paulo:
Companhia das Letras, 2010. (\emph{Edição Standart Brasileira das Obras
Psicológicas Completas de Sigmund Freud}, vol. \versal{XII}).

\versal{FURTADO}. Celso. \emph{Formação econômica do Brasil.} São Paulo:
Companhia das letras, 2007.

\versal{GAGNEBIN}, J. M. Do conceito de \emph{mímesis} no pensamento de Adorno e
Benjamin. In: \emph{Revista Perspectiva}, Vol. 16, p. 67-86, São
Paulo, 1993.

\versal{GIMPEL}, Jean. \emph{La révolution industrielle du Moyen Âge.} Paris:
Éditions du Seuil, 1975.

\versal{GOLDMANN}, Lucien. \emph{Pour une sociologie du roman.} Paris:
Gallimard"-Collection Tel, 1986.

\versal{GOTO}, Roberto. Malandragem revisitada. Campinas: Pontes editora, 1988.

\versal{GOUREVICHT}, Aron. \emph{Le marchand.} \emph{In: L'Homme médieval.}
Paris: Éditions du Seuil, 1989.

\versal{GRACIÁN}, Baltasar. \emph{L'homme universel.} Trad. Joseph de
Courbeville. Paris: Éditions Ivrea, 1994.

\versal{GUICHARNAUD}, Jacques. \emph{Molière, une aventure théâtrale.} Paris:
Gallimard, 1963.

\versal{GYORY}, Jean. \emph{Homo Austriacus. In:} \emph{Cahier de L'herne.}
Paris: Éditions de L'herne, 1981.

\versal{HABBAKUK}, John. \emph{La disparition du paysan anglais. In:}
\emph{Annales, Économie, Société, Civilisation.} Vol. 20. Número 4.
pp.649-663. Ano 1965.

\versal{HANKE}, Michel. A qualidade de O homem sem qualidades de Robert Musil.
In~: Revista \versal{ALCEU} - v.4 - n.8 - p. 128 a 140 - jan./jun. 2004.

\versal{HARDMAN}, Francisco Foot. \emph{Antigos Modernistas.} \emph{In:} A
vingança de Hileia. São Paulo: Unesp. 2009.

\_\_\_\_\_\_. \emph{Algumas fantasias de Brasil: o modernismo paulista e
a nova naturalidade de nação. In.} Pelas Margens, outros caminhos da
história e da literatura. \versal{DECCA}, Salvadoria \& \versal{LEMAIRE}, Ria (\versal{ORG}.).
Campinas: Editora da Unicamp. 2000.

\versal{HEGEL}, G. W. F. \emph{La raison dans l'histoire} [A razão na
história] Trad. Kostas P.. Paris: \emph{Éditions} 10/18, 2007.

\versal{HOBSBAWM}, Eric. \emph{Da revolução industrial inglesa ao
imperialismo.} Tradução Donaldson Magalhães. 6ª ed. Rio de Janeiro:
Forense Universitária, 2014.

\_\_\_\_\_\_. \emph{Era dos Extremos. O breve séc. \versal{XX}, 1914-1991.}
Trad. Marcos Santarrita. São Paulo: Companhia das Letras, 1995.

\versal{HOLANDA}, Sérgio Buarque. \emph{Raízes do Brasil.} São Paulo: Companhia
das Letras, 2014.

\versal{HUIZINGA}, Johan. \emph{O outono da Idade Média.} Trad. Francis Petra
Hansen. São Paulo: Cosac Naify, 2010.

\versal{JACOBY}, Russel. \emph{Imagem Imperfeita: pensamento utópico para uma
época antiutópica}. Tradução de Carolina de Melo Bomfim Araújo.
Rio de Janeiro: Civilização Brasileira, 2007.

\versal{JAMESON}, Fredric. \emph{Modernidade Singular.} Rio de Janeiro:
Civilização Brasileira, 2005.

\_\_\_\_\_\_. \emph{O marxismo tardio. Adorno ou a persistência da
dialética.} Trad. Luiz Paulo Rouanet. São Paulo: Unesp, Boitempo, 1997.

\versal{JANCSÓ}, Istvan. \emph{A sedução da liberdade: cotidiano e contestação
política no final do século \versal{XVIII}. In:} \emph{História da vida privada
no Brasil.} Org. Laura de Mello Sousa. São Paulo: Companhia das Letras,
1997.

\versal{JAPPE}, Anselm. Será que o dinheiro nos pensa? In: Revista da Soc. Bras.
Economia Política, São Paulo, nº 33, p. 177-194, outubro 2012.

\_\_\_\_\_\_. \emph{Crédit à mort. La décomposition du capitalisme et
de ses critiques}. Fécamp~: Nouvelles éditions Lignes, 2011. [Crédito
à morte. São Paulo: Hedra, 2013]

\_\_\_\_\_\_\_. \emph{As aventuras da mercadoria}. Trad. José
Miranda Justo. Lisboa~: Antígona, 2006.

\versal{JÚNIOR}, Caio Prado. \emph{A revolução brasileira. A questão agrária no
Brasil.} São Paulo: Companhia das Letras, 2014

\_\_\_\_\_\_. \emph{Formação do Brasil contemporâneo.} São Paulo:
Companhia das Letras, 2011.

\versal{KANT}, I. \emph{Textos seletos}. (Raimundo V. e Floriano F. Trad.).
Petrópolis: Vozes, 2005.

\versal{KEEGAN}, John. \emph{Uma história da guerra.}Tradução Pedro Maia
Soares. São Paulo: Companhia das Letras, 2006.

\versal{KEHL}, Maria Rita. \emph{18 crônicas e mais algumas.} São Paulo:
Boitempo, 2011.

\_\_\_\_\_\_. \emph{O tempo e o cão. A atualidade das depressões.} São
Paulo: Boitempo, 2009.

\versal{KNOWLSON}, James. \emph{Note sur les images visuelles de \emph{Oh les
beaux jours}.} In: Littérature. Paris: Armand Collin. Nº 16, 2012/3.

\versal{KRAUS}, Karl. \emph{La littérature démolie.} Tradução Yves Kobry.
Paris: Payot, 1993.

\_\_\_\_\_\_ \emph{Les derniers jours de l'humanité.} Trad. Jean"-Louis
Besson \& Henri Christophe. Marseille: Agone, 2003.

\versal{KURZ}, Robert. \emph{Dinheiro sem valor. Linhas gerais para uma
transformação da crítica da economia política.} Tradução Lumir Nahodil.
Lisboa: Antígona, 2014.

\_\_\_\_\_\_. Dominação sem sujeito. Trad. Fernando Barros. In:
\_\_\_\_\_\_\_\_\_\_. \emph{Razão Sangrenta} (p. 129-211). São Paulo:
Hedra, 2010a.

\_\_\_\_\_\_. Ontologia negativa. Trad. Fernando Barros. In:
\_\_\_\_\_\_\_\_\_\_. \emph{Razão Sangrenta} (pp. 213-297). São Paulo:
Hedra, 2010b.

\_\_\_\_\_\_\_ . \emph{Cinzenta é a árvore dourada da vida e verde é a
teoria}. Trad. Tito Lívio R. Em Revista \emph{\versal{EXIT}!} 04. Acesso em 13 de
junho de 2013 de:
\emph{http://obeco.planetaclix.pt/exit\_indice\_geral.htm}.

\_\_\_\_\_\_. \emph{Lire Marx. Les textes les plus importants de Karl
Marx pour le \versal{XXI}ème siècle}. Tradução Hélène e Lucien Steinberg.
1. Ed. Paris: La Balustrade, 2002.

\_\_\_\_\_\_. \emph{O estouro da modernidade. Com tostões e canhões.}
Tradução de Lumir Nahodil. In: revista \emph{Jungle} \emph{World},
09.01.2002.

\_\_\_\_\_\_. \emph{Até a última gota}. Trad. José M. Macêdo\emph{.
Folha} \emph{de} \emph{São} \emph{Paulo -- Caderno} \emph{Mais}!, São
Paulo, 24 de agosto de 1997.

\versal{LAPLANCHE} \& \versal{PONTALIS}. \emph{Vocabulário de psicanálise.} Tradução
Pedro Tamen. São Paulo: Martins Fontes, 2001.

\versal{LASCH}, Christopher. Le Seul et Vrai Paradis : Une histoire de
l'idéologie du progrès et de ses critiques. Tradução Frédéric Joly.
Paris~: Flamarion,
2006. \emph{lasch-christopher.-le-seul-et-vrai-paradis-une-histoire-de-liduxe9ologie-du-progruxe8s-et-de-ses-critiques.-traduuxe7uxe3o-fruxe9duxe9ric-joly.-paris-flamarion-2006.}

\versal{LEBRUN}, Jean"-Pierre. \emph{Um mundo sem limite: ensaio para uma
crítica psicanalítica do social}. Tradução de Sandra Felgueiras .
Rio de Janeiro: Companhia de Freud editora, 2004.

\versal{LE GOFF}, Jacques. \emph{Marchands et banquiers du Moyen Âge}. Paris:
Presses Universitaires de France, 2011.

\_\_\_\_\_\_. \emph{La civilisation de l'occident médieval.} Paris:
Éditions Flammarion, 2008.

\_\_\_\_\_\_. Pour un autre moyen âge. \emph{In:} \emph{Un autre Moyen
Âge.} Collection Quarto. Paris: Gallimard, 1999a.

\_\_\_\_\_\_. La bourse et la vie. \emph{In:} \emph{Un autre Moyen
Âge.} Collection Quarto. Paris: Gallimard, 1999b.

\_\_\_\_\_\_\_\_\_\_\emph{Os intelectuais na Idade Média.} Trad.
Margarida Correia. 2ª ed. Lisboa: Gradiva, 1984.

\versal{LE RIDER}, Jacques. \emph{A modernidade vienense e as crises de
identidade.} São Paulo: Civilização Brasileira, 1993.

\versal{LESSA}, Bia. \emph{Sem qualidade. In:} \versal{MUSIL}, Robert. \emph{O homem sem
qualidades.} Rio de Janeiro: Nova Fronteira, 2006.

\versal{LIPOVETSKY}, Gilles. \emph{L'ère du vide} [A era do vazio]. Paris:
Gallimard, 2008.

\versal{LITTRÉ}, Émille. \emph{Dictionnaire de la langue française}. Paris:
Édition Pauvert, 1957. Tomo \versal{IV}.

\versal{LUCKÁCS}, Georg. \emph{A teoria do romance.} Trad. José Marcos Mariani.
São Paulo: Duas Cidades/Editora 34, 2009.

\versal{MAAR}, Leo W.. Educação Crítica, Formação Cultural e Emancipação Política
na Escola de Frankfurt. In: \versal{PUCCI}, Bruno (Org.). \emph{Teoria Crítica
e Educação}. Santa Catarina: \versal{UFSCAR}, 1994.

\versal{MAGRIS}, Claudio. \emph{L'Odyssée rectiligne de Robert Musil. In:}
\emph{Cahier de L'herne.} Paris: Éditions de L'herne, 1981.

\versal{MANDEVILLE}, Bernard. \emph{Uma investigação sobre a origem da virtude
moral. In:} \emph{Filosofia moral britânica. Textos do século \versal{XVIII}.}
Trad. Álvaro Cabral. Campinas: editora unicamp, 2013.

\versal{MARCUSE}, Herbert. \emph{Eros e Civilização}. Trad. De Álvaro Cabral.
8. Ed. [reimpr.]. Rio de Janeiro: \versal{LTC}, 2009.

\_\_\_\_\_\_. \emph{Política e psicanálise, o fim da utopia}. Trad. De
Bernardo Frederico. 2. Ed. Lisboa: Moraes Editores, 1980.

\_\_\_\_\_\_. \emph{Razão e} \emph{revolução}. Trad. Marília B.. Rio
de Janeiro~: Paz e Terra, 1978.

\_\_\_\_\_\_. \emph{A ideologia da sociedade industrial}. Tradução de
Giasone Rebuá. 4. ed. Rio de Janeiro: Zahar Editores, 1973.

\versal{MARFUZ}, Luiz. \emph{Beckett e a implosão da cena}. São Paulo: editora
Perspectiva, 2014.

\versal{MARX}, Karl. \emph{Grundrisse}. Tradução de Mário Duayer e Nélio
Schneider. São Paulo: Boitempo-\versal{UFRJ}, 2011.

\_\_\_\_\_\_. \emph{O capital}. Tradução Régis Barbosa e Flávio R.
Kothe. 2. Ed. São Paulo: Nova Cultural, 1985.

\_\_\_\_\_\_. \emph{Para a Crítica da Economia Política}. São Paulo:
Abril Cultural, 1982.

\_\_\_\_\_\_. \emph{Contribution à la critique de l'économie
politique}. Tradução de Maurice Husson e Gilbert Badia. Paris: Éditions
Sociales, 1972.

\versal{MARX}, Karl; \versal{ENGELS}, F. \emph{A ideologia alemã.} Trad. Rubens Enderle,
Nélio Schneider e Luciano Cavini Martorano. São Paulo: Boitempo, 2007.

\versal{MELMAN}, Charles. \emph{La nouvelle économie psychique: la façon de
penser et de jouir aujourd'hui} [\emph{A nova economia psíquica~: a
forma de pensar e de gozar hoje}]. Toulouse~: Éditions Ères, 2009.

\versal{MENEGAT}, Marildo. \emph{No labirinto espelhado: reflexões de Borges
sobre a barbárie.} In: \emph{Estados da plebe no capitalismo
contemporâneo.} São Paulo: Outra expressões, 2013.

\_\_\_\_\_\_. \emph{Estudo sobre a dialética civilização x barbárie na
tradição crítica brasileira.} Relatório de pós"-doutorado. São Paulo,
2011. Disponível em
http://www.fflch.usp.br/df/site/posdoc/docs/marildo\_menegat.pdf

\_\_\_\_\_\_. \emph{Depois do fim do mundo. A crise da modernidade e a
barbárie.} Rio de Janeiro: Relume Dumará, 2003.

\versal{MICHÉA}, Jean"-Claude. \emph{L'enseignement de l'ignorance et ses
conditions modernes.} Paris: Climats 2006.

\versal{MICHELET}, Jules. \emph{Histoire de France. Tome \versal{XV}ème.} Paris: Lacroix
\& C. Éditeurs, 1877.

\versal{MIRANDOLA}, Pico della. \emph{Discurso sobre a dignidade do homem.}
Trad. Maria de Lurdes Sirgado. Lisboa: Edições 70, 2006.

\versal{MOLIÈRE}, Jean Baptiste. \emph{Oeuvres complètes.} Paris: Bibliothèque
de la Pléiâde, 1962.

\_\_\_\_\_\_. \emph{Le bourgeois gentilhomme.} Paris: E. Dentu
Éditeur, 1893.

\_\_\_\_\_\_. \emph{Précieuses Ridicules}. Paris: E. Dentu Éditeur,
1893.

\_\_\_\_\_\_. \emph{L'Avare.} Paris: E. Dentu Éditeur, 1893.

\versal{MOTT}, Luiz. \emph{Cotidiano e vivência religiosa: entre a capela e o
calundu.} \emph{In:} \emph{História da vida privada no Brasil.} Org.
Laura de Mello Sousa. São Paulo: Companhia das Letras, 1997.

\versal{MUSIL}, Robert. \emph{O homem sem qualidades}. Tradução de Lya Luft e
Carlos Abbenseth. 1. ed. especial. Rio de Janeiro: Nova Fronteira, 2006.

\_\_\_\_\_\_. \emph{O jovem Törless}. Trad. Lya Luft. Rio de Janeiro:
Ed. Globo, 2003.

\_\_\_\_\_\_. \emph{Essais.} Trad. Phillippe Jacottet. Paris: Éditions
du Seuil, 1984.

\_\_\_\_\_\_. \emph{Journaux.} Trad. Philippe Jaccottet. Paris:
Éditions du Seuil, 1981a. [Tomo I]

\_\_\_\_\_\_. \emph{Journaux}. Trad. Philippe Jaccottet. Paris:
Éditions du Seuil, 1981b. [Tomo \versal{II}]

\_\_\_\_\_\_. \emph{Amour (Ébauche). In:} \emph{Cahier de L'herne.}
Paris: Éditions de L'herne, 1981.

\versal{NEF}, John. \emph{La guerre et le progrès humain.} Tradução Armand
Rebillon. Paris: Alsatia, 1954a.

\_\_\_\_\_\_. \emph{La naissance de la civilisation industrielle et le
monde contemporain}. Paris: Armand Colin, 1954b.

\versal{PARKER}, Geoffrey. \emph{La révolution militaire. La guerre et l'essor
de l'occident entre 1500 et 1800.} Paris: Gallimard"-Folio, 2013.

\versal{POSTONE}, Moishe. \emph{Temps, travail et domination sociale.} Tradução
Luc Mercier e Olivier Galtier. Paris: Fayard, 2009.

\versal{PRADO}, Antônio Arnoni. Intinerário de uma falsa vanguarda. São Paulo:
Editora 34. 2010.

\versal{PRONKO}, Léonard. \emph{Théâtre de l'avant"-garde. Beckett, Ionesco et
le théâtre expérimental em France.} Paris: Dénoël, 1963.

\versal{QUESNAY}, François. \emph{Cereais.} Tradução Sara Albieri. São Paulo:
Nova Cultural, 1983.

\versal{RACINE}, Jean. \emph{Phèdre. In:} \emph{Oeuvres complètes.} Paris:
Bibliothèque de la Pléiâde, 1969.

\versal{RAT}, Maurice. \emph{Molière, l'homme et l'oeuvre. In:} \versal{MOLIÈRE},
Jean"-Baptiste. \emph{Oeuvres complètes.} Paris: Bibliothèque de la
Pléiâde, 1962.

\_\_\_\_\_\_. \emph{Notes et variantes. In:} \versal{MOLIÈRE},
Jean"-Baptiste\emph{. Oeuvres complètes.} Paris: Bibliothèque de la
Pléiâde, 1962.

\versal{RICCI}, Maria Teresa. \emph{Du cortegiano au Discreto: l'homme accompli
chez Castiglione et Gracián.} Paris: Honoré Champion, 2009.

\versal{RIOUX}, Jean"-Pierre. \emph{La révolution industrielle -- 1780-1880.}
Paris: Éditions du seuil, 1971.

\versal{ROBERT}, Marthe. \emph{Médiation. In:} \emph{Cahier de L'herne.} Paris:
Éditions de L'herne, 1981.

\versal{ROBERT}, Paul. \emph{Dictionnaire alphabétique et analogique de la
langue française}. Paris: Société du nouveau Littré, 1962. Tomo V.

\versal{ROTH}, Marie"-Louise. \emph{Dans le carnaval de l'histoire.} \emph{In:}
\emph{Cahier de L'herne Musil.} Paris: Éditions de L'herne, 1981.

\versal{ROUANET}, Sérgio Paulo. \emph{Teoria crítica e psicanálise}. Rio de
Janeiro: Tempo Brasileiro, 2001.

\versal{ROUSSIAUD}, Jacques. \emph{Le citadin.} \emph{In: L'Homme médieval.}
Paris: Éditions du Seuil, 1989.

\versal{ROUX}, Edmonde Charles. \emph{Aspects de l'austrianité de Musil. In:}
\emph{Cahier de L'herne.} Paris: Éditions de L'herne, 1981.

\versal{SARTRE}, Jean"-Paul. \emph{Apresentação de Les Temps Modernes.} Revista
Praga. São Paulo: Editora Hucitec, 1999.

\versal{SCHOLZ}, Roswitha. \emph{O Valor é o homem}. Trad. De José Marcos
Macêdo. In: \emph{Novos} \emph{estudos} -- \versal{CEBRAP}, nº. 45. Jul"-1996, p.
15-36. Disponível em
\emph{www.obeco.planetaclix.pt}. Acesso em julho 2013.

\versal{SCHWARZ}, Roberto. \emph{Ao vencedor as batatas.} São Paulo: Duas
cidades; ed. 34, 2012

\_\_\_\_\_\_. \emph{Um mestre na periferia do capitalismo.} São Paulo:
Duas cidades; ed. 34, 2000.

\_\_\_\_\_\_. \emph{A carroça, o bonde e o poeta modernista. In:}
\emph{Que horas são?} São Paulo: Duas cidades; ed. 34, 1987.

\_\_\_\_\_\_. \emph{Pressupostos, salvo engano, de ``Dialética da
malandragem''. In}: \emph{Que horas são?} São Paulo: Duas cidades; ed.
34, 1987.

\versal{SEVERIANO}, Maria de Fátima. \emph{Narcisismo e publicidade. Uma
análise psicossocial dos ideais de consumo na contemporaneidade}. 2. Ed.
São Paulo: Annablume, 2007.

\versal{SIMMEL}. Georg. \emph{Philosophie de l'argent.} Trad. de Sabine Cornile
e Philippe Ivernel. 2ª edição, 4ª reimp. Paris: Presses Universitaires
de France, 2009.

\versal{SMITH}, Adam. \emph{A riqueza das nações. Vol I.} Tradução Luiz João
Baraúna. São Paulo: Nova Cultural, 1983.

\versal{SOARES}, J. C. Sobrevivendo como vaga"-lumes. Reflexões sobre o tempo
d'\emph{O homem sem qualidades} de Robert Musil e o homem ``2.0'',
versão acelerada, hipermoderna. In: \emph{Tempo e subjetividades:
perspectivas plurais.} Org. \versal{EWALD}, Ariane; \versal{SOARES}, J.C.; \versal{SEVERIANO}, M.
Fátima; \versal{AQUINO}, Cássio B. Rio de Janeiro: 7 Letras, 2013.

\_\_\_\_\_\_\_. Escola de Frankfurt: unindo materialismo e psicanálise
na construção de uma psicologia social marginal. Em Ana Maria J., Arthur
A. e Francisco T. (Orgs.), \emph{História da psicologia: rumos e
percursos} (p. 473-501). Rio de Janeiro: Nau Editora, 2008.

\versal{SOARES}, Jorge C. \& \versal{EWALD}, Ariane. \emph{Reflexões à sombra de
Adorno}. \emph{In: Revista Nomadas,} 2004. p. 1-12.

\versal{SOKEL}. Walter. \emph{Musil et l'existencialisme. In:} \emph{Cahier de
L'herne.} Paris: Éditions de L'herne, 1981.

\versal{SOUZA}, Gilda de Melo e. O tupi e o alaúde: uma interpretação de
Macunaíma. São Paulo: Duas Cidades, 1979.

\versal{TIEDEMANN}, Rolf. \emph{Introdução.}In: \emph{Passages.} \versal{BENJAMIN}, W.
Paris: Éditions du Cerf, 1997.

\versal{TODOROV}, Tzvetan. \emph{A literatura em perigo.} Tra. Caio Meira. Rio
de Janeiro: Difel, 2009.

\versal{TRENKLE}, Norbert. Negatividade interrompida, notas sobre a crítica de
Horkheimer e Adorno a Kant e ao Esclarecimento. Trad. José P. V. In:
\emph{Revista Krisis} \emph{25} (2002). Disponível em
\emph{http://adorno.planetaclix.pt/ntrenkle.htm}.
Acesso em março de 2013.

Velloso, Mônica. A brasilidade verde"-amarela: nacionalismo e
regionalismo paulista. In: \emph{Revista Estudos históricos.} Vol. 6,
ed.11. pp. 15-36, 1993.

\versal{WEBB}, Eugène. \emph{As peças de Samuel Beckett.} Tradução de Pedro
Sette"-Câmara. São Paulo: Editora Biblioteca Teatral, 2012.

\versal{WEBER}, Max. \emph{A ética protestante e o espírito do capitalismo.}
Tradução José Marcos Mariani. São Paulo: Companhia das Letras, 2013.

\versal{WOOD}, Ellen. \emph{As origens agrárias do capitalismo. In:}
\emph{Crítica marxista\emph{. }}Nº 10. pp. 12-29. 2000.
\end{Parskip}